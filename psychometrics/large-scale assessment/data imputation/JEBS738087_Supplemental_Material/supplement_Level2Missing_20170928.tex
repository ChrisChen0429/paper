\documentclass[doc,12pt]{apa6}

% main font and language
\usepackage[T1]{fontenc}
\usepackage[american]{babel}
\usepackage{newtxtext,newtxmath}
\usepackage{setspace} % on for custom linespacing
\setstretch{1.5} % set linespacing

%** interactive PDF
\usepackage{hyperref}
%\hypersetup{ % links in colored text
%     colorlinks = true,
%     linkcolor = cyan,
%     citecolor = cyan
%}
\hypersetup{ % links in colored boxes/lines
    pdfborderstyle={/S/U/W 1}, % underline of width 1pt
    colorlinks = false,
    linkbordercolor = {0 0.5 1},
    citebordercolor = {0 0.5 1},
    urlbordercolor = {1 1 1}
} 

%** reference list
\usepackage[natbibapa]{apacite}
\usepackage{url}

%** math mode
\usepackage{amsmath}
\usepackage{array}
\usepackage[low-sup]{subdepth}

%** table tools
\usepackage{booktabs}
\usepackage{caption}
\usepackage{dcolumn}
\usepackage[figuresleft]{rotating}
\captionsetup[table]{font={stretch=1.25}, aboveskip=2pt}
% tables at top on floats-only pages
\makeatletter
\setlength{\@fptop}{0pt}
\setlength{\@fpbot}{0pt plus 1fil}
\makeatother
% multi-line cells
\usepackage{makecell}
\renewcommand\cellset{\renewcommand\arraystretch{0.8}%
\setlength\extrarowheight{0pt}}

%** graphic tools
%\usepackage{graphicx}
%\usepackage[export]{adjustbox} % large figure scaling
\captionsetup[figure]{font={normalsize, stretch=1.5}}

%** code tools
\usepackage{color}
\usepackage{zi4}
\usepackage{listings}

%** code blocks ans inline code
\definecolor{white}{gray}{1.00}
\definecolor{mediumgray}{gray}{0.44}
\definecolor{mediumgreen}{RGB}{85,130,0}
\definecolor{mediumcyan}{RGB}{0,115,115}
\definecolor{darkorange}{RGB}{130,75,0}

\lstdefinelanguage{Rsupp}{%
  language = R,
  otherkeywords = {!=,$,\&,\%/\%,\%*\%,\%\%},
  morekeywords = [1]{data.frame, as.integer, is.na, with, within, lmer, mice, panImpute, jomoImpute, mids2mitml.list, jomo2mitml.list, testEstimates, testModels, testConstraints, clusterMeans, mitmlComplete},
  morekeywords = [2]{!=, $, \&, \%/\%, \%*\%, \%\%, NULL, NA, NaN, TRUE, FALSE},
  deletekeywords = {drop, package, trace, print, formula},
  morecomment = [f][\color{mediumgray}][0]>,
  alsoletter = {.}
}
\lstdefinelanguage{mplus}{
  morekeywords={data,variable,define,analysis,model},
  sensitive=false,
  morecomment=[l]{!}, % inline comment
}
\lstdefinelanguage{bugs}{
  language = R,
  keywords = {for, dnorm, dmnorm, dunif, dwish, pow, inverse, mean},
  otherkeywords={},
  morekeywords = {},
  morekeywords = [2]{},
  deletekeywords = {},
  alsoletter={.}
}

\lstset{ 
  basewidth=0.99ex,
  aboveskip=2.8ex plus 0.2ex minus 0.5ex,
  belowskip=0.6ex plus 0.2ex minus 0.5ex,
  lineskip=-0.0ex,
  breaklines=true,
  breakatwhitespace=true,
  breakindent=10pt,
  keepspaces=true,
  showstringspaces=false,
  language=Rsupp
}
\lstdefinestyle{rcode}{
  basicstyle=\linespread{0.95}\footnotesize\ttfamily,
  backgroundcolor=\color{white},
  commentstyle=\color{mediumgray}\mdseries\footnotesize\ttfamily,
  stringstyle=\color{darkorange}\mdseries\footnotesize\ttfamily,
  keywordstyle=[1]\color{mediumgreen}\mdseries\footnotesize\ttfamily,
  keywordstyle=[2]\color{mediumcyan}\mdseries\footnotesize\ttfamily,
  rulecolor=\color{white},
  frame=trbl,
  framesep=6pt,
  xleftmargin=0pt
}
\lstdefinestyle{mplus}{
  basicstyle=\linespread{0.95}\footnotesize\ttfamily,
  backgroundcolor=\color{white},
  rulecolor=\color{white},
  frame=trbl,
  framesep=6pt,
  xleftmargin=0pt,
  language=mplus
}
\lstdefinestyle{bugs}{
  basicstyle=\linespread{0.95}\footnotesize\ttfamily,
  backgroundcolor=\color{white},
  commentstyle=\color{mediumgray}\mdseries\footnotesize\ttfamily,
  stringstyle=\color{darkorange}\mdseries\footnotesize\ttfamily,
  keywordstyle=\color{mediumgreen}\mdseries\footnotesize\ttfamily,
  rulecolor=\color{white},
  frame=trbl,
  framesep=6pt,
  xleftmargin=0pt,
  language=bugs
}
\lstdefinestyle{plain}{
  basicstyle=\linespread{0.95}\footnotesize\ttfamily,
  commentstyle=\color{black}\mdseries\footnotesize\ttfamily,
  keywordstyle=\color{black}\mdseries\footnotesize\ttfamily,
  rulecolor=\color{white},
  frame=trbl,
  framesep=6pt,
  xleftmargin=0pt
}\newcommand{\mystrut}{\vrule height 7pt depth 2pt width 0pt}
\newcommand{\lstbox}[1]{\colorbox{background}{\mystrut\lstinline[style=rcode]?#1?}}

% breiter Balken in mathmode
\newcommand\widebar[1]{%
  \hbox{%
    \vbox{%
      \hrule height 0.6pt% 
      \kern0.256ex%
      \hbox{%
        \kern-0.15em%
        \ensuremath{#1}%
        \kern-0.1em%
      }%
    }%
  }%
} 

% griechische Buchstaben APA-konform (aus txgreeks-Paket)
\DeclareMathSymbol{\beta}{\mathalpha}{lettersA}{12}
\DeclareMathSymbol{\delta}{\mathalpha}{lettersA}{14}
\DeclareMathSymbol{\epsilon}{\mathalpha}{lettersA}{15}
\DeclareMathSymbol{\varepsilon}{\mathalpha}{lettersA}{34}
\DeclareMathSymbol{\theta}{\mathalpha}{lettersA}{18}
\DeclareMathSymbol{\omega}{\mathalpha}{lettersA}{33}
\DeclareMathSymbol{\rho}{\mathalpha}{lettersA}{26}
\DeclareMathSymbol{\upsilon}{\mathalpha}{lettersA}{29}
\DeclareMathSymbol{\tau}{\mathalpha}{lettersA}{28}
\DeclareMathSymbol{\mu}{\mathalpha}{lettersA}{22}
\DeclareMathSymbol{\nu}{\mathalpha}{lettersA}{23}
\DeclareMathSymbol{\psi}{\mathalpha}{lettersA}{32}
\DeclareMathSymbol{\kappa}{\mathalpha}{lettersA}{20}
\DeclareMathSymbol{\lambda}{\mathalpha}{lettersA}{21}
\DeclareMathSymbol{\phi}{\mathalpha}{lettersA}{30}
\DeclareMathSymbol{\pi}{\mathalpha}{lettersA}{25}
\DeclareMathSymbol{\sigma}{\mathalpha}{lettersA}{27}
\DeclareMathSymbol{\alpha}{\mathalpha}{lettersA}{11}
\DeclareMathSymbol{\gamma}{\mathalpha}{lettersA}{13}
\DeclareMathSymbol{\chi}{\mathalpha}{lettersA}{31}
\DeclareMathSymbol{\eta}{\mathalpha}{lettersA}{17}
\DeclareMathSymbol{\zeta}{\mathalpha}{lettersA}{16}

\title{Multiple Imputation of Missing Data at Level 2:\\ A Comparison of Fully Conditional and Joint Modeling in Multilevel Designs}
\author{\\[1ex]Simon Grund$^{1,2}$, Oliver L\"udtke$^{1,2}$, and Alexander Robitzsch$^{1,2}$}
\affiliation{\\[3ex]$^1$Leibniz Institute for Science and Mathematics Education, Kiel, Germany\\[1ex]
$^2$Centre for International Student Assessment, Germany}

\shorttitle{Supplemental Online Material}

%%%
\begin{document}

~\\[4ex]\doublespace{ \maketitle }

\begin{center}\begin{Large}~\\[8ex]
\bf{Supplemental Online Material}
\end{Large}\end{center}

\newpage 
\setcounter{page}{1}

\noindent
Enclosed in this document are the supplemental materials for our article entitled ``Multiple Imputation of Missing Data at Level 2: A Comparison of Fully Conditional and Joint Modeling in Multilevel Designs.''
\hyperref[sec:supc]{Supplement A} contains the computer code and M\emph{plus} syntax file used in the empirical example.
\hyperref[sec:supb]{Supplement B} contains an additional simulation study that compares an ``empirical Bayes'' and a fully Bayesian procedure for FCS-LAT.
\hyperref[sec:supa]{Supplement C} contains an additional simulation study that contrasts the use of least informative and data-dependent priors in JM.
\hyperref[sec:supd]{Supplement D} contains additional tables that include the complete results of Studies 1 and 2.

\section{Supplement A: Computer Code for the Empirical Example}
\label{sec:supa}

Below, we provide the computer code for the statistical software R \citep{RCoreTeam2016} and the syntax file for the statistical software M\emph{plus} that were used in the empirical example in the main article.
The data set comprised the German subsample of the Programme for International Student Assessment (PISA; \citealp{OECD2014}). These data are available online free of charge (\url{https://www.oecd.org/pisa}).

\lstinputlisting[style=rcode]{listings/sup-Example_PISA.R} % \end{verb}

\noindent
Following the imputation, the completed data sets were saved in a series of text files.
These files were analyzed with the statistical software M\emph{plus} \citep{Muthen2012} using the syntax file given below.

\lstinputlisting[style=mplus]{listings/sup-Example_MplusSyntax.inp} % \end{verb}

\clearpage

\section{Supplement B: Fully Bayesian Procedure for FCS-LAT}
\label{sec:supb}

As an alternative to using an ``empirical Bayes'' procedure during the generation of plausible values of latent cluster means, a fully Bayesian procedure may be used that draws the model parameters from their posterior distributions.
The resulting procedure is ``Bayesianly proper'' in the sense of \citet{Rubin1987} and may improve efficiency and coverage properties in smaller samples.
In the following simulation study, we evaluated the differences between the empirical Bayes and the fully Bayesian procedure for the generation of latent cluster means.
Both procedures were implemented using the R package \texttt{miceadds} \citep{Robitzsch2017}.
\begin{table}[bt]
\begin{threeparttable}
\setlength{\tabcolsep}{1.4pt}
\renewcommand{\arraystretch}{1.00}
\small
\caption{\small Bias (in \%), Relative RMSE, and Coverage of the 95\% Confidence Interval for the Covariance of $y$ with $z$ ($\hat{\sigma}_{yz}$) and the Regression Coefficients of $y$ on $z$ and $z$ on $y$ ($\hat\beta_{yz}$ and $\hat\beta_{zy}$) for Small ICC of $y$ ($\rho_{Iy}=.10$) and 20\% Missing Data (MAR, $\lambda=0.5$)\label{tab:sup-fb}}
\begin{tabular}{lccccccccc}
\hline\\[-2.2ex]
 & \multicolumn{3}{c}{Bias (\%)} & \multicolumn{3}{c}{Rel. RMSE} & \multicolumn{3}{c}{Coverage (\%)} \\ \cmidrule(r){2-4}\cmidrule(r){5-7}\cmidrule(r){8-10}
  & \makecell{\footnotesize FCS-MAN} & \makecell{\footnotesize FCS-LAT\\(EB)} & \makecell{\footnotesize FCS-LAT\\(FB)} & \makecell{\footnotesize FCS-MAN} & \makecell{\footnotesize FCS-LAT\\(EB)} & \makecell{\footnotesize FCS-LAT\\(FB)} & \makecell{\footnotesize FCS-MAN} & \makecell{\footnotesize FCS-LAT\\(EB)} & \multicolumn{1}{c}{\makecell{\footnotesize FCS-LAT\\(FB)}} \\ 
[0.2ex]\hline\\[-2.0ex]
& \multicolumn{9}{c}{Covariance $y$ with $z$ ($\hat\sigma_{yz}$)} \\[0.4ex]\hline\\[-2.0ex]
\nopagebreak $J=30$  & $\phantom{-}0.9\phantom{0}$ & $\phantom{-}5.0\phantom{0}$ & ${-}3.7\phantom{0}$ & $\phantom{0}0.789\phantom{0}$ & $\phantom{0}0.809\phantom{0}$ & $\phantom{0}0.743\phantom{0}$ & $\phantom{0}93.8\phantom{0}$ & $\phantom{0}93.4\phantom{0}$ & $\phantom{0}94.1\phantom{0}$ \\
\nopagebreak $J=50$  & ${-}0.3\phantom{0}$ & $\phantom{-}3.6\phantom{0}$ & ${-}1.6\phantom{0}$ & $\phantom{0}0.584\phantom{0}$ & $\phantom{0}0.604\phantom{0}$ & $\phantom{0}0.565\phantom{0}$ & $\phantom{0}94.5\phantom{0}$ & $\phantom{0}93.9\phantom{0}$ & $\phantom{0}95.0\phantom{0}$ \\
\nopagebreak $J=100$  & $\phantom{-}0.1\phantom{0}$ & $\phantom{-}3.0\phantom{0}$ & $\phantom{-}0.3\phantom{0}$ & $\phantom{0}0.419\phantom{0}$ & $\phantom{0}0.426\phantom{0}$ & $\phantom{0}0.406\phantom{0}$ & $\phantom{0}94.3\phantom{0}$ & $\phantom{0}93.8\phantom{0}$ & $\phantom{0}94.7\phantom{0}$ \\
[0.5ex]\hline\\[-2.0ex] 
& \multicolumn{9}{c}{Regression $y \sim z$ ($\hat\beta_{yz}$)} \\[0.4ex]\hline\\[-2.0ex]
\nopagebreak $J=30$  & ${-}4.7\phantom{0}$ & ${-}1.2\phantom{0}$ & ${-}7.5\phantom{0}$ & $\phantom{0}0.707\phantom{0}$ & $\phantom{0}0.726\phantom{0}$ & $\phantom{0}0.688\phantom{0}$ & $\phantom{0}91.5\phantom{0}$ & $\phantom{0}90.1\phantom{0}$ & $\phantom{0}93.6\phantom{0}$ \\
\nopagebreak $J=50$  & ${-}3.0\phantom{0}$ & $\phantom{-}0.1\phantom{0}$ & ${-}4.0\phantom{0}$ & $\phantom{0}0.532\phantom{0}$ & $\phantom{0}0.544\phantom{0}$ & $\phantom{0}0.519\phantom{0}$ & $\phantom{0}92.4\phantom{0}$ & $\phantom{0}91.7\phantom{0}$ & $\phantom{0}94.4\phantom{0}$ \\
\nopagebreak $J=100$  & ${-}1.5\phantom{0}$ & $\phantom{-}0.8\phantom{0}$ & ${-}1.4\phantom{0}$ & $\phantom{0}0.382\phantom{0}$ & $\phantom{0}0.389\phantom{0}$ & $\phantom{0}0.376\phantom{0}$ & $\phantom{0}93.4\phantom{0}$ & $\phantom{0}92.7\phantom{0}$ & $\phantom{0}94.3\phantom{0}$ \\
[0.5ex]\hline\\[-2.0ex] 
& \multicolumn{9}{c}{Regression $z \sim y$ ($\hat\beta_{zy}$)} \\[0.4ex]\hline\\[-2.0ex]
\nopagebreak $J=30$  & ${-}10.2\phantom{0}$ & $\phantom{0}{-}6.7\phantom{0}$ & ${-}10.8\phantom{0}$ & $\phantom{0}0.957\phantom{0}$ & $\phantom{0}0.989\phantom{0}$ & $\phantom{0}0.968\phantom{0}$ & $\phantom{0}92.8\phantom{0}$ & $\phantom{0}93.3\phantom{0}$ & $\phantom{0}94.5\phantom{0}$ \\
\nopagebreak $J=50$  & $\phantom{0}{-}4.3\phantom{0}$ & $\phantom{0}{-}0.9\phantom{0}$ & $\phantom{0}{-}4.3\phantom{0}$ & $\phantom{0}0.791\phantom{0}$ & $\phantom{0}0.793\phantom{0}$ & $\phantom{0}0.795\phantom{0}$ & $\phantom{0}94.0\phantom{0}$ & $\phantom{0}94.0\phantom{0}$ & $\phantom{0}95.0\phantom{0}$ \\
\nopagebreak $J=100$  & $\phantom{0}{-}1.3\phantom{0}$ & $\phantom{0}\phantom{-}1.4\phantom{0}$ & $\phantom{0}{-}1.1\phantom{0}$ & $\phantom{0}0.601\phantom{0}$ & $\phantom{0}0.596\phantom{0}$ & $\phantom{0}0.582\phantom{0}$ & $\phantom{0}93.9\phantom{0}$ & $\phantom{0}94.0\phantom{0}$ & $\phantom{0}94.5\phantom{0}$ \\
[0.5ex]\hline\\[-1.8ex] 
\end{tabular}
\begin{tablenotes}[para,flushleft]{\footnotesize \textit{Note.} $J$ = number of clusters; FCS-MAN = two-level FCS with manifest cluster means; FCS-LAT = two-level FCS with latent cluster means (empirical Bayes); FCS-LAT = two-level FCS with latent cluster means (fully Bayesian).}\end{tablenotes}
\end{threeparttable}
\end{table}

The simulated conditions were similar to those in Study 1 in the main article and were intended to match the conditions displayed in Figure 2 and Table 2 with smaller samples ($n=5$, $J=30$, 50, 100).

The results of this simulation are provided in Table \ref{tab:sup-fb}.
When FCS-LAT was based on an empirical Bayes procedure (EB), we observed coverage rates for the regression coefficient of $y$ on $z$ ($\hat{\beta}_{yz}$) below the nominal value of 95\% in very small samples ($J \leq 50$).
On the other hand, when FCS-LAT was based on a fully Bayesian procedure (FB), the coverage rates were nearly optimal for all parameters and even in very small samples.
In addition, FCS-LAT (FB) yielded lower values for the RMSE than FCS-MAN or FCS-LAT (EB) in these conditions, indicating that the parameters were estimated with greater accuracy overall.
By contrast, the results for the bias were usually best under FCS-LAT (EB), whereas the bias under FCS-LAT (FB) was slightly higher for the two regression coefficients ($\hat{\beta}_{yz}$ and $\hat{\beta}_{zy}$).

\section{Supplement C: Data-Dependent Priors in JM}
\label{sec:supc}

\noindent
Even though the use of data-dependent priors (DDPs) is not without criticism \citep[e.g., see][]{Gelman2014}, it has been recommended that DDPs be used to mitigate problems in the estimation of multilevel models in smaller samples \citep[e.g.,][]{McNeish2016,Grund2016}.
In the following simulation study, we explored the effects of using DDPs for MI of missing data at Level 2 using JM.
In order to specify DDPs, we estimated the covariance matrix of the variables at Level 1 ($\hat{\boldsymbol\Sigma}_1$) and Level 2 ($\hat{\boldsymbol\Sigma}_2$) from the complete data using the formulae provided by \citet{Muthen1994}.
To ensure that these matrices were positive-definite, the variances were constrained to be larger than zero, and the covariances were constrained in such a way that they implied a correlation between -1 and 1.
Using these estimates, we set the scale matrices of the inverse-Wishart priors to $\mathbf{S}_1=\nu_1 \hat{\boldsymbol\Sigma}_1$ for the covariance matrix at Level 1 and $\mathbf{S}_2=\nu_2 \hat{\boldsymbol\Sigma}_2$ for the covariance matrix at Level 2, where the degrees of freedom $\nu_1$ and $\nu_2$ were set to match the dimensions of $\hat{\boldsymbol\Sigma}_1$ and $\hat{\boldsymbol\Sigma}_2$, respectively (i.e., largest possible dispersion; see \citealp{Schafer2002a}).
The remaining parameters of the simulation were a subset of the conditions in Study 1 ($n=5$; $J=30$, 50, 100, 200, 500, 1000; $\rho_{Iy}=.10$, .30; $\lambda=0.5$; 20\% missing data).

\begin{figure}[t]
  \centering
  \fitfigure{figures/sup-Figure_Bias_Mean-Var-Cov_md20-lam50.pdf}
  \caption{\small Estimated bias for the mean and the variance of $z$ ($\hat\mu_{z}$ and $\hat\sigma_{z}^{2}$) and the covariance of $y$ with $z$ ($\hat\sigma_{yz}$) for varying numbers of clusters ($J$) and ICCs of $y$ ($\rho_{Iy}$) and 20\% missing data (MAR, $\lambda=0.5$). LD = listwise deletion; FCS-MAN = two-level FCS with manifest group means; JM = joint modeling with least-informative priors; JM-DDP = joint modeling with data-dependent priors.}
  \label{fig:sup-ddp}
\end{figure}

The results for the bias in the mean and variance of $z$ ($\hat\mu_z$ and $\hat\sigma_z^2$), and the covariance of $y$ with $z$ ($\hat\sigma_{yz}$) are presented in Figure \ref{fig:sup-ddp}.
As in the main study, all procedures provided essentially unbiased estimates of the mean and variance of $z$.
However, to provide unbiased estimates of the covariance of $y$ with $z$, JM required slightly larger samples than FCS-MAN, especially when the ICC of $y$ was low ($J\geq200$ with $\rho_{Iy}$).
By contrast, the use of data-dependent priors (JM-DDP) led to a noticeable decrease in bias; that is, much smaller samples were required for the bias in the parameter estimates to vanish ($J\geq100$).
These results illustrate that the performance of JM may be substantially improved by employing data-dependent priors or, alternatively, by formulating a more reasonable ``prior guess'' of the variances and covariances at Level 1 and 2 on the basis of prior knowledge rather than relying on the standard least-informative priors.

\section{Supplement D: Additional Tables}
\label{sec:supd}

\noindent
This section presents the complete results of the simulation studies reported in the main article.
Note that the bias is calculated here in a ``raw'' metric for all procedures, that is, with the ``true'' values in the data-generating model as a point of reference.
This is in contrast to the main article, where the bias was calculated on the basis of the results obtained from the complete data sets.
The results are organized in a series of tables numbered according to the simulation study. Tables 1-1 to 1-30 contain the results of Study 1, and Tables 2-1 to 2-40 contain the results of Study 2.

\clearpage
\captionsetup[table]{font={stretch=1.00}, aboveskip=4pt}

\setcounter{table}{0}
\renewcommand{\thetable}{1-\arabic{table}}
\begin{sidewaystable}
\begin{threeparttable}
\setlength{\tabcolsep}{1.2pt}
\renewcommand{\arraystretch}{0.95}
\footnotesize
\caption{\small Study 1: Bias, RMSE, and Coverage of the 95\% Confidence Interval for the Mean of $z$ ($\hat\mu_z$) With 20\% Missing Data (MCAR, $\lambda=0$)}
\begin{tabular}{llcccccccccccccccccc}
\hline\\[-1.8ex]
& & \multicolumn{6}{c}{Bias (\%)} & \multicolumn{6}{c}{RMSE} & \multicolumn{6}{c}{Coverage (\%)} \\ \cmidrule(r){3-8}\cmidrule(r){9-14}\cmidrule(r){15-20}
 &  & CD & LD & \makecell{FCS-\\SL} & \makecell{FCS-\\MAN} & \makecell{FCS-\\LAT} & JM & CD & LD & \makecell{FCS-\\SL} & \makecell{FCS-\\MAN} & \makecell{FCS-\\LAT} & JM & CD & LD & \makecell{FCS-\\SL} & \makecell{FCS-\\MAN} & \makecell{FCS-\\LAT} & \multicolumn{1}{c}{JM} \\ 
[0.4ex]\hline\\[-1.8ex]
& & \multicolumn{18}{c}{Small intraclass correlation $(\rho_{Iy}=.10)$} \\[0.6ex]\hline\\[-1.8ex]
\multicolumn{4}{l}{$n=5$} \\  & \nopagebreak $\;J=30$  & ${-}0.00\phantom{0}$ & $\phantom{-}0.00\phantom{0}$ & $\phantom{-}0.00\phantom{0}$ & $\phantom{-}0.00\phantom{0}$ & $\phantom{-}0.00\phantom{0}$ & $\phantom{-}0.00\phantom{0}$ & $\phantom{0}0.18\phantom{0}$ & $\phantom{0}0.20\phantom{0}$ & $\phantom{0}0.20\phantom{0}$ & $\phantom{0}0.20\phantom{0}$ & $\phantom{0}0.20\phantom{0}$ & $\phantom{0}0.20\phantom{0}$ & $\phantom{0}94.1\phantom{0}$ & $\phantom{0}93.8\phantom{0}$ & $\phantom{0}89.5\phantom{0}$ & $\phantom{0}94.5\phantom{0}$ & $\phantom{0}94.5\phantom{0}$ & $\phantom{0}94.2\phantom{0}$ \\
 & \nopagebreak $\;J=50$  & ${-}0.00\phantom{0}$ & ${-}0.00\phantom{0}$ & ${-}0.00\phantom{0}$ & ${-}0.00\phantom{0}$ & ${-}0.00\phantom{0}$ & ${-}0.00\phantom{0}$ & $\phantom{0}0.14\phantom{0}$ & $\phantom{0}0.15\phantom{0}$ & $\phantom{0}0.15\phantom{0}$ & $\phantom{0}0.15\phantom{0}$ & $\phantom{0}0.15\phantom{0}$ & $\phantom{0}0.15\phantom{0}$ & $\phantom{0}94.9\phantom{0}$ & $\phantom{0}94.9\phantom{0}$ & $\phantom{0}91.9\phantom{0}$ & $\phantom{0}95.3\phantom{0}$ & $\phantom{0}95.1\phantom{0}$ & $\phantom{0}94.9\phantom{0}$ \\
 & \nopagebreak $\;J=100$  & ${-}0.00\phantom{0}$ & ${-}0.00\phantom{0}$ & ${-}0.00\phantom{0}$ & ${-}0.00\phantom{0}$ & ${-}0.00\phantom{0}$ & ${-}0.00\phantom{0}$ & $\phantom{0}0.10\phantom{0}$ & $\phantom{0}0.11\phantom{0}$ & $\phantom{0}0.11\phantom{0}$ & $\phantom{0}0.11\phantom{0}$ & $\phantom{0}0.11\phantom{0}$ & $\phantom{0}0.11\phantom{0}$ & $\phantom{0}95.3\phantom{0}$ & $\phantom{0}94.7\phantom{0}$ & $\phantom{0}90.7\phantom{0}$ & $\phantom{0}94.9\phantom{0}$ & $\phantom{0}94.7\phantom{0}$ & $\phantom{0}95.2\phantom{0}$ \\
 & \nopagebreak $\;J=200$  & $\phantom{-}0.00\phantom{0}$ & ${-}0.00\phantom{0}$ & ${-}0.00\phantom{0}$ & $\phantom{-}0.00\phantom{0}$ & $\phantom{-}0.00\phantom{0}$ & ${-}0.00\phantom{0}$ & $\phantom{0}0.07\phantom{0}$ & $\phantom{0}0.08\phantom{0}$ & $\phantom{0}0.08\phantom{0}$ & $\phantom{0}0.08\phantom{0}$ & $\phantom{0}0.08\phantom{0}$ & $\phantom{0}0.08\phantom{0}$ & $\phantom{0}94.6\phantom{0}$ & $\phantom{0}94.6\phantom{0}$ & $\phantom{0}89.9\phantom{0}$ & $\phantom{0}94.5\phantom{0}$ & $\phantom{0}94.6\phantom{0}$ & $\phantom{0}95.2\phantom{0}$ \\
 & \nopagebreak $\;J=500$  & $\phantom{-}0.00\phantom{0}$ & ${-}0.00\phantom{0}$ & ${-}0.00\phantom{0}$ & ${-}0.00\phantom{0}$ & ${-}0.00\phantom{0}$ & ${-}0.00\phantom{0}$ & $\phantom{0}0.04\phantom{0}$ & $\phantom{0}0.05\phantom{0}$ & $\phantom{0}0.05\phantom{0}$ & $\phantom{0}0.05\phantom{0}$ & $\phantom{0}0.05\phantom{0}$ & $\phantom{0}0.05\phantom{0}$ & $\phantom{0}95.5\phantom{0}$ & $\phantom{0}95.0\phantom{0}$ & $\phantom{0}91.3\phantom{0}$ & $\phantom{0}95.4\phantom{0}$ & $\phantom{0}95.4\phantom{0}$ & $\phantom{0}95.5\phantom{0}$ \\
 & \nopagebreak $\;J=1000$  & ${-}0.00\phantom{0}$ & ${-}0.00\phantom{0}$ & ${-}0.00\phantom{0}$ & ${-}0.00\phantom{0}$ & ${-}0.00\phantom{0}$ & ${-}0.00\phantom{0}$ & $\phantom{0}0.03\phantom{0}$ & $\phantom{0}0.04\phantom{0}$ & $\phantom{0}0.04\phantom{0}$ & $\phantom{0}0.04\phantom{0}$ & $\phantom{0}0.04\phantom{0}$ & $\phantom{0}0.04\phantom{0}$ & $\phantom{0}94.7\phantom{0}$ & $\phantom{0}94.4\phantom{0}$ & $\phantom{0}89.7\phantom{0}$ & $\phantom{0}94.3\phantom{0}$ & $\phantom{0}93.7\phantom{0}$ & $\phantom{0}94.8\phantom{0}$ \\
\multicolumn{4}{l}{$n=20$} \\  & \nopagebreak $\;J=30$  & ${-}0.00\phantom{0}$ & ${-}0.00\phantom{0}$ & ${-}0.00\phantom{0}$ & ${-}0.00\phantom{0}$ & ${-}0.00\phantom{0}$ & ${-}0.00\phantom{0}$ & $\phantom{0}0.18\phantom{0}$ & $\phantom{0}0.20\phantom{0}$ & $\phantom{0}0.20\phantom{0}$ & $\phantom{0}0.20\phantom{0}$ & $\phantom{0}0.20\phantom{0}$ & $\phantom{0}0.20\phantom{0}$ & $\phantom{0}93.9\phantom{0}$ & $\phantom{0}93.2\phantom{0}$ & $\phantom{0}87.7\phantom{0}$ & $\phantom{0}94.1\phantom{0}$ & $\phantom{0}94.4\phantom{0}$ & $\phantom{0}94.3\phantom{0}$ \\
 & \nopagebreak $\;J=50$  & ${-}0.00\phantom{0}$ & ${-}0.00\phantom{0}$ & ${-}0.00\phantom{0}$ & ${-}0.00\phantom{0}$ & ${-}0.00\phantom{0}$ & ${-}0.00\phantom{0}$ & $\phantom{0}0.15\phantom{0}$ & $\phantom{0}0.16\phantom{0}$ & $\phantom{0}0.16\phantom{0}$ & $\phantom{0}0.16\phantom{0}$ & $\phantom{0}0.16\phantom{0}$ & $\phantom{0}0.16\phantom{0}$ & $\phantom{0}93.7\phantom{0}$ & $\phantom{0}93.1\phantom{0}$ & $\phantom{0}86.9\phantom{0}$ & $\phantom{0}93.9\phantom{0}$ & $\phantom{0}93.3\phantom{0}$ & $\phantom{0}93.9\phantom{0}$ \\
 & \nopagebreak $\;J=100$  & $\phantom{-}0.00\phantom{0}$ & $\phantom{-}0.00\phantom{0}$ & $\phantom{-}0.00\phantom{0}$ & $\phantom{-}0.00\phantom{0}$ & $\phantom{-}0.00\phantom{0}$ & $\phantom{-}0.00\phantom{0}$ & $\phantom{0}0.10\phantom{0}$ & $\phantom{0}0.11\phantom{0}$ & $\phantom{0}0.11\phantom{0}$ & $\phantom{0}0.11\phantom{0}$ & $\phantom{0}0.11\phantom{0}$ & $\phantom{0}0.11\phantom{0}$ & $\phantom{0}93.9\phantom{0}$ & $\phantom{0}94.3\phantom{0}$ & $\phantom{0}86.7\phantom{0}$ & $\phantom{0}94.7\phantom{0}$ & $\phantom{0}93.8\phantom{0}$ & $\phantom{0}93.9\phantom{0}$ \\
 & \nopagebreak $\;J=200$  & ${-}0.00\phantom{0}$ & ${-}0.00\phantom{0}$ & ${-}0.00\phantom{0}$ & ${-}0.00\phantom{0}$ & ${-}0.00\phantom{0}$ & ${-}0.00\phantom{0}$ & $\phantom{0}0.07\phantom{0}$ & $\phantom{0}0.08\phantom{0}$ & $\phantom{0}0.08\phantom{0}$ & $\phantom{0}0.08\phantom{0}$ & $\phantom{0}0.08\phantom{0}$ & $\phantom{0}0.08\phantom{0}$ & $\phantom{0}95.1\phantom{0}$ & $\phantom{0}95.1\phantom{0}$ & $\phantom{0}89.0\phantom{0}$ & $\phantom{0}95.3\phantom{0}$ & $\phantom{0}95.6\phantom{0}$ & $\phantom{0}96.0\phantom{0}$ \\
 & \nopagebreak $\;J=500$  & ${-}0.00\phantom{0}$ & ${-}0.00\phantom{0}$ & ${-}0.00\phantom{0}$ & ${-}0.00\phantom{0}$ & ${-}0.00\phantom{0}$ & ${-}0.00\phantom{0}$ & $\phantom{0}0.04\phantom{0}$ & $\phantom{0}0.05\phantom{0}$ & $\phantom{0}0.05\phantom{0}$ & $\phantom{0}0.05\phantom{0}$ & $\phantom{0}0.05\phantom{0}$ & $\phantom{0}0.05\phantom{0}$ & $\phantom{0}96.7\phantom{0}$ & $\phantom{0}95.4\phantom{0}$ & $\phantom{0}89.2\phantom{0}$ & $\phantom{0}96.1\phantom{0}$ & $\phantom{0}95.5\phantom{0}$ & $\phantom{0}96.1\phantom{0}$ \\
 & \nopagebreak $\;J=1000$  & $\phantom{-}0.00\phantom{0}$ & $\phantom{-}0.00\phantom{0}$ & $\phantom{-}0.00\phantom{0}$ & $\phantom{-}0.00\phantom{0}$ & $\phantom{-}0.00\phantom{0}$ & $\phantom{-}0.00\phantom{0}$ & $\phantom{0}0.03\phantom{0}$ & $\phantom{0}0.03\phantom{0}$ & $\phantom{0}0.03\phantom{0}$ & $\phantom{0}0.03\phantom{0}$ & $\phantom{0}0.03\phantom{0}$ & $\phantom{0}0.03\phantom{0}$ & $\phantom{0}94.8\phantom{0}$ & $\phantom{0}95.8\phantom{0}$ & $\phantom{0}88.5\phantom{0}$ & $\phantom{0}95.1\phantom{0}$ & $\phantom{0}96.0\phantom{0}$ & $\phantom{0}95.7\phantom{0}$ \\
[0.5ex]\hline\\[-1.6ex] 
& & \multicolumn{18}{c}{Moderate intraclass correlation $(\rho_{Iy}=.30)$} \\[0.6ex]\hline\\[-1.8ex]
\multicolumn{4}{l}{$n=5$} \\  & \nopagebreak $\;J=30$  & $\phantom{-}0.00\phantom{0}$ & $\phantom{-}0.00\phantom{0}$ & $\phantom{-}0.00\phantom{0}$ & $\phantom{-}0.00\phantom{0}$ & $\phantom{-}0.00\phantom{0}$ & $\phantom{-}0.00\phantom{0}$ & $\phantom{0}0.19\phantom{0}$ & $\phantom{0}0.21\phantom{0}$ & $\phantom{0}0.21\phantom{0}$ & $\phantom{0}0.21\phantom{0}$ & $\phantom{0}0.21\phantom{0}$ & $\phantom{0}0.21\phantom{0}$ & $\phantom{0}92.5\phantom{0}$ & $\phantom{0}92.5\phantom{0}$ & $\phantom{0}88.1\phantom{0}$ & $\phantom{0}93.4\phantom{0}$ & $\phantom{0}93.0\phantom{0}$ & $\phantom{0}92.5\phantom{0}$ \\
 & \nopagebreak $\;J=50$  & $\phantom{-}0.00\phantom{0}$ & ${-}0.00\phantom{0}$ & $\phantom{-}0.00\phantom{0}$ & $\phantom{-}0.00\phantom{0}$ & $\phantom{-}0.00\phantom{0}$ & $\phantom{-}0.00\phantom{0}$ & $\phantom{0}0.14\phantom{0}$ & $\phantom{0}0.15\phantom{0}$ & $\phantom{0}0.15\phantom{0}$ & $\phantom{0}0.15\phantom{0}$ & $\phantom{0}0.15\phantom{0}$ & $\phantom{0}0.15\phantom{0}$ & $\phantom{0}94.4\phantom{0}$ & $\phantom{0}95.2\phantom{0}$ & $\phantom{0}90.9\phantom{0}$ & $\phantom{0}95.1\phantom{0}$ & $\phantom{0}95.1\phantom{0}$ & $\phantom{0}94.7\phantom{0}$ \\
 & \nopagebreak $\;J=100$  & $\phantom{-}0.00\phantom{0}$ & $\phantom{-}0.00\phantom{0}$ & $\phantom{-}0.00\phantom{0}$ & $\phantom{-}0.00\phantom{0}$ & $\phantom{-}0.00\phantom{0}$ & $\phantom{-}0.00\phantom{0}$ & $\phantom{0}0.10\phantom{0}$ & $\phantom{0}0.12\phantom{0}$ & $\phantom{0}0.12\phantom{0}$ & $\phantom{0}0.12\phantom{0}$ & $\phantom{0}0.12\phantom{0}$ & $\phantom{0}0.12\phantom{0}$ & $\phantom{0}94.1\phantom{0}$ & $\phantom{0}93.3\phantom{0}$ & $\phantom{0}88.3\phantom{0}$ & $\phantom{0}92.6\phantom{0}$ & $\phantom{0}93.3\phantom{0}$ & $\phantom{0}93.1\phantom{0}$ \\
 & \nopagebreak $\;J=200$  & $\phantom{-}0.00\phantom{0}$ & $\phantom{-}0.00\phantom{0}$ & $\phantom{-}0.00\phantom{0}$ & $\phantom{-}0.00\phantom{0}$ & $\phantom{-}0.00\phantom{0}$ & $\phantom{-}0.00\phantom{0}$ & $\phantom{0}0.07\phantom{0}$ & $\phantom{0}0.08\phantom{0}$ & $\phantom{0}0.08\phantom{0}$ & $\phantom{0}0.08\phantom{0}$ & $\phantom{0}0.08\phantom{0}$ & $\phantom{0}0.08\phantom{0}$ & $\phantom{0}95.5\phantom{0}$ & $\phantom{0}95.7\phantom{0}$ & $\phantom{0}91.8\phantom{0}$ & $\phantom{0}95.1\phantom{0}$ & $\phantom{0}95.1\phantom{0}$ & $\phantom{0}95.3\phantom{0}$ \\
 & \nopagebreak $\;J=500$  & ${-}0.00\phantom{0}$ & ${-}0.00\phantom{0}$ & ${-}0.00\phantom{0}$ & ${-}0.00\phantom{0}$ & ${-}0.00\phantom{0}$ & ${-}0.00\phantom{0}$ & $\phantom{0}0.05\phantom{0}$ & $\phantom{0}0.05\phantom{0}$ & $\phantom{0}0.05\phantom{0}$ & $\phantom{0}0.05\phantom{0}$ & $\phantom{0}0.05\phantom{0}$ & $\phantom{0}0.05\phantom{0}$ & $\phantom{0}95.3\phantom{0}$ & $\phantom{0}95.5\phantom{0}$ & $\phantom{0}90.3\phantom{0}$ & $\phantom{0}94.8\phantom{0}$ & $\phantom{0}95.1\phantom{0}$ & $\phantom{0}94.1\phantom{0}$ \\
 & \nopagebreak $\;J=1000$  & $\phantom{-}0.00\phantom{0}$ & $\phantom{-}0.00\phantom{0}$ & $\phantom{-}0.00\phantom{0}$ & $\phantom{-}0.00\phantom{0}$ & $\phantom{-}0.00\phantom{0}$ & $\phantom{-}0.00\phantom{0}$ & $\phantom{0}0.03\phantom{0}$ & $\phantom{0}0.04\phantom{0}$ & $\phantom{0}0.04\phantom{0}$ & $\phantom{0}0.04\phantom{0}$ & $\phantom{0}0.04\phantom{0}$ & $\phantom{0}0.04\phantom{0}$ & $\phantom{0}93.9\phantom{0}$ & $\phantom{0}95.1\phantom{0}$ & $\phantom{0}88.8\phantom{0}$ & $\phantom{0}95.2\phantom{0}$ & $\phantom{0}95.6\phantom{0}$ & $\phantom{0}94.8\phantom{0}$ \\
\multicolumn{4}{l}{$n=20$} \\  & \nopagebreak $\;J=30$  & ${-}0.00\phantom{0}$ & ${-}0.00\phantom{0}$ & ${-}0.00\phantom{0}$ & ${-}0.00\phantom{0}$ & ${-}0.00\phantom{0}$ & ${-}0.00\phantom{0}$ & $\phantom{0}0.18\phantom{0}$ & $\phantom{0}0.20\phantom{0}$ & $\phantom{0}0.20\phantom{0}$ & $\phantom{0}0.20\phantom{0}$ & $\phantom{0}0.20\phantom{0}$ & $\phantom{0}0.20\phantom{0}$ & $\phantom{0}93.7\phantom{0}$ & $\phantom{0}93.8\phantom{0}$ & $\phantom{0}86.2\phantom{0}$ & $\phantom{0}94.1\phantom{0}$ & $\phantom{0}94.3\phantom{0}$ & $\phantom{0}93.9\phantom{0}$ \\
 & \nopagebreak $\;J=50$  & ${-}0.00\phantom{0}$ & ${-}0.00\phantom{0}$ & ${-}0.00\phantom{0}$ & ${-}0.00\phantom{0}$ & ${-}0.00\phantom{0}$ & ${-}0.00\phantom{0}$ & $\phantom{0}0.14\phantom{0}$ & $\phantom{0}0.16\phantom{0}$ & $\phantom{0}0.16\phantom{0}$ & $\phantom{0}0.16\phantom{0}$ & $\phantom{0}0.16\phantom{0}$ & $\phantom{0}0.16\phantom{0}$ & $\phantom{0}93.2\phantom{0}$ & $\phantom{0}93.9\phantom{0}$ & $\phantom{0}87.6\phantom{0}$ & $\phantom{0}93.9\phantom{0}$ & $\phantom{0}94.0\phantom{0}$ & $\phantom{0}93.9\phantom{0}$ \\
 & \nopagebreak $\;J=100$  & ${-}0.00\phantom{0}$ & $\phantom{-}0.00\phantom{0}$ & $\phantom{-}0.00\phantom{0}$ & $\phantom{-}0.00\phantom{0}$ & $\phantom{-}0.00\phantom{0}$ & ${-}0.00\phantom{0}$ & $\phantom{0}0.10\phantom{0}$ & $\phantom{0}0.11\phantom{0}$ & $\phantom{0}0.11\phantom{0}$ & $\phantom{0}0.11\phantom{0}$ & $\phantom{0}0.11\phantom{0}$ & $\phantom{0}0.11\phantom{0}$ & $\phantom{0}94.8\phantom{0}$ & $\phantom{0}93.9\phantom{0}$ & $\phantom{0}88.3\phantom{0}$ & $\phantom{0}93.9\phantom{0}$ & $\phantom{0}94.5\phantom{0}$ & $\phantom{0}94.9\phantom{0}$ \\
 & \nopagebreak $\;J=200$  & ${-}0.00\phantom{0}$ & ${-}0.00\phantom{0}$ & ${-}0.00\phantom{0}$ & ${-}0.00\phantom{0}$ & ${-}0.00\phantom{0}$ & ${-}0.00\phantom{0}$ & $\phantom{0}0.07\phantom{0}$ & $\phantom{0}0.08\phantom{0}$ & $\phantom{0}0.08\phantom{0}$ & $\phantom{0}0.08\phantom{0}$ & $\phantom{0}0.08\phantom{0}$ & $\phantom{0}0.08\phantom{0}$ & $\phantom{0}93.9\phantom{0}$ & $\phantom{0}92.9\phantom{0}$ & $\phantom{0}86.9\phantom{0}$ & $\phantom{0}93.3\phantom{0}$ & $\phantom{0}93.2\phantom{0}$ & $\phantom{0}93.5\phantom{0}$ \\
 & \nopagebreak $\;J=500$  & ${-}0.00\phantom{0}$ & ${-}0.00\phantom{0}$ & ${-}0.00\phantom{0}$ & ${-}0.00\phantom{0}$ & ${-}0.00\phantom{0}$ & ${-}0.00\phantom{0}$ & $\phantom{0}0.04\phantom{0}$ & $\phantom{0}0.05\phantom{0}$ & $\phantom{0}0.05\phantom{0}$ & $\phantom{0}0.05\phantom{0}$ & $\phantom{0}0.05\phantom{0}$ & $\phantom{0}0.05\phantom{0}$ & $\phantom{0}95.3\phantom{0}$ & $\phantom{0}94.6\phantom{0}$ & $\phantom{0}89.3\phantom{0}$ & $\phantom{0}94.7\phantom{0}$ & $\phantom{0}94.4\phantom{0}$ & $\phantom{0}94.4\phantom{0}$ \\
 & \nopagebreak $\;J=1000$  & $\phantom{-}0.00\phantom{0}$ & $\phantom{-}0.00\phantom{0}$ & $\phantom{-}0.00\phantom{0}$ & $\phantom{-}0.00\phantom{0}$ & $\phantom{-}0.00\phantom{0}$ & $\phantom{-}0.00\phantom{0}$ & $\phantom{0}0.03\phantom{0}$ & $\phantom{0}0.03\phantom{0}$ & $\phantom{0}0.03\phantom{0}$ & $\phantom{0}0.03\phantom{0}$ & $\phantom{0}0.03\phantom{0}$ & $\phantom{0}0.03\phantom{0}$ & $\phantom{0}95.6\phantom{0}$ & $\phantom{0}95.1\phantom{0}$ & $\phantom{0}89.9\phantom{0}$ & $\phantom{0}95.0\phantom{0}$ & $\phantom{0}95.3\phantom{0}$ & $\phantom{0}95.7\phantom{0}$ \\
[0.5ex]\hline\\[-1.6ex] 
\end{tabular}
\begin{tablenotes}[para,flushleft]{\footnotesize \textit{Note.} $n$ = cluster size; $J$ = number of clusters; CD = complete data sets; LD = listwise deletion; FCS-SL = single-level FCS; FCS-MAN = two-level FCS with manifest cluster means; FCS-LAT = two-level FCS with latent cluster means; JM = joint modeling.}\end{tablenotes}
\end{threeparttable}
\end{sidewaystable}
\begin{sidewaystable}
\begin{threeparttable}
\setlength{\tabcolsep}{1.2pt}
\renewcommand{\arraystretch}{0.95}
\footnotesize
\caption{\small Study 1: Bias, RMSE, and Coverage of the 95\% Confidence Interval for the Mean of $z$ ($\hat\mu_z$) With 20\% Missing Data (MAR, $\lambda=0.5$)}
\begin{tabular}{llcccccccccccccccccc}
\hline\\[-1.8ex]
& & \multicolumn{6}{c}{Bias (\%)} & \multicolumn{6}{c}{RMSE} & \multicolumn{6}{c}{Coverage (\%)} \\ \cmidrule(r){3-8}\cmidrule(r){9-14}\cmidrule(r){15-20}
 &  & CD & LD & \makecell{FCS-\\SL} & \makecell{FCS-\\MAN} & \makecell{FCS-\\LAT} & JM & CD & LD & \makecell{FCS-\\SL} & \makecell{FCS-\\MAN} & \makecell{FCS-\\LAT} & JM & CD & LD & \makecell{FCS-\\SL} & \makecell{FCS-\\MAN} & \makecell{FCS-\\LAT} & \multicolumn{1}{c}{JM} \\ 
[0.4ex]\hline\\[-1.8ex]
& & \multicolumn{18}{c}{Small intraclass correlation $(\rho_{Iy}=.10)$} \\[0.6ex]\hline\\[-1.8ex]
\multicolumn{4}{l}{$n=5$} \\  & \nopagebreak $\;J=30$  & ${-}0.00\phantom{0}$ & ${-}0.06\phantom{0}$ & ${-}0.05\phantom{0}$ & ${-}0.01\phantom{0}$ & ${-}0.00\phantom{0}$ & ${-}0.03\phantom{0}$ & $\phantom{0}0.18\phantom{0}$ & $\phantom{0}0.22\phantom{0}$ & $\phantom{0}0.21\phantom{0}$ & $\phantom{0}0.21\phantom{0}$ & $\phantom{0}0.21\phantom{0}$ & $\phantom{0}0.21\phantom{0}$ & $\phantom{0}92.4\phantom{0}$ & $\phantom{0}91.9\phantom{0}$ & $\phantom{0}86.6\phantom{0}$ & $\phantom{0}93.6\phantom{0}$ & $\phantom{0}93.4\phantom{0}$ & $\phantom{0}93.0\phantom{0}$ \\
 & \nopagebreak $\;J=50$  & $\phantom{-}0.00\phantom{0}$ & ${-}0.05\phantom{0}$ & ${-}0.04\phantom{0}$ & $\phantom{-}0.00\phantom{0}$ & $\phantom{-}0.01\phantom{0}$ & ${-}0.02\phantom{0}$ & $\phantom{0}0.14\phantom{0}$ & $\phantom{0}0.16\phantom{0}$ & $\phantom{0}0.16\phantom{0}$ & $\phantom{0}0.16\phantom{0}$ & $\phantom{0}0.16\phantom{0}$ & $\phantom{0}0.16\phantom{0}$ & $\phantom{0}94.5\phantom{0}$ & $\phantom{0}93.6\phantom{0}$ & $\phantom{0}88.4\phantom{0}$ & $\phantom{0}95.5\phantom{0}$ & $\phantom{0}95.9\phantom{0}$ & $\phantom{0}95.4\phantom{0}$ \\
 & \nopagebreak $\;J=100$  & ${-}0.00\phantom{0}$ & ${-}0.05\phantom{0}$ & ${-}0.04\phantom{0}$ & ${-}0.00\phantom{0}$ & $\phantom{-}0.00\phantom{0}$ & ${-}0.01\phantom{0}$ & $\phantom{0}0.10\phantom{0}$ & $\phantom{0}0.12\phantom{0}$ & $\phantom{0}0.12\phantom{0}$ & $\phantom{0}0.11\phantom{0}$ & $\phantom{0}0.11\phantom{0}$ & $\phantom{0}0.11\phantom{0}$ & $\phantom{0}94.5\phantom{0}$ & $\phantom{0}91.8\phantom{0}$ & $\phantom{0}87.6\phantom{0}$ & $\phantom{0}94.5\phantom{0}$ & $\phantom{0}94.5\phantom{0}$ & $\phantom{0}94.7\phantom{0}$ \\
 & \nopagebreak $\;J=200$  & ${-}0.00\phantom{0}$ & ${-}0.05\phantom{0}$ & ${-}0.04\phantom{0}$ & ${-}0.00\phantom{0}$ & $\phantom{-}0.00\phantom{0}$ & ${-}0.01\phantom{0}$ & $\phantom{0}0.07\phantom{0}$ & $\phantom{0}0.09\phantom{0}$ & $\phantom{0}0.09\phantom{0}$ & $\phantom{0}0.08\phantom{0}$ & $\phantom{0}0.08\phantom{0}$ & $\phantom{0}0.08\phantom{0}$ & $\phantom{0}95.5\phantom{0}$ & $\phantom{0}90.2\phantom{0}$ & $\phantom{0}87.7\phantom{0}$ & $\phantom{0}94.7\phantom{0}$ & $\phantom{0}94.8\phantom{0}$ & $\phantom{0}95.5\phantom{0}$ \\
 & \nopagebreak $\;J=500$  & ${-}0.00\phantom{0}$ & ${-}0.05\phantom{0}$ & ${-}0.04\phantom{0}$ & ${-}0.00\phantom{0}$ & ${-}0.00\phantom{0}$ & ${-}0.00\phantom{0}$ & $\phantom{0}0.04\phantom{0}$ & $\phantom{0}0.07\phantom{0}$ & $\phantom{0}0.06\phantom{0}$ & $\phantom{0}0.05\phantom{0}$ & $\phantom{0}0.05\phantom{0}$ & $\phantom{0}0.05\phantom{0}$ & $\phantom{0}96.8\phantom{0}$ & $\phantom{0}81.8\phantom{0}$ & $\phantom{0}80.6\phantom{0}$ & $\phantom{0}95.4\phantom{0}$ & $\phantom{0}95.0\phantom{0}$ & $\phantom{0}95.2\phantom{0}$ \\
 & \nopagebreak $\;J=1000$  & ${-}0.00\phantom{0}$ & ${-}0.05\phantom{0}$ & ${-}0.04\phantom{0}$ & ${-}0.00\phantom{0}$ & ${-}0.00\phantom{0}$ & ${-}0.00\phantom{0}$ & $\phantom{0}0.03\phantom{0}$ & $\phantom{0}0.06\phantom{0}$ & $\phantom{0}0.05\phantom{0}$ & $\phantom{0}0.04\phantom{0}$ & $\phantom{0}0.04\phantom{0}$ & $\phantom{0}0.04\phantom{0}$ & $\phantom{0}94.7\phantom{0}$ & $\phantom{0}67.7\phantom{0}$ & $\phantom{0}70.1\phantom{0}$ & $\phantom{0}94.8\phantom{0}$ & $\phantom{0}94.6\phantom{0}$ & $\phantom{0}95.0\phantom{0}$ \\
\multicolumn{4}{l}{$n=20$} \\  & \nopagebreak $\;J=30$  & ${-}0.00\phantom{0}$ & ${-}0.07\phantom{0}$ & ${-}0.06\phantom{0}$ & $\phantom{-}0.00\phantom{0}$ & $\phantom{-}0.01\phantom{0}$ & ${-}0.03\phantom{0}$ & $\phantom{0}0.18\phantom{0}$ & $\phantom{0}0.22\phantom{0}$ & $\phantom{0}0.21\phantom{0}$ & $\phantom{0}0.21\phantom{0}$ & $\phantom{0}0.21\phantom{0}$ & $\phantom{0}0.21\phantom{0}$ & $\phantom{0}93.1\phantom{0}$ & $\phantom{0}91.9\phantom{0}$ & $\phantom{0}83.7\phantom{0}$ & $\phantom{0}93.5\phantom{0}$ & $\phantom{0}93.2\phantom{0}$ & $\phantom{0}92.6\phantom{0}$ \\
 & \nopagebreak $\;J=50$  & $\phantom{-}0.00\phantom{0}$ & ${-}0.07\phantom{0}$ & ${-}0.06\phantom{0}$ & $\phantom{-}0.00\phantom{0}$ & $\phantom{-}0.01\phantom{0}$ & ${-}0.02\phantom{0}$ & $\phantom{0}0.15\phantom{0}$ & $\phantom{0}0.18\phantom{0}$ & $\phantom{0}0.17\phantom{0}$ & $\phantom{0}0.16\phantom{0}$ & $\phantom{0}0.17\phantom{0}$ & $\phantom{0}0.16\phantom{0}$ & $\phantom{0}94.1\phantom{0}$ & $\phantom{0}90.7\phantom{0}$ & $\phantom{0}83.1\phantom{0}$ & $\phantom{0}94.5\phantom{0}$ & $\phantom{0}94.3\phantom{0}$ & $\phantom{0}94.1\phantom{0}$ \\
 & \nopagebreak $\;J=100$  & $\phantom{-}0.00\phantom{0}$ & ${-}0.07\phantom{0}$ & ${-}0.06\phantom{0}$ & $\phantom{-}0.00\phantom{0}$ & $\phantom{-}0.00\phantom{0}$ & ${-}0.01\phantom{0}$ & $\phantom{0}0.10\phantom{0}$ & $\phantom{0}0.13\phantom{0}$ & $\phantom{0}0.12\phantom{0}$ & $\phantom{0}0.11\phantom{0}$ & $\phantom{0}0.11\phantom{0}$ & $\phantom{0}0.11\phantom{0}$ & $\phantom{0}94.7\phantom{0}$ & $\phantom{0}90.5\phantom{0}$ & $\phantom{0}82.7\phantom{0}$ & $\phantom{0}95.5\phantom{0}$ & $\phantom{0}95.1\phantom{0}$ & $\phantom{0}95.1\phantom{0}$ \\
 & \nopagebreak $\;J=200$  & $\phantom{-}0.01\phantom{0}$ & ${-}0.07\phantom{0}$ & ${-}0.06\phantom{0}$ & $\phantom{-}0.00\phantom{0}$ & $\phantom{-}0.00\phantom{0}$ & ${-}0.00\phantom{0}$ & $\phantom{0}0.07\phantom{0}$ & $\phantom{0}0.11\phantom{0}$ & $\phantom{0}0.10\phantom{0}$ & $\phantom{0}0.08\phantom{0}$ & $\phantom{0}0.08\phantom{0}$ & $\phantom{0}0.08\phantom{0}$ & $\phantom{0}94.2\phantom{0}$ & $\phantom{0}84.1\phantom{0}$ & $\phantom{0}76.8\phantom{0}$ & $\phantom{0}94.4\phantom{0}$ & $\phantom{0}93.8\phantom{0}$ & $\phantom{0}94.4\phantom{0}$ \\
 & \nopagebreak $\;J=500$  & ${-}0.00\phantom{0}$ & ${-}0.08\phantom{0}$ & ${-}0.07\phantom{0}$ & ${-}0.00\phantom{0}$ & ${-}0.00\phantom{0}$ & ${-}0.01\phantom{0}$ & $\phantom{0}0.04\phantom{0}$ & $\phantom{0}0.09\phantom{0}$ & $\phantom{0}0.08\phantom{0}$ & $\phantom{0}0.05\phantom{0}$ & $\phantom{0}0.05\phantom{0}$ & $\phantom{0}0.05\phantom{0}$ & $\phantom{0}95.4\phantom{0}$ & $\phantom{0}66.4\phantom{0}$ & $\phantom{0}59.4\phantom{0}$ & $\phantom{0}94.5\phantom{0}$ & $\phantom{0}93.8\phantom{0}$ & $\phantom{0}94.3\phantom{0}$ \\
 & \nopagebreak $\;J=1000$  & $\phantom{-}0.00\phantom{0}$ & ${-}0.07\phantom{0}$ & ${-}0.06\phantom{0}$ & ${-}0.00\phantom{0}$ & $\phantom{-}0.00\phantom{0}$ & ${-}0.00\phantom{0}$ & $\phantom{0}0.03\phantom{0}$ & $\phantom{0}0.08\phantom{0}$ & $\phantom{0}0.07\phantom{0}$ & $\phantom{0}0.04\phantom{0}$ & $\phantom{0}0.04\phantom{0}$ & $\phantom{0}0.04\phantom{0}$ & $\phantom{0}96.1\phantom{0}$ & $\phantom{0}45.7\phantom{0}$ & $\phantom{0}41.2\phantom{0}$ & $\phantom{0}95.0\phantom{0}$ & $\phantom{0}95.0\phantom{0}$ & $\phantom{0}95.4\phantom{0}$ \\
[0.5ex]\hline\\[-1.6ex] 
& & \multicolumn{18}{c}{Moderate intraclass correlation $(\rho_{Iy}=.30)$} \\[0.6ex]\hline\\[-1.8ex]
\multicolumn{4}{l}{$n=5$} \\  & \nopagebreak $\;J=30$  & $\phantom{-}0.00\phantom{0}$ & ${-}0.07\phantom{0}$ & ${-}0.04\phantom{0}$ & $\phantom{-}0.00\phantom{0}$ & $\phantom{-}0.01\phantom{0}$ & ${-}0.02\phantom{0}$ & $\phantom{0}0.19\phantom{0}$ & $\phantom{0}0.22\phantom{0}$ & $\phantom{0}0.21\phantom{0}$ & $\phantom{0}0.22\phantom{0}$ & $\phantom{0}0.22\phantom{0}$ & $\phantom{0}0.21\phantom{0}$ & $\phantom{0}92.1\phantom{0}$ & $\phantom{0}90.9\phantom{0}$ & $\phantom{0}85.7\phantom{0}$ & $\phantom{0}93.0\phantom{0}$ & $\phantom{0}92.5\phantom{0}$ & $\phantom{0}92.5\phantom{0}$ \\
 & \nopagebreak $\;J=50$  & $\phantom{-}0.00\phantom{0}$ & ${-}0.07\phantom{0}$ & ${-}0.04\phantom{0}$ & $\phantom{-}0.00\phantom{0}$ & $\phantom{-}0.01\phantom{0}$ & ${-}0.01\phantom{0}$ & $\phantom{0}0.14\phantom{0}$ & $\phantom{0}0.17\phantom{0}$ & $\phantom{0}0.16\phantom{0}$ & $\phantom{0}0.16\phantom{0}$ & $\phantom{0}0.16\phantom{0}$ & $\phantom{0}0.15\phantom{0}$ & $\phantom{0}94.3\phantom{0}$ & $\phantom{0}92.2\phantom{0}$ & $\phantom{0}88.3\phantom{0}$ & $\phantom{0}94.1\phantom{0}$ & $\phantom{0}94.3\phantom{0}$ & $\phantom{0}95.1\phantom{0}$ \\
 & \nopagebreak $\;J=100$  & $\phantom{-}0.00\phantom{0}$ & ${-}0.07\phantom{0}$ & ${-}0.04\phantom{0}$ & ${-}0.00\phantom{0}$ & ${-}0.00\phantom{0}$ & ${-}0.01\phantom{0}$ & $\phantom{0}0.10\phantom{0}$ & $\phantom{0}0.13\phantom{0}$ & $\phantom{0}0.12\phantom{0}$ & $\phantom{0}0.11\phantom{0}$ & $\phantom{0}0.11\phantom{0}$ & $\phantom{0}0.11\phantom{0}$ & $\phantom{0}93.9\phantom{0}$ & $\phantom{0}88.3\phantom{0}$ & $\phantom{0}86.1\phantom{0}$ & $\phantom{0}94.2\phantom{0}$ & $\phantom{0}94.0\phantom{0}$ & $\phantom{0}94.1\phantom{0}$ \\
 & \nopagebreak $\;J=200$  & $\phantom{-}0.00\phantom{0}$ & ${-}0.07\phantom{0}$ & ${-}0.04\phantom{0}$ & $\phantom{-}0.00\phantom{0}$ & $\phantom{-}0.00\phantom{0}$ & $\phantom{-}0.00\phantom{0}$ & $\phantom{0}0.07\phantom{0}$ & $\phantom{0}0.10\phantom{0}$ & $\phantom{0}0.09\phantom{0}$ & $\phantom{0}0.08\phantom{0}$ & $\phantom{0}0.08\phantom{0}$ & $\phantom{0}0.08\phantom{0}$ & $\phantom{0}95.6\phantom{0}$ & $\phantom{0}85.0\phantom{0}$ & $\phantom{0}85.4\phantom{0}$ & $\phantom{0}95.0\phantom{0}$ & $\phantom{0}95.4\phantom{0}$ & $\phantom{0}95.4\phantom{0}$ \\
 & \nopagebreak $\;J=500$  & ${-}0.00\phantom{0}$ & ${-}0.07\phantom{0}$ & ${-}0.04\phantom{0}$ & ${-}0.00\phantom{0}$ & ${-}0.00\phantom{0}$ & ${-}0.00\phantom{0}$ & $\phantom{0}0.04\phantom{0}$ & $\phantom{0}0.09\phantom{0}$ & $\phantom{0}0.07\phantom{0}$ & $\phantom{0}0.05\phantom{0}$ & $\phantom{0}0.05\phantom{0}$ & $\phantom{0}0.05\phantom{0}$ & $\phantom{0}94.7\phantom{0}$ & $\phantom{0}68.1\phantom{0}$ & $\phantom{0}77.1\phantom{0}$ & $\phantom{0}94.7\phantom{0}$ & $\phantom{0}94.7\phantom{0}$ & $\phantom{0}95.3\phantom{0}$ \\
 & \nopagebreak $\;J=1000$  & $\phantom{-}0.00\phantom{0}$ & ${-}0.07\phantom{0}$ & ${-}0.04\phantom{0}$ & $\phantom{-}0.00\phantom{0}$ & $\phantom{-}0.00\phantom{0}$ & ${-}0.00\phantom{0}$ & $\phantom{0}0.03\phantom{0}$ & $\phantom{0}0.08\phantom{0}$ & $\phantom{0}0.05\phantom{0}$ & $\phantom{0}0.04\phantom{0}$ & $\phantom{0}0.04\phantom{0}$ & $\phantom{0}0.04\phantom{0}$ & $\phantom{0}94.1\phantom{0}$ & $\phantom{0}46.4\phantom{0}$ & $\phantom{0}67.1\phantom{0}$ & $\phantom{0}94.8\phantom{0}$ & $\phantom{0}94.8\phantom{0}$ & $\phantom{0}95.5\phantom{0}$ \\
\multicolumn{4}{l}{$n=20$} \\  & \nopagebreak $\;J=30$  & $\phantom{-}0.01\phantom{0}$ & ${-}0.08\phantom{0}$ & ${-}0.05\phantom{0}$ & $\phantom{-}0.01\phantom{0}$ & $\phantom{-}0.01\phantom{0}$ & ${-}0.01\phantom{0}$ & $\phantom{0}0.18\phantom{0}$ & $\phantom{0}0.21\phantom{0}$ & $\phantom{0}0.20\phantom{0}$ & $\phantom{0}0.20\phantom{0}$ & $\phantom{0}0.20\phantom{0}$ & $\phantom{0}0.20\phantom{0}$ & $\phantom{0}94.4\phantom{0}$ & $\phantom{0}92.1\phantom{0}$ & $\phantom{0}86.6\phantom{0}$ & $\phantom{0}94.9\phantom{0}$ & $\phantom{0}95.1\phantom{0}$ & $\phantom{0}94.5\phantom{0}$ \\
 & \nopagebreak $\;J=50$  & ${-}0.00\phantom{0}$ & ${-}0.08\phantom{0}$ & ${-}0.05\phantom{0}$ & $\phantom{-}0.00\phantom{0}$ & $\phantom{-}0.00\phantom{0}$ & ${-}0.01\phantom{0}$ & $\phantom{0}0.14\phantom{0}$ & $\phantom{0}0.17\phantom{0}$ & $\phantom{0}0.16\phantom{0}$ & $\phantom{0}0.16\phantom{0}$ & $\phantom{0}0.16\phantom{0}$ & $\phantom{0}0.16\phantom{0}$ & $\phantom{0}94.8\phantom{0}$ & $\phantom{0}90.9\phantom{0}$ & $\phantom{0}86.8\phantom{0}$ & $\phantom{0}94.7\phantom{0}$ & $\phantom{0}94.2\phantom{0}$ & $\phantom{0}94.3\phantom{0}$ \\
 & \nopagebreak $\;J=100$  & $\phantom{-}0.00\phantom{0}$ & ${-}0.08\phantom{0}$ & ${-}0.06\phantom{0}$ & $\phantom{-}0.00\phantom{0}$ & $\phantom{-}0.00\phantom{0}$ & ${-}0.00\phantom{0}$ & $\phantom{0}0.10\phantom{0}$ & $\phantom{0}0.14\phantom{0}$ & $\phantom{0}0.12\phantom{0}$ & $\phantom{0}0.11\phantom{0}$ & $\phantom{0}0.11\phantom{0}$ & $\phantom{0}0.11\phantom{0}$ & $\phantom{0}94.9\phantom{0}$ & $\phantom{0}90.1\phantom{0}$ & $\phantom{0}85.7\phantom{0}$ & $\phantom{0}94.6\phantom{0}$ & $\phantom{0}95.1\phantom{0}$ & $\phantom{0}94.4\phantom{0}$ \\
 & \nopagebreak $\;J=200$  & $\phantom{-}0.00\phantom{0}$ & ${-}0.08\phantom{0}$ & ${-}0.05\phantom{0}$ & $\phantom{-}0.00\phantom{0}$ & $\phantom{-}0.00\phantom{0}$ & $\phantom{-}0.00\phantom{0}$ & $\phantom{0}0.07\phantom{0}$ & $\phantom{0}0.11\phantom{0}$ & $\phantom{0}0.10\phantom{0}$ & $\phantom{0}0.08\phantom{0}$ & $\phantom{0}0.08\phantom{0}$ & $\phantom{0}0.08\phantom{0}$ & $\phantom{0}94.4\phantom{0}$ & $\phantom{0}81.4\phantom{0}$ & $\phantom{0}79.3\phantom{0}$ & $\phantom{0}94.1\phantom{0}$ & $\phantom{0}94.7\phantom{0}$ & $\phantom{0}94.9\phantom{0}$ \\
 & \nopagebreak $\;J=500$  & $\phantom{-}0.00\phantom{0}$ & ${-}0.08\phantom{0}$ & ${-}0.06\phantom{0}$ & $\phantom{-}0.00\phantom{0}$ & $\phantom{-}0.00\phantom{0}$ & $\phantom{-}0.00\phantom{0}$ & $\phantom{0}0.05\phantom{0}$ & $\phantom{0}0.10\phantom{0}$ & $\phantom{0}0.08\phantom{0}$ & $\phantom{0}0.05\phantom{0}$ & $\phantom{0}0.05\phantom{0}$ & $\phantom{0}0.05\phantom{0}$ & $\phantom{0}94.8\phantom{0}$ & $\phantom{0}61.7\phantom{0}$ & $\phantom{0}67.3\phantom{0}$ & $\phantom{0}93.7\phantom{0}$ & $\phantom{0}94.3\phantom{0}$ & $\phantom{0}93.9\phantom{0}$ \\
 & \nopagebreak $\;J=1000$  & ${-}0.00\phantom{0}$ & ${-}0.08\phantom{0}$ & ${-}0.06\phantom{0}$ & $\phantom{-}0.00\phantom{0}$ & $\phantom{-}0.00\phantom{0}$ & ${-}0.00\phantom{0}$ & $\phantom{0}0.03\phantom{0}$ & $\phantom{0}0.09\phantom{0}$ & $\phantom{0}0.07\phantom{0}$ & $\phantom{0}0.04\phantom{0}$ & $\phantom{0}0.04\phantom{0}$ & $\phantom{0}0.04\phantom{0}$ & $\phantom{0}94.9\phantom{0}$ & $\phantom{0}35.8\phantom{0}$ & $\phantom{0}49.1\phantom{0}$ & $\phantom{0}93.6\phantom{0}$ & $\phantom{0}93.8\phantom{0}$ & $\phantom{0}94.8\phantom{0}$ \\
[0.5ex]\hline\\[-1.6ex] 
\end{tabular}
\begin{tablenotes}[para,flushleft]{\footnotesize \textit{Note.} $n$ = cluster size; $J$ = number of clusters; CD = complete data sets; LD = listwise deletion; FCS-SL = single-level FCS; FCS-MAN = two-level FCS with manifest cluster means; FCS-LAT = two-level FCS with latent cluster means; JM = joint modeling.}\end{tablenotes}
\end{threeparttable}
\end{sidewaystable}
\begin{sidewaystable}
\begin{threeparttable}
\setlength{\tabcolsep}{1.2pt}
\renewcommand{\arraystretch}{0.95}
\footnotesize
\caption{\small Study 1: Bias, RMSE, and Coverage of the 95\% Confidence Interval for the Mean of $z$ ($\hat\mu_z$) With 20\% Missing Data (MAR, $\lambda=1$)}
\begin{tabular}{llcccccccccccccccccc}
\hline\\[-1.8ex]
& & \multicolumn{6}{c}{Bias (\%)} & \multicolumn{6}{c}{RMSE} & \multicolumn{6}{c}{Coverage (\%)} \\ \cmidrule(r){3-8}\cmidrule(r){9-14}\cmidrule(r){15-20}
 &  & CD & LD & \makecell{FCS-\\SL} & \makecell{FCS-\\MAN} & \makecell{FCS-\\LAT} & JM & CD & LD & \makecell{FCS-\\SL} & \makecell{FCS-\\MAN} & \makecell{FCS-\\LAT} & JM & CD & LD & \makecell{FCS-\\SL} & \makecell{FCS-\\MAN} & \makecell{FCS-\\LAT} & \multicolumn{1}{c}{JM} \\ 
[0.4ex]\hline\\[-1.8ex]
& & \multicolumn{18}{c}{Small intraclass correlation $(\rho_{Iy}=.10)$} \\[0.6ex]\hline\\[-1.8ex]
\multicolumn{4}{l}{$n=5$} \\  & \nopagebreak $\;J=30$  & $\phantom{-}0.00\phantom{0}$ & ${-}0.10\phantom{0}$ & ${-}0.09\phantom{0}$ & ${-}0.00\phantom{0}$ & $\phantom{-}0.01\phantom{0}$ & ${-}0.06\phantom{0}$ & $\phantom{0}0.19\phantom{0}$ & $\phantom{0}0.23\phantom{0}$ & $\phantom{0}0.22\phantom{0}$ & $\phantom{0}0.24\phantom{0}$ & $\phantom{0}0.24\phantom{0}$ & $\phantom{0}0.22\phantom{0}$ & $\phantom{0}92.6\phantom{0}$ & $\phantom{0}89.6\phantom{0}$ & $\phantom{0}85.0\phantom{0}$ & $\phantom{0}93.8\phantom{0}$ & $\phantom{0}92.3\phantom{0}$ & $\phantom{0}93.2\phantom{0}$ \\
 & \nopagebreak $\;J=50$  & ${-}0.00\phantom{0}$ & ${-}0.10\phantom{0}$ & ${-}0.08\phantom{0}$ & $\phantom{-}0.00\phantom{0}$ & $\phantom{-}0.02\phantom{0}$ & ${-}0.04\phantom{0}$ & $\phantom{0}0.14\phantom{0}$ & $\phantom{0}0.19\phantom{0}$ & $\phantom{0}0.18\phantom{0}$ & $\phantom{0}0.18\phantom{0}$ & $\phantom{0}0.18\phantom{0}$ & $\phantom{0}0.17\phantom{0}$ & $\phantom{0}95.3\phantom{0}$ & $\phantom{0}89.2\phantom{0}$ & $\phantom{0}85.2\phantom{0}$ & $\phantom{0}95.7\phantom{0}$ & $\phantom{0}93.5\phantom{0}$ & $\phantom{0}95.3\phantom{0}$ \\
 & \nopagebreak $\;J=100$  & $\phantom{-}0.00\phantom{0}$ & ${-}0.10\phantom{0}$ & ${-}0.08\phantom{0}$ & $\phantom{-}0.00\phantom{0}$ & $\phantom{-}0.02\phantom{0}$ & ${-}0.03\phantom{0}$ & $\phantom{0}0.10\phantom{0}$ & $\phantom{0}0.15\phantom{0}$ & $\phantom{0}0.14\phantom{0}$ & $\phantom{0}0.12\phantom{0}$ & $\phantom{0}0.12\phantom{0}$ & $\phantom{0}0.12\phantom{0}$ & $\phantom{0}95.0\phantom{0}$ & $\phantom{0}84.6\phantom{0}$ & $\phantom{0}81.3\phantom{0}$ & $\phantom{0}95.2\phantom{0}$ & $\phantom{0}94.0\phantom{0}$ & $\phantom{0}94.8\phantom{0}$ \\
 & \nopagebreak $\;J=200$  & ${-}0.00\phantom{0}$ & ${-}0.11\phantom{0}$ & ${-}0.09\phantom{0}$ & ${-}0.00\phantom{0}$ & $\phantom{-}0.01\phantom{0}$ & ${-}0.02\phantom{0}$ & $\phantom{0}0.07\phantom{0}$ & $\phantom{0}0.13\phantom{0}$ & $\phantom{0}0.12\phantom{0}$ & $\phantom{0}0.09\phantom{0}$ & $\phantom{0}0.09\phantom{0}$ & $\phantom{0}0.09\phantom{0}$ & $\phantom{0}94.9\phantom{0}$ & $\phantom{0}71.0\phantom{0}$ & $\phantom{0}68.7\phantom{0}$ & $\phantom{0}94.5\phantom{0}$ & $\phantom{0}93.3\phantom{0}$ & $\phantom{0}93.4\phantom{0}$ \\
 & \nopagebreak $\;J=500$  & $\phantom{-}0.00\phantom{0}$ & ${-}0.10\phantom{0}$ & ${-}0.09\phantom{0}$ & $\phantom{-}0.00\phantom{0}$ & $\phantom{-}0.00\phantom{0}$ & ${-}0.01\phantom{0}$ & $\phantom{0}0.04\phantom{0}$ & $\phantom{0}0.12\phantom{0}$ & $\phantom{0}0.10\phantom{0}$ & $\phantom{0}0.05\phantom{0}$ & $\phantom{0}0.05\phantom{0}$ & $\phantom{0}0.05\phantom{0}$ & $\phantom{0}95.9\phantom{0}$ & $\phantom{0}42.5\phantom{0}$ & $\phantom{0}47.2\phantom{0}$ & $\phantom{0}95.2\phantom{0}$ & $\phantom{0}94.1\phantom{0}$ & $\phantom{0}96.1\phantom{0}$ \\
 & \nopagebreak $\;J=1000$  & $\phantom{-}0.00\phantom{0}$ & ${-}0.10\phantom{0}$ & ${-}0.08\phantom{0}$ & $\phantom{-}0.00\phantom{0}$ & $\phantom{-}0.00\phantom{0}$ & ${-}0.01\phantom{0}$ & $\phantom{0}0.03\phantom{0}$ & $\phantom{0}0.11\phantom{0}$ & $\phantom{0}0.09\phantom{0}$ & $\phantom{0}0.04\phantom{0}$ & $\phantom{0}0.04\phantom{0}$ & $\phantom{0}0.04\phantom{0}$ & $\phantom{0}94.5\phantom{0}$ & $\phantom{0}14.6\phantom{0}$ & $\phantom{0}23.0\phantom{0}$ & $\phantom{0}94.3\phantom{0}$ & $\phantom{0}93.9\phantom{0}$ & $\phantom{0}94.4\phantom{0}$ \\
\multicolumn{4}{l}{$n=20$} \\  & \nopagebreak $\;J=30$  & ${-}0.00\phantom{0}$ & ${-}0.14\phantom{0}$ & ${-}0.13\phantom{0}$ & $\phantom{-}0.00\phantom{0}$ & $\phantom{-}0.01\phantom{0}$ & ${-}0.07\phantom{0}$ & $\phantom{0}0.18\phantom{0}$ & $\phantom{0}0.24\phantom{0}$ & $\phantom{0}0.24\phantom{0}$ & $\phantom{0}0.23\phantom{0}$ & $\phantom{0}0.24\phantom{0}$ & $\phantom{0}0.22\phantom{0}$ & $\phantom{0}94.2\phantom{0}$ & $\phantom{0}86.4\phantom{0}$ & $\phantom{0}79.0\phantom{0}$ & $\phantom{0}94.2\phantom{0}$ & $\phantom{0}93.5\phantom{0}$ & $\phantom{0}92.4\phantom{0}$ \\
 & \nopagebreak $\;J=50$  & $\phantom{-}0.01\phantom{0}$ & ${-}0.14\phantom{0}$ & ${-}0.13\phantom{0}$ & $\phantom{-}0.01\phantom{0}$ & $\phantom{-}0.02\phantom{0}$ & ${-}0.05\phantom{0}$ & $\phantom{0}0.14\phantom{0}$ & $\phantom{0}0.21\phantom{0}$ & $\phantom{0}0.20\phantom{0}$ & $\phantom{0}0.18\phantom{0}$ & $\phantom{0}0.18\phantom{0}$ & $\phantom{0}0.17\phantom{0}$ & $\phantom{0}94.4\phantom{0}$ & $\phantom{0}81.5\phantom{0}$ & $\phantom{0}74.3\phantom{0}$ & $\phantom{0}94.0\phantom{0}$ & $\phantom{0}94.6\phantom{0}$ & $\phantom{0}93.5\phantom{0}$ \\
 & \nopagebreak $\;J=100$  & $\phantom{-}0.00\phantom{0}$ & ${-}0.14\phantom{0}$ & ${-}0.13\phantom{0}$ & $\phantom{-}0.00\phantom{0}$ & $\phantom{-}0.01\phantom{0}$ & ${-}0.03\phantom{0}$ & $\phantom{0}0.10\phantom{0}$ & $\phantom{0}0.18\phantom{0}$ & $\phantom{0}0.17\phantom{0}$ & $\phantom{0}0.12\phantom{0}$ & $\phantom{0}0.13\phantom{0}$ & $\phantom{0}0.12\phantom{0}$ & $\phantom{0}95.1\phantom{0}$ & $\phantom{0}71.9\phantom{0}$ & $\phantom{0}63.3\phantom{0}$ & $\phantom{0}94.2\phantom{0}$ & $\phantom{0}93.8\phantom{0}$ & $\phantom{0}93.7\phantom{0}$ \\
 & \nopagebreak $\;J=200$  & $\phantom{-}0.01\phantom{0}$ & ${-}0.14\phantom{0}$ & ${-}0.13\phantom{0}$ & $\phantom{-}0.00\phantom{0}$ & $\phantom{-}0.01\phantom{0}$ & ${-}0.02\phantom{0}$ & $\phantom{0}0.07\phantom{0}$ & $\phantom{0}0.16\phantom{0}$ & $\phantom{0}0.15\phantom{0}$ & $\phantom{0}0.08\phantom{0}$ & $\phantom{0}0.08\phantom{0}$ & $\phantom{0}0.08\phantom{0}$ & $\phantom{0}94.7\phantom{0}$ & $\phantom{0}53.7\phantom{0}$ & $\phantom{0}46.5\phantom{0}$ & $\phantom{0}94.7\phantom{0}$ & $\phantom{0}95.1\phantom{0}$ & $\phantom{0}94.8\phantom{0}$ \\
 & \nopagebreak $\;J=500$  & ${-}0.00\phantom{0}$ & ${-}0.14\phantom{0}$ & ${-}0.13\phantom{0}$ & ${-}0.00\phantom{0}$ & ${-}0.00\phantom{0}$ & ${-}0.01\phantom{0}$ & $\phantom{0}0.04\phantom{0}$ & $\phantom{0}0.15\phantom{0}$ & $\phantom{0}0.14\phantom{0}$ & $\phantom{0}0.05\phantom{0}$ & $\phantom{0}0.05\phantom{0}$ & $\phantom{0}0.05\phantom{0}$ & $\phantom{0}95.5\phantom{0}$ & $\phantom{0}13.0\phantom{0}$ & $\phantom{0}11.3\phantom{0}$ & $\phantom{0}95.0\phantom{0}$ & $\phantom{0}94.6\phantom{0}$ & $\phantom{0}94.2\phantom{0}$ \\
 & \nopagebreak $\;J=1000$  & ${-}0.00\phantom{0}$ & ${-}0.15\phantom{0}$ & ${-}0.13\phantom{0}$ & ${-}0.00\phantom{0}$ & ${-}0.00\phantom{0}$ & ${-}0.01\phantom{0}$ & $\phantom{0}0.03\phantom{0}$ & $\phantom{0}0.15\phantom{0}$ & $\phantom{0}0.14\phantom{0}$ & $\phantom{0}0.04\phantom{0}$ & $\phantom{0}0.04\phantom{0}$ & $\phantom{0}0.04\phantom{0}$ & $\phantom{0}93.6\phantom{0}$ & $\phantom{0}\phantom{0}0.9\phantom{0}$ & $\phantom{0}\phantom{0}1.0\phantom{0}$ & $\phantom{0}94.1\phantom{0}$ & $\phantom{0}94.1\phantom{0}$ & $\phantom{0}94.1\phantom{0}$ \\
[0.5ex]\hline\\[-1.6ex] 
& & \multicolumn{18}{c}{Moderate intraclass correlation $(\rho_{Iy}=.30)$} \\[0.6ex]\hline\\[-1.8ex]
\multicolumn{4}{l}{$n=5$} \\  & \nopagebreak $\;J=30$  & ${-}0.00\phantom{0}$ & ${-}0.15\phantom{0}$ & ${-}0.10\phantom{0}$ & ${-}0.00\phantom{0}$ & $\phantom{-}0.01\phantom{0}$ & ${-}0.05\phantom{0}$ & $\phantom{0}0.19\phantom{0}$ & $\phantom{0}0.25\phantom{0}$ & $\phantom{0}0.22\phantom{0}$ & $\phantom{0}0.23\phantom{0}$ & $\phantom{0}0.24\phantom{0}$ & $\phantom{0}0.22\phantom{0}$ & $\phantom{0}93.0\phantom{0}$ & $\phantom{0}85.7\phantom{0}$ & $\phantom{0}82.8\phantom{0}$ & $\phantom{0}93.9\phantom{0}$ & $\phantom{0}92.9\phantom{0}$ & $\phantom{0}92.9\phantom{0}$ \\
 & \nopagebreak $\;J=50$  & ${-}0.00\phantom{0}$ & ${-}0.15\phantom{0}$ & ${-}0.10\phantom{0}$ & ${-}0.00\phantom{0}$ & $\phantom{-}0.01\phantom{0}$ & ${-}0.03\phantom{0}$ & $\phantom{0}0.14\phantom{0}$ & $\phantom{0}0.21\phantom{0}$ & $\phantom{0}0.18\phantom{0}$ & $\phantom{0}0.18\phantom{0}$ & $\phantom{0}0.18\phantom{0}$ & $\phantom{0}0.17\phantom{0}$ & $\phantom{0}94.1\phantom{0}$ & $\phantom{0}81.6\phantom{0}$ & $\phantom{0}80.6\phantom{0}$ & $\phantom{0}93.8\phantom{0}$ & $\phantom{0}93.8\phantom{0}$ & $\phantom{0}93.7\phantom{0}$ \\
 & \nopagebreak $\;J=100$  & $\phantom{-}0.00\phantom{0}$ & ${-}0.14\phantom{0}$ & ${-}0.10\phantom{0}$ & $\phantom{-}0.00\phantom{0}$ & $\phantom{-}0.00\phantom{0}$ & ${-}0.01\phantom{0}$ & $\phantom{0}0.10\phantom{0}$ & $\phantom{0}0.18\phantom{0}$ & $\phantom{0}0.15\phantom{0}$ & $\phantom{0}0.12\phantom{0}$ & $\phantom{0}0.12\phantom{0}$ & $\phantom{0}0.12\phantom{0}$ & $\phantom{0}95.1\phantom{0}$ & $\phantom{0}71.3\phantom{0}$ & $\phantom{0}75.8\phantom{0}$ & $\phantom{0}94.3\phantom{0}$ & $\phantom{0}94.3\phantom{0}$ & $\phantom{0}94.9\phantom{0}$ \\
 & \nopagebreak $\;J=200$  & ${-}0.00\phantom{0}$ & ${-}0.15\phantom{0}$ & ${-}0.10\phantom{0}$ & ${-}0.00\phantom{0}$ & ${-}0.00\phantom{0}$ & ${-}0.01\phantom{0}$ & $\phantom{0}0.07\phantom{0}$ & $\phantom{0}0.16\phantom{0}$ & $\phantom{0}0.12\phantom{0}$ & $\phantom{0}0.08\phantom{0}$ & $\phantom{0}0.08\phantom{0}$ & $\phantom{0}0.08\phantom{0}$ & $\phantom{0}96.1\phantom{0}$ & $\phantom{0}52.2\phantom{0}$ & $\phantom{0}65.0\phantom{0}$ & $\phantom{0}95.6\phantom{0}$ & $\phantom{0}94.5\phantom{0}$ & $\phantom{0}95.7\phantom{0}$ \\
 & \nopagebreak $\;J=500$  & ${-}0.00\phantom{0}$ & ${-}0.15\phantom{0}$ & ${-}0.10\phantom{0}$ & ${-}0.00\phantom{0}$ & ${-}0.00\phantom{0}$ & ${-}0.00\phantom{0}$ & $\phantom{0}0.04\phantom{0}$ & $\phantom{0}0.15\phantom{0}$ & $\phantom{0}0.11\phantom{0}$ & $\phantom{0}0.05\phantom{0}$ & $\phantom{0}0.05\phantom{0}$ & $\phantom{0}0.05\phantom{0}$ & $\phantom{0}95.5\phantom{0}$ & $\phantom{0}13.2\phantom{0}$ & $\phantom{0}34.9\phantom{0}$ & $\phantom{0}95.2\phantom{0}$ & $\phantom{0}95.6\phantom{0}$ & $\phantom{0}95.0\phantom{0}$ \\
 & \nopagebreak $\;J=1000$  & $\phantom{-}0.00\phantom{0}$ & ${-}0.14\phantom{0}$ & ${-}0.10\phantom{0}$ & ${-}0.00\phantom{0}$ & ${-}0.00\phantom{0}$ & ${-}0.00\phantom{0}$ & $\phantom{0}0.03\phantom{0}$ & $\phantom{0}0.15\phantom{0}$ & $\phantom{0}0.11\phantom{0}$ & $\phantom{0}0.04\phantom{0}$ & $\phantom{0}0.04\phantom{0}$ & $\phantom{0}0.04\phantom{0}$ & $\phantom{0}94.8\phantom{0}$ & $\phantom{0}\phantom{0}1.1\phantom{0}$ & $\phantom{0}10.5\phantom{0}$ & $\phantom{0}95.0\phantom{0}$ & $\phantom{0}93.6\phantom{0}$ & $\phantom{0}94.9\phantom{0}$ \\
\multicolumn{4}{l}{$n=20$} \\  & \nopagebreak $\;J=30$  & $\phantom{-}0.00\phantom{0}$ & ${-}0.16\phantom{0}$ & ${-}0.12\phantom{0}$ & $\phantom{-}0.01\phantom{0}$ & $\phantom{-}0.01\phantom{0}$ & ${-}0.03\phantom{0}$ & $\phantom{0}0.18\phantom{0}$ & $\phantom{0}0.25\phantom{0}$ & $\phantom{0}0.23\phantom{0}$ & $\phantom{0}0.22\phantom{0}$ & $\phantom{0}0.22\phantom{0}$ & $\phantom{0}0.21\phantom{0}$ & $\phantom{0}94.3\phantom{0}$ & $\phantom{0}84.5\phantom{0}$ & $\phantom{0}78.9\phantom{0}$ & $\phantom{0}94.8\phantom{0}$ & $\phantom{0}94.1\phantom{0}$ & $\phantom{0}94.1\phantom{0}$ \\
 & \nopagebreak $\;J=50$  & $\phantom{-}0.00\phantom{0}$ & ${-}0.16\phantom{0}$ & ${-}0.13\phantom{0}$ & $\phantom{-}0.00\phantom{0}$ & $\phantom{-}0.00\phantom{0}$ & ${-}0.03\phantom{0}$ & $\phantom{0}0.14\phantom{0}$ & $\phantom{0}0.22\phantom{0}$ & $\phantom{0}0.20\phantom{0}$ & $\phantom{0}0.18\phantom{0}$ & $\phantom{0}0.18\phantom{0}$ & $\phantom{0}0.17\phantom{0}$ & $\phantom{0}93.8\phantom{0}$ & $\phantom{0}79.8\phantom{0}$ & $\phantom{0}76.3\phantom{0}$ & $\phantom{0}93.9\phantom{0}$ & $\phantom{0}94.2\phantom{0}$ & $\phantom{0}93.0\phantom{0}$ \\
 & \nopagebreak $\;J=100$  & $\phantom{-}0.00\phantom{0}$ & ${-}0.16\phantom{0}$ & ${-}0.12\phantom{0}$ & $\phantom{-}0.01\phantom{0}$ & $\phantom{-}0.01\phantom{0}$ & ${-}0.01\phantom{0}$ & $\phantom{0}0.10\phantom{0}$ & $\phantom{0}0.19\phantom{0}$ & $\phantom{0}0.16\phantom{0}$ & $\phantom{0}0.12\phantom{0}$ & $\phantom{0}0.12\phantom{0}$ & $\phantom{0}0.12\phantom{0}$ & $\phantom{0}94.3\phantom{0}$ & $\phantom{0}67.5\phantom{0}$ & $\phantom{0}67.1\phantom{0}$ & $\phantom{0}94.5\phantom{0}$ & $\phantom{0}94.2\phantom{0}$ & $\phantom{0}94.6\phantom{0}$ \\
 & \nopagebreak $\;J=200$  & $\phantom{-}0.00\phantom{0}$ & ${-}0.17\phantom{0}$ & ${-}0.13\phantom{0}$ & $\phantom{-}0.00\phantom{0}$ & $\phantom{-}0.00\phantom{0}$ & ${-}0.01\phantom{0}$ & $\phantom{0}0.07\phantom{0}$ & $\phantom{0}0.18\phantom{0}$ & $\phantom{0}0.15\phantom{0}$ & $\phantom{0}0.08\phantom{0}$ & $\phantom{0}0.08\phantom{0}$ & $\phantom{0}0.08\phantom{0}$ & $\phantom{0}95.2\phantom{0}$ & $\phantom{0}39.1\phantom{0}$ & $\phantom{0}44.8\phantom{0}$ & $\phantom{0}95.2\phantom{0}$ & $\phantom{0}95.5\phantom{0}$ & $\phantom{0}95.0\phantom{0}$ \\
 & \nopagebreak $\;J=500$  & $\phantom{-}0.00\phantom{0}$ & ${-}0.17\phantom{0}$ & ${-}0.13\phantom{0}$ & ${-}0.00\phantom{0}$ & ${-}0.00\phantom{0}$ & ${-}0.00\phantom{0}$ & $\phantom{0}0.04\phantom{0}$ & $\phantom{0}0.17\phantom{0}$ & $\phantom{0}0.14\phantom{0}$ & $\phantom{0}0.05\phantom{0}$ & $\phantom{0}0.05\phantom{0}$ & $\phantom{0}0.05\phantom{0}$ & $\phantom{0}96.0\phantom{0}$ & $\phantom{0}\phantom{0}5.1\phantom{0}$ & $\phantom{0}11.0\phantom{0}$ & $\phantom{0}95.3\phantom{0}$ & $\phantom{0}94.5\phantom{0}$ & $\phantom{0}94.8\phantom{0}$ \\
 & \nopagebreak $\;J=1000$  & $\phantom{-}0.00\phantom{0}$ & ${-}0.16\phantom{0}$ & ${-}0.13\phantom{0}$ & $\phantom{-}0.00\phantom{0}$ & $\phantom{-}0.00\phantom{0}$ & ${-}0.00\phantom{0}$ & $\phantom{0}0.03\phantom{0}$ & $\phantom{0}0.17\phantom{0}$ & $\phantom{0}0.13\phantom{0}$ & $\phantom{0}0.04\phantom{0}$ & $\phantom{0}0.04\phantom{0}$ & $\phantom{0}0.04\phantom{0}$ & $\phantom{0}94.1\phantom{0}$ & $\phantom{0}\phantom{0}0.0\phantom{0}$ & $\phantom{0}\phantom{0}1.3\phantom{0}$ & $\phantom{0}94.9\phantom{0}$ & $\phantom{0}95.0\phantom{0}$ & $\phantom{0}95.3\phantom{0}$ \\
[0.5ex]\hline\\[-1.6ex] 
\end{tabular}
\begin{tablenotes}[para,flushleft]{\footnotesize \textit{Note.} $n$ = cluster size; $J$ = number of clusters; CD = complete data sets; LD = listwise deletion; FCS-SL = single-level FCS; FCS-MAN = two-level FCS with manifest cluster means; FCS-LAT = two-level FCS with latent cluster means; JM = joint modeling.}\end{tablenotes}
\end{threeparttable}
\end{sidewaystable}
\begin{sidewaystable}
\begin{threeparttable}
\setlength{\tabcolsep}{1.2pt}
\renewcommand{\arraystretch}{0.95}
\footnotesize
\caption{\small Study 1: Bias, RMSE, and Coverage of the 95\% Confidence Interval for the Mean of $z$ ($\hat\mu_z$) With 40\% Missing Data (MCAR, $\lambda=0$)}
\begin{tabular}{llcccccccccccccccccc}
\hline\\[-1.8ex]
& & \multicolumn{6}{c}{Bias (\%)} & \multicolumn{6}{c}{RMSE} & \multicolumn{6}{c}{Coverage (\%)} \\ \cmidrule(r){3-8}\cmidrule(r){9-14}\cmidrule(r){15-20}
 &  & CD & LD & \makecell{FCS-\\SL} & \makecell{FCS-\\MAN} & \makecell{FCS-\\LAT} & JM & CD & LD & \makecell{FCS-\\SL} & \makecell{FCS-\\MAN} & \makecell{FCS-\\LAT} & JM & CD & LD & \makecell{FCS-\\SL} & \makecell{FCS-\\MAN} & \makecell{FCS-\\LAT} & \multicolumn{1}{c}{JM} \\ 
[0.4ex]\hline\\[-1.8ex]
& & \multicolumn{18}{c}{Small intraclass correlation $(\rho_{Iy}=.10)$} \\[0.6ex]\hline\\[-1.8ex]
\multicolumn{4}{l}{$n=5$} \\  & \nopagebreak $\;J=30$  & ${-}0.00\phantom{0}$ & ${-}0.00\phantom{0}$ & ${-}0.00\phantom{0}$ & ${-}0.00\phantom{0}$ & ${-}0.00\phantom{0}$ & ${-}0.00\phantom{0}$ & $\phantom{0}0.18\phantom{0}$ & $\phantom{0}0.23\phantom{0}$ & $\phantom{0}0.23\phantom{0}$ & $\phantom{0}0.24\phantom{0}$ & $\phantom{0}0.24\phantom{0}$ & $\phantom{0}0.24\phantom{0}$ & $\phantom{0}94.1\phantom{0}$ & $\phantom{0}92.7\phantom{0}$ & $\phantom{0}81.1\phantom{0}$ & $\phantom{0}92.9\phantom{0}$ & $\phantom{0}93.9\phantom{0}$ & $\phantom{0}93.1\phantom{0}$ \\
 & \nopagebreak $\;J=50$  & $\phantom{-}0.00\phantom{0}$ & ${-}0.00\phantom{0}$ & ${-}0.00\phantom{0}$ & ${-}0.00\phantom{0}$ & ${-}0.00\phantom{0}$ & ${-}0.00\phantom{0}$ & $\phantom{0}0.14\phantom{0}$ & $\phantom{0}0.19\phantom{0}$ & $\phantom{0}0.18\phantom{0}$ & $\phantom{0}0.18\phantom{0}$ & $\phantom{0}0.18\phantom{0}$ & $\phantom{0}0.19\phantom{0}$ & $\phantom{0}93.5\phantom{0}$ & $\phantom{0}93.6\phantom{0}$ & $\phantom{0}82.4\phantom{0}$ & $\phantom{0}93.8\phantom{0}$ & $\phantom{0}93.9\phantom{0}$ & $\phantom{0}93.7\phantom{0}$ \\
 & \nopagebreak $\;J=100$  & ${-}0.00\phantom{0}$ & ${-}0.00\phantom{0}$ & ${-}0.00\phantom{0}$ & ${-}0.00\phantom{0}$ & ${-}0.00\phantom{0}$ & ${-}0.00\phantom{0}$ & $\phantom{0}0.10\phantom{0}$ & $\phantom{0}0.13\phantom{0}$ & $\phantom{0}0.13\phantom{0}$ & $\phantom{0}0.13\phantom{0}$ & $\phantom{0}0.13\phantom{0}$ & $\phantom{0}0.13\phantom{0}$ & $\phantom{0}94.7\phantom{0}$ & $\phantom{0}94.1\phantom{0}$ & $\phantom{0}81.5\phantom{0}$ & $\phantom{0}95.4\phantom{0}$ & $\phantom{0}94.8\phantom{0}$ & $\phantom{0}94.9\phantom{0}$ \\
 & \nopagebreak $\;J=200$  & ${-}0.00\phantom{0}$ & $\phantom{-}0.00\phantom{0}$ & $\phantom{-}0.00\phantom{0}$ & ${-}0.00\phantom{0}$ & ${-}0.00\phantom{0}$ & ${-}0.00\phantom{0}$ & $\phantom{0}0.07\phantom{0}$ & $\phantom{0}0.09\phantom{0}$ & $\phantom{0}0.09\phantom{0}$ & $\phantom{0}0.09\phantom{0}$ & $\phantom{0}0.09\phantom{0}$ & $\phantom{0}0.09\phantom{0}$ & $\phantom{0}94.8\phantom{0}$ & $\phantom{0}93.8\phantom{0}$ & $\phantom{0}82.6\phantom{0}$ & $\phantom{0}93.7\phantom{0}$ & $\phantom{0}94.4\phantom{0}$ & $\phantom{0}94.1\phantom{0}$ \\
 & \nopagebreak $\;J=500$  & ${-}0.00\phantom{0}$ & ${-}0.00\phantom{0}$ & ${-}0.00\phantom{0}$ & ${-}0.00\phantom{0}$ & ${-}0.00\phantom{0}$ & ${-}0.00\phantom{0}$ & $\phantom{0}0.04\phantom{0}$ & $\phantom{0}0.06\phantom{0}$ & $\phantom{0}0.06\phantom{0}$ & $\phantom{0}0.06\phantom{0}$ & $\phantom{0}0.06\phantom{0}$ & $\phantom{0}0.06\phantom{0}$ & $\phantom{0}95.5\phantom{0}$ & $\phantom{0}94.4\phantom{0}$ & $\phantom{0}83.1\phantom{0}$ & $\phantom{0}94.6\phantom{0}$ & $\phantom{0}94.8\phantom{0}$ & $\phantom{0}95.1\phantom{0}$ \\
 & \nopagebreak $\;J=1000$  & ${-}0.00\phantom{0}$ & ${-}0.00\phantom{0}$ & ${-}0.00\phantom{0}$ & ${-}0.00\phantom{0}$ & ${-}0.00\phantom{0}$ & ${-}0.00\phantom{0}$ & $\phantom{0}0.03\phantom{0}$ & $\phantom{0}0.04\phantom{0}$ & $\phantom{0}0.04\phantom{0}$ & $\phantom{0}0.04\phantom{0}$ & $\phantom{0}0.04\phantom{0}$ & $\phantom{0}0.04\phantom{0}$ & $\phantom{0}93.6\phantom{0}$ & $\phantom{0}93.9\phantom{0}$ & $\phantom{0}81.7\phantom{0}$ & $\phantom{0}93.2\phantom{0}$ & $\phantom{0}93.2\phantom{0}$ & $\phantom{0}93.9\phantom{0}$ \\
\multicolumn{4}{l}{$n=20$} \\  & \nopagebreak $\;J=30$  & $\phantom{-}0.00\phantom{0}$ & $\phantom{-}0.01\phantom{0}$ & $\phantom{-}0.01\phantom{0}$ & $\phantom{-}0.01\phantom{0}$ & $\phantom{-}0.01\phantom{0}$ & $\phantom{-}0.01\phantom{0}$ & $\phantom{0}0.19\phantom{0}$ & $\phantom{0}0.24\phantom{0}$ & $\phantom{0}0.24\phantom{0}$ & $\phantom{0}0.24\phantom{0}$ & $\phantom{0}0.24\phantom{0}$ & $\phantom{0}0.24\phantom{0}$ & $\phantom{0}92.9\phantom{0}$ & $\phantom{0}92.4\phantom{0}$ & $\phantom{0}74.4\phantom{0}$ & $\phantom{0}93.5\phantom{0}$ & $\phantom{0}93.8\phantom{0}$ & $\phantom{0}93.1\phantom{0}$ \\
 & \nopagebreak $\;J=50$  & ${-}0.00\phantom{0}$ & ${-}0.00\phantom{0}$ & ${-}0.00\phantom{0}$ & ${-}0.00\phantom{0}$ & ${-}0.00\phantom{0}$ & ${-}0.00\phantom{0}$ & $\phantom{0}0.14\phantom{0}$ & $\phantom{0}0.19\phantom{0}$ & $\phantom{0}0.19\phantom{0}$ & $\phantom{0}0.19\phantom{0}$ & $\phantom{0}0.18\phantom{0}$ & $\phantom{0}0.19\phantom{0}$ & $\phantom{0}94.3\phantom{0}$ & $\phantom{0}92.9\phantom{0}$ & $\phantom{0}74.9\phantom{0}$ & $\phantom{0}94.1\phantom{0}$ & $\phantom{0}94.5\phantom{0}$ & $\phantom{0}93.4\phantom{0}$ \\
 & \nopagebreak $\;J=100$  & $\phantom{-}0.00\phantom{0}$ & $\phantom{-}0.00\phantom{0}$ & $\phantom{-}0.00\phantom{0}$ & $\phantom{-}0.00\phantom{0}$ & $\phantom{-}0.00\phantom{0}$ & $\phantom{-}0.00\phantom{0}$ & $\phantom{0}0.10\phantom{0}$ & $\phantom{0}0.13\phantom{0}$ & $\phantom{0}0.13\phantom{0}$ & $\phantom{0}0.13\phantom{0}$ & $\phantom{0}0.13\phantom{0}$ & $\phantom{0}0.13\phantom{0}$ & $\phantom{0}94.5\phantom{0}$ & $\phantom{0}94.3\phantom{0}$ & $\phantom{0}76.5\phantom{0}$ & $\phantom{0}94.5\phantom{0}$ & $\phantom{0}94.1\phantom{0}$ & $\phantom{0}94.3\phantom{0}$ \\
 & \nopagebreak $\;J=200$  & ${-}0.00\phantom{0}$ & $\phantom{-}0.00\phantom{0}$ & $\phantom{-}0.00\phantom{0}$ & $\phantom{-}0.00\phantom{0}$ & $\phantom{-}0.00\phantom{0}$ & $\phantom{-}0.00\phantom{0}$ & $\phantom{0}0.07\phantom{0}$ & $\phantom{0}0.09\phantom{0}$ & $\phantom{0}0.09\phantom{0}$ & $\phantom{0}0.09\phantom{0}$ & $\phantom{0}0.09\phantom{0}$ & $\phantom{0}0.09\phantom{0}$ & $\phantom{0}93.9\phantom{0}$ & $\phantom{0}94.5\phantom{0}$ & $\phantom{0}76.5\phantom{0}$ & $\phantom{0}94.5\phantom{0}$ & $\phantom{0}93.6\phantom{0}$ & $\phantom{0}94.4\phantom{0}$ \\
 & \nopagebreak $\;J=500$  & $\phantom{-}0.00\phantom{0}$ & $\phantom{-}0.00\phantom{0}$ & ${-}0.00\phantom{0}$ & ${-}0.00\phantom{0}$ & ${-}0.00\phantom{0}$ & $\phantom{-}0.00\phantom{0}$ & $\phantom{0}0.04\phantom{0}$ & $\phantom{0}0.06\phantom{0}$ & $\phantom{0}0.06\phantom{0}$ & $\phantom{0}0.06\phantom{0}$ & $\phantom{0}0.06\phantom{0}$ & $\phantom{0}0.06\phantom{0}$ & $\phantom{0}95.0\phantom{0}$ & $\phantom{0}95.1\phantom{0}$ & $\phantom{0}81.2\phantom{0}$ & $\phantom{0}94.5\phantom{0}$ & $\phantom{0}94.5\phantom{0}$ & $\phantom{0}94.3\phantom{0}$ \\
 & \nopagebreak $\;J=1000$  & ${-}0.00\phantom{0}$ & ${-}0.00\phantom{0}$ & ${-}0.00\phantom{0}$ & ${-}0.00\phantom{0}$ & ${-}0.00\phantom{0}$ & ${-}0.00\phantom{0}$ & $\phantom{0}0.03\phantom{0}$ & $\phantom{0}0.04\phantom{0}$ & $\phantom{0}0.04\phantom{0}$ & $\phantom{0}0.04\phantom{0}$ & $\phantom{0}0.04\phantom{0}$ & $\phantom{0}0.04\phantom{0}$ & $\phantom{0}94.5\phantom{0}$ & $\phantom{0}94.2\phantom{0}$ & $\phantom{0}76.5\phantom{0}$ & $\phantom{0}93.9\phantom{0}$ & $\phantom{0}93.6\phantom{0}$ & $\phantom{0}94.9\phantom{0}$ \\
[0.5ex]\hline\\[-1.6ex] 
& & \multicolumn{18}{c}{Moderate intraclass correlation $(\rho_{Iy}=.30)$} \\[0.6ex]\hline\\[-1.8ex]
\multicolumn{4}{l}{$n=5$} \\  & \nopagebreak $\;J=30$  & ${-}0.00\phantom{0}$ & ${-}0.00\phantom{0}$ & ${-}0.00\phantom{0}$ & ${-}0.00\phantom{0}$ & ${-}0.00\phantom{0}$ & ${-}0.00\phantom{0}$ & $\phantom{0}0.18\phantom{0}$ & $\phantom{0}0.24\phantom{0}$ & $\phantom{0}0.23\phantom{0}$ & $\phantom{0}0.24\phantom{0}$ & $\phantom{0}0.24\phantom{0}$ & $\phantom{0}0.24\phantom{0}$ & $\phantom{0}93.5\phantom{0}$ & $\phantom{0}93.4\phantom{0}$ & $\phantom{0}81.2\phantom{0}$ & $\phantom{0}94.3\phantom{0}$ & $\phantom{0}94.1\phantom{0}$ & $\phantom{0}94.5\phantom{0}$ \\
 & \nopagebreak $\;J=50$  & $\phantom{-}0.00\phantom{0}$ & $\phantom{-}0.00\phantom{0}$ & $\phantom{-}0.00\phantom{0}$ & $\phantom{-}0.00\phantom{0}$ & $\phantom{-}0.01\phantom{0}$ & $\phantom{-}0.01\phantom{0}$ & $\phantom{0}0.14\phantom{0}$ & $\phantom{0}0.18\phantom{0}$ & $\phantom{0}0.18\phantom{0}$ & $\phantom{0}0.18\phantom{0}$ & $\phantom{0}0.18\phantom{0}$ & $\phantom{0}0.18\phantom{0}$ & $\phantom{0}94.1\phantom{0}$ & $\phantom{0}94.4\phantom{0}$ & $\phantom{0}82.9\phantom{0}$ & $\phantom{0}94.6\phantom{0}$ & $\phantom{0}94.1\phantom{0}$ & $\phantom{0}94.1\phantom{0}$ \\
 & \nopagebreak $\;J=100$  & ${-}0.00\phantom{0}$ & ${-}0.00\phantom{0}$ & ${-}0.00\phantom{0}$ & ${-}0.00\phantom{0}$ & ${-}0.00\phantom{0}$ & ${-}0.00\phantom{0}$ & $\phantom{0}0.10\phantom{0}$ & $\phantom{0}0.12\phantom{0}$ & $\phantom{0}0.12\phantom{0}$ & $\phantom{0}0.13\phantom{0}$ & $\phantom{0}0.13\phantom{0}$ & $\phantom{0}0.13\phantom{0}$ & $\phantom{0}94.5\phantom{0}$ & $\phantom{0}94.9\phantom{0}$ & $\phantom{0}84.6\phantom{0}$ & $\phantom{0}95.1\phantom{0}$ & $\phantom{0}94.9\phantom{0}$ & $\phantom{0}94.7\phantom{0}$ \\
 & \nopagebreak $\;J=200$  & $\phantom{-}0.00\phantom{0}$ & ${-}0.00\phantom{0}$ & ${-}0.00\phantom{0}$ & $\phantom{-}0.00\phantom{0}$ & $\phantom{-}0.00\phantom{0}$ & $\phantom{-}0.00\phantom{0}$ & $\phantom{0}0.07\phantom{0}$ & $\phantom{0}0.09\phantom{0}$ & $\phantom{0}0.09\phantom{0}$ & $\phantom{0}0.09\phantom{0}$ & $\phantom{0}0.09\phantom{0}$ & $\phantom{0}0.09\phantom{0}$ & $\phantom{0}94.6\phantom{0}$ & $\phantom{0}94.4\phantom{0}$ & $\phantom{0}82.9\phantom{0}$ & $\phantom{0}94.7\phantom{0}$ & $\phantom{0}94.3\phantom{0}$ & $\phantom{0}94.7\phantom{0}$ \\
 & \nopagebreak $\;J=500$  & $\phantom{-}0.00\phantom{0}$ & $\phantom{-}0.00\phantom{0}$ & $\phantom{-}0.00\phantom{0}$ & $\phantom{-}0.00\phantom{0}$ & $\phantom{-}0.00\phantom{0}$ & $\phantom{-}0.00\phantom{0}$ & $\phantom{0}0.05\phantom{0}$ & $\phantom{0}0.06\phantom{0}$ & $\phantom{0}0.06\phantom{0}$ & $\phantom{0}0.06\phantom{0}$ & $\phantom{0}0.06\phantom{0}$ & $\phantom{0}0.06\phantom{0}$ & $\phantom{0}95.5\phantom{0}$ & $\phantom{0}95.6\phantom{0}$ & $\phantom{0}84.4\phantom{0}$ & $\phantom{0}95.6\phantom{0}$ & $\phantom{0}95.3\phantom{0}$ & $\phantom{0}95.9\phantom{0}$ \\
 & \nopagebreak $\;J=1000$  & $\phantom{-}0.00\phantom{0}$ & $\phantom{-}0.00\phantom{0}$ & $\phantom{-}0.00\phantom{0}$ & $\phantom{-}0.00\phantom{0}$ & $\phantom{-}0.00\phantom{0}$ & $\phantom{-}0.00\phantom{0}$ & $\phantom{0}0.03\phantom{0}$ & $\phantom{0}0.04\phantom{0}$ & $\phantom{0}0.04\phantom{0}$ & $\phantom{0}0.04\phantom{0}$ & $\phantom{0}0.04\phantom{0}$ & $\phantom{0}0.04\phantom{0}$ & $\phantom{0}94.9\phantom{0}$ & $\phantom{0}94.9\phantom{0}$ & $\phantom{0}82.2\phantom{0}$ & $\phantom{0}94.1\phantom{0}$ & $\phantom{0}94.2\phantom{0}$ & $\phantom{0}95.7\phantom{0}$ \\
\multicolumn{4}{l}{$n=20$} \\  & \nopagebreak $\;J=30$  & $\phantom{-}0.01\phantom{0}$ & $\phantom{-}0.01\phantom{0}$ & $\phantom{-}0.01\phantom{0}$ & $\phantom{-}0.01\phantom{0}$ & $\phantom{-}0.01\phantom{0}$ & $\phantom{-}0.00\phantom{0}$ & $\phantom{0}0.19\phantom{0}$ & $\phantom{0}0.24\phantom{0}$ & $\phantom{0}0.23\phantom{0}$ & $\phantom{0}0.24\phantom{0}$ & $\phantom{0}0.24\phantom{0}$ & $\phantom{0}0.24\phantom{0}$ & $\phantom{0}93.8\phantom{0}$ & $\phantom{0}92.9\phantom{0}$ & $\phantom{0}75.6\phantom{0}$ & $\phantom{0}93.7\phantom{0}$ & $\phantom{0}94.1\phantom{0}$ & $\phantom{0}94.1\phantom{0}$ \\
 & \nopagebreak $\;J=50$  & $\phantom{-}0.00\phantom{0}$ & $\phantom{-}0.01\phantom{0}$ & $\phantom{-}0.01\phantom{0}$ & $\phantom{-}0.00\phantom{0}$ & $\phantom{-}0.01\phantom{0}$ & $\phantom{-}0.01\phantom{0}$ & $\phantom{0}0.15\phantom{0}$ & $\phantom{0}0.19\phantom{0}$ & $\phantom{0}0.18\phantom{0}$ & $\phantom{0}0.18\phantom{0}$ & $\phantom{0}0.18\phantom{0}$ & $\phantom{0}0.18\phantom{0}$ & $\phantom{0}93.3\phantom{0}$ & $\phantom{0}92.3\phantom{0}$ & $\phantom{0}76.7\phantom{0}$ & $\phantom{0}93.6\phantom{0}$ & $\phantom{0}94.5\phantom{0}$ & $\phantom{0}93.9\phantom{0}$ \\
 & \nopagebreak $\;J=100$  & ${-}0.00\phantom{0}$ & ${-}0.00\phantom{0}$ & ${-}0.00\phantom{0}$ & ${-}0.00\phantom{0}$ & $\phantom{-}0.00\phantom{0}$ & ${-}0.00\phantom{0}$ & $\phantom{0}0.10\phantom{0}$ & $\phantom{0}0.13\phantom{0}$ & $\phantom{0}0.13\phantom{0}$ & $\phantom{0}0.13\phantom{0}$ & $\phantom{0}0.13\phantom{0}$ & $\phantom{0}0.13\phantom{0}$ & $\phantom{0}94.6\phantom{0}$ & $\phantom{0}94.5\phantom{0}$ & $\phantom{0}78.5\phantom{0}$ & $\phantom{0}94.7\phantom{0}$ & $\phantom{0}94.9\phantom{0}$ & $\phantom{0}95.5\phantom{0}$ \\
 & \nopagebreak $\;J=200$  & $\phantom{-}0.00\phantom{0}$ & ${-}0.00\phantom{0}$ & ${-}0.00\phantom{0}$ & ${-}0.00\phantom{0}$ & ${-}0.00\phantom{0}$ & ${-}0.00\phantom{0}$ & $\phantom{0}0.07\phantom{0}$ & $\phantom{0}0.09\phantom{0}$ & $\phantom{0}0.09\phantom{0}$ & $\phantom{0}0.09\phantom{0}$ & $\phantom{0}0.09\phantom{0}$ & $\phantom{0}0.09\phantom{0}$ & $\phantom{0}96.3\phantom{0}$ & $\phantom{0}95.5\phantom{0}$ & $\phantom{0}81.7\phantom{0}$ & $\phantom{0}95.6\phantom{0}$ & $\phantom{0}96.0\phantom{0}$ & $\phantom{0}96.2\phantom{0}$ \\
 & \nopagebreak $\;J=500$  & ${-}0.00\phantom{0}$ & ${-}0.00\phantom{0}$ & ${-}0.00\phantom{0}$ & ${-}0.00\phantom{0}$ & ${-}0.00\phantom{0}$ & ${-}0.00\phantom{0}$ & $\phantom{0}0.04\phantom{0}$ & $\phantom{0}0.06\phantom{0}$ & $\phantom{0}0.05\phantom{0}$ & $\phantom{0}0.05\phantom{0}$ & $\phantom{0}0.05\phantom{0}$ & $\phantom{0}0.05\phantom{0}$ & $\phantom{0}95.7\phantom{0}$ & $\phantom{0}95.5\phantom{0}$ & $\phantom{0}80.6\phantom{0}$ & $\phantom{0}95.3\phantom{0}$ & $\phantom{0}95.1\phantom{0}$ & $\phantom{0}96.4\phantom{0}$ \\
 & \nopagebreak $\;J=1000$  & ${-}0.00\phantom{0}$ & ${-}0.00\phantom{0}$ & ${-}0.00\phantom{0}$ & $\phantom{-}0.00\phantom{0}$ & $\phantom{-}0.00\phantom{0}$ & $\phantom{-}0.00\phantom{0}$ & $\phantom{0}0.03\phantom{0}$ & $\phantom{0}0.04\phantom{0}$ & $\phantom{0}0.04\phantom{0}$ & $\phantom{0}0.04\phantom{0}$ & $\phantom{0}0.04\phantom{0}$ & $\phantom{0}0.04\phantom{0}$ & $\phantom{0}94.9\phantom{0}$ & $\phantom{0}95.2\phantom{0}$ & $\phantom{0}80.6\phantom{0}$ & $\phantom{0}95.0\phantom{0}$ & $\phantom{0}94.9\phantom{0}$ & $\phantom{0}95.7\phantom{0}$ \\
[0.5ex]\hline\\[-1.6ex] 
\end{tabular}
\begin{tablenotes}[para,flushleft]{\footnotesize \textit{Note.} $n$ = cluster size; $J$ = number of clusters; CD = complete data sets; LD = listwise deletion; FCS-SL = single-level FCS; FCS-MAN = two-level FCS with manifest cluster means; FCS-LAT = two-level FCS with latent cluster means; JM = joint modeling.}\end{tablenotes}
\end{threeparttable}
\end{sidewaystable}
\begin{sidewaystable}
\begin{threeparttable}
\setlength{\tabcolsep}{1.2pt}
\renewcommand{\arraystretch}{0.95}
\footnotesize
\caption{\small Study 1: Bias, RMSE, and Coverage of the 95\% Confidence Interval for the Mean of $z$ ($\hat\mu_z$) With 40\% Missing Data (MAR, $\lambda=0.5$)}
\begin{tabular}{llcccccccccccccccccc}
\hline\\[-1.8ex]
& & \multicolumn{6}{c}{Bias (\%)} & \multicolumn{6}{c}{RMSE} & \multicolumn{6}{c}{Coverage (\%)} \\ \cmidrule(r){3-8}\cmidrule(r){9-14}\cmidrule(r){15-20}
 &  & CD & LD & \makecell{FCS-\\SL} & \makecell{FCS-\\MAN} & \makecell{FCS-\\LAT} & JM & CD & LD & \makecell{FCS-\\SL} & \makecell{FCS-\\MAN} & \makecell{FCS-\\LAT} & JM & CD & LD & \makecell{FCS-\\SL} & \makecell{FCS-\\MAN} & \makecell{FCS-\\LAT} & \multicolumn{1}{c}{JM} \\ 
[0.4ex]\hline\\[-1.8ex]
& & \multicolumn{18}{c}{Small intraclass correlation $(\rho_{Iy}=.10)$} \\[0.6ex]\hline\\[-1.8ex]
\multicolumn{4}{l}{$n=5$} \\  & \nopagebreak $\;J=30$  & ${-}0.00\phantom{0}$ & ${-}0.09\phantom{0}$ & ${-}0.06\phantom{0}$ & $\phantom{-}0.01\phantom{0}$ & $\phantom{-}0.02\phantom{0}$ & ${-}0.04\phantom{0}$ & $\phantom{0}0.19\phantom{0}$ & $\phantom{0}0.25\phantom{0}$ & $\phantom{0}0.25\phantom{0}$ & $\phantom{0}0.27\phantom{0}$ & $\phantom{0}0.27\phantom{0}$ & $\phantom{0}0.25\phantom{0}$ & $\phantom{0}92.5\phantom{0}$ & $\phantom{0}90.6\phantom{0}$ & $\phantom{0}79.1\phantom{0}$ & $\phantom{0}93.5\phantom{0}$ & $\phantom{0}94.2\phantom{0}$ & $\phantom{0}93.4\phantom{0}$ \\
 & \nopagebreak $\;J=50$  & ${-}0.00\phantom{0}$ & ${-}0.10\phantom{0}$ & ${-}0.08\phantom{0}$ & ${-}0.00\phantom{0}$ & $\phantom{-}0.01\phantom{0}$ & ${-}0.05\phantom{0}$ & $\phantom{0}0.14\phantom{0}$ & $\phantom{0}0.21\phantom{0}$ & $\phantom{0}0.20\phantom{0}$ & $\phantom{0}0.20\phantom{0}$ & $\phantom{0}0.20\phantom{0}$ & $\phantom{0}0.20\phantom{0}$ & $\phantom{0}93.6\phantom{0}$ & $\phantom{0}89.3\phantom{0}$ & $\phantom{0}78.0\phantom{0}$ & $\phantom{0}93.3\phantom{0}$ & $\phantom{0}94.1\phantom{0}$ & $\phantom{0}92.6\phantom{0}$ \\
 & \nopagebreak $\;J=100$  & $\phantom{-}0.00\phantom{0}$ & ${-}0.10\phantom{0}$ & ${-}0.07\phantom{0}$ & $\phantom{-}0.00\phantom{0}$ & $\phantom{-}0.01\phantom{0}$ & ${-}0.03\phantom{0}$ & $\phantom{0}0.10\phantom{0}$ & $\phantom{0}0.16\phantom{0}$ & $\phantom{0}0.15\phantom{0}$ & $\phantom{0}0.14\phantom{0}$ & $\phantom{0}0.14\phantom{0}$ & $\phantom{0}0.14\phantom{0}$ & $\phantom{0}95.5\phantom{0}$ & $\phantom{0}87.7\phantom{0}$ & $\phantom{0}75.1\phantom{0}$ & $\phantom{0}94.0\phantom{0}$ & $\phantom{0}94.3\phantom{0}$ & $\phantom{0}93.7\phantom{0}$ \\
 & \nopagebreak $\;J=200$  & ${-}0.00\phantom{0}$ & ${-}0.10\phantom{0}$ & ${-}0.08\phantom{0}$ & ${-}0.00\phantom{0}$ & $\phantom{-}0.00\phantom{0}$ & ${-}0.02\phantom{0}$ & $\phantom{0}0.07\phantom{0}$ & $\phantom{0}0.14\phantom{0}$ & $\phantom{0}0.12\phantom{0}$ & $\phantom{0}0.10\phantom{0}$ & $\phantom{0}0.10\phantom{0}$ & $\phantom{0}0.10\phantom{0}$ & $\phantom{0}95.5\phantom{0}$ & $\phantom{0}80.9\phantom{0}$ & $\phantom{0}67.7\phantom{0}$ & $\phantom{0}93.9\phantom{0}$ & $\phantom{0}92.9\phantom{0}$ & $\phantom{0}93.7\phantom{0}$ \\
 & \nopagebreak $\;J=500$  & ${-}0.00\phantom{0}$ & ${-}0.10\phantom{0}$ & ${-}0.07\phantom{0}$ & ${-}0.00\phantom{0}$ & $\phantom{-}0.00\phantom{0}$ & ${-}0.01\phantom{0}$ & $\phantom{0}0.04\phantom{0}$ & $\phantom{0}0.11\phantom{0}$ & $\phantom{0}0.09\phantom{0}$ & $\phantom{0}0.06\phantom{0}$ & $\phantom{0}0.06\phantom{0}$ & $\phantom{0}0.06\phantom{0}$ & $\phantom{0}95.1\phantom{0}$ & $\phantom{0}59.7\phantom{0}$ & $\phantom{0}53.5\phantom{0}$ & $\phantom{0}94.5\phantom{0}$ & $\phantom{0}94.7\phantom{0}$ & $\phantom{0}94.8\phantom{0}$ \\
 & \nopagebreak $\;J=1000$  & $\phantom{-}0.00\phantom{0}$ & ${-}0.10\phantom{0}$ & ${-}0.07\phantom{0}$ & ${-}0.00\phantom{0}$ & $\phantom{-}0.00\phantom{0}$ & ${-}0.01\phantom{0}$ & $\phantom{0}0.03\phantom{0}$ & $\phantom{0}0.11\phantom{0}$ & $\phantom{0}0.08\phantom{0}$ & $\phantom{0}0.04\phantom{0}$ & $\phantom{0}0.04\phantom{0}$ & $\phantom{0}0.04\phantom{0}$ & $\phantom{0}95.1\phantom{0}$ & $\phantom{0}32.5\phantom{0}$ & $\phantom{0}32.6\phantom{0}$ & $\phantom{0}95.3\phantom{0}$ & $\phantom{0}95.3\phantom{0}$ & $\phantom{0}95.3\phantom{0}$ \\
\multicolumn{4}{l}{$n=20$} \\  & \nopagebreak $\;J=30$  & ${-}0.01\phantom{0}$ & ${-}0.15\phantom{0}$ & ${-}0.13\phantom{0}$ & ${-}0.01\phantom{0}$ & $\phantom{-}0.00\phantom{0}$ & ${-}0.08\phantom{0}$ & $\phantom{0}0.18\phantom{0}$ & $\phantom{0}0.28\phantom{0}$ & $\phantom{0}0.27\phantom{0}$ & $\phantom{0}0.25\phantom{0}$ & $\phantom{0}0.26\phantom{0}$ & $\phantom{0}0.25\phantom{0}$ & $\phantom{0}93.1\phantom{0}$ & $\phantom{0}87.5\phantom{0}$ & $\phantom{0}69.0\phantom{0}$ & $\phantom{0}93.9\phantom{0}$ & $\phantom{0}93.8\phantom{0}$ & $\phantom{0}92.3\phantom{0}$ \\
 & \nopagebreak $\;J=50$  & ${-}0.00\phantom{0}$ & ${-}0.13\phantom{0}$ & ${-}0.12\phantom{0}$ & $\phantom{-}0.00\phantom{0}$ & $\phantom{-}0.01\phantom{0}$ & ${-}0.05\phantom{0}$ & $\phantom{0}0.14\phantom{0}$ & $\phantom{0}0.23\phantom{0}$ & $\phantom{0}0.22\phantom{0}$ & $\phantom{0}0.20\phantom{0}$ & $\phantom{0}0.20\phantom{0}$ & $\phantom{0}0.19\phantom{0}$ & $\phantom{0}93.5\phantom{0}$ & $\phantom{0}86.5\phantom{0}$ & $\phantom{0}66.5\phantom{0}$ & $\phantom{0}93.9\phantom{0}$ & $\phantom{0}93.2\phantom{0}$ & $\phantom{0}92.3\phantom{0}$ \\
 & \nopagebreak $\;J=100$  & ${-}0.00\phantom{0}$ & ${-}0.13\phantom{0}$ & ${-}0.12\phantom{0}$ & $\phantom{-}0.00\phantom{0}$ & $\phantom{-}0.01\phantom{0}$ & ${-}0.03\phantom{0}$ & $\phantom{0}0.10\phantom{0}$ & $\phantom{0}0.19\phantom{0}$ & $\phantom{0}0.17\phantom{0}$ & $\phantom{0}0.14\phantom{0}$ & $\phantom{0}0.14\phantom{0}$ & $\phantom{0}0.14\phantom{0}$ & $\phantom{0}94.8\phantom{0}$ & $\phantom{0}80.7\phantom{0}$ & $\phantom{0}59.1\phantom{0}$ & $\phantom{0}93.6\phantom{0}$ & $\phantom{0}93.5\phantom{0}$ & $\phantom{0}93.3\phantom{0}$ \\
 & \nopagebreak $\;J=200$  & ${-}0.00\phantom{0}$ & ${-}0.14\phantom{0}$ & ${-}0.12\phantom{0}$ & ${-}0.00\phantom{0}$ & ${-}0.00\phantom{0}$ & ${-}0.02\phantom{0}$ & $\phantom{0}0.07\phantom{0}$ & $\phantom{0}0.17\phantom{0}$ & $\phantom{0}0.15\phantom{0}$ & $\phantom{0}0.10\phantom{0}$ & $\phantom{0}0.10\phantom{0}$ & $\phantom{0}0.10\phantom{0}$ & $\phantom{0}95.5\phantom{0}$ & $\phantom{0}66.3\phantom{0}$ & $\phantom{0}46.2\phantom{0}$ & $\phantom{0}94.3\phantom{0}$ & $\phantom{0}93.5\phantom{0}$ & $\phantom{0}93.8\phantom{0}$ \\
 & \nopagebreak $\;J=500$  & $\phantom{-}0.00\phantom{0}$ & ${-}0.13\phantom{0}$ & ${-}0.12\phantom{0}$ & ${-}0.00\phantom{0}$ & ${-}0.00\phantom{0}$ & ${-}0.01\phantom{0}$ & $\phantom{0}0.04\phantom{0}$ & $\phantom{0}0.15\phantom{0}$ & $\phantom{0}0.13\phantom{0}$ & $\phantom{0}0.06\phantom{0}$ & $\phantom{0}0.06\phantom{0}$ & $\phantom{0}0.06\phantom{0}$ & $\phantom{0}94.7\phantom{0}$ & $\phantom{0}33.3\phantom{0}$ & $\phantom{0}20.3\phantom{0}$ & $\phantom{0}95.3\phantom{0}$ & $\phantom{0}95.7\phantom{0}$ & $\phantom{0}94.9\phantom{0}$ \\
 & \nopagebreak $\;J=1000$  & $\phantom{-}0.00\phantom{0}$ & ${-}0.13\phantom{0}$ & ${-}0.12\phantom{0}$ & $\phantom{-}0.00\phantom{0}$ & $\phantom{-}0.00\phantom{0}$ & ${-}0.00\phantom{0}$ & $\phantom{0}0.03\phantom{0}$ & $\phantom{0}0.14\phantom{0}$ & $\phantom{0}0.12\phantom{0}$ & $\phantom{0}0.04\phantom{0}$ & $\phantom{0}0.04\phantom{0}$ & $\phantom{0}0.04\phantom{0}$ & $\phantom{0}95.5\phantom{0}$ & $\phantom{0}\phantom{0}9.5\phantom{0}$ & $\phantom{0}\phantom{0}4.9\phantom{0}$ & $\phantom{0}93.2\phantom{0}$ & $\phantom{0}94.8\phantom{0}$ & $\phantom{0}95.1\phantom{0}$ \\
[0.5ex]\hline\\[-1.6ex] 
& & \multicolumn{18}{c}{Moderate intraclass correlation $(\rho_{Iy}=.30)$} \\[0.6ex]\hline\\[-1.8ex]
\multicolumn{4}{l}{$n=5$} \\  & \nopagebreak $\;J=30$  & $\phantom{-}0.00\phantom{0}$ & ${-}0.13\phantom{0}$ & ${-}0.08\phantom{0}$ & $\phantom{-}0.00\phantom{0}$ & $\phantom{-}0.01\phantom{0}$ & ${-}0.04\phantom{0}$ & $\phantom{0}0.19\phantom{0}$ & $\phantom{0}0.27\phantom{0}$ & $\phantom{0}0.25\phantom{0}$ & $\phantom{0}0.26\phantom{0}$ & $\phantom{0}0.26\phantom{0}$ & $\phantom{0}0.25\phantom{0}$ & $\phantom{0}94.3\phantom{0}$ & $\phantom{0}89.3\phantom{0}$ & $\phantom{0}79.1\phantom{0}$ & $\phantom{0}93.9\phantom{0}$ & $\phantom{0}94.1\phantom{0}$ & $\phantom{0}93.9\phantom{0}$ \\
 & \nopagebreak $\;J=50$  & ${-}0.00\phantom{0}$ & ${-}0.14\phantom{0}$ & ${-}0.09\phantom{0}$ & ${-}0.00\phantom{0}$ & $\phantom{-}0.00\phantom{0}$ & ${-}0.03\phantom{0}$ & $\phantom{0}0.14\phantom{0}$ & $\phantom{0}0.23\phantom{0}$ & $\phantom{0}0.20\phantom{0}$ & $\phantom{0}0.19\phantom{0}$ & $\phantom{0}0.20\phantom{0}$ & $\phantom{0}0.19\phantom{0}$ & $\phantom{0}94.0\phantom{0}$ & $\phantom{0}87.2\phantom{0}$ & $\phantom{0}78.3\phantom{0}$ & $\phantom{0}93.7\phantom{0}$ & $\phantom{0}95.0\phantom{0}$ & $\phantom{0}94.5\phantom{0}$ \\
 & \nopagebreak $\;J=100$  & $\phantom{-}0.00\phantom{0}$ & ${-}0.13\phantom{0}$ & ${-}0.08\phantom{0}$ & $\phantom{-}0.00\phantom{0}$ & $\phantom{-}0.00\phantom{0}$ & ${-}0.02\phantom{0}$ & $\phantom{0}0.10\phantom{0}$ & $\phantom{0}0.18\phantom{0}$ & $\phantom{0}0.15\phantom{0}$ & $\phantom{0}0.14\phantom{0}$ & $\phantom{0}0.14\phantom{0}$ & $\phantom{0}0.14\phantom{0}$ & $\phantom{0}94.5\phantom{0}$ & $\phantom{0}81.3\phantom{0}$ & $\phantom{0}75.1\phantom{0}$ & $\phantom{0}93.8\phantom{0}$ & $\phantom{0}93.2\phantom{0}$ & $\phantom{0}93.9\phantom{0}$ \\
 & \nopagebreak $\;J=200$  & ${-}0.00\phantom{0}$ & ${-}0.13\phantom{0}$ & ${-}0.08\phantom{0}$ & ${-}0.00\phantom{0}$ & ${-}0.00\phantom{0}$ & ${-}0.01\phantom{0}$ & $\phantom{0}0.07\phantom{0}$ & $\phantom{0}0.16\phantom{0}$ & $\phantom{0}0.12\phantom{0}$ & $\phantom{0}0.09\phantom{0}$ & $\phantom{0}0.09\phantom{0}$ & $\phantom{0}0.09\phantom{0}$ & $\phantom{0}95.6\phantom{0}$ & $\phantom{0}68.7\phantom{0}$ & $\phantom{0}68.6\phantom{0}$ & $\phantom{0}94.6\phantom{0}$ & $\phantom{0}94.8\phantom{0}$ & $\phantom{0}94.9\phantom{0}$ \\
 & \nopagebreak $\;J=500$  & ${-}0.00\phantom{0}$ & ${-}0.13\phantom{0}$ & ${-}0.08\phantom{0}$ & $\phantom{-}0.00\phantom{0}$ & $\phantom{-}0.00\phantom{0}$ & ${-}0.00\phantom{0}$ & $\phantom{0}0.05\phantom{0}$ & $\phantom{0}0.15\phantom{0}$ & $\phantom{0}0.10\phantom{0}$ & $\phantom{0}0.06\phantom{0}$ & $\phantom{0}0.06\phantom{0}$ & $\phantom{0}0.06\phantom{0}$ & $\phantom{0}94.2\phantom{0}$ & $\phantom{0}35.4\phantom{0}$ & $\phantom{0}51.6\phantom{0}$ & $\phantom{0}93.9\phantom{0}$ & $\phantom{0}94.2\phantom{0}$ & $\phantom{0}94.8\phantom{0}$ \\
 & \nopagebreak $\;J=1000$  & ${-}0.00\phantom{0}$ & ${-}0.13\phantom{0}$ & ${-}0.08\phantom{0}$ & $\phantom{-}0.00\phantom{0}$ & ${-}0.00\phantom{0}$ & ${-}0.00\phantom{0}$ & $\phantom{0}0.03\phantom{0}$ & $\phantom{0}0.14\phantom{0}$ & $\phantom{0}0.09\phantom{0}$ & $\phantom{0}0.04\phantom{0}$ & $\phantom{0}0.04\phantom{0}$ & $\phantom{0}0.04\phantom{0}$ & $\phantom{0}95.6\phantom{0}$ & $\phantom{0}\phantom{0}8.0\phantom{0}$ & $\phantom{0}27.0\phantom{0}$ & $\phantom{0}93.8\phantom{0}$ & $\phantom{0}94.4\phantom{0}$ & $\phantom{0}95.5\phantom{0}$ \\
\multicolumn{4}{l}{$n=20$} \\  & \nopagebreak $\;J=30$  & ${-}0.00\phantom{0}$ & ${-}0.15\phantom{0}$ & ${-}0.11\phantom{0}$ & ${-}0.00\phantom{0}$ & $\phantom{-}0.00\phantom{0}$ & ${-}0.04\phantom{0}$ & $\phantom{0}0.18\phantom{0}$ & $\phantom{0}0.28\phantom{0}$ & $\phantom{0}0.25\phantom{0}$ & $\phantom{0}0.24\phantom{0}$ & $\phantom{0}0.24\phantom{0}$ & $\phantom{0}0.24\phantom{0}$ & $\phantom{0}93.7\phantom{0}$ & $\phantom{0}88.1\phantom{0}$ & $\phantom{0}70.1\phantom{0}$ & $\phantom{0}94.3\phantom{0}$ & $\phantom{0}94.9\phantom{0}$ & $\phantom{0}94.6\phantom{0}$ \\
 & \nopagebreak $\;J=50$  & $\phantom{-}0.00\phantom{0}$ & ${-}0.15\phantom{0}$ & ${-}0.11\phantom{0}$ & $\phantom{-}0.00\phantom{0}$ & $\phantom{-}0.00\phantom{0}$ & ${-}0.02\phantom{0}$ & $\phantom{0}0.14\phantom{0}$ & $\phantom{0}0.24\phantom{0}$ & $\phantom{0}0.21\phantom{0}$ & $\phantom{0}0.20\phantom{0}$ & $\phantom{0}0.20\phantom{0}$ & $\phantom{0}0.19\phantom{0}$ & $\phantom{0}94.0\phantom{0}$ & $\phantom{0}85.0\phantom{0}$ & $\phantom{0}68.6\phantom{0}$ & $\phantom{0}93.5\phantom{0}$ & $\phantom{0}93.9\phantom{0}$ & $\phantom{0}94.6\phantom{0}$ \\
 & \nopagebreak $\;J=100$  & $\phantom{-}0.00\phantom{0}$ & ${-}0.15\phantom{0}$ & ${-}0.10\phantom{0}$ & $\phantom{-}0.01\phantom{0}$ & $\phantom{-}0.01\phantom{0}$ & ${-}0.01\phantom{0}$ & $\phantom{0}0.10\phantom{0}$ & $\phantom{0}0.20\phantom{0}$ & $\phantom{0}0.16\phantom{0}$ & $\phantom{0}0.14\phantom{0}$ & $\phantom{0}0.14\phantom{0}$ & $\phantom{0}0.14\phantom{0}$ & $\phantom{0}94.1\phantom{0}$ & $\phantom{0}77.1\phantom{0}$ & $\phantom{0}63.1\phantom{0}$ & $\phantom{0}93.8\phantom{0}$ & $\phantom{0}93.4\phantom{0}$ & $\phantom{0}94.9\phantom{0}$ \\
 & \nopagebreak $\;J=200$  & $\phantom{-}0.01\phantom{0}$ & ${-}0.15\phantom{0}$ & ${-}0.10\phantom{0}$ & $\phantom{-}0.00\phantom{0}$ & $\phantom{-}0.01\phantom{0}$ & ${-}0.00\phantom{0}$ & $\phantom{0}0.07\phantom{0}$ & $\phantom{0}0.17\phantom{0}$ & $\phantom{0}0.14\phantom{0}$ & $\phantom{0}0.10\phantom{0}$ & $\phantom{0}0.10\phantom{0}$ & $\phantom{0}0.10\phantom{0}$ & $\phantom{0}93.9\phantom{0}$ & $\phantom{0}62.1\phantom{0}$ & $\phantom{0}54.4\phantom{0}$ & $\phantom{0}93.7\phantom{0}$ & $\phantom{0}93.9\phantom{0}$ & $\phantom{0}94.6\phantom{0}$ \\
 & \nopagebreak $\;J=500$  & $\phantom{-}0.00\phantom{0}$ & ${-}0.15\phantom{0}$ & ${-}0.11\phantom{0}$ & $\phantom{-}0.00\phantom{0}$ & $\phantom{-}0.00\phantom{0}$ & ${-}0.00\phantom{0}$ & $\phantom{0}0.05\phantom{0}$ & $\phantom{0}0.16\phantom{0}$ & $\phantom{0}0.12\phantom{0}$ & $\phantom{0}0.06\phantom{0}$ & $\phantom{0}0.06\phantom{0}$ & $\phantom{0}0.06\phantom{0}$ & $\phantom{0}94.8\phantom{0}$ & $\phantom{0}23.9\phantom{0}$ & $\phantom{0}26.1\phantom{0}$ & $\phantom{0}94.3\phantom{0}$ & $\phantom{0}94.1\phantom{0}$ & $\phantom{0}94.5\phantom{0}$ \\
 & \nopagebreak $\;J=1000$  & ${-}0.00\phantom{0}$ & ${-}0.15\phantom{0}$ & ${-}0.11\phantom{0}$ & ${-}0.00\phantom{0}$ & ${-}0.00\phantom{0}$ & ${-}0.00\phantom{0}$ & $\phantom{0}0.03\phantom{0}$ & $\phantom{0}0.16\phantom{0}$ & $\phantom{0}0.11\phantom{0}$ & $\phantom{0}0.04\phantom{0}$ & $\phantom{0}0.04\phantom{0}$ & $\phantom{0}0.04\phantom{0}$ & $\phantom{0}94.2\phantom{0}$ & $\phantom{0}\phantom{0}3.0\phantom{0}$ & $\phantom{0}\phantom{0}7.0\phantom{0}$ & $\phantom{0}94.8\phantom{0}$ & $\phantom{0}94.4\phantom{0}$ & $\phantom{0}95.4\phantom{0}$ \\
[0.5ex]\hline\\[-1.6ex] 
\end{tabular}
\begin{tablenotes}[para,flushleft]{\footnotesize \textit{Note.} $n$ = cluster size; $J$ = number of clusters; CD = complete data sets; LD = listwise deletion; FCS-SL = single-level FCS; FCS-MAN = two-level FCS with manifest cluster means; FCS-LAT = two-level FCS with latent cluster means; JM = joint modeling.}\end{tablenotes}
\end{threeparttable}
\end{sidewaystable}
\begin{sidewaystable}
\begin{threeparttable}
\setlength{\tabcolsep}{1.2pt}
\renewcommand{\arraystretch}{0.95}
\footnotesize
\caption{\small Study 1: Bias, RMSE, and Coverage of the 95\% Confidence Interval for the Mean of $z$ ($\hat\mu_z$) With 40\% Missing Data (MAR, $\lambda=1$)}
\begin{tabular}{llcccccccccccccccccc}
\hline\\[-1.8ex]
& & \multicolumn{6}{c}{Bias (\%)} & \multicolumn{6}{c}{RMSE} & \multicolumn{6}{c}{Coverage (\%)} \\ \cmidrule(r){3-8}\cmidrule(r){9-14}\cmidrule(r){15-20}
 &  & CD & LD & \makecell{FCS-\\SL} & \makecell{FCS-\\MAN} & \makecell{FCS-\\LAT} & JM & CD & LD & \makecell{FCS-\\SL} & \makecell{FCS-\\MAN} & \makecell{FCS-\\LAT} & JM & CD & LD & \makecell{FCS-\\SL} & \makecell{FCS-\\MAN} & \makecell{FCS-\\LAT} & \multicolumn{1}{c}{JM} \\ 
[0.4ex]\hline\\[-1.8ex]
& & \multicolumn{18}{c}{Small intraclass correlation $(\rho_{Iy}=.10)$} \\[0.6ex]\hline\\[-1.8ex]
\multicolumn{4}{l}{$n=5$} \\  & \nopagebreak $\;J=30$  & $\phantom{-}0.00\phantom{0}$ & ${-}0.19\phantom{0}$ & ${-}0.16\phantom{0}$ & $\phantom{-}0.01\phantom{0}$ & $\phantom{-}0.00\phantom{0}$ & ${-}0.13\phantom{0}$ & $\phantom{0}0.18\phantom{0}$ & $\phantom{0}0.30\phantom{0}$ & $\phantom{0}0.28\phantom{0}$ & $\phantom{0}0.37\phantom{0}$ & $\phantom{0}0.35\phantom{0}$ & $\phantom{0}0.28\phantom{0}$ & $\phantom{0}95.1\phantom{0}$ & $\phantom{0}83.7\phantom{0}$ & $\phantom{0}72.6\phantom{0}$ & $\phantom{0}92.7\phantom{0}$ & $\phantom{0}92.1\phantom{0}$ & $\phantom{0}92.1\phantom{0}$ \\
 & \nopagebreak $\;J=50$  & $\phantom{-}0.00\phantom{0}$ & ${-}0.20\phantom{0}$ & ${-}0.17\phantom{0}$ & ${-}0.00\phantom{0}$ & $\phantom{-}0.02\phantom{0}$ & ${-}0.12\phantom{0}$ & $\phantom{0}0.14\phantom{0}$ & $\phantom{0}0.26\phantom{0}$ & $\phantom{0}0.24\phantom{0}$ & $\phantom{0}0.26\phantom{0}$ & $\phantom{0}0.26\phantom{0}$ & $\phantom{0}0.23\phantom{0}$ & $\phantom{0}95.3\phantom{0}$ & $\phantom{0}78.2\phantom{0}$ & $\phantom{0}65.7\phantom{0}$ & $\phantom{0}95.0\phantom{0}$ & $\phantom{0}92.9\phantom{0}$ & $\phantom{0}92.1\phantom{0}$ \\
 & \nopagebreak $\;J=100$  & $\phantom{-}0.00\phantom{0}$ & ${-}0.19\phantom{0}$ & ${-}0.16\phantom{0}$ & $\phantom{-}0.01\phantom{0}$ & $\phantom{-}0.04\phantom{0}$ & ${-}0.09\phantom{0}$ & $\phantom{0}0.10\phantom{0}$ & $\phantom{0}0.23\phantom{0}$ & $\phantom{0}0.21\phantom{0}$ & $\phantom{0}0.19\phantom{0}$ & $\phantom{0}0.19\phantom{0}$ & $\phantom{0}0.17\phantom{0}$ & $\phantom{0}93.7\phantom{0}$ & $\phantom{0}65.4\phantom{0}$ & $\phantom{0}53.2\phantom{0}$ & $\phantom{0}93.5\phantom{0}$ & $\phantom{0}90.7\phantom{0}$ & $\phantom{0}91.8\phantom{0}$ \\
 & \nopagebreak $\;J=200$  & $\phantom{-}0.00\phantom{0}$ & ${-}0.19\phantom{0}$ & ${-}0.16\phantom{0}$ & $\phantom{-}0.00\phantom{0}$ & $\phantom{-}0.03\phantom{0}$ & ${-}0.06\phantom{0}$ & $\phantom{0}0.07\phantom{0}$ & $\phantom{0}0.21\phantom{0}$ & $\phantom{0}0.19\phantom{0}$ & $\phantom{0}0.13\phantom{0}$ & $\phantom{0}0.14\phantom{0}$ & $\phantom{0}0.12\phantom{0}$ & $\phantom{0}95.3\phantom{0}$ & $\phantom{0}41.0\phantom{0}$ & $\phantom{0}33.2\phantom{0}$ & $\phantom{0}93.9\phantom{0}$ & $\phantom{0}91.6\phantom{0}$ & $\phantom{0}92.0\phantom{0}$ \\
 & \nopagebreak $\;J=500$  & ${-}0.00\phantom{0}$ & ${-}0.19\phantom{0}$ & ${-}0.17\phantom{0}$ & $\phantom{-}0.00\phantom{0}$ & $\phantom{-}0.01\phantom{0}$ & ${-}0.04\phantom{0}$ & $\phantom{0}0.05\phantom{0}$ & $\phantom{0}0.20\phantom{0}$ & $\phantom{0}0.18\phantom{0}$ & $\phantom{0}0.08\phantom{0}$ & $\phantom{0}0.08\phantom{0}$ & $\phantom{0}0.08\phantom{0}$ & $\phantom{0}94.3\phantom{0}$ & $\phantom{0}\phantom{0}6.3\phantom{0}$ & $\phantom{0}\phantom{0}6.1\phantom{0}$ & $\phantom{0}94.1\phantom{0}$ & $\phantom{0}92.6\phantom{0}$ & $\phantom{0}92.5\phantom{0}$ \\
 & \nopagebreak $\;J=1000$  & $\phantom{-}0.00\phantom{0}$ & ${-}0.19\phantom{0}$ & ${-}0.16\phantom{0}$ & ${-}0.00\phantom{0}$ & $\phantom{-}0.00\phantom{0}$ & ${-}0.02\phantom{0}$ & $\phantom{0}0.03\phantom{0}$ & $\phantom{0}0.20\phantom{0}$ & $\phantom{0}0.17\phantom{0}$ & $\phantom{0}0.06\phantom{0}$ & $\phantom{0}0.06\phantom{0}$ & $\phantom{0}0.06\phantom{0}$ & $\phantom{0}93.6\phantom{0}$ & $\phantom{0}\phantom{0}0.2\phantom{0}$ & $\phantom{0}\phantom{0}0.5\phantom{0}$ & $\phantom{0}94.2\phantom{0}$ & $\phantom{0}93.2\phantom{0}$ & $\phantom{0}92.5\phantom{0}$ \\
\multicolumn{4}{l}{$n=20$} \\  & \nopagebreak $\;J=30$  & ${-}0.00\phantom{0}$ & ${-}0.27\phantom{0}$ & ${-}0.25\phantom{0}$ & ${-}0.01\phantom{0}$ & $\phantom{-}0.02\phantom{0}$ & ${-}0.18\phantom{0}$ & $\phantom{0}0.18\phantom{0}$ & $\phantom{0}0.35\phantom{0}$ & $\phantom{0}0.33\phantom{0}$ & $\phantom{0}0.35\phantom{0}$ & $\phantom{0}0.37\phantom{0}$ & $\phantom{0}0.30\phantom{0}$ & $\phantom{0}94.2\phantom{0}$ & $\phantom{0}74.9\phantom{0}$ & $\phantom{0}51.5\phantom{0}$ & $\phantom{0}93.7\phantom{0}$ & $\phantom{0}91.8\phantom{0}$ & $\phantom{0}90.1\phantom{0}$ \\
 & \nopagebreak $\;J=50$  & $\phantom{-}0.00\phantom{0}$ & ${-}0.26\phantom{0}$ & ${-}0.25\phantom{0}$ & $\phantom{-}0.00\phantom{0}$ & $\phantom{-}0.03\phantom{0}$ & ${-}0.14\phantom{0}$ & $\phantom{0}0.14\phantom{0}$ & $\phantom{0}0.32\phantom{0}$ & $\phantom{0}0.30\phantom{0}$ & $\phantom{0}0.27\phantom{0}$ & $\phantom{0}0.29\phantom{0}$ & $\phantom{0}0.24\phantom{0}$ & $\phantom{0}94.7\phantom{0}$ & $\phantom{0}63.9\phantom{0}$ & $\phantom{0}42.3\phantom{0}$ & $\phantom{0}92.8\phantom{0}$ & $\phantom{0}92.2\phantom{0}$ & $\phantom{0}90.3\phantom{0}$ \\
 & \nopagebreak $\;J=100$  & ${-}0.00\phantom{0}$ & ${-}0.27\phantom{0}$ & ${-}0.25\phantom{0}$ & ${-}0.00\phantom{0}$ & $\phantom{-}0.01\phantom{0}$ & ${-}0.10\phantom{0}$ & $\phantom{0}0.10\phantom{0}$ & $\phantom{0}0.29\phantom{0}$ & $\phantom{0}0.28\phantom{0}$ & $\phantom{0}0.18\phantom{0}$ & $\phantom{0}0.18\phantom{0}$ & $\phantom{0}0.18\phantom{0}$ & $\phantom{0}94.9\phantom{0}$ & $\phantom{0}41.1\phantom{0}$ & $\phantom{0}20.9\phantom{0}$ & $\phantom{0}94.8\phantom{0}$ & $\phantom{0}92.9\phantom{0}$ & $\phantom{0}90.2\phantom{0}$ \\
 & \nopagebreak $\;J=200$  & $\phantom{-}0.00\phantom{0}$ & ${-}0.27\phantom{0}$ & ${-}0.25\phantom{0}$ & $\phantom{-}0.00\phantom{0}$ & $\phantom{-}0.01\phantom{0}$ & ${-}0.06\phantom{0}$ & $\phantom{0}0.07\phantom{0}$ & $\phantom{0}0.28\phantom{0}$ & $\phantom{0}0.26\phantom{0}$ & $\phantom{0}0.12\phantom{0}$ & $\phantom{0}0.13\phantom{0}$ & $\phantom{0}0.13\phantom{0}$ & $\phantom{0}95.0\phantom{0}$ & $\phantom{0}13.3\phantom{0}$ & $\phantom{0}\phantom{0}5.3\phantom{0}$ & $\phantom{0}94.1\phantom{0}$ & $\phantom{0}92.7\phantom{0}$ & $\phantom{0}91.6\phantom{0}$ \\
 & \nopagebreak $\;J=500$  & ${-}0.00\phantom{0}$ & ${-}0.27\phantom{0}$ & ${-}0.25\phantom{0}$ & $\phantom{-}0.00\phantom{0}$ & $\phantom{-}0.00\phantom{0}$ & ${-}0.03\phantom{0}$ & $\phantom{0}0.04\phantom{0}$ & $\phantom{0}0.27\phantom{0}$ & $\phantom{0}0.25\phantom{0}$ & $\phantom{0}0.08\phantom{0}$ & $\phantom{0}0.08\phantom{0}$ & $\phantom{0}0.08\phantom{0}$ & $\phantom{0}95.3\phantom{0}$ & $\phantom{0}\phantom{0}0.3\phantom{0}$ & $\phantom{0}\phantom{0}0.1\phantom{0}$ & $\phantom{0}93.8\phantom{0}$ & $\phantom{0}92.8\phantom{0}$ & $\phantom{0}92.0\phantom{0}$ \\
 & \nopagebreak $\;J=1000$  & ${-}0.00\phantom{0}$ & ${-}0.27\phantom{0}$ & ${-}0.25\phantom{0}$ & ${-}0.00\phantom{0}$ & ${-}0.00\phantom{0}$ & ${-}0.02\phantom{0}$ & $\phantom{0}0.03\phantom{0}$ & $\phantom{0}0.27\phantom{0}$ & $\phantom{0}0.25\phantom{0}$ & $\phantom{0}0.05\phantom{0}$ & $\phantom{0}0.05\phantom{0}$ & $\phantom{0}0.06\phantom{0}$ & $\phantom{0}94.8\phantom{0}$ & $\phantom{0}\phantom{0}0.0\phantom{0}$ & $\phantom{0}\phantom{0}0.0\phantom{0}$ & $\phantom{0}94.3\phantom{0}$ & $\phantom{0}93.9\phantom{0}$ & $\phantom{0}94.6\phantom{0}$ \\
[0.5ex]\hline\\[-1.6ex] 
& & \multicolumn{18}{c}{Moderate intraclass correlation $(\rho_{Iy}=.30)$} \\[0.6ex]\hline\\[-1.8ex]
\multicolumn{4}{l}{$n=5$} \\  & \nopagebreak $\;J=30$  & $\phantom{-}0.00\phantom{0}$ & ${-}0.27\phantom{0}$ & ${-}0.21\phantom{0}$ & ${-}0.01\phantom{0}$ & $\phantom{-}0.04\phantom{0}$ & ${-}0.13\phantom{0}$ & $\phantom{0}0.19\phantom{0}$ & $\phantom{0}0.35\phantom{0}$ & $\phantom{0}0.31\phantom{0}$ & $\phantom{0}0.36\phantom{0}$ & $\phantom{0}0.38\phantom{0}$ & $\phantom{0}0.29\phantom{0}$ & $\phantom{0}93.2\phantom{0}$ & $\phantom{0}74.4\phantom{0}$ & $\phantom{0}66.3\phantom{0}$ & $\phantom{0}93.9\phantom{0}$ & $\phantom{0}92.7\phantom{0}$ & $\phantom{0}92.3\phantom{0}$ \\
 & \nopagebreak $\;J=50$  & $\phantom{-}0.00\phantom{0}$ & ${-}0.27\phantom{0}$ & ${-}0.20\phantom{0}$ & ${-}0.00\phantom{0}$ & $\phantom{-}0.03\phantom{0}$ & ${-}0.09\phantom{0}$ & $\phantom{0}0.14\phantom{0}$ & $\phantom{0}0.32\phantom{0}$ & $\phantom{0}0.27\phantom{0}$ & $\phantom{0}0.26\phantom{0}$ & $\phantom{0}0.28\phantom{0}$ & $\phantom{0}0.23\phantom{0}$ & $\phantom{0}93.7\phantom{0}$ & $\phantom{0}65.0\phantom{0}$ & $\phantom{0}59.1\phantom{0}$ & $\phantom{0}93.9\phantom{0}$ & $\phantom{0}91.5\phantom{0}$ & $\phantom{0}92.9\phantom{0}$ \\
 & \nopagebreak $\;J=100$  & $\phantom{-}0.01\phantom{0}$ & ${-}0.26\phantom{0}$ & ${-}0.19\phantom{0}$ & $\phantom{-}0.01\phantom{0}$ & $\phantom{-}0.02\phantom{0}$ & ${-}0.04\phantom{0}$ & $\phantom{0}0.10\phantom{0}$ & $\phantom{0}0.29\phantom{0}$ & $\phantom{0}0.23\phantom{0}$ & $\phantom{0}0.18\phantom{0}$ & $\phantom{0}0.19\phantom{0}$ & $\phantom{0}0.17\phantom{0}$ & $\phantom{0}94.7\phantom{0}$ & $\phantom{0}43.1\phantom{0}$ & $\phantom{0}44.6\phantom{0}$ & $\phantom{0}92.8\phantom{0}$ & $\phantom{0}92.9\phantom{0}$ & $\phantom{0}94.5\phantom{0}$ \\
 & \nopagebreak $\;J=200$  & ${-}0.00\phantom{0}$ & ${-}0.27\phantom{0}$ & ${-}0.20\phantom{0}$ & ${-}0.00\phantom{0}$ & $\phantom{-}0.00\phantom{0}$ & ${-}0.03\phantom{0}$ & $\phantom{0}0.07\phantom{0}$ & $\phantom{0}0.28\phantom{0}$ & $\phantom{0}0.22\phantom{0}$ & $\phantom{0}0.13\phantom{0}$ & $\phantom{0}0.13\phantom{0}$ & $\phantom{0}0.13\phantom{0}$ & $\phantom{0}94.7\phantom{0}$ & $\phantom{0}13.4\phantom{0}$ & $\phantom{0}21.5\phantom{0}$ & $\phantom{0}93.8\phantom{0}$ & $\phantom{0}92.3\phantom{0}$ & $\phantom{0}93.3\phantom{0}$ \\
 & \nopagebreak $\;J=500$  & $\phantom{-}0.00\phantom{0}$ & ${-}0.26\phantom{0}$ & ${-}0.20\phantom{0}$ & $\phantom{-}0.00\phantom{0}$ & $\phantom{-}0.00\phantom{0}$ & ${-}0.01\phantom{0}$ & $\phantom{0}0.05\phantom{0}$ & $\phantom{0}0.27\phantom{0}$ & $\phantom{0}0.21\phantom{0}$ & $\phantom{0}0.08\phantom{0}$ & $\phantom{0}0.08\phantom{0}$ & $\phantom{0}0.08\phantom{0}$ & $\phantom{0}94.3\phantom{0}$ & $\phantom{0}\phantom{0}0.6\phantom{0}$ & $\phantom{0}\phantom{0}2.3\phantom{0}$ & $\phantom{0}92.8\phantom{0}$ & $\phantom{0}93.4\phantom{0}$ & $\phantom{0}93.0\phantom{0}$ \\
 & \nopagebreak $\;J=1000$  & ${-}0.00\phantom{0}$ & ${-}0.27\phantom{0}$ & ${-}0.20\phantom{0}$ & $\phantom{-}0.00\phantom{0}$ & $\phantom{-}0.00\phantom{0}$ & ${-}0.00\phantom{0}$ & $\phantom{0}0.03\phantom{0}$ & $\phantom{0}0.27\phantom{0}$ & $\phantom{0}0.20\phantom{0}$ & $\phantom{0}0.06\phantom{0}$ & $\phantom{0}0.06\phantom{0}$ & $\phantom{0}0.06\phantom{0}$ & $\phantom{0}96.1\phantom{0}$ & $\phantom{0}\phantom{0}0.0\phantom{0}$ & $\phantom{0}\phantom{0}0.0\phantom{0}$ & $\phantom{0}93.9\phantom{0}$ & $\phantom{0}93.4\phantom{0}$ & $\phantom{0}94.4\phantom{0}$ \\
\multicolumn{4}{l}{$n=20$} \\  & \nopagebreak $\;J=30$  & ${-}0.01\phantom{0}$ & ${-}0.31\phantom{0}$ & ${-}0.26\phantom{0}$ & ${-}0.01\phantom{0}$ & $\phantom{-}0.00\phantom{0}$ & ${-}0.12\phantom{0}$ & $\phantom{0}0.18\phantom{0}$ & $\phantom{0}0.38\phantom{0}$ & $\phantom{0}0.34\phantom{0}$ & $\phantom{0}0.33\phantom{0}$ & $\phantom{0}0.35\phantom{0}$ & $\phantom{0}0.30\phantom{0}$ & $\phantom{0}94.6\phantom{0}$ & $\phantom{0}67.1\phantom{0}$ & $\phantom{0}49.4\phantom{0}$ & $\phantom{0}95.2\phantom{0}$ & $\phantom{0}93.1\phantom{0}$ & $\phantom{0}92.7\phantom{0}$ \\
 & \nopagebreak $\;J=50$  & ${-}0.00\phantom{0}$ & ${-}0.31\phantom{0}$ & ${-}0.25\phantom{0}$ & ${-}0.00\phantom{0}$ & $\phantom{-}0.00\phantom{0}$ & ${-}0.08\phantom{0}$ & $\phantom{0}0.14\phantom{0}$ & $\phantom{0}0.35\phantom{0}$ & $\phantom{0}0.31\phantom{0}$ & $\phantom{0}0.26\phantom{0}$ & $\phantom{0}0.26\phantom{0}$ & $\phantom{0}0.23\phantom{0}$ & $\phantom{0}94.5\phantom{0}$ & $\phantom{0}54.8\phantom{0}$ & $\phantom{0}38.1\phantom{0}$ & $\phantom{0}93.3\phantom{0}$ & $\phantom{0}94.3\phantom{0}$ & $\phantom{0}93.1\phantom{0}$ \\
 & \nopagebreak $\;J=100$  & $\phantom{-}0.00\phantom{0}$ & ${-}0.31\phantom{0}$ & ${-}0.26\phantom{0}$ & ${-}0.00\phantom{0}$ & ${-}0.00\phantom{0}$ & ${-}0.05\phantom{0}$ & $\phantom{0}0.10\phantom{0}$ & $\phantom{0}0.33\phantom{0}$ & $\phantom{0}0.28\phantom{0}$ & $\phantom{0}0.17\phantom{0}$ & $\phantom{0}0.17\phantom{0}$ & $\phantom{0}0.17\phantom{0}$ & $\phantom{0}95.1\phantom{0}$ & $\phantom{0}25.5\phantom{0}$ & $\phantom{0}18.1\phantom{0}$ & $\phantom{0}94.8\phantom{0}$ & $\phantom{0}94.0\phantom{0}$ & $\phantom{0}95.0\phantom{0}$ \\
 & \nopagebreak $\;J=200$  & ${-}0.00\phantom{0}$ & ${-}0.30\phantom{0}$ & ${-}0.25\phantom{0}$ & ${-}0.00\phantom{0}$ & ${-}0.00\phantom{0}$ & ${-}0.03\phantom{0}$ & $\phantom{0}0.07\phantom{0}$ & $\phantom{0}0.32\phantom{0}$ & $\phantom{0}0.27\phantom{0}$ & $\phantom{0}0.12\phantom{0}$ & $\phantom{0}0.12\phantom{0}$ & $\phantom{0}0.12\phantom{0}$ & $\phantom{0}93.6\phantom{0}$ & $\phantom{0}\phantom{0}6.2\phantom{0}$ & $\phantom{0}\phantom{0}5.6\phantom{0}$ & $\phantom{0}92.7\phantom{0}$ & $\phantom{0}93.1\phantom{0}$ & $\phantom{0}95.3\phantom{0}$ \\
 & \nopagebreak $\;J=500$  & $\phantom{-}0.00\phantom{0}$ & ${-}0.30\phantom{0}$ & ${-}0.25\phantom{0}$ & $\phantom{-}0.01\phantom{0}$ & $\phantom{-}0.01\phantom{0}$ & ${-}0.00\phantom{0}$ & $\phantom{0}0.04\phantom{0}$ & $\phantom{0}0.31\phantom{0}$ & $\phantom{0}0.25\phantom{0}$ & $\phantom{0}0.08\phantom{0}$ & $\phantom{0}0.08\phantom{0}$ & $\phantom{0}0.07\phantom{0}$ & $\phantom{0}96.0\phantom{0}$ & $\phantom{0}\phantom{0}0.0\phantom{0}$ & $\phantom{0}\phantom{0}0.0\phantom{0}$ & $\phantom{0}94.7\phantom{0}$ & $\phantom{0}93.9\phantom{0}$ & $\phantom{0}95.8\phantom{0}$ \\
 & \nopagebreak $\;J=1000$  & ${-}0.00\phantom{0}$ & ${-}0.31\phantom{0}$ & ${-}0.25\phantom{0}$ & $\phantom{-}0.00\phantom{0}$ & $\phantom{-}0.00\phantom{0}$ & ${-}0.00\phantom{0}$ & $\phantom{0}0.03\phantom{0}$ & $\phantom{0}0.31\phantom{0}$ & $\phantom{0}0.25\phantom{0}$ & $\phantom{0}0.06\phantom{0}$ & $\phantom{0}0.06\phantom{0}$ & $\phantom{0}0.06\phantom{0}$ & $\phantom{0}95.7\phantom{0}$ & $\phantom{0}\phantom{0}0.0\phantom{0}$ & $\phantom{0}\phantom{0}0.0\phantom{0}$ & $\phantom{0}93.1\phantom{0}$ & $\phantom{0}93.1\phantom{0}$ & $\phantom{0}94.7\phantom{0}$ \\
[0.5ex]\hline\\[-1.6ex] 
\end{tabular}
\begin{tablenotes}[para,flushleft]{\footnotesize \textit{Note.} $n$ = cluster size; $J$ = number of clusters; CD = complete data sets; LD = listwise deletion; FCS-SL = single-level FCS; FCS-MAN = two-level FCS with manifest cluster means; FCS-LAT = two-level FCS with latent cluster means; JM = joint modeling.}\end{tablenotes}
\end{threeparttable}
\end{sidewaystable}
\begin{sidewaystable}
\begin{threeparttable}
\setlength{\tabcolsep}{1.2pt}
\renewcommand{\arraystretch}{0.95}
\footnotesize
\caption{\small Study 1: Bias (in \%), RMSE, and Coverage of the 95\% Confidence Interval for the Variance of $z$ ($\hat\sigma_z^2$) With 20\% Missing Data (MCAR, $\lambda=0$)}
\begin{tabular}{llcccccccccccccccccc}
\hline\\[-1.8ex]
& & \multicolumn{6}{c}{Bias (\%)} & \multicolumn{6}{c}{RMSE} & \multicolumn{6}{c}{Coverage (\%)} \\ \cmidrule(r){3-8}\cmidrule(r){9-14}\cmidrule(r){15-20}
 &  & CD & LD & \makecell{FCS-\\SL} & \makecell{FCS-\\MAN} & \makecell{FCS-\\LAT} & JM & CD & LD & \makecell{FCS-\\SL} & \makecell{FCS-\\MAN} & \makecell{FCS-\\LAT} & JM & CD & LD & \makecell{FCS-\\SL} & \makecell{FCS-\\MAN} & \makecell{FCS-\\LAT} & \multicolumn{1}{c}{JM} \\ 
[0.4ex]\hline\\[-1.8ex]
& & \multicolumn{18}{c}{Small intraclass correlation $(\rho_{Iy}=.10)$} \\[0.6ex]\hline\\[-1.8ex]
\multicolumn{4}{l}{$n=5$} \\  & \nopagebreak $\;J=30$  & $\phantom{0}{-}2.6\phantom{0}$ & $\phantom{0}{-}3.5\phantom{0}$ & ${-}18.6\phantom{0}$ & $\phantom{0}\phantom{-}1.5\phantom{0}$ & $\phantom{0}\phantom{-}1.4\phantom{0}$ & $\phantom{0}{-}1.6\phantom{0}$ & $\phantom{0}0.26\phantom{0}$ & $\phantom{0}0.29\phantom{0}$ & $\phantom{0}0.31\phantom{0}$ & $\phantom{0}0.31\phantom{0}$ & $\phantom{0}0.31\phantom{0}$ & $\phantom{0}0.30\phantom{0}$ & $\phantom{0}87.5\phantom{0}$ & $\phantom{0}84.8\phantom{0}$ & $\phantom{0}71.1\phantom{0}$ & $\phantom{0}89.7\phantom{0}$ & $\phantom{0}89.8\phantom{0}$ & $\phantom{0}87.7\phantom{0}$ \\
 & \nopagebreak $\;J=50$  & $\phantom{0}{-}2.1\phantom{0}$ & $\phantom{0}{-}2.6\phantom{0}$ & ${-}17.9\phantom{0}$ & $\phantom{0}\phantom{-}0.2\phantom{0}$ & $\phantom{0}\phantom{-}0.3\phantom{0}$ & $\phantom{0}{-}1.6\phantom{0}$ & $\phantom{0}0.20\phantom{0}$ & $\phantom{0}0.23\phantom{0}$ & $\phantom{0}0.27\phantom{0}$ & $\phantom{0}0.24\phantom{0}$ & $\phantom{0}0.23\phantom{0}$ & $\phantom{0}0.23\phantom{0}$ & $\phantom{0}89.5\phantom{0}$ & $\phantom{0}87.5\phantom{0}$ & $\phantom{0}69.8\phantom{0}$ & $\phantom{0}91.8\phantom{0}$ & $\phantom{0}90.9\phantom{0}$ & $\phantom{0}89.7\phantom{0}$ \\
 & \nopagebreak $\;J=100$  & $\phantom{0}{-}1.5\phantom{0}$ & $\phantom{0}{-}1.7\phantom{0}$ & ${-}17.3\phantom{0}$ & $\phantom{0}{-}0.6\phantom{0}$ & $\phantom{0}{-}0.3\phantom{0}$ & $\phantom{0}{-}1.4\phantom{0}$ & $\phantom{0}0.14\phantom{0}$ & $\phantom{0}0.15\phantom{0}$ & $\phantom{0}0.22\phantom{0}$ & $\phantom{0}0.16\phantom{0}$ & $\phantom{0}0.16\phantom{0}$ & $\phantom{0}0.16\phantom{0}$ & $\phantom{0}91.5\phantom{0}$ & $\phantom{0}92.0\phantom{0}$ & $\phantom{0}66.3\phantom{0}$ & $\phantom{0}93.4\phantom{0}$ & $\phantom{0}92.5\phantom{0}$ & $\phantom{0}92.8\phantom{0}$ \\
 & \nopagebreak $\;J=200$  & $\phantom{0}{-}0.3\phantom{0}$ & $\phantom{0}{-}0.4\phantom{0}$ & ${-}16.2\phantom{0}$ & $\phantom{0}\phantom{-}0.3\phantom{0}$ & $\phantom{0}\phantom{-}0.3\phantom{0}$ & $\phantom{0}{-}0.2\phantom{0}$ & $\phantom{0}0.10\phantom{0}$ & $\phantom{0}0.11\phantom{0}$ & $\phantom{0}0.19\phantom{0}$ & $\phantom{0}0.11\phantom{0}$ & $\phantom{0}0.11\phantom{0}$ & $\phantom{0}0.11\phantom{0}$ & $\phantom{0}93.8\phantom{0}$ & $\phantom{0}93.5\phantom{0}$ & $\phantom{0}54.1\phantom{0}$ & $\phantom{0}94.7\phantom{0}$ & $\phantom{0}94.5\phantom{0}$ & $\phantom{0}94.3\phantom{0}$ \\
 & \nopagebreak $\;J=500$  & $\phantom{0}{-}0.0\phantom{0}$ & $\phantom{0}{-}0.1\phantom{0}$ & ${-}16.1\phantom{0}$ & $\phantom{0}\phantom{-}0.2\phantom{0}$ & $\phantom{0}\phantom{-}0.1\phantom{0}$ & $\phantom{0}{-}0.1\phantom{0}$ & $\phantom{0}0.06\phantom{0}$ & $\phantom{0}0.07\phantom{0}$ & $\phantom{0}0.17\phantom{0}$ & $\phantom{0}0.07\phantom{0}$ & $\phantom{0}0.07\phantom{0}$ & $\phantom{0}0.07\phantom{0}$ & $\phantom{0}94.3\phantom{0}$ & $\phantom{0}94.6\phantom{0}$ & $\phantom{0}24.9\phantom{0}$ & $\phantom{0}94.0\phantom{0}$ & $\phantom{0}94.4\phantom{0}$ & $\phantom{0}94.1\phantom{0}$ \\
 & \nopagebreak $\;J=1000$  & $\phantom{0}\phantom{-}0.2\phantom{0}$ & $\phantom{0}\phantom{-}0.2\phantom{0}$ & ${-}15.8\phantom{0}$ & $\phantom{0}\phantom{-}0.3\phantom{0}$ & $\phantom{0}\phantom{-}0.3\phantom{0}$ & $\phantom{0}\phantom{-}0.2\phantom{0}$ & $\phantom{0}0.04\phantom{0}$ & $\phantom{0}0.05\phantom{0}$ & $\phantom{0}0.16\phantom{0}$ & $\phantom{0}0.05\phantom{0}$ & $\phantom{0}0.05\phantom{0}$ & $\phantom{0}0.05\phantom{0}$ & $\phantom{0}94.6\phantom{0}$ & $\phantom{0}95.3\phantom{0}$ & $\phantom{0}\phantom{0}5.8\phantom{0}$ & $\phantom{0}95.2\phantom{0}$ & $\phantom{0}94.8\phantom{0}$ & $\phantom{0}95.1\phantom{0}$ \\
\multicolumn{4}{l}{$n=20$} \\  & \nopagebreak $\;J=30$  & $\phantom{0}{-}3.3\phantom{0}$ & $\phantom{0}{-}4.2\phantom{0}$ & ${-}22.4\phantom{0}$ & $\phantom{0}\phantom{-}0.1\phantom{0}$ & $\phantom{0}\phantom{-}0.6\phantom{0}$ & $\phantom{0}{-}2.4\phantom{0}$ & $\phantom{0}0.25\phantom{0}$ & $\phantom{0}0.28\phantom{0}$ & $\phantom{0}0.32\phantom{0}$ & $\phantom{0}0.30\phantom{0}$ & $\phantom{0}0.29\phantom{0}$ & $\phantom{0}0.28\phantom{0}$ & $\phantom{0}86.6\phantom{0}$ & $\phantom{0}84.8\phantom{0}$ & $\phantom{0}67.5\phantom{0}$ & $\phantom{0}88.8\phantom{0}$ & $\phantom{0}89.7\phantom{0}$ & $\phantom{0}88.0\phantom{0}$ \\
 & \nopagebreak $\;J=50$  & $\phantom{0}{-}1.6\phantom{0}$ & $\phantom{0}{-}2.3\phantom{0}$ & ${-}20.9\phantom{0}$ & $\phantom{0}\phantom{-}0.2\phantom{0}$ & $\phantom{0}\phantom{-}0.2\phantom{0}$ & $\phantom{0}{-}1.5\phantom{0}$ & $\phantom{0}0.20\phantom{0}$ & $\phantom{0}0.23\phantom{0}$ & $\phantom{0}0.28\phantom{0}$ & $\phantom{0}0.24\phantom{0}$ & $\phantom{0}0.24\phantom{0}$ & $\phantom{0}0.23\phantom{0}$ & $\phantom{0}89.5\phantom{0}$ & $\phantom{0}88.7\phantom{0}$ & $\phantom{0}66.1\phantom{0}$ & $\phantom{0}90.9\phantom{0}$ & $\phantom{0}91.3\phantom{0}$ & $\phantom{0}90.2\phantom{0}$ \\
 & \nopagebreak $\;J=100$  & $\phantom{0}{-}0.6\phantom{0}$ & $\phantom{0}{-}0.8\phantom{0}$ & ${-}19.6\phantom{0}$ & $\phantom{0}\phantom{-}0.4\phantom{0}$ & $\phantom{0}\phantom{-}0.5\phantom{0}$ & $\phantom{0}{-}0.6\phantom{0}$ & $\phantom{0}0.14\phantom{0}$ & $\phantom{0}0.16\phantom{0}$ & $\phantom{0}0.24\phantom{0}$ & $\phantom{0}0.16\phantom{0}$ & $\phantom{0}0.16\phantom{0}$ & $\phantom{0}0.16\phantom{0}$ & $\phantom{0}92.3\phantom{0}$ & $\phantom{0}91.9\phantom{0}$ & $\phantom{0}60.1\phantom{0}$ & $\phantom{0}92.9\phantom{0}$ & $\phantom{0}93.2\phantom{0}$ & $\phantom{0}92.9\phantom{0}$ \\
 & \nopagebreak $\;J=200$  & $\phantom{0}{-}0.5\phantom{0}$ & $\phantom{0}{-}0.6\phantom{0}$ & ${-}19.5\phantom{0}$ & $\phantom{0}{-}0.0\phantom{0}$ & $\phantom{0}\phantom{-}0.1\phantom{0}$ & $\phantom{0}{-}0.5\phantom{0}$ & $\phantom{0}0.10\phantom{0}$ & $\phantom{0}0.11\phantom{0}$ & $\phantom{0}0.22\phantom{0}$ & $\phantom{0}0.11\phantom{0}$ & $\phantom{0}0.11\phantom{0}$ & $\phantom{0}0.11\phantom{0}$ & $\phantom{0}94.0\phantom{0}$ & $\phantom{0}94.4\phantom{0}$ & $\phantom{0}43.3\phantom{0}$ & $\phantom{0}95.0\phantom{0}$ & $\phantom{0}94.5\phantom{0}$ & $\phantom{0}94.7\phantom{0}$ \\
 & \nopagebreak $\;J=500$  & $\phantom{0}{-}0.3\phantom{0}$ & $\phantom{0}{-}0.3\phantom{0}$ & ${-}19.3\phantom{0}$ & $\phantom{0}{-}0.2\phantom{0}$ & $\phantom{0}{-}0.2\phantom{0}$ & $\phantom{0}{-}0.3\phantom{0}$ & $\phantom{0}0.06\phantom{0}$ & $\phantom{0}0.07\phantom{0}$ & $\phantom{0}0.20\phantom{0}$ & $\phantom{0}0.07\phantom{0}$ & $\phantom{0}0.07\phantom{0}$ & $\phantom{0}0.07\phantom{0}$ & $\phantom{0}94.4\phantom{0}$ & $\phantom{0}93.8\phantom{0}$ & $\phantom{0}12.4\phantom{0}$ & $\phantom{0}93.8\phantom{0}$ & $\phantom{0}93.6\phantom{0}$ & $\phantom{0}94.0\phantom{0}$ \\
 & \nopagebreak $\;J=1000$  & $\phantom{0}{-}0.1\phantom{0}$ & $\phantom{0}\phantom{-}0.0\phantom{0}$ & ${-}19.0\phantom{0}$ & $\phantom{0}\phantom{-}0.2\phantom{0}$ & $\phantom{0}\phantom{-}0.2\phantom{0}$ & $\phantom{0}\phantom{-}0.0\phantom{0}$ & $\phantom{0}0.05\phantom{0}$ & $\phantom{0}0.05\phantom{0}$ & $\phantom{0}0.19\phantom{0}$ & $\phantom{0}0.05\phantom{0}$ & $\phantom{0}0.05\phantom{0}$ & $\phantom{0}0.05\phantom{0}$ & $\phantom{0}94.4\phantom{0}$ & $\phantom{0}95.0\phantom{0}$ & $\phantom{0}\phantom{0}1.2\phantom{0}$ & $\phantom{0}95.0\phantom{0}$ & $\phantom{0}94.9\phantom{0}$ & $\phantom{0}94.8\phantom{0}$ \\
[0.5ex]\hline\\[-1.6ex] 
& & \multicolumn{18}{c}{Moderate intraclass correlation $(\rho_{Iy}=.30)$} \\[0.6ex]\hline\\[-1.8ex]
\multicolumn{4}{l}{$n=5$} \\  & \nopagebreak $\;J=30$  & $\phantom{0}{-}3.3\phantom{0}$ & $\phantom{0}{-}4.4\phantom{0}$ & ${-}18.8\phantom{0}$ & $\phantom{0}\phantom{-}0.3\phantom{0}$ & $\phantom{0}\phantom{-}0.8\phantom{0}$ & $\phantom{0}{-}2.1\phantom{0}$ & $\phantom{0}0.25\phantom{0}$ & $\phantom{0}0.28\phantom{0}$ & $\phantom{0}0.31\phantom{0}$ & $\phantom{0}0.30\phantom{0}$ & $\phantom{0}0.30\phantom{0}$ & $\phantom{0}0.29\phantom{0}$ & $\phantom{0}87.5\phantom{0}$ & $\phantom{0}86.1\phantom{0}$ & $\phantom{0}71.6\phantom{0}$ & $\phantom{0}90.5\phantom{0}$ & $\phantom{0}90.9\phantom{0}$ & $\phantom{0}89.8\phantom{0}$ \\
 & \nopagebreak $\;J=50$  & $\phantom{0}{-}3.2\phantom{0}$ & $\phantom{0}{-}3.4\phantom{0}$ & ${-}18.2\phantom{0}$ & $\phantom{0}{-}1.1\phantom{0}$ & $\phantom{0}{-}1.0\phantom{0}$ & $\phantom{0}{-}2.1\phantom{0}$ & $\phantom{0}0.19\phantom{0}$ & $\phantom{0}0.21\phantom{0}$ & $\phantom{0}0.26\phantom{0}$ & $\phantom{0}0.22\phantom{0}$ & $\phantom{0}0.22\phantom{0}$ & $\phantom{0}0.22\phantom{0}$ & $\phantom{0}90.6\phantom{0}$ & $\phantom{0}88.2\phantom{0}$ & $\phantom{0}70.9\phantom{0}$ & $\phantom{0}91.5\phantom{0}$ & $\phantom{0}91.1\phantom{0}$ & $\phantom{0}90.1\phantom{0}$ \\
 & \nopagebreak $\;J=100$  & $\phantom{0}{-}0.9\phantom{0}$ & $\phantom{0}{-}1.1\phantom{0}$ & ${-}16.4\phantom{0}$ & $\phantom{0}\phantom{-}0.0\phantom{0}$ & $\phantom{0}\phantom{-}0.1\phantom{0}$ & $\phantom{0}{-}0.5\phantom{0}$ & $\phantom{0}0.14\phantom{0}$ & $\phantom{0}0.16\phantom{0}$ & $\phantom{0}0.21\phantom{0}$ & $\phantom{0}0.16\phantom{0}$ & $\phantom{0}0.16\phantom{0}$ & $\phantom{0}0.16\phantom{0}$ & $\phantom{0}92.3\phantom{0}$ & $\phantom{0}92.7\phantom{0}$ & $\phantom{0}66.9\phantom{0}$ & $\phantom{0}93.9\phantom{0}$ & $\phantom{0}93.7\phantom{0}$ & $\phantom{0}93.3\phantom{0}$ \\
 & \nopagebreak $\;J=200$  & $\phantom{0}{-}0.4\phantom{0}$ & $\phantom{0}{-}0.4\phantom{0}$ & ${-}15.9\phantom{0}$ & $\phantom{0}\phantom{-}0.2\phantom{0}$ & $\phantom{0}\phantom{-}0.2\phantom{0}$ & $\phantom{0}{-}0.1\phantom{0}$ & $\phantom{0}0.10\phantom{0}$ & $\phantom{0}0.11\phantom{0}$ & $\phantom{0}0.19\phantom{0}$ & $\phantom{0}0.11\phantom{0}$ & $\phantom{0}0.11\phantom{0}$ & $\phantom{0}0.11\phantom{0}$ & $\phantom{0}93.3\phantom{0}$ & $\phantom{0}93.2\phantom{0}$ & $\phantom{0}56.4\phantom{0}$ & $\phantom{0}94.4\phantom{0}$ & $\phantom{0}94.4\phantom{0}$ & $\phantom{0}93.8\phantom{0}$ \\
 & \nopagebreak $\;J=500$  & $\phantom{0}{-}0.3\phantom{0}$ & $\phantom{0}{-}0.2\phantom{0}$ & ${-}15.8\phantom{0}$ & $\phantom{0}\phantom{-}0.0\phantom{0}$ & $\phantom{0}{-}0.0\phantom{0}$ & $\phantom{0}{-}0.1\phantom{0}$ & $\phantom{0}0.06\phantom{0}$ & $\phantom{0}0.07\phantom{0}$ & $\phantom{0}0.17\phantom{0}$ & $\phantom{0}0.07\phantom{0}$ & $\phantom{0}0.07\phantom{0}$ & $\phantom{0}0.07\phantom{0}$ & $\phantom{0}95.3\phantom{0}$ & $\phantom{0}95.8\phantom{0}$ & $\phantom{0}24.8\phantom{0}$ & $\phantom{0}96.1\phantom{0}$ & $\phantom{0}96.1\phantom{0}$ & $\phantom{0}95.4\phantom{0}$ \\
 & \nopagebreak $\;J=1000$  & $\phantom{0}{-}0.1\phantom{0}$ & $\phantom{0}{-}0.1\phantom{0}$ & ${-}15.8\phantom{0}$ & $\phantom{0}{-}0.0\phantom{0}$ & $\phantom{0}{-}0.0\phantom{0}$ & $\phantom{0}{-}0.1\phantom{0}$ & $\phantom{0}0.04\phantom{0}$ & $\phantom{0}0.05\phantom{0}$ & $\phantom{0}0.16\phantom{0}$ & $\phantom{0}0.05\phantom{0}$ & $\phantom{0}0.05\phantom{0}$ & $\phantom{0}0.05\phantom{0}$ & $\phantom{0}94.1\phantom{0}$ & $\phantom{0}94.9\phantom{0}$ & $\phantom{0}\phantom{0}5.9\phantom{0}$ & $\phantom{0}94.4\phantom{0}$ & $\phantom{0}94.5\phantom{0}$ & $\phantom{0}95.2\phantom{0}$ \\
\multicolumn{4}{l}{$n=20$} \\  & \nopagebreak $\;J=30$  & $\phantom{0}{-}2.8\phantom{0}$ & $\phantom{0}{-}3.8\phantom{0}$ & ${-}21.3\phantom{0}$ & $\phantom{0}\phantom{-}0.3\phantom{0}$ & $\phantom{0}\phantom{-}0.4\phantom{0}$ & $\phantom{0}{-}1.7\phantom{0}$ & $\phantom{0}0.25\phantom{0}$ & $\phantom{0}0.28\phantom{0}$ & $\phantom{0}0.32\phantom{0}$ & $\phantom{0}0.29\phantom{0}$ & $\phantom{0}0.29\phantom{0}$ & $\phantom{0}0.28\phantom{0}$ & $\phantom{0}87.3\phantom{0}$ & $\phantom{0}84.8\phantom{0}$ & $\phantom{0}68.2\phantom{0}$ & $\phantom{0}90.1\phantom{0}$ & $\phantom{0}90.1\phantom{0}$ & $\phantom{0}88.3\phantom{0}$ \\
 & \nopagebreak $\;J=50$  & $\phantom{0}{-}1.2\phantom{0}$ & $\phantom{0}{-}0.8\phantom{0}$ & ${-}19.2\phantom{0}$ & $\phantom{0}\phantom{-}1.3\phantom{0}$ & $\phantom{0}\phantom{-}1.5\phantom{0}$ & $\phantom{0}\phantom{-}0.2\phantom{0}$ & $\phantom{0}0.20\phantom{0}$ & $\phantom{0}0.23\phantom{0}$ & $\phantom{0}0.27\phantom{0}$ & $\phantom{0}0.23\phantom{0}$ & $\phantom{0}0.24\phantom{0}$ & $\phantom{0}0.23\phantom{0}$ & $\phantom{0}90.3\phantom{0}$ & $\phantom{0}89.3\phantom{0}$ & $\phantom{0}69.3\phantom{0}$ & $\phantom{0}92.5\phantom{0}$ & $\phantom{0}92.8\phantom{0}$ & $\phantom{0}91.3\phantom{0}$ \\
 & \nopagebreak $\;J=100$  & $\phantom{0}{-}0.7\phantom{0}$ & $\phantom{0}{-}1.0\phantom{0}$ & ${-}19.3\phantom{0}$ & $\phantom{0}\phantom{-}0.1\phantom{0}$ & $\phantom{0}\phantom{-}0.1\phantom{0}$ & $\phantom{0}{-}0.3\phantom{0}$ & $\phantom{0}0.14\phantom{0}$ & $\phantom{0}0.15\phantom{0}$ & $\phantom{0}0.23\phantom{0}$ & $\phantom{0}0.16\phantom{0}$ & $\phantom{0}0.16\phantom{0}$ & $\phantom{0}0.16\phantom{0}$ & $\phantom{0}93.5\phantom{0}$ & $\phantom{0}93.2\phantom{0}$ & $\phantom{0}61.5\phantom{0}$ & $\phantom{0}94.1\phantom{0}$ & $\phantom{0}94.1\phantom{0}$ & $\phantom{0}94.2\phantom{0}$ \\
 & \nopagebreak $\;J=200$  & $\phantom{0}{-}0.4\phantom{0}$ & $\phantom{0}{-}0.2\phantom{0}$ & ${-}18.7\phantom{0}$ & $\phantom{0}\phantom{-}0.2\phantom{0}$ & $\phantom{0}\phantom{-}0.2\phantom{0}$ & $\phantom{0}{-}0.0\phantom{0}$ & $\phantom{0}0.10\phantom{0}$ & $\phantom{0}0.12\phantom{0}$ & $\phantom{0}0.21\phantom{0}$ & $\phantom{0}0.12\phantom{0}$ & $\phantom{0}0.12\phantom{0}$ & $\phantom{0}0.12\phantom{0}$ & $\phantom{0}93.2\phantom{0}$ & $\phantom{0}93.1\phantom{0}$ & $\phantom{0}46.1\phantom{0}$ & $\phantom{0}93.7\phantom{0}$ & $\phantom{0}93.4\phantom{0}$ & $\phantom{0}92.7\phantom{0}$ \\
 & \nopagebreak $\;J=500$  & $\phantom{0}{-}0.4\phantom{0}$ & $\phantom{0}{-}0.4\phantom{0}$ & ${-}18.9\phantom{0}$ & $\phantom{0}{-}0.2\phantom{0}$ & $\phantom{0}{-}0.2\phantom{0}$ & $\phantom{0}{-}0.3\phantom{0}$ & $\phantom{0}0.06\phantom{0}$ & $\phantom{0}0.07\phantom{0}$ & $\phantom{0}0.20\phantom{0}$ & $\phantom{0}0.07\phantom{0}$ & $\phantom{0}0.07\phantom{0}$ & $\phantom{0}0.07\phantom{0}$ & $\phantom{0}94.2\phantom{0}$ & $\phantom{0}93.6\phantom{0}$ & $\phantom{0}14.6\phantom{0}$ & $\phantom{0}93.9\phantom{0}$ & $\phantom{0}94.3\phantom{0}$ & $\phantom{0}94.2\phantom{0}$ \\
 & \nopagebreak $\;J=1000$  & $\phantom{0}\phantom{-}0.1\phantom{0}$ & $\phantom{0}\phantom{-}0.1\phantom{0}$ & ${-}18.5\phantom{0}$ & $\phantom{0}\phantom{-}0.2\phantom{0}$ & $\phantom{0}\phantom{-}0.2\phantom{0}$ & $\phantom{0}\phantom{-}0.2\phantom{0}$ & $\phantom{0}0.05\phantom{0}$ & $\phantom{0}0.05\phantom{0}$ & $\phantom{0}0.19\phantom{0}$ & $\phantom{0}0.05\phantom{0}$ & $\phantom{0}0.05\phantom{0}$ & $\phantom{0}0.05\phantom{0}$ & $\phantom{0}94.6\phantom{0}$ & $\phantom{0}94.8\phantom{0}$ & $\phantom{0}\phantom{0}2.1\phantom{0}$ & $\phantom{0}94.3\phantom{0}$ & $\phantom{0}95.0\phantom{0}$ & $\phantom{0}94.7\phantom{0}$ \\
[0.5ex]\hline\\[-1.6ex] 
\end{tabular}
\begin{tablenotes}[para,flushleft]{\footnotesize \textit{Note.} $n$ = cluster size; $J$ = number of clusters; CD = complete data sets; LD = listwise deletion; FCS-SL = single-level FCS; FCS-MAN = two-level FCS with manifest cluster means; FCS-LAT = two-level FCS with latent cluster means; JM = joint modeling.}\end{tablenotes}
\end{threeparttable}
\end{sidewaystable}
\begin{sidewaystable}
\begin{threeparttable}
\setlength{\tabcolsep}{1.2pt}
\renewcommand{\arraystretch}{0.95}
\footnotesize
\caption{\small Study 1: Bias (in \%), RMSE, and Coverage of the 95\% Confidence Interval for the Variance of $z$ ($\hat\sigma_z^2$) With 20\% Missing Data (MAR, $\lambda=0.5$)}
\begin{tabular}{llcccccccccccccccccc}
\hline\\[-1.8ex]
& & \multicolumn{6}{c}{Bias (\%)} & \multicolumn{6}{c}{RMSE} & \multicolumn{6}{c}{Coverage (\%)} \\ \cmidrule(r){3-8}\cmidrule(r){9-14}\cmidrule(r){15-20}
 &  & CD & LD & \makecell{FCS-\\SL} & \makecell{FCS-\\MAN} & \makecell{FCS-\\LAT} & JM & CD & LD & \makecell{FCS-\\SL} & \makecell{FCS-\\MAN} & \makecell{FCS-\\LAT} & JM & CD & LD & \makecell{FCS-\\SL} & \makecell{FCS-\\MAN} & \makecell{FCS-\\LAT} & \multicolumn{1}{c}{JM} \\ 
[0.4ex]\hline\\[-1.8ex]
& & \multicolumn{18}{c}{Small intraclass correlation $(\rho_{Iy}=.10)$} \\[0.6ex]\hline\\[-1.8ex]
\multicolumn{4}{l}{$n=5$} \\  & \nopagebreak $\;J=30$  & $\phantom{0}{-}3.3\phantom{0}$ & $\phantom{0}{-}5.1\phantom{0}$ & ${-}19.9\phantom{0}$ & $\phantom{0}\phantom{-}1.6\phantom{0}$ & $\phantom{0}\phantom{-}1.9\phantom{0}$ & $\phantom{0}{-}2.6\phantom{0}$ & $\phantom{0}0.25\phantom{0}$ & $\phantom{0}0.28\phantom{0}$ & $\phantom{0}0.31\phantom{0}$ & $\phantom{0}0.31\phantom{0}$ & $\phantom{0}0.31\phantom{0}$ & $\phantom{0}0.29\phantom{0}$ & $\phantom{0}87.6\phantom{0}$ & $\phantom{0}85.8\phantom{0}$ & $\phantom{0}70.5\phantom{0}$ & $\phantom{0}90.6\phantom{0}$ & $\phantom{0}90.2\phantom{0}$ & $\phantom{0}88.6\phantom{0}$ \\
 & \nopagebreak $\;J=50$  & $\phantom{0}{-}2.3\phantom{0}$ & $\phantom{0}{-}4.0\phantom{0}$ & ${-}19.2\phantom{0}$ & $\phantom{0}\phantom{-}0.1\phantom{0}$ & $\phantom{0}\phantom{-}0.6\phantom{0}$ & $\phantom{0}{-}2.4\phantom{0}$ & $\phantom{0}0.19\phantom{0}$ & $\phantom{0}0.22\phantom{0}$ & $\phantom{0}0.27\phantom{0}$ & $\phantom{0}0.23\phantom{0}$ & $\phantom{0}0.23\phantom{0}$ & $\phantom{0}0.22\phantom{0}$ & $\phantom{0}90.9\phantom{0}$ & $\phantom{0}87.7\phantom{0}$ & $\phantom{0}68.5\phantom{0}$ & $\phantom{0}92.5\phantom{0}$ & $\phantom{0}92.1\phantom{0}$ & $\phantom{0}90.3\phantom{0}$ \\
 & \nopagebreak $\;J=100$  & $\phantom{0}{-}1.9\phantom{0}$ & $\phantom{0}{-}2.9\phantom{0}$ & ${-}18.3\phantom{0}$ & $\phantom{0}{-}0.4\phantom{0}$ & $\phantom{0}{-}0.1\phantom{0}$ & $\phantom{0}{-}1.6\phantom{0}$ & $\phantom{0}0.14\phantom{0}$ & $\phantom{0}0.16\phantom{0}$ & $\phantom{0}0.23\phantom{0}$ & $\phantom{0}0.16\phantom{0}$ & $\phantom{0}0.17\phantom{0}$ & $\phantom{0}0.16\phantom{0}$ & $\phantom{0}92.1\phantom{0}$ & $\phantom{0}90.7\phantom{0}$ & $\phantom{0}62.5\phantom{0}$ & $\phantom{0}92.7\phantom{0}$ & $\phantom{0}92.5\phantom{0}$ & $\phantom{0}91.9\phantom{0}$ \\
 & \nopagebreak $\;J=200$  & $\phantom{0}{-}0.7\phantom{0}$ & $\phantom{0}{-}2.0\phantom{0}$ & ${-}17.3\phantom{0}$ & $\phantom{0}{-}0.2\phantom{0}$ & $\phantom{0}{-}0.1\phantom{0}$ & $\phantom{0}{-}1.0\phantom{0}$ & $\phantom{0}0.10\phantom{0}$ & $\phantom{0}0.11\phantom{0}$ & $\phantom{0}0.20\phantom{0}$ & $\phantom{0}0.11\phantom{0}$ & $\phantom{0}0.11\phantom{0}$ & $\phantom{0}0.11\phantom{0}$ & $\phantom{0}94.4\phantom{0}$ & $\phantom{0}92.5\phantom{0}$ & $\phantom{0}51.4\phantom{0}$ & $\phantom{0}93.7\phantom{0}$ & $\phantom{0}93.9\phantom{0}$ & $\phantom{0}94.3\phantom{0}$ \\
 & \nopagebreak $\;J=500$  & $\phantom{0}{-}0.1\phantom{0}$ & $\phantom{0}{-}1.1\phantom{0}$ & ${-}16.8\phantom{0}$ & $\phantom{0}\phantom{-}0.1\phantom{0}$ & $\phantom{0}\phantom{-}0.1\phantom{0}$ & $\phantom{0}{-}0.2\phantom{0}$ & $\phantom{0}0.06\phantom{0}$ & $\phantom{0}0.07\phantom{0}$ & $\phantom{0}0.18\phantom{0}$ & $\phantom{0}0.07\phantom{0}$ & $\phantom{0}0.07\phantom{0}$ & $\phantom{0}0.07\phantom{0}$ & $\phantom{0}94.2\phantom{0}$ & $\phantom{0}92.7\phantom{0}$ & $\phantom{0}22.2\phantom{0}$ & $\phantom{0}93.9\phantom{0}$ & $\phantom{0}93.2\phantom{0}$ & $\phantom{0}94.5\phantom{0}$ \\
 & \nopagebreak $\;J=1000$  & $\phantom{0}{-}0.2\phantom{0}$ & $\phantom{0}{-}1.2\phantom{0}$ & ${-}16.9\phantom{0}$ & $\phantom{0}{-}0.0\phantom{0}$ & $\phantom{0}{-}0.1\phantom{0}$ & $\phantom{0}{-}0.2\phantom{0}$ & $\phantom{0}0.05\phantom{0}$ & $\phantom{0}0.05\phantom{0}$ & $\phantom{0}0.17\phantom{0}$ & $\phantom{0}0.05\phantom{0}$ & $\phantom{0}0.05\phantom{0}$ & $\phantom{0}0.05\phantom{0}$ & $\phantom{0}94.1\phantom{0}$ & $\phantom{0}93.3\phantom{0}$ & $\phantom{0}\phantom{0}3.5\phantom{0}$ & $\phantom{0}94.2\phantom{0}$ & $\phantom{0}94.7\phantom{0}$ & $\phantom{0}93.7\phantom{0}$ \\
\multicolumn{4}{l}{$n=20$} \\  & \nopagebreak $\;J=30$  & $\phantom{0}{-}3.0\phantom{0}$ & $\phantom{0}{-}5.6\phantom{0}$ & ${-}23.4\phantom{0}$ & $\phantom{0}\phantom{-}1.5\phantom{0}$ & $\phantom{0}\phantom{-}2.2\phantom{0}$ & $\phantom{0}{-}2.9\phantom{0}$ & $\phantom{0}0.26\phantom{0}$ & $\phantom{0}0.29\phantom{0}$ & $\phantom{0}0.33\phantom{0}$ & $\phantom{0}0.32\phantom{0}$ & $\phantom{0}0.32\phantom{0}$ & $\phantom{0}0.29\phantom{0}$ & $\phantom{0}87.0\phantom{0}$ & $\phantom{0}83.4\phantom{0}$ & $\phantom{0}66.3\phantom{0}$ & $\phantom{0}89.9\phantom{0}$ & $\phantom{0}89.9\phantom{0}$ & $\phantom{0}87.2\phantom{0}$ \\
 & \nopagebreak $\;J=50$  & $\phantom{0}{-}1.8\phantom{0}$ & $\phantom{0}{-}4.0\phantom{0}$ & ${-}22.2\phantom{0}$ & $\phantom{0}\phantom{-}0.6\phantom{0}$ & $\phantom{0}\phantom{-}1.0\phantom{0}$ & $\phantom{0}{-}2.0\phantom{0}$ & $\phantom{0}0.20\phantom{0}$ & $\phantom{0}0.22\phantom{0}$ & $\phantom{0}0.29\phantom{0}$ & $\phantom{0}0.23\phantom{0}$ & $\phantom{0}0.24\phantom{0}$ & $\phantom{0}0.22\phantom{0}$ & $\phantom{0}90.1\phantom{0}$ & $\phantom{0}89.2\phantom{0}$ & $\phantom{0}65.8\phantom{0}$ & $\phantom{0}93.3\phantom{0}$ & $\phantom{0}93.8\phantom{0}$ & $\phantom{0}91.7\phantom{0}$ \\
 & \nopagebreak $\;J=100$  & $\phantom{0}{-}1.2\phantom{0}$ & $\phantom{0}{-}3.1\phantom{0}$ & ${-}21.4\phantom{0}$ & $\phantom{0}\phantom{-}0.1\phantom{0}$ & $\phantom{0}\phantom{-}0.1\phantom{0}$ & $\phantom{0}{-}1.4\phantom{0}$ & $\phantom{0}0.14\phantom{0}$ & $\phantom{0}0.15\phantom{0}$ & $\phantom{0}0.25\phantom{0}$ & $\phantom{0}0.16\phantom{0}$ & $\phantom{0}0.16\phantom{0}$ & $\phantom{0}0.16\phantom{0}$ & $\phantom{0}92.7\phantom{0}$ & $\phantom{0}91.9\phantom{0}$ & $\phantom{0}54.9\phantom{0}$ & $\phantom{0}94.1\phantom{0}$ & $\phantom{0}93.7\phantom{0}$ & $\phantom{0}93.6\phantom{0}$ \\
 & \nopagebreak $\;J=200$  & $\phantom{0}{-}0.7\phantom{0}$ & $\phantom{0}{-}2.6\phantom{0}$ & ${-}21.1\phantom{0}$ & $\phantom{0}{-}0.3\phantom{0}$ & $\phantom{0}{-}0.2\phantom{0}$ & $\phantom{0}{-}1.0\phantom{0}$ & $\phantom{0}0.10\phantom{0}$ & $\phantom{0}0.12\phantom{0}$ & $\phantom{0}0.23\phantom{0}$ & $\phantom{0}0.12\phantom{0}$ & $\phantom{0}0.12\phantom{0}$ & $\phantom{0}0.12\phantom{0}$ & $\phantom{0}92.9\phantom{0}$ & $\phantom{0}90.9\phantom{0}$ & $\phantom{0}36.8\phantom{0}$ & $\phantom{0}93.1\phantom{0}$ & $\phantom{0}93.1\phantom{0}$ & $\phantom{0}92.6\phantom{0}$ \\
 & \nopagebreak $\;J=500$  & $\phantom{0}{-}0.3\phantom{0}$ & $\phantom{0}{-}2.0\phantom{0}$ & ${-}20.6\phantom{0}$ & $\phantom{0}\phantom{-}0.0\phantom{0}$ & $\phantom{0}\phantom{-}0.1\phantom{0}$ & $\phantom{0}{-}0.3\phantom{0}$ & $\phantom{0}0.06\phantom{0}$ & $\phantom{0}0.07\phantom{0}$ & $\phantom{0}0.21\phantom{0}$ & $\phantom{0}0.07\phantom{0}$ & $\phantom{0}0.07\phantom{0}$ & $\phantom{0}0.07\phantom{0}$ & $\phantom{0}95.3\phantom{0}$ & $\phantom{0}93.4\phantom{0}$ & $\phantom{0}\phantom{0}7.4\phantom{0}$ & $\phantom{0}95.9\phantom{0}$ & $\phantom{0}95.7\phantom{0}$ & $\phantom{0}95.2\phantom{0}$ \\
 & \nopagebreak $\;J=1000$  & $\phantom{0}\phantom{-}0.1\phantom{0}$ & $\phantom{0}{-}1.7\phantom{0}$ & ${-}20.3\phantom{0}$ & $\phantom{0}\phantom{-}0.2\phantom{0}$ & $\phantom{0}\phantom{-}0.2\phantom{0}$ & $\phantom{0}\phantom{-}0.0\phantom{0}$ & $\phantom{0}0.04\phantom{0}$ & $\phantom{0}0.05\phantom{0}$ & $\phantom{0}0.21\phantom{0}$ & $\phantom{0}0.05\phantom{0}$ & $\phantom{0}0.05\phantom{0}$ & $\phantom{0}0.05\phantom{0}$ & $\phantom{0}95.3\phantom{0}$ & $\phantom{0}91.8\phantom{0}$ & $\phantom{0}\phantom{0}0.6\phantom{0}$ & $\phantom{0}95.4\phantom{0}$ & $\phantom{0}95.4\phantom{0}$ & $\phantom{0}94.3\phantom{0}$ \\
[0.5ex]\hline\\[-1.6ex] 
& & \multicolumn{18}{c}{Moderate intraclass correlation $(\rho_{Iy}=.30)$} \\[0.6ex]\hline\\[-1.8ex]
\multicolumn{4}{l}{$n=5$} \\  & \nopagebreak $\;J=30$  & $\phantom{0}{-}3.5\phantom{0}$ & $\phantom{0}{-}5.8\phantom{0}$ & ${-}20.2\phantom{0}$ & $\phantom{0}\phantom{-}1.4\phantom{0}$ & $\phantom{0}\phantom{-}2.0\phantom{0}$ & $\phantom{0}{-}2.2\phantom{0}$ & $\phantom{0}0.25\phantom{0}$ & $\phantom{0}0.27\phantom{0}$ & $\phantom{0}0.31\phantom{0}$ & $\phantom{0}0.30\phantom{0}$ & $\phantom{0}0.30\phantom{0}$ & $\phantom{0}0.28\phantom{0}$ & $\phantom{0}88.2\phantom{0}$ & $\phantom{0}85.0\phantom{0}$ & $\phantom{0}69.8\phantom{0}$ & $\phantom{0}90.5\phantom{0}$ & $\phantom{0}90.8\phantom{0}$ & $\phantom{0}89.1\phantom{0}$ \\
 & \nopagebreak $\;J=50$  & $\phantom{0}{-}2.6\phantom{0}$ & $\phantom{0}{-}4.5\phantom{0}$ & ${-}18.9\phantom{0}$ & $\phantom{0}\phantom{-}0.0\phantom{0}$ & $\phantom{0}\phantom{-}0.2\phantom{0}$ & $\phantom{0}{-}1.6\phantom{0}$ & $\phantom{0}0.20\phantom{0}$ & $\phantom{0}0.23\phantom{0}$ & $\phantom{0}0.27\phantom{0}$ & $\phantom{0}0.23\phantom{0}$ & $\phantom{0}0.24\phantom{0}$ & $\phantom{0}0.23\phantom{0}$ & $\phantom{0}88.7\phantom{0}$ & $\phantom{0}86.6\phantom{0}$ & $\phantom{0}68.4\phantom{0}$ & $\phantom{0}91.4\phantom{0}$ & $\phantom{0}91.5\phantom{0}$ & $\phantom{0}89.6\phantom{0}$ \\
 & \nopagebreak $\;J=100$  & $\phantom{0}{-}1.6\phantom{0}$ & $\phantom{0}{-}3.9\phantom{0}$ & ${-}18.7\phantom{0}$ & $\phantom{0}{-}0.9\phantom{0}$ & $\phantom{0}{-}0.8\phantom{0}$ & $\phantom{0}{-}1.7\phantom{0}$ & $\phantom{0}0.14\phantom{0}$ & $\phantom{0}0.16\phantom{0}$ & $\phantom{0}0.23\phantom{0}$ & $\phantom{0}0.16\phantom{0}$ & $\phantom{0}0.16\phantom{0}$ & $\phantom{0}0.16\phantom{0}$ & $\phantom{0}91.3\phantom{0}$ & $\phantom{0}89.1\phantom{0}$ & $\phantom{0}61.8\phantom{0}$ & $\phantom{0}92.2\phantom{0}$ & $\phantom{0}92.8\phantom{0}$ & $\phantom{0}91.5\phantom{0}$ \\
 & \nopagebreak $\;J=200$  & $\phantom{0}{-}0.8\phantom{0}$ & $\phantom{0}{-}2.7\phantom{0}$ & ${-}17.8\phantom{0}$ & $\phantom{0}{-}0.2\phantom{0}$ & $\phantom{0}{-}0.2\phantom{0}$ & $\phantom{0}{-}0.6\phantom{0}$ & $\phantom{0}0.11\phantom{0}$ & $\phantom{0}0.12\phantom{0}$ & $\phantom{0}0.20\phantom{0}$ & $\phantom{0}0.12\phantom{0}$ & $\phantom{0}0.12\phantom{0}$ & $\phantom{0}0.12\phantom{0}$ & $\phantom{0}91.3\phantom{0}$ & $\phantom{0}88.3\phantom{0}$ & $\phantom{0}48.8\phantom{0}$ & $\phantom{0}92.3\phantom{0}$ & $\phantom{0}92.4\phantom{0}$ & $\phantom{0}91.8\phantom{0}$ \\
 & \nopagebreak $\;J=500$  & $\phantom{0}{-}0.3\phantom{0}$ & $\phantom{0}{-}2.2\phantom{0}$ & ${-}17.2\phantom{0}$ & $\phantom{0}{-}0.2\phantom{0}$ & $\phantom{0}{-}0.1\phantom{0}$ & $\phantom{0}{-}0.3\phantom{0}$ & $\phantom{0}0.06\phantom{0}$ & $\phantom{0}0.07\phantom{0}$ & $\phantom{0}0.18\phantom{0}$ & $\phantom{0}0.07\phantom{0}$ & $\phantom{0}0.07\phantom{0}$ & $\phantom{0}0.07\phantom{0}$ & $\phantom{0}93.3\phantom{0}$ & $\phantom{0}92.7\phantom{0}$ & $\phantom{0}17.9\phantom{0}$ & $\phantom{0}94.4\phantom{0}$ & $\phantom{0}94.3\phantom{0}$ & $\phantom{0}94.7\phantom{0}$ \\
 & \nopagebreak $\;J=1000$  & $\phantom{0}\phantom{-}0.1\phantom{0}$ & $\phantom{0}{-}1.7\phantom{0}$ & ${-}16.8\phantom{0}$ & $\phantom{0}\phantom{-}0.2\phantom{0}$ & $\phantom{0}\phantom{-}0.2\phantom{0}$ & $\phantom{0}\phantom{-}0.1\phantom{0}$ & $\phantom{0}0.05\phantom{0}$ & $\phantom{0}0.05\phantom{0}$ & $\phantom{0}0.17\phantom{0}$ & $\phantom{0}0.05\phantom{0}$ & $\phantom{0}0.05\phantom{0}$ & $\phantom{0}0.05\phantom{0}$ & $\phantom{0}95.2\phantom{0}$ & $\phantom{0}93.0\phantom{0}$ & $\phantom{0}\phantom{0}3.7\phantom{0}$ & $\phantom{0}94.9\phantom{0}$ & $\phantom{0}94.7\phantom{0}$ & $\phantom{0}94.8\phantom{0}$ \\
\multicolumn{4}{l}{$n=20$} \\  & \nopagebreak $\;J=30$  & $\phantom{0}{-}4.4\phantom{0}$ & $\phantom{0}{-}7.2\phantom{0}$ & ${-}24.3\phantom{0}$ & $\phantom{0}\phantom{-}0.0\phantom{0}$ & $\phantom{0}{-}0.2\phantom{0}$ & $\phantom{0}{-}2.9\phantom{0}$ & $\phantom{0}0.25\phantom{0}$ & $\phantom{0}0.29\phantom{0}$ & $\phantom{0}0.34\phantom{0}$ & $\phantom{0}0.31\phantom{0}$ & $\phantom{0}0.31\phantom{0}$ & $\phantom{0}0.30\phantom{0}$ & $\phantom{0}87.2\phantom{0}$ & $\phantom{0}82.2\phantom{0}$ & $\phantom{0}64.2\phantom{0}$ & $\phantom{0}88.9\phantom{0}$ & $\phantom{0}89.3\phantom{0}$ & $\phantom{0}86.9\phantom{0}$ \\
 & \nopagebreak $\;J=50$  & $\phantom{0}{-}2.2\phantom{0}$ & $\phantom{0}{-}5.0\phantom{0}$ & ${-}22.1\phantom{0}$ & $\phantom{0}{-}0.0\phantom{0}$ & $\phantom{0}\phantom{-}0.1\phantom{0}$ & $\phantom{0}{-}1.6\phantom{0}$ & $\phantom{0}0.20\phantom{0}$ & $\phantom{0}0.22\phantom{0}$ & $\phantom{0}0.29\phantom{0}$ & $\phantom{0}0.23\phantom{0}$ & $\phantom{0}0.23\phantom{0}$ & $\phantom{0}0.23\phantom{0}$ & $\phantom{0}90.1\phantom{0}$ & $\phantom{0}87.7\phantom{0}$ & $\phantom{0}63.6\phantom{0}$ & $\phantom{0}91.1\phantom{0}$ & $\phantom{0}91.2\phantom{0}$ & $\phantom{0}90.2\phantom{0}$ \\
 & \nopagebreak $\;J=100$  & $\phantom{0}{-}0.9\phantom{0}$ & $\phantom{0}{-}3.2\phantom{0}$ & ${-}21.1\phantom{0}$ & $\phantom{0}\phantom{-}0.4\phantom{0}$ & $\phantom{0}\phantom{-}0.4\phantom{0}$ & $\phantom{0}{-}0.4\phantom{0}$ & $\phantom{0}0.14\phantom{0}$ & $\phantom{0}0.15\phantom{0}$ & $\phantom{0}0.25\phantom{0}$ & $\phantom{0}0.16\phantom{0}$ & $\phantom{0}0.16\phantom{0}$ & $\phantom{0}0.16\phantom{0}$ & $\phantom{0}92.6\phantom{0}$ & $\phantom{0}90.2\phantom{0}$ & $\phantom{0}55.5\phantom{0}$ & $\phantom{0}93.5\phantom{0}$ & $\phantom{0}93.7\phantom{0}$ & $\phantom{0}93.3\phantom{0}$ \\
 & \nopagebreak $\;J=200$  & $\phantom{0}{-}0.1\phantom{0}$ & $\phantom{0}{-}2.7\phantom{0}$ & ${-}20.6\phantom{0}$ & $\phantom{0}\phantom{-}0.4\phantom{0}$ & $\phantom{0}\phantom{-}0.4\phantom{0}$ & $\phantom{0}{-}0.1\phantom{0}$ & $\phantom{0}0.10\phantom{0}$ & $\phantom{0}0.11\phantom{0}$ & $\phantom{0}0.23\phantom{0}$ & $\phantom{0}0.12\phantom{0}$ & $\phantom{0}0.12\phantom{0}$ & $\phantom{0}0.11\phantom{0}$ & $\phantom{0}93.5\phantom{0}$ & $\phantom{0}90.9\phantom{0}$ & $\phantom{0}36.7\phantom{0}$ & $\phantom{0}94.2\phantom{0}$ & $\phantom{0}94.1\phantom{0}$ & $\phantom{0}93.0\phantom{0}$ \\
 & \nopagebreak $\;J=500$  & $\phantom{0}{-}0.2\phantom{0}$ & $\phantom{0}{-}2.4\phantom{0}$ & ${-}20.4\phantom{0}$ & $\phantom{0}\phantom{-}0.2\phantom{0}$ & $\phantom{0}\phantom{-}0.2\phantom{0}$ & $\phantom{0}\phantom{-}0.1\phantom{0}$ & $\phantom{0}0.06\phantom{0}$ & $\phantom{0}0.08\phantom{0}$ & $\phantom{0}0.21\phantom{0}$ & $\phantom{0}0.08\phantom{0}$ & $\phantom{0}0.08\phantom{0}$ & $\phantom{0}0.08\phantom{0}$ & $\phantom{0}93.7\phantom{0}$ & $\phantom{0}90.8\phantom{0}$ & $\phantom{0}\phantom{0}9.7\phantom{0}$ & $\phantom{0}93.9\phantom{0}$ & $\phantom{0}93.9\phantom{0}$ & $\phantom{0}94.4\phantom{0}$ \\
 & \nopagebreak $\;J=1000$  & $\phantom{0}{-}0.2\phantom{0}$ & $\phantom{0}{-}2.6\phantom{0}$ & ${-}20.6\phantom{0}$ & $\phantom{0}{-}0.1\phantom{0}$ & $\phantom{0}{-}0.1\phantom{0}$ & $\phantom{0}{-}0.2\phantom{0}$ & $\phantom{0}0.04\phantom{0}$ & $\phantom{0}0.05\phantom{0}$ & $\phantom{0}0.21\phantom{0}$ & $\phantom{0}0.05\phantom{0}$ & $\phantom{0}0.05\phantom{0}$ & $\phantom{0}0.05\phantom{0}$ & $\phantom{0}95.5\phantom{0}$ & $\phantom{0}91.1\phantom{0}$ & $\phantom{0}\phantom{0}0.2\phantom{0}$ & $\phantom{0}95.8\phantom{0}$ & $\phantom{0}95.8\phantom{0}$ & $\phantom{0}95.6\phantom{0}$ \\
[0.5ex]\hline\\[-1.6ex] 
\end{tabular}
\begin{tablenotes}[para,flushleft]{\footnotesize \textit{Note.} $n$ = cluster size; $J$ = number of clusters; CD = complete data sets; LD = listwise deletion; FCS-SL = single-level FCS; FCS-MAN = two-level FCS with manifest cluster means; FCS-LAT = two-level FCS with latent cluster means; JM = joint modeling.}\end{tablenotes}
\end{threeparttable}
\end{sidewaystable}
\begin{sidewaystable}
\begin{threeparttable}
\setlength{\tabcolsep}{1.2pt}
\renewcommand{\arraystretch}{0.95}
\footnotesize
\caption{\small Study 1: Bias (in \%), RMSE, and Coverage of the 95\% Confidence Interval for the Variance of $z$ ($\hat\sigma_z^2$) With 20\% Missing Data (MAR, $\lambda=1$)}
\begin{tabular}{llcccccccccccccccccc}
\hline\\[-1.8ex]
& & \multicolumn{6}{c}{Bias (\%)} & \multicolumn{6}{c}{RMSE} & \multicolumn{6}{c}{Coverage (\%)} \\ \cmidrule(r){3-8}\cmidrule(r){9-14}\cmidrule(r){15-20}
 &  & CD & LD & \makecell{FCS-\\SL} & \makecell{FCS-\\MAN} & \makecell{FCS-\\LAT} & JM & CD & LD & \makecell{FCS-\\SL} & \makecell{FCS-\\MAN} & \makecell{FCS-\\LAT} & JM & CD & LD & \makecell{FCS-\\SL} & \makecell{FCS-\\MAN} & \makecell{FCS-\\LAT} & \multicolumn{1}{c}{JM} \\ 
[0.4ex]\hline\\[-1.8ex]
& & \multicolumn{18}{c}{Small intraclass correlation $(\rho_{Iy}=.10)$} \\[0.6ex]\hline\\[-1.8ex]
\multicolumn{4}{l}{$n=5$} \\  & \nopagebreak $\;J=30$  & $\phantom{0}{-}3.4\phantom{0}$ & $\phantom{0}{-}7.4\phantom{0}$ & ${-}21.7\phantom{0}$ & $\phantom{0}\phantom{-}7.4\phantom{0}$ & $\phantom{0}\phantom{-}8.8\phantom{0}$ & $\phantom{0}{-}2.5\phantom{0}$ & $\phantom{0}0.25\phantom{0}$ & $\phantom{0}0.28\phantom{0}$ & $\phantom{0}0.32\phantom{0}$ & $\phantom{0}0.36\phantom{0}$ & $\phantom{0}0.37\phantom{0}$ & $\phantom{0}0.29\phantom{0}$ & $\phantom{0}86.7\phantom{0}$ & $\phantom{0}82.9\phantom{0}$ & $\phantom{0}68.5\phantom{0}$ & $\phantom{0}92.8\phantom{0}$ & $\phantom{0}93.1\phantom{0}$ & $\phantom{0}89.5\phantom{0}$ \\
 & \nopagebreak $\;J=50$  & $\phantom{0}{-}1.8\phantom{0}$ & $\phantom{0}{-}5.8\phantom{0}$ & ${-}20.6\phantom{0}$ & $\phantom{0}\phantom{-}4.0\phantom{0}$ & $\phantom{0}\phantom{-}5.5\phantom{0}$ & $\phantom{0}{-}1.9\phantom{0}$ & $\phantom{0}0.21\phantom{0}$ & $\phantom{0}0.22\phantom{0}$ & $\phantom{0}0.28\phantom{0}$ & $\phantom{0}0.26\phantom{0}$ & $\phantom{0}0.27\phantom{0}$ & $\phantom{0}0.23\phantom{0}$ & $\phantom{0}89.2\phantom{0}$ & $\phantom{0}86.7\phantom{0}$ & $\phantom{0}65.4\phantom{0}$ & $\phantom{0}93.6\phantom{0}$ & $\phantom{0}93.8\phantom{0}$ & $\phantom{0}91.4\phantom{0}$ \\
 & \nopagebreak $\;J=100$  & $\phantom{0}{-}0.8\phantom{0}$ & $\phantom{0}{-}4.5\phantom{0}$ & ${-}19.4\phantom{0}$ & $\phantom{0}\phantom{-}2.1\phantom{0}$ & $\phantom{0}\phantom{-}3.4\phantom{0}$ & $\phantom{0}{-}1.3\phantom{0}$ & $\phantom{0}0.14\phantom{0}$ & $\phantom{0}0.15\phantom{0}$ & $\phantom{0}0.23\phantom{0}$ & $\phantom{0}0.17\phantom{0}$ & $\phantom{0}0.17\phantom{0}$ & $\phantom{0}0.15\phantom{0}$ & $\phantom{0}92.5\phantom{0}$ & $\phantom{0}89.8\phantom{0}$ & $\phantom{0}60.3\phantom{0}$ & $\phantom{0}94.9\phantom{0}$ & $\phantom{0}95.1\phantom{0}$ & $\phantom{0}93.3\phantom{0}$ \\
 & \nopagebreak $\;J=200$  & $\phantom{0}{-}0.5\phantom{0}$ & $\phantom{0}{-}4.4\phantom{0}$ & ${-}19.5\phantom{0}$ & $\phantom{0}\phantom{-}0.7\phantom{0}$ & $\phantom{0}\phantom{-}1.4\phantom{0}$ & $\phantom{0}{-}1.3\phantom{0}$ & $\phantom{0}0.10\phantom{0}$ & $\phantom{0}0.11\phantom{0}$ & $\phantom{0}0.21\phantom{0}$ & $\phantom{0}0.12\phantom{0}$ & $\phantom{0}0.12\phantom{0}$ & $\phantom{0}0.11\phantom{0}$ & $\phantom{0}94.4\phantom{0}$ & $\phantom{0}89.0\phantom{0}$ & $\phantom{0}40.7\phantom{0}$ & $\phantom{0}95.6\phantom{0}$ & $\phantom{0}95.4\phantom{0}$ & $\phantom{0}93.7\phantom{0}$ \\
 & \nopagebreak $\;J=500$  & $\phantom{0}{-}0.1\phantom{0}$ & $\phantom{0}{-}3.8\phantom{0}$ & ${-}19.1\phantom{0}$ & $\phantom{0}\phantom{-}0.5\phantom{0}$ & $\phantom{0}\phantom{-}0.7\phantom{0}$ & $\phantom{0}{-}0.5\phantom{0}$ & $\phantom{0}0.06\phantom{0}$ & $\phantom{0}0.08\phantom{0}$ & $\phantom{0}0.20\phantom{0}$ & $\phantom{0}0.08\phantom{0}$ & $\phantom{0}0.08\phantom{0}$ & $\phantom{0}0.07\phantom{0}$ & $\phantom{0}95.3\phantom{0}$ & $\phantom{0}90.1\phantom{0}$ & $\phantom{0}11.0\phantom{0}$ & $\phantom{0}95.0\phantom{0}$ & $\phantom{0}94.0\phantom{0}$ & $\phantom{0}94.4\phantom{0}$ \\
 & \nopagebreak $\;J=1000$  & $\phantom{0}{-}0.2\phantom{0}$ & $\phantom{0}{-}4.0\phantom{0}$ & ${-}19.2\phantom{0}$ & $\phantom{0}\phantom{-}0.0\phantom{0}$ & $\phantom{0}\phantom{-}0.2\phantom{0}$ & $\phantom{0}{-}0.6\phantom{0}$ & $\phantom{0}0.04\phantom{0}$ & $\phantom{0}0.06\phantom{0}$ & $\phantom{0}0.20\phantom{0}$ & $\phantom{0}0.06\phantom{0}$ & $\phantom{0}0.06\phantom{0}$ & $\phantom{0}0.05\phantom{0}$ & $\phantom{0}95.5\phantom{0}$ & $\phantom{0}84.7\phantom{0}$ & $\phantom{0}\phantom{0}1.1\phantom{0}$ & $\phantom{0}95.0\phantom{0}$ & $\phantom{0}94.4\phantom{0}$ & $\phantom{0}94.4\phantom{0}$ \\
\multicolumn{4}{l}{$n=20$} \\  & \nopagebreak $\;J=30$  & $\phantom{0}{-}1.9\phantom{0}$ & $\phantom{0}{-}9.4\phantom{0}$ & ${-}26.3\phantom{0}$ & $\phantom{0}\phantom{-}7.9\phantom{0}$ & $\phantom{0}\phantom{-}9.6\phantom{0}$ & $\phantom{0}{-}4.0\phantom{0}$ & $\phantom{0}0.26\phantom{0}$ & $\phantom{0}0.28\phantom{0}$ & $\phantom{0}0.34\phantom{0}$ & $\phantom{0}0.37\phantom{0}$ & $\phantom{0}0.39\phantom{0}$ & $\phantom{0}0.29\phantom{0}$ & $\phantom{0}87.8\phantom{0}$ & $\phantom{0}82.1\phantom{0}$ & $\phantom{0}61.5\phantom{0}$ & $\phantom{0}92.9\phantom{0}$ & $\phantom{0}92.5\phantom{0}$ & $\phantom{0}88.1\phantom{0}$ \\
 & \nopagebreak $\;J=50$  & $\phantom{0}{-}2.1\phantom{0}$ & $\phantom{0}{-}9.7\phantom{0}$ & ${-}26.8\phantom{0}$ & $\phantom{0}\phantom{-}3.2\phantom{0}$ & $\phantom{0}\phantom{-}4.4\phantom{0}$ & $\phantom{0}{-}4.3\phantom{0}$ & $\phantom{0}0.20\phantom{0}$ & $\phantom{0}0.22\phantom{0}$ & $\phantom{0}0.32\phantom{0}$ & $\phantom{0}0.26\phantom{0}$ & $\phantom{0}0.27\phantom{0}$ & $\phantom{0}0.23\phantom{0}$ & $\phantom{0}90.2\phantom{0}$ & $\phantom{0}82.9\phantom{0}$ & $\phantom{0}55.0\phantom{0}$ & $\phantom{0}92.7\phantom{0}$ & $\phantom{0}92.7\phantom{0}$ & $\phantom{0}89.4\phantom{0}$ \\
 & \nopagebreak $\;J=100$  & $\phantom{0}{-}0.9\phantom{0}$ & $\phantom{0}{-}8.2\phantom{0}$ & ${-}25.7\phantom{0}$ & $\phantom{0}\phantom{-}1.7\phantom{0}$ & $\phantom{0}\phantom{-}1.9\phantom{0}$ & $\phantom{0}{-}2.7\phantom{0}$ & $\phantom{0}0.14\phantom{0}$ & $\phantom{0}0.17\phantom{0}$ & $\phantom{0}0.28\phantom{0}$ & $\phantom{0}0.18\phantom{0}$ & $\phantom{0}0.19\phantom{0}$ & $\phantom{0}0.17\phantom{0}$ & $\phantom{0}92.3\phantom{0}$ & $\phantom{0}84.6\phantom{0}$ & $\phantom{0}41.9\phantom{0}$ & $\phantom{0}93.5\phantom{0}$ & $\phantom{0}93.2\phantom{0}$ & $\phantom{0}90.7\phantom{0}$ \\
 & \nopagebreak $\;J=200$  & $\phantom{0}{-}1.1\phantom{0}$ & $\phantom{0}{-}8.7\phantom{0}$ & ${-}26.1\phantom{0}$ & $\phantom{0}{-}0.4\phantom{0}$ & $\phantom{0}{-}0.2\phantom{0}$ & $\phantom{0}{-}2.8\phantom{0}$ & $\phantom{0}0.10\phantom{0}$ & $\phantom{0}0.13\phantom{0}$ & $\phantom{0}0.27\phantom{0}$ & $\phantom{0}0.13\phantom{0}$ & $\phantom{0}0.13\phantom{0}$ & $\phantom{0}0.12\phantom{0}$ & $\phantom{0}92.7\phantom{0}$ & $\phantom{0}80.5\phantom{0}$ & $\phantom{0}17.0\phantom{0}$ & $\phantom{0}92.6\phantom{0}$ & $\phantom{0}93.4\phantom{0}$ & $\phantom{0}91.4\phantom{0}$ \\
 & \nopagebreak $\;J=500$  & $\phantom{0}{-}0.4\phantom{0}$ & $\phantom{0}{-}7.7\phantom{0}$ & ${-}25.2\phantom{0}$ & $\phantom{0}{-}0.1\phantom{0}$ & $\phantom{0}{-}0.1\phantom{0}$ & $\phantom{0}{-}1.2\phantom{0}$ & $\phantom{0}0.06\phantom{0}$ & $\phantom{0}0.10\phantom{0}$ & $\phantom{0}0.26\phantom{0}$ & $\phantom{0}0.08\phantom{0}$ & $\phantom{0}0.08\phantom{0}$ & $\phantom{0}0.08\phantom{0}$ & $\phantom{0}93.3\phantom{0}$ & $\phantom{0}74.4\phantom{0}$ & $\phantom{0}\phantom{0}2.1\phantom{0}$ & $\phantom{0}93.9\phantom{0}$ & $\phantom{0}93.5\phantom{0}$ & $\phantom{0}93.0\phantom{0}$ \\
 & \nopagebreak $\;J=1000$  & $\phantom{0}{-}0.1\phantom{0}$ & $\phantom{0}{-}7.2\phantom{0}$ & ${-}24.8\phantom{0}$ & $\phantom{0}\phantom{-}0.2\phantom{0}$ & $\phantom{0}\phantom{-}0.2\phantom{0}$ & $\phantom{0}{-}0.4\phantom{0}$ & $\phantom{0}0.05\phantom{0}$ & $\phantom{0}0.09\phantom{0}$ & $\phantom{0}0.25\phantom{0}$ & $\phantom{0}0.06\phantom{0}$ & $\phantom{0}0.06\phantom{0}$ & $\phantom{0}0.06\phantom{0}$ & $\phantom{0}95.1\phantom{0}$ & $\phantom{0}64.3\phantom{0}$ & $\phantom{0}\phantom{0}0.0\phantom{0}$ & $\phantom{0}94.2\phantom{0}$ & $\phantom{0}94.6\phantom{0}$ & $\phantom{0}93.8\phantom{0}$ \\
[0.5ex]\hline\\[-1.6ex] 
& & \multicolumn{18}{c}{Moderate intraclass correlation $(\rho_{Iy}=.30)$} \\[0.6ex]\hline\\[-1.8ex]
\multicolumn{4}{l}{$n=5$} \\  & \nopagebreak $\;J=30$  & $\phantom{0}{-}3.5\phantom{0}$ & ${-}11.2\phantom{0}$ & ${-}24.1\phantom{0}$ & $\phantom{0}\phantom{-}5.6\phantom{0}$ & $\phantom{0}\phantom{-}8.1\phantom{0}$ & $\phantom{0}{-}2.8\phantom{0}$ & $\phantom{0}0.25\phantom{0}$ & $\phantom{0}0.28\phantom{0}$ & $\phantom{0}0.33\phantom{0}$ & $\phantom{0}0.35\phantom{0}$ & $\phantom{0}0.37\phantom{0}$ & $\phantom{0}0.30\phantom{0}$ & $\phantom{0}87.1\phantom{0}$ & $\phantom{0}80.9\phantom{0}$ & $\phantom{0}62.9\phantom{0}$ & $\phantom{0}92.2\phantom{0}$ & $\phantom{0}92.6\phantom{0}$ & $\phantom{0}88.3\phantom{0}$ \\
 & \nopagebreak $\;J=50$  & $\phantom{0}{-}1.9\phantom{0}$ & $\phantom{0}{-}9.7\phantom{0}$ & ${-}23.0\phantom{0}$ & $\phantom{0}\phantom{-}2.8\phantom{0}$ & $\phantom{0}\phantom{-}4.2\phantom{0}$ & $\phantom{0}{-}2.2\phantom{0}$ & $\phantom{0}0.20\phantom{0}$ & $\phantom{0}0.23\phantom{0}$ & $\phantom{0}0.29\phantom{0}$ & $\phantom{0}0.26\phantom{0}$ & $\phantom{0}0.27\phantom{0}$ & $\phantom{0}0.23\phantom{0}$ & $\phantom{0}90.1\phantom{0}$ & $\phantom{0}82.5\phantom{0}$ & $\phantom{0}61.7\phantom{0}$ & $\phantom{0}92.0\phantom{0}$ & $\phantom{0}92.1\phantom{0}$ & $\phantom{0}90.5\phantom{0}$ \\
 & \nopagebreak $\;J=100$  & $\phantom{0}{-}0.9\phantom{0}$ & $\phantom{0}{-}8.2\phantom{0}$ & ${-}22.1\phantom{0}$ & $\phantom{0}\phantom{-}1.5\phantom{0}$ & $\phantom{0}\phantom{-}1.8\phantom{0}$ & $\phantom{0}{-}0.8\phantom{0}$ & $\phantom{0}0.14\phantom{0}$ & $\phantom{0}0.17\phantom{0}$ & $\phantom{0}0.26\phantom{0}$ & $\phantom{0}0.18\phantom{0}$ & $\phantom{0}0.19\phantom{0}$ & $\phantom{0}0.17\phantom{0}$ & $\phantom{0}91.6\phantom{0}$ & $\phantom{0}84.7\phantom{0}$ & $\phantom{0}51.2\phantom{0}$ & $\phantom{0}93.1\phantom{0}$ & $\phantom{0}93.7\phantom{0}$ & $\phantom{0}92.6\phantom{0}$ \\
 & \nopagebreak $\;J=200$  & $\phantom{0}{-}0.7\phantom{0}$ & $\phantom{0}{-}7.9\phantom{0}$ & ${-}21.8\phantom{0}$ & $\phantom{0}\phantom{-}0.2\phantom{0}$ & $\phantom{0}\phantom{-}0.3\phantom{0}$ & $\phantom{0}{-}0.7\phantom{0}$ & $\phantom{0}0.10\phantom{0}$ & $\phantom{0}0.13\phantom{0}$ & $\phantom{0}0.24\phantom{0}$ & $\phantom{0}0.12\phantom{0}$ & $\phantom{0}0.13\phantom{0}$ & $\phantom{0}0.12\phantom{0}$ & $\phantom{0}93.0\phantom{0}$ & $\phantom{0}83.8\phantom{0}$ & $\phantom{0}31.2\phantom{0}$ & $\phantom{0}93.1\phantom{0}$ & $\phantom{0}92.9\phantom{0}$ & $\phantom{0}93.1\phantom{0}$ \\
 & \nopagebreak $\;J=500$  & $\phantom{0}{-}0.1\phantom{0}$ & $\phantom{0}{-}7.1\phantom{0}$ & ${-}21.2\phantom{0}$ & $\phantom{0}\phantom{-}0.5\phantom{0}$ & $\phantom{0}\phantom{-}0.6\phantom{0}$ & $\phantom{0}\phantom{-}0.1\phantom{0}$ & $\phantom{0}0.06\phantom{0}$ & $\phantom{0}0.09\phantom{0}$ & $\phantom{0}0.22\phantom{0}$ & $\phantom{0}0.08\phantom{0}$ & $\phantom{0}0.08\phantom{0}$ & $\phantom{0}0.08\phantom{0}$ & $\phantom{0}95.0\phantom{0}$ & $\phantom{0}79.0\phantom{0}$ & $\phantom{0}\phantom{0}5.3\phantom{0}$ & $\phantom{0}94.0\phantom{0}$ & $\phantom{0}94.3\phantom{0}$ & $\phantom{0}95.3\phantom{0}$ \\
 & \nopagebreak $\;J=1000$  & $\phantom{0}{-}0.0\phantom{0}$ & $\phantom{0}{-}7.2\phantom{0}$ & ${-}21.3\phantom{0}$ & $\phantom{0}\phantom{-}0.1\phantom{0}$ & $\phantom{0}\phantom{-}0.2\phantom{0}$ & $\phantom{0}{-}0.1\phantom{0}$ & $\phantom{0}0.05\phantom{0}$ & $\phantom{0}0.09\phantom{0}$ & $\phantom{0}0.22\phantom{0}$ & $\phantom{0}0.06\phantom{0}$ & $\phantom{0}0.06\phantom{0}$ & $\phantom{0}0.06\phantom{0}$ & $\phantom{0}93.7\phantom{0}$ & $\phantom{0}64.0\phantom{0}$ & $\phantom{0}\phantom{0}0.3\phantom{0}$ & $\phantom{0}93.5\phantom{0}$ & $\phantom{0}93.7\phantom{0}$ & $\phantom{0}93.2\phantom{0}$ \\
\multicolumn{4}{l}{$n=20$} \\  & \nopagebreak $\;J=30$  & $\phantom{0}{-}4.2\phantom{0}$ & ${-}13.4\phantom{0}$ & ${-}29.1\phantom{0}$ & $\phantom{0}\phantom{-}4.9\phantom{0}$ & $\phantom{0}\phantom{-}4.9\phantom{0}$ & $\phantom{0}{-}2.8\phantom{0}$ & $\phantom{0}0.25\phantom{0}$ & $\phantom{0}0.28\phantom{0}$ & $\phantom{0}0.36\phantom{0}$ & $\phantom{0}0.36\phantom{0}$ & $\phantom{0}0.35\phantom{0}$ & $\phantom{0}0.30\phantom{0}$ & $\phantom{0}87.9\phantom{0}$ & $\phantom{0}79.1\phantom{0}$ & $\phantom{0}58.3\phantom{0}$ & $\phantom{0}90.9\phantom{0}$ & $\phantom{0}91.1\phantom{0}$ & $\phantom{0}88.1\phantom{0}$ \\
 & \nopagebreak $\;J=50$  & $\phantom{0}{-}1.0\phantom{0}$ & ${-}10.4\phantom{0}$ & ${-}26.6\phantom{0}$ & $\phantom{0}\phantom{-}4.2\phantom{0}$ & $\phantom{0}\phantom{-}4.3\phantom{0}$ & $\phantom{0}{-}0.3\phantom{0}$ & $\phantom{0}0.20\phantom{0}$ & $\phantom{0}0.23\phantom{0}$ & $\phantom{0}0.32\phantom{0}$ & $\phantom{0}0.28\phantom{0}$ & $\phantom{0}0.28\phantom{0}$ & $\phantom{0}0.25\phantom{0}$ & $\phantom{0}90.1\phantom{0}$ & $\phantom{0}81.9\phantom{0}$ & $\phantom{0}54.5\phantom{0}$ & $\phantom{0}92.9\phantom{0}$ & $\phantom{0}91.9\phantom{0}$ & $\phantom{0}89.5\phantom{0}$ \\
 & \nopagebreak $\;J=100$  & $\phantom{0}{-}1.3\phantom{0}$ & ${-}10.6\phantom{0}$ & ${-}26.9\phantom{0}$ & $\phantom{0}\phantom{-}1.2\phantom{0}$ & $\phantom{0}\phantom{-}1.4\phantom{0}$ & $\phantom{0}{-}0.8\phantom{0}$ & $\phantom{0}0.15\phantom{0}$ & $\phantom{0}0.18\phantom{0}$ & $\phantom{0}0.30\phantom{0}$ & $\phantom{0}0.19\phantom{0}$ & $\phantom{0}0.19\phantom{0}$ & $\phantom{0}0.18\phantom{0}$ & $\phantom{0}90.8\phantom{0}$ & $\phantom{0}80.5\phantom{0}$ & $\phantom{0}38.6\phantom{0}$ & $\phantom{0}93.1\phantom{0}$ & $\phantom{0}92.7\phantom{0}$ & $\phantom{0}91.9\phantom{0}$ \\
 & \nopagebreak $\;J=200$  & $\phantom{0}{-}0.1\phantom{0}$ & $\phantom{0}{-}9.6\phantom{0}$ & ${-}26.3\phantom{0}$ & $\phantom{0}\phantom{-}1.3\phantom{0}$ & $\phantom{0}\phantom{-}1.2\phantom{0}$ & $\phantom{0}\phantom{-}0.0\phantom{0}$ & $\phantom{0}0.10\phantom{0}$ & $\phantom{0}0.14\phantom{0}$ & $\phantom{0}0.28\phantom{0}$ & $\phantom{0}0.13\phantom{0}$ & $\phantom{0}0.13\phantom{0}$ & $\phantom{0}0.13\phantom{0}$ & $\phantom{0}94.5\phantom{0}$ & $\phantom{0}78.2\phantom{0}$ & $\phantom{0}16.8\phantom{0}$ & $\phantom{0}94.6\phantom{0}$ & $\phantom{0}94.4\phantom{0}$ & $\phantom{0}93.9\phantom{0}$ \\
 & \nopagebreak $\;J=500$  & $\phantom{0}{-}0.2\phantom{0}$ & $\phantom{0}{-}9.7\phantom{0}$ & ${-}26.3\phantom{0}$ & $\phantom{0}\phantom{-}0.0\phantom{0}$ & $\phantom{0}\phantom{-}0.1\phantom{0}$ & $\phantom{0}{-}0.3\phantom{0}$ & $\phantom{0}0.06\phantom{0}$ & $\phantom{0}0.12\phantom{0}$ & $\phantom{0}0.27\phantom{0}$ & $\phantom{0}0.08\phantom{0}$ & $\phantom{0}0.08\phantom{0}$ & $\phantom{0}0.08\phantom{0}$ & $\phantom{0}93.3\phantom{0}$ & $\phantom{0}63.7\phantom{0}$ & $\phantom{0}\phantom{0}0.6\phantom{0}$ & $\phantom{0}92.9\phantom{0}$ & $\phantom{0}93.3\phantom{0}$ & $\phantom{0}93.5\phantom{0}$ \\
 & \nopagebreak $\;J=1000$  & $\phantom{0}{-}0.2\phantom{0}$ & $\phantom{0}{-}9.6\phantom{0}$ & ${-}26.1\phantom{0}$ & $\phantom{0}{-}0.1\phantom{0}$ & $\phantom{0}{-}0.0\phantom{0}$ & $\phantom{0}{-}0.2\phantom{0}$ & $\phantom{0}0.04\phantom{0}$ & $\phantom{0}0.11\phantom{0}$ & $\phantom{0}0.26\phantom{0}$ & $\phantom{0}0.06\phantom{0}$ & $\phantom{0}0.06\phantom{0}$ & $\phantom{0}0.06\phantom{0}$ & $\phantom{0}95.1\phantom{0}$ & $\phantom{0}43.0\phantom{0}$ & $\phantom{0}\phantom{0}0.1\phantom{0}$ & $\phantom{0}94.3\phantom{0}$ & $\phantom{0}94.5\phantom{0}$ & $\phantom{0}94.0\phantom{0}$ \\
[0.5ex]\hline\\[-1.6ex] 
\end{tabular}
\begin{tablenotes}[para,flushleft]{\footnotesize \textit{Note.} $n$ = cluster size; $J$ = number of clusters; CD = complete data sets; LD = listwise deletion; FCS-SL = single-level FCS; FCS-MAN = two-level FCS with manifest cluster means; FCS-LAT = two-level FCS with latent cluster means; JM = joint modeling.}\end{tablenotes}
\end{threeparttable}
\end{sidewaystable}
\begin{sidewaystable}
\begin{threeparttable}
\setlength{\tabcolsep}{1.2pt}
\renewcommand{\arraystretch}{0.95}
\footnotesize
\caption{\small Study 1: Bias (in \%), RMSE, and Coverage of the 95\% Confidence Interval for the Variance of $z$ ($\hat\sigma_z^2$) With 40\% Missing Data (MCAR, $\lambda=0$)}
\begin{tabular}{llcccccccccccccccccc}
\hline\\[-1.8ex]
& & \multicolumn{6}{c}{Bias (\%)} & \multicolumn{6}{c}{RMSE} & \multicolumn{6}{c}{Coverage (\%)} \\ \cmidrule(r){3-8}\cmidrule(r){9-14}\cmidrule(r){15-20}
 &  & CD & LD & \makecell{FCS-\\SL} & \makecell{FCS-\\MAN} & \makecell{FCS-\\LAT} & JM & CD & LD & \makecell{FCS-\\SL} & \makecell{FCS-\\MAN} & \makecell{FCS-\\LAT} & JM & CD & LD & \makecell{FCS-\\SL} & \makecell{FCS-\\MAN} & \makecell{FCS-\\LAT} & \multicolumn{1}{c}{JM} \\ 
[0.4ex]\hline\\[-1.8ex]
& & \multicolumn{18}{c}{Small intraclass correlation $(\rho_{Iy}=.10)$} \\[0.6ex]\hline\\[-1.8ex]
\multicolumn{4}{l}{$n=5$} \\  & \nopagebreak $\;J=30$  & $\phantom{0}{-}3.1\phantom{0}$ & $\phantom{0}{-}5.7\phantom{0}$ & ${-}35.4\phantom{0}$ & $\phantom{0}\phantom{-}7.9\phantom{0}$ & $\phantom{0}\phantom{-}8.2\phantom{0}$ & $\phantom{0}{-}0.4\phantom{0}$ & $\phantom{0}0.25\phantom{0}$ & $\phantom{0}0.33\phantom{0}$ & $\phantom{0}0.42\phantom{0}$ & $\phantom{0}0.41\phantom{0}$ & $\phantom{0}0.42\phantom{0}$ & $\phantom{0}0.34\phantom{0}$ & $\phantom{0}87.4\phantom{0}$ & $\phantom{0}82.0\phantom{0}$ & $\phantom{0}47.6\phantom{0}$ & $\phantom{0}91.5\phantom{0}$ & $\phantom{0}90.9\phantom{0}$ & $\phantom{0}87.9\phantom{0}$ \\
 & \nopagebreak $\;J=50$  & $\phantom{0}{-}1.7\phantom{0}$ & $\phantom{0}{-}3.3\phantom{0}$ & ${-}33.6\phantom{0}$ & $\phantom{0}\phantom{-}3.8\phantom{0}$ & $\phantom{0}\phantom{-}4.4\phantom{0}$ & $\phantom{0}{-}0.2\phantom{0}$ & $\phantom{0}0.20\phantom{0}$ & $\phantom{0}0.26\phantom{0}$ & $\phantom{0}0.38\phantom{0}$ & $\phantom{0}0.28\phantom{0}$ & $\phantom{0}0.29\phantom{0}$ & $\phantom{0}0.26\phantom{0}$ & $\phantom{0}90.1\phantom{0}$ & $\phantom{0}86.7\phantom{0}$ & $\phantom{0}42.2\phantom{0}$ & $\phantom{0}93.4\phantom{0}$ & $\phantom{0}93.9\phantom{0}$ & $\phantom{0}91.4\phantom{0}$ \\
 & \nopagebreak $\;J=100$  & $\phantom{0}{-}0.6\phantom{0}$ & $\phantom{0}{-}1.3\phantom{0}$ & ${-}32.7\phantom{0}$ & $\phantom{0}\phantom{-}2.0\phantom{0}$ & $\phantom{0}\phantom{-}2.4\phantom{0}$ & $\phantom{0}{-}0.2\phantom{0}$ & $\phantom{0}0.15\phantom{0}$ & $\phantom{0}0.18\phantom{0}$ & $\phantom{0}0.35\phantom{0}$ & $\phantom{0}0.19\phantom{0}$ & $\phantom{0}0.20\phantom{0}$ & $\phantom{0}0.19\phantom{0}$ & $\phantom{0}91.3\phantom{0}$ & $\phantom{0}91.0\phantom{0}$ & $\phantom{0}27.2\phantom{0}$ & $\phantom{0}93.4\phantom{0}$ & $\phantom{0}93.7\phantom{0}$ & $\phantom{0}92.3\phantom{0}$ \\
 & \nopagebreak $\;J=200$  & $\phantom{0}{-}0.5\phantom{0}$ & $\phantom{0}{-}1.0\phantom{0}$ & ${-}32.6\phantom{0}$ & $\phantom{0}\phantom{-}0.7\phantom{0}$ & $\phantom{0}\phantom{-}1.0\phantom{0}$ & $\phantom{0}{-}0.3\phantom{0}$ & $\phantom{0}0.10\phantom{0}$ & $\phantom{0}0.13\phantom{0}$ & $\phantom{0}0.34\phantom{0}$ & $\phantom{0}0.14\phantom{0}$ & $\phantom{0}0.14\phantom{0}$ & $\phantom{0}0.13\phantom{0}$ & $\phantom{0}93.1\phantom{0}$ & $\phantom{0}91.8\phantom{0}$ & $\phantom{0}\phantom{0}8.4\phantom{0}$ & $\phantom{0}93.3\phantom{0}$ & $\phantom{0}92.6\phantom{0}$ & $\phantom{0}92.2\phantom{0}$ \\
 & \nopagebreak $\;J=500$  & $\phantom{0}{-}0.1\phantom{0}$ & $\phantom{0}{-}0.1\phantom{0}$ & ${-}32.0\phantom{0}$ & $\phantom{0}\phantom{-}0.5\phantom{0}$ & $\phantom{0}\phantom{-}0.6\phantom{0}$ & $\phantom{0}{-}0.0\phantom{0}$ & $\phantom{0}0.06\phantom{0}$ & $\phantom{0}0.08\phantom{0}$ & $\phantom{0}0.33\phantom{0}$ & $\phantom{0}0.08\phantom{0}$ & $\phantom{0}0.08\phantom{0}$ & $\phantom{0}0.08\phantom{0}$ & $\phantom{0}95.3\phantom{0}$ & $\phantom{0}94.6\phantom{0}$ & $\phantom{0}\phantom{0}0.3\phantom{0}$ & $\phantom{0}95.2\phantom{0}$ & $\phantom{0}95.4\phantom{0}$ & $\phantom{0}94.6\phantom{0}$ \\
 & \nopagebreak $\;J=1000$  & $\phantom{0}{-}0.1\phantom{0}$ & $\phantom{0}{-}0.2\phantom{0}$ & ${-}32.1\phantom{0}$ & $\phantom{0}\phantom{-}0.1\phantom{0}$ & $\phantom{0}\phantom{-}0.1\phantom{0}$ & $\phantom{0}{-}0.2\phantom{0}$ & $\phantom{0}0.04\phantom{0}$ & $\phantom{0}0.06\phantom{0}$ & $\phantom{0}0.32\phantom{0}$ & $\phantom{0}0.06\phantom{0}$ & $\phantom{0}0.06\phantom{0}$ & $\phantom{0}0.06\phantom{0}$ & $\phantom{0}95.3\phantom{0}$ & $\phantom{0}94.0\phantom{0}$ & $\phantom{0}\phantom{0}0.0\phantom{0}$ & $\phantom{0}93.9\phantom{0}$ & $\phantom{0}93.6\phantom{0}$ & $\phantom{0}93.9\phantom{0}$ \\
\multicolumn{4}{l}{$n=20$} \\  & \nopagebreak $\;J=30$  & $\phantom{0}{-}3.8\phantom{0}$ & $\phantom{0}{-}7.4\phantom{0}$ & ${-}42.5\phantom{0}$ & $\phantom{0}\phantom{-}4.8\phantom{0}$ & $\phantom{0}\phantom{-}6.6\phantom{0}$ & $\phantom{0}{-}2.9\phantom{0}$ & $\phantom{0}0.25\phantom{0}$ & $\phantom{0}0.32\phantom{0}$ & $\phantom{0}0.48\phantom{0}$ & $\phantom{0}0.38\phantom{0}$ & $\phantom{0}0.40\phantom{0}$ & $\phantom{0}0.33\phantom{0}$ & $\phantom{0}88.4\phantom{0}$ & $\phantom{0}81.9\phantom{0}$ & $\phantom{0}36.8\phantom{0}$ & $\phantom{0}91.7\phantom{0}$ & $\phantom{0}92.7\phantom{0}$ & $\phantom{0}87.9\phantom{0}$ \\
 & \nopagebreak $\;J=50$  & $\phantom{0}{-}1.9\phantom{0}$ & $\phantom{0}{-}3.6\phantom{0}$ & ${-}39.9\phantom{0}$ & $\phantom{0}\phantom{-}3.1\phantom{0}$ & $\phantom{0}\phantom{-}3.5\phantom{0}$ & $\phantom{0}{-}1.5\phantom{0}$ & $\phantom{0}0.19\phantom{0}$ & $\phantom{0}0.25\phantom{0}$ & $\phantom{0}0.43\phantom{0}$ & $\phantom{0}0.28\phantom{0}$ & $\phantom{0}0.28\phantom{0}$ & $\phantom{0}0.26\phantom{0}$ & $\phantom{0}90.9\phantom{0}$ & $\phantom{0}87.2\phantom{0}$ & $\phantom{0}32.3\phantom{0}$ & $\phantom{0}92.8\phantom{0}$ & $\phantom{0}92.5\phantom{0}$ & $\phantom{0}90.1\phantom{0}$ \\
 & \nopagebreak $\;J=100$  & $\phantom{0}{-}0.3\phantom{0}$ & $\phantom{0}{-}1.3\phantom{0}$ & ${-}38.7\phantom{0}$ & $\phantom{0}\phantom{-}1.9\phantom{0}$ & $\phantom{0}\phantom{-}2.1\phantom{0}$ & $\phantom{0}{-}0.5\phantom{0}$ & $\phantom{0}0.14\phantom{0}$ & $\phantom{0}0.18\phantom{0}$ & $\phantom{0}0.41\phantom{0}$ & $\phantom{0}0.19\phantom{0}$ & $\phantom{0}0.19\phantom{0}$ & $\phantom{0}0.18\phantom{0}$ & $\phantom{0}94.0\phantom{0}$ & $\phantom{0}91.7\phantom{0}$ & $\phantom{0}14.7\phantom{0}$ & $\phantom{0}94.5\phantom{0}$ & $\phantom{0}93.9\phantom{0}$ & $\phantom{0}92.5\phantom{0}$ \\
 & \nopagebreak $\;J=200$  & $\phantom{0}{-}1.0\phantom{0}$ & $\phantom{0}{-}1.1\phantom{0}$ & ${-}38.6\phantom{0}$ & $\phantom{0}\phantom{-}0.4\phantom{0}$ & $\phantom{0}\phantom{-}0.5\phantom{0}$ & $\phantom{0}{-}0.7\phantom{0}$ & $\phantom{0}0.10\phantom{0}$ & $\phantom{0}0.13\phantom{0}$ & $\phantom{0}0.40\phantom{0}$ & $\phantom{0}0.13\phantom{0}$ & $\phantom{0}0.13\phantom{0}$ & $\phantom{0}0.13\phantom{0}$ & $\phantom{0}94.0\phantom{0}$ & $\phantom{0}93.3\phantom{0}$ & $\phantom{0}\phantom{0}2.8\phantom{0}$ & $\phantom{0}94.1\phantom{0}$ & $\phantom{0}95.0\phantom{0}$ & $\phantom{0}94.0\phantom{0}$ \\
 & \nopagebreak $\;J=500$  & $\phantom{0}{-}0.3\phantom{0}$ & $\phantom{0}{-}0.4\phantom{0}$ & ${-}38.2\phantom{0}$ & $\phantom{0}\phantom{-}0.1\phantom{0}$ & $\phantom{0}\phantom{-}0.2\phantom{0}$ & $\phantom{0}{-}0.3\phantom{0}$ & $\phantom{0}0.06\phantom{0}$ & $\phantom{0}0.08\phantom{0}$ & $\phantom{0}0.39\phantom{0}$ & $\phantom{0}0.08\phantom{0}$ & $\phantom{0}0.08\phantom{0}$ & $\phantom{0}0.08\phantom{0}$ & $\phantom{0}95.7\phantom{0}$ & $\phantom{0}94.1\phantom{0}$ & $\phantom{0}\phantom{0}0.0\phantom{0}$ & $\phantom{0}94.9\phantom{0}$ & $\phantom{0}93.5\phantom{0}$ & $\phantom{0}94.5\phantom{0}$ \\
 & \nopagebreak $\;J=1000$  & $\phantom{0}\phantom{-}0.1\phantom{0}$ & $\phantom{0}{-}0.1\phantom{0}$ & ${-}38.0\phantom{0}$ & $\phantom{0}\phantom{-}0.2\phantom{0}$ & $\phantom{0}\phantom{-}0.2\phantom{0}$ & $\phantom{0}{-}0.1\phantom{0}$ & $\phantom{0}0.04\phantom{0}$ & $\phantom{0}0.06\phantom{0}$ & $\phantom{0}0.38\phantom{0}$ & $\phantom{0}0.06\phantom{0}$ & $\phantom{0}0.06\phantom{0}$ & $\phantom{0}0.06\phantom{0}$ & $\phantom{0}95.2\phantom{0}$ & $\phantom{0}95.2\phantom{0}$ & $\phantom{0}\phantom{0}0.0\phantom{0}$ & $\phantom{0}94.6\phantom{0}$ & $\phantom{0}94.3\phantom{0}$ & $\phantom{0}95.0\phantom{0}$ \\
[0.5ex]\hline\\[-1.6ex] 
& & \multicolumn{18}{c}{Moderate intraclass correlation $(\rho_{Iy}=.30)$} \\[0.6ex]\hline\\[-1.8ex]
\multicolumn{4}{l}{$n=5$} \\  & \nopagebreak $\;J=30$  & $\phantom{0}{-}2.8\phantom{0}$ & $\phantom{0}{-}5.3\phantom{0}$ & ${-}34.1\phantom{0}$ & $\phantom{0}\phantom{-}6.8\phantom{0}$ & $\phantom{0}\phantom{-}7.6\phantom{0}$ & $\phantom{0}\phantom{-}0.5\phantom{0}$ & $\phantom{0}0.26\phantom{0}$ & $\phantom{0}0.33\phantom{0}$ & $\phantom{0}0.41\phantom{0}$ & $\phantom{0}0.39\phantom{0}$ & $\phantom{0}0.39\phantom{0}$ & $\phantom{0}0.35\phantom{0}$ & $\phantom{0}86.1\phantom{0}$ & $\phantom{0}82.9\phantom{0}$ & $\phantom{0}49.7\phantom{0}$ & $\phantom{0}91.9\phantom{0}$ & $\phantom{0}92.0\phantom{0}$ & $\phantom{0}89.9\phantom{0}$ \\
 & \nopagebreak $\;J=50$  & $\phantom{0}{-}2.8\phantom{0}$ & $\phantom{0}{-}3.3\phantom{0}$ & ${-}33.0\phantom{0}$ & $\phantom{0}\phantom{-}3.4\phantom{0}$ & $\phantom{0}\phantom{-}3.8\phantom{0}$ & $\phantom{0}{-}0.0\phantom{0}$ & $\phantom{0}0.20\phantom{0}$ & $\phantom{0}0.25\phantom{0}$ & $\phantom{0}0.38\phantom{0}$ & $\phantom{0}0.28\phantom{0}$ & $\phantom{0}0.28\phantom{0}$ & $\phantom{0}0.26\phantom{0}$ & $\phantom{0}89.2\phantom{0}$ & $\phantom{0}87.1\phantom{0}$ & $\phantom{0}43.5\phantom{0}$ & $\phantom{0}92.1\phantom{0}$ & $\phantom{0}92.7\phantom{0}$ & $\phantom{0}91.5\phantom{0}$ \\
 & \nopagebreak $\;J=100$  & $\phantom{0}{-}1.2\phantom{0}$ & $\phantom{0}{-}2.1\phantom{0}$ & ${-}32.6\phantom{0}$ & $\phantom{0}\phantom{-}1.0\phantom{0}$ & $\phantom{0}\phantom{-}1.1\phantom{0}$ & $\phantom{0}{-}0.3\phantom{0}$ & $\phantom{0}0.14\phantom{0}$ & $\phantom{0}0.19\phantom{0}$ & $\phantom{0}0.35\phantom{0}$ & $\phantom{0}0.19\phantom{0}$ & $\phantom{0}0.19\phantom{0}$ & $\phantom{0}0.19\phantom{0}$ & $\phantom{0}92.7\phantom{0}$ & $\phantom{0}89.3\phantom{0}$ & $\phantom{0}27.0\phantom{0}$ & $\phantom{0}92.9\phantom{0}$ & $\phantom{0}92.3\phantom{0}$ & $\phantom{0}92.2\phantom{0}$ \\
 & \nopagebreak $\;J=200$  & $\phantom{0}{-}1.4\phantom{0}$ & $\phantom{0}{-}1.2\phantom{0}$ & ${-}31.9\phantom{0}$ & $\phantom{0}\phantom{-}0.2\phantom{0}$ & $\phantom{0}\phantom{-}0.3\phantom{0}$ & $\phantom{0}{-}0.5\phantom{0}$ & $\phantom{0}0.10\phantom{0}$ & $\phantom{0}0.12\phantom{0}$ & $\phantom{0}0.33\phantom{0}$ & $\phantom{0}0.13\phantom{0}$ & $\phantom{0}0.13\phantom{0}$ & $\phantom{0}0.13\phantom{0}$ & $\phantom{0}93.7\phantom{0}$ & $\phantom{0}92.9\phantom{0}$ & $\phantom{0}\phantom{0}8.4\phantom{0}$ & $\phantom{0}93.9\phantom{0}$ & $\phantom{0}94.4\phantom{0}$ & $\phantom{0}93.7\phantom{0}$ \\
 & \nopagebreak $\;J=500$  & $\phantom{0}{-}0.0\phantom{0}$ & $\phantom{0}\phantom{-}0.0\phantom{0}$ & ${-}31.1\phantom{0}$ & $\phantom{0}\phantom{-}0.6\phantom{0}$ & $\phantom{0}\phantom{-}0.5\phantom{0}$ & $\phantom{0}\phantom{-}0.3\phantom{0}$ & $\phantom{0}0.06\phantom{0}$ & $\phantom{0}0.08\phantom{0}$ & $\phantom{0}0.32\phantom{0}$ & $\phantom{0}0.09\phantom{0}$ & $\phantom{0}0.08\phantom{0}$ & $\phantom{0}0.08\phantom{0}$ & $\phantom{0}93.4\phantom{0}$ & $\phantom{0}93.5\phantom{0}$ & $\phantom{0}\phantom{0}0.1\phantom{0}$ & $\phantom{0}94.2\phantom{0}$ & $\phantom{0}93.8\phantom{0}$ & $\phantom{0}93.9\phantom{0}$ \\
 & \nopagebreak $\;J=1000$  & $\phantom{0}{-}0.0\phantom{0}$ & $\phantom{0}{-}0.2\phantom{0}$ & ${-}31.4\phantom{0}$ & $\phantom{0}\phantom{-}0.1\phantom{0}$ & $\phantom{0}\phantom{-}0.1\phantom{0}$ & $\phantom{0}\phantom{-}0.0\phantom{0}$ & $\phantom{0}0.04\phantom{0}$ & $\phantom{0}0.06\phantom{0}$ & $\phantom{0}0.32\phantom{0}$ & $\phantom{0}0.06\phantom{0}$ & $\phantom{0}0.06\phantom{0}$ & $\phantom{0}0.06\phantom{0}$ & $\phantom{0}94.6\phantom{0}$ & $\phantom{0}95.4\phantom{0}$ & $\phantom{0}\phantom{0}0.0\phantom{0}$ & $\phantom{0}95.0\phantom{0}$ & $\phantom{0}95.2\phantom{0}$ & $\phantom{0}95.0\phantom{0}$ \\
\multicolumn{4}{l}{$n=20$} \\  & \nopagebreak $\;J=30$  & $\phantom{0}{-}4.0\phantom{0}$ & $\phantom{0}{-}6.5\phantom{0}$ & ${-}40.9\phantom{0}$ & $\phantom{0}\phantom{-}5.6\phantom{0}$ & $\phantom{0}\phantom{-}5.8\phantom{0}$ & $\phantom{0}{-}0.2\phantom{0}$ & $\phantom{0}0.26\phantom{0}$ & $\phantom{0}0.34\phantom{0}$ & $\phantom{0}0.47\phantom{0}$ & $\phantom{0}0.40\phantom{0}$ & $\phantom{0}0.39\phantom{0}$ & $\phantom{0}0.36\phantom{0}$ & $\phantom{0}86.1\phantom{0}$ & $\phantom{0}80.2\phantom{0}$ & $\phantom{0}40.3\phantom{0}$ & $\phantom{0}90.6\phantom{0}$ & $\phantom{0}90.9\phantom{0}$ & $\phantom{0}89.1\phantom{0}$ \\
 & \nopagebreak $\;J=50$  & $\phantom{0}{-}2.3\phantom{0}$ & $\phantom{0}{-}3.6\phantom{0}$ & ${-}38.9\phantom{0}$ & $\phantom{0}\phantom{-}2.7\phantom{0}$ & $\phantom{0}\phantom{-}3.5\phantom{0}$ & $\phantom{0}{-}0.1\phantom{0}$ & $\phantom{0}0.19\phantom{0}$ & $\phantom{0}0.24\phantom{0}$ & $\phantom{0}0.42\phantom{0}$ & $\phantom{0}0.26\phantom{0}$ & $\phantom{0}0.27\phantom{0}$ & $\phantom{0}0.25\phantom{0}$ & $\phantom{0}90.2\phantom{0}$ & $\phantom{0}87.5\phantom{0}$ & $\phantom{0}32.8\phantom{0}$ & $\phantom{0}92.7\phantom{0}$ & $\phantom{0}92.6\phantom{0}$ & $\phantom{0}91.7\phantom{0}$ \\
 & \nopagebreak $\;J=100$  & $\phantom{0}{-}1.1\phantom{0}$ & $\phantom{0}{-}1.6\phantom{0}$ & ${-}38.2\phantom{0}$ & $\phantom{0}\phantom{-}1.3\phantom{0}$ & $\phantom{0}\phantom{-}1.2\phantom{0}$ & $\phantom{0}\phantom{-}0.1\phantom{0}$ & $\phantom{0}0.14\phantom{0}$ & $\phantom{0}0.19\phantom{0}$ & $\phantom{0}0.40\phantom{0}$ & $\phantom{0}0.20\phantom{0}$ & $\phantom{0}0.20\phantom{0}$ & $\phantom{0}0.19\phantom{0}$ & $\phantom{0}92.0\phantom{0}$ & $\phantom{0}89.7\phantom{0}$ & $\phantom{0}17.1\phantom{0}$ & $\phantom{0}92.6\phantom{0}$ & $\phantom{0}93.2\phantom{0}$ & $\phantom{0}93.1\phantom{0}$ \\
 & \nopagebreak $\;J=200$  & $\phantom{0}{-}0.4\phantom{0}$ & $\phantom{0}{-}0.7\phantom{0}$ & ${-}37.5\phantom{0}$ & $\phantom{0}\phantom{-}0.9\phantom{0}$ & $\phantom{0}\phantom{-}0.9\phantom{0}$ & $\phantom{0}\phantom{-}0.0\phantom{0}$ & $\phantom{0}0.10\phantom{0}$ & $\phantom{0}0.13\phantom{0}$ & $\phantom{0}0.38\phantom{0}$ & $\phantom{0}0.13\phantom{0}$ & $\phantom{0}0.13\phantom{0}$ & $\phantom{0}0.13\phantom{0}$ & $\phantom{0}94.0\phantom{0}$ & $\phantom{0}92.8\phantom{0}$ & $\phantom{0}\phantom{0}2.1\phantom{0}$ & $\phantom{0}94.9\phantom{0}$ & $\phantom{0}94.3\phantom{0}$ & $\phantom{0}94.1\phantom{0}$ \\
 & \nopagebreak $\;J=500$  & $\phantom{0}{-}0.3\phantom{0}$ & $\phantom{0}{-}0.3\phantom{0}$ & ${-}37.3\phantom{0}$ & $\phantom{0}\phantom{-}0.2\phantom{0}$ & $\phantom{0}\phantom{-}0.1\phantom{0}$ & $\phantom{0}\phantom{-}0.0\phantom{0}$ & $\phantom{0}0.06\phantom{0}$ & $\phantom{0}0.08\phantom{0}$ & $\phantom{0}0.38\phantom{0}$ & $\phantom{0}0.08\phantom{0}$ & $\phantom{0}0.08\phantom{0}$ & $\phantom{0}0.08\phantom{0}$ & $\phantom{0}94.9\phantom{0}$ & $\phantom{0}94.7\phantom{0}$ & $\phantom{0}\phantom{0}0.0\phantom{0}$ & $\phantom{0}95.1\phantom{0}$ & $\phantom{0}95.4\phantom{0}$ & $\phantom{0}94.6\phantom{0}$ \\
 & \nopagebreak $\;J=1000$  & $\phantom{0}{-}0.0\phantom{0}$ & $\phantom{0}{-}0.3\phantom{0}$ & ${-}37.4\phantom{0}$ & $\phantom{0}\phantom{-}0.1\phantom{0}$ & $\phantom{0}{-}0.0\phantom{0}$ & $\phantom{0}{-}0.1\phantom{0}$ & $\phantom{0}0.05\phantom{0}$ & $\phantom{0}0.06\phantom{0}$ & $\phantom{0}0.38\phantom{0}$ & $\phantom{0}0.06\phantom{0}$ & $\phantom{0}0.06\phantom{0}$ & $\phantom{0}0.06\phantom{0}$ & $\phantom{0}93.8\phantom{0}$ & $\phantom{0}93.7\phantom{0}$ & $\phantom{0}\phantom{0}0.0\phantom{0}$ & $\phantom{0}94.6\phantom{0}$ & $\phantom{0}93.1\phantom{0}$ & $\phantom{0}94.1\phantom{0}$ \\
[0.5ex]\hline\\[-1.6ex] 
\end{tabular}
\begin{tablenotes}[para,flushleft]{\footnotesize \textit{Note.} $n$ = cluster size; $J$ = number of clusters; CD = complete data sets; LD = listwise deletion; FCS-SL = single-level FCS; FCS-MAN = two-level FCS with manifest cluster means; FCS-LAT = two-level FCS with latent cluster means; JM = joint modeling.}\end{tablenotes}
\end{threeparttable}
\end{sidewaystable}
\begin{sidewaystable}
\begin{threeparttable}
\setlength{\tabcolsep}{1.2pt}
\renewcommand{\arraystretch}{0.95}
\footnotesize
\caption{\small Study 1: Bias (in \%), RMSE, and Coverage of the 95\% Confidence Interval for the Variance of $z$ ($\hat\sigma_z^2$) With 40\% Missing Data (MAR, $\lambda=0.5$)}
\begin{tabular}{llcccccccccccccccccc}
\hline\\[-1.8ex]
& & \multicolumn{6}{c}{Bias (\%)} & \multicolumn{6}{c}{RMSE} & \multicolumn{6}{c}{Coverage (\%)} \\ \cmidrule(r){3-8}\cmidrule(r){9-14}\cmidrule(r){15-20}
 &  & CD & LD & \makecell{FCS-\\SL} & \makecell{FCS-\\MAN} & \makecell{FCS-\\LAT} & JM & CD & LD & \makecell{FCS-\\SL} & \makecell{FCS-\\MAN} & \makecell{FCS-\\LAT} & JM & CD & LD & \makecell{FCS-\\SL} & \makecell{FCS-\\MAN} & \makecell{FCS-\\LAT} & \multicolumn{1}{c}{JM} \\ 
[0.4ex]\hline\\[-1.8ex]
& & \multicolumn{18}{c}{Small intraclass correlation $(\rho_{Iy}=.10)$} \\[0.6ex]\hline\\[-1.8ex]
\multicolumn{4}{l}{$n=5$} \\  & \nopagebreak $\;J=30$  & $\phantom{0}{-}4.7\phantom{0}$ & $\phantom{0}{-}7.2\phantom{0}$ & ${-}36.0\phantom{0}$ & $\phantom{-}10.6\phantom{0}$ & $\phantom{-}10.0\phantom{0}$ & $\phantom{0}{-}1.0\phantom{0}$ & $\phantom{0}0.25\phantom{0}$ & $\phantom{0}0.32\phantom{0}$ & $\phantom{0}0.43\phantom{0}$ & $\phantom{0}0.44\phantom{0}$ & $\phantom{0}0.41\phantom{0}$ & $\phantom{0}0.34\phantom{0}$ & $\phantom{0}86.7\phantom{0}$ & $\phantom{0}83.7\phantom{0}$ & $\phantom{0}45.9\phantom{0}$ & $\phantom{0}92.7\phantom{0}$ & $\phantom{0}93.4\phantom{0}$ & $\phantom{0}89.5\phantom{0}$ \\
 & \nopagebreak $\;J=50$  & $\phantom{0}{-}1.3\phantom{0}$ & $\phantom{0}{-}4.7\phantom{0}$ & ${-}34.8\phantom{0}$ & $\phantom{0}\phantom{-}4.6\phantom{0}$ & $\phantom{0}\phantom{-}5.6\phantom{0}$ & $\phantom{0}{-}1.1\phantom{0}$ & $\phantom{0}0.20\phantom{0}$ & $\phantom{0}0.26\phantom{0}$ & $\phantom{0}0.39\phantom{0}$ & $\phantom{0}0.30\phantom{0}$ & $\phantom{0}0.30\phantom{0}$ & $\phantom{0}0.27\phantom{0}$ & $\phantom{0}89.7\phantom{0}$ & $\phantom{0}84.7\phantom{0}$ & $\phantom{0}39.3\phantom{0}$ & $\phantom{0}92.1\phantom{0}$ & $\phantom{0}92.6\phantom{0}$ & $\phantom{0}90.1\phantom{0}$ \\
 & \nopagebreak $\;J=100$  & $\phantom{0}{-}0.7\phantom{0}$ & $\phantom{0}{-}2.4\phantom{0}$ & ${-}33.3\phantom{0}$ & $\phantom{0}\phantom{-}2.8\phantom{0}$ & $\phantom{0}\phantom{-}3.5\phantom{0}$ & $\phantom{0}{-}0.3\phantom{0}$ & $\phantom{0}0.14\phantom{0}$ & $\phantom{0}0.18\phantom{0}$ & $\phantom{0}0.36\phantom{0}$ & $\phantom{0}0.20\phantom{0}$ & $\phantom{0}0.20\phantom{0}$ & $\phantom{0}0.18\phantom{0}$ & $\phantom{0}93.3\phantom{0}$ & $\phantom{0}90.9\phantom{0}$ & $\phantom{0}24.4\phantom{0}$ & $\phantom{0}94.9\phantom{0}$ & $\phantom{0}95.3\phantom{0}$ & $\phantom{0}93.3\phantom{0}$ \\
 & \nopagebreak $\;J=200$  & $\phantom{0}{-}0.1\phantom{0}$ & $\phantom{0}{-}1.7\phantom{0}$ & ${-}32.9\phantom{0}$ & $\phantom{0}\phantom{-}1.6\phantom{0}$ & $\phantom{0}\phantom{-}1.8\phantom{0}$ & $\phantom{0}{-}0.1\phantom{0}$ & $\phantom{0}0.10\phantom{0}$ & $\phantom{0}0.13\phantom{0}$ & $\phantom{0}0.34\phantom{0}$ & $\phantom{0}0.14\phantom{0}$ & $\phantom{0}0.14\phantom{0}$ & $\phantom{0}0.14\phantom{0}$ & $\phantom{0}93.4\phantom{0}$ & $\phantom{0}90.9\phantom{0}$ & $\phantom{0}\phantom{0}7.8\phantom{0}$ & $\phantom{0}94.4\phantom{0}$ & $\phantom{0}93.5\phantom{0}$ & $\phantom{0}92.8\phantom{0}$ \\
 & \nopagebreak $\;J=500$  & $\phantom{0}{-}0.0\phantom{0}$ & $\phantom{0}{-}1.6\phantom{0}$ & ${-}32.9\phantom{0}$ & $\phantom{0}\phantom{-}0.5\phantom{0}$ & $\phantom{0}\phantom{-}0.6\phantom{0}$ & $\phantom{0}{-}0.2\phantom{0}$ & $\phantom{0}0.06\phantom{0}$ & $\phantom{0}0.08\phantom{0}$ & $\phantom{0}0.33\phantom{0}$ & $\phantom{0}0.09\phantom{0}$ & $\phantom{0}0.09\phantom{0}$ & $\phantom{0}0.08\phantom{0}$ & $\phantom{0}95.1\phantom{0}$ & $\phantom{0}93.0\phantom{0}$ & $\phantom{0}\phantom{0}0.1\phantom{0}$ & $\phantom{0}94.5\phantom{0}$ & $\phantom{0}93.4\phantom{0}$ & $\phantom{0}93.0\phantom{0}$ \\
 & \nopagebreak $\;J=1000$  & $\phantom{0}{-}0.1\phantom{0}$ & $\phantom{0}{-}1.5\phantom{0}$ & ${-}33.0\phantom{0}$ & $\phantom{0}\phantom{-}0.1\phantom{0}$ & $\phantom{0}\phantom{-}0.2\phantom{0}$ & $\phantom{0}{-}0.2\phantom{0}$ & $\phantom{0}0.04\phantom{0}$ & $\phantom{0}0.06\phantom{0}$ & $\phantom{0}0.33\phantom{0}$ & $\phantom{0}0.06\phantom{0}$ & $\phantom{0}0.06\phantom{0}$ & $\phantom{0}0.06\phantom{0}$ & $\phantom{0}95.3\phantom{0}$ & $\phantom{0}92.8\phantom{0}$ & $\phantom{0}\phantom{0}0.0\phantom{0}$ & $\phantom{0}94.5\phantom{0}$ & $\phantom{0}94.5\phantom{0}$ & $\phantom{0}94.3\phantom{0}$ \\
\multicolumn{4}{l}{$n=20$} \\  & \nopagebreak $\;J=30$  & $\phantom{0}{-}3.5\phantom{0}$ & $\phantom{0}{-}8.1\phantom{0}$ & ${-}42.9\phantom{0}$ & $\phantom{0}\phantom{-}9.2\phantom{0}$ & $\phantom{-}10.9\phantom{0}$ & $\phantom{0}{-}2.3\phantom{0}$ & $\phantom{0}0.26\phantom{0}$ & $\phantom{0}0.33\phantom{0}$ & $\phantom{0}0.48\phantom{0}$ & $\phantom{0}0.43\phantom{0}$ & $\phantom{0}0.44\phantom{0}$ & $\phantom{0}0.34\phantom{0}$ & $\phantom{0}86.3\phantom{0}$ & $\phantom{0}81.9\phantom{0}$ & $\phantom{0}37.1\phantom{0}$ & $\phantom{0}93.7\phantom{0}$ & $\phantom{0}93.3\phantom{0}$ & $\phantom{0}89.5\phantom{0}$ \\
 & \nopagebreak $\;J=50$  & $\phantom{0}{-}2.1\phantom{0}$ & $\phantom{0}{-}6.4\phantom{0}$ & ${-}41.7\phantom{0}$ & $\phantom{0}\phantom{-}3.4\phantom{0}$ & $\phantom{0}\phantom{-}4.3\phantom{0}$ & $\phantom{0}{-}2.7\phantom{0}$ & $\phantom{0}0.20\phantom{0}$ & $\phantom{0}0.25\phantom{0}$ & $\phantom{0}0.45\phantom{0}$ & $\phantom{0}0.28\phantom{0}$ & $\phantom{0}0.29\phantom{0}$ & $\phantom{0}0.26\phantom{0}$ & $\phantom{0}89.7\phantom{0}$ & $\phantom{0}84.4\phantom{0}$ & $\phantom{0}28.1\phantom{0}$ & $\phantom{0}92.3\phantom{0}$ & $\phantom{0}92.9\phantom{0}$ & $\phantom{0}90.1\phantom{0}$ \\
 & \nopagebreak $\;J=100$  & $\phantom{0}{-}0.8\phantom{0}$ & $\phantom{0}{-}3.6\phantom{0}$ & ${-}40.1\phantom{0}$ & $\phantom{0}\phantom{-}2.4\phantom{0}$ & $\phantom{0}\phantom{-}2.4\phantom{0}$ & $\phantom{0}{-}0.8\phantom{0}$ & $\phantom{0}0.14\phantom{0}$ & $\phantom{0}0.18\phantom{0}$ & $\phantom{0}0.42\phantom{0}$ & $\phantom{0}0.19\phantom{0}$ & $\phantom{0}0.19\phantom{0}$ & $\phantom{0}0.18\phantom{0}$ & $\phantom{0}92.3\phantom{0}$ & $\phantom{0}88.8\phantom{0}$ & $\phantom{0}13.2\phantom{0}$ & $\phantom{0}93.7\phantom{0}$ & $\phantom{0}94.0\phantom{0}$ & $\phantom{0}92.1\phantom{0}$ \\
 & \nopagebreak $\;J=200$  & $\phantom{0}{-}0.2\phantom{0}$ & $\phantom{0}{-}2.8\phantom{0}$ & ${-}39.6\phantom{0}$ & $\phantom{0}\phantom{-}1.2\phantom{0}$ & $\phantom{0}\phantom{-}1.5\phantom{0}$ & $\phantom{0}{-}0.5\phantom{0}$ & $\phantom{0}0.10\phantom{0}$ & $\phantom{0}0.13\phantom{0}$ & $\phantom{0}0.41\phantom{0}$ & $\phantom{0}0.13\phantom{0}$ & $\phantom{0}0.14\phantom{0}$ & $\phantom{0}0.13\phantom{0}$ & $\phantom{0}94.8\phantom{0}$ & $\phantom{0}92.2\phantom{0}$ & $\phantom{0}\phantom{0}1.2\phantom{0}$ & $\phantom{0}94.6\phantom{0}$ & $\phantom{0}94.3\phantom{0}$ & $\phantom{0}93.7\phantom{0}$ \\
 & \nopagebreak $\;J=500$  & $\phantom{0}{-}0.0\phantom{0}$ & $\phantom{0}{-}2.8\phantom{0}$ & ${-}39.6\phantom{0}$ & $\phantom{0}\phantom{-}0.3\phantom{0}$ & $\phantom{0}\phantom{-}0.5\phantom{0}$ & $\phantom{0}{-}0.3\phantom{0}$ & $\phantom{0}0.06\phantom{0}$ & $\phantom{0}0.08\phantom{0}$ & $\phantom{0}0.40\phantom{0}$ & $\phantom{0}0.08\phantom{0}$ & $\phantom{0}0.08\phantom{0}$ & $\phantom{0}0.08\phantom{0}$ & $\phantom{0}94.9\phantom{0}$ & $\phantom{0}92.1\phantom{0}$ & $\phantom{0}\phantom{0}0.0\phantom{0}$ & $\phantom{0}93.9\phantom{0}$ & $\phantom{0}95.4\phantom{0}$ & $\phantom{0}94.0\phantom{0}$ \\
 & \nopagebreak $\;J=1000$  & $\phantom{0}{-}0.0\phantom{0}$ & $\phantom{0}{-}2.4\phantom{0}$ & ${-}39.4\phantom{0}$ & $\phantom{0}\phantom{-}0.4\phantom{0}$ & $\phantom{0}\phantom{-}0.4\phantom{0}$ & $\phantom{0}\phantom{-}0.1\phantom{0}$ & $\phantom{0}0.05\phantom{0}$ & $\phantom{0}0.06\phantom{0}$ & $\phantom{0}0.40\phantom{0}$ & $\phantom{0}0.06\phantom{0}$ & $\phantom{0}0.06\phantom{0}$ & $\phantom{0}0.06\phantom{0}$ & $\phantom{0}94.3\phantom{0}$ & $\phantom{0}91.9\phantom{0}$ & $\phantom{0}\phantom{0}0.0\phantom{0}$ & $\phantom{0}94.4\phantom{0}$ & $\phantom{0}93.5\phantom{0}$ & $\phantom{0}93.9\phantom{0}$ \\
[0.5ex]\hline\\[-1.6ex] 
& & \multicolumn{18}{c}{Moderate intraclass correlation $(\rho_{Iy}=.30)$} \\[0.6ex]\hline\\[-1.8ex]
\multicolumn{4}{l}{$n=5$} \\  & \nopagebreak $\;J=30$  & $\phantom{0}{-}2.2\phantom{0}$ & $\phantom{0}{-}6.9\phantom{0}$ & ${-}34.5\phantom{0}$ & $\phantom{-}10.8\phantom{0}$ & $\phantom{-}12.3\phantom{0}$ & $\phantom{0}\phantom{-}1.7\phantom{0}$ & $\phantom{0}0.26\phantom{0}$ & $\phantom{0}0.33\phantom{0}$ & $\phantom{0}0.42\phantom{0}$ & $\phantom{0}0.43\phantom{0}$ & $\phantom{0}0.44\phantom{0}$ & $\phantom{0}0.35\phantom{0}$ & $\phantom{0}87.7\phantom{0}$ & $\phantom{0}80.6\phantom{0}$ & $\phantom{0}49.7\phantom{0}$ & $\phantom{0}92.3\phantom{0}$ & $\phantom{0}92.3\phantom{0}$ & $\phantom{0}89.7\phantom{0}$ \\
 & \nopagebreak $\;J=50$  & $\phantom{0}{-}2.4\phantom{0}$ & $\phantom{0}{-}6.4\phantom{0}$ & ${-}35.1\phantom{0}$ & $\phantom{0}\phantom{-}3.9\phantom{0}$ & $\phantom{0}\phantom{-}4.7\phantom{0}$ & $\phantom{0}{-}0.5\phantom{0}$ & $\phantom{0}0.20\phantom{0}$ & $\phantom{0}0.26\phantom{0}$ & $\phantom{0}0.39\phantom{0}$ & $\phantom{0}0.29\phantom{0}$ & $\phantom{0}0.30\phantom{0}$ & $\phantom{0}0.27\phantom{0}$ & $\phantom{0}90.3\phantom{0}$ & $\phantom{0}84.4\phantom{0}$ & $\phantom{0}38.9\phantom{0}$ & $\phantom{0}92.1\phantom{0}$ & $\phantom{0}93.6\phantom{0}$ & $\phantom{0}89.9\phantom{0}$ \\
 & \nopagebreak $\;J=100$  & $\phantom{0}{-}1.9\phantom{0}$ & $\phantom{0}{-}5.0\phantom{0}$ & ${-}34.2\phantom{0}$ & $\phantom{0}\phantom{-}1.1\phantom{0}$ & $\phantom{0}\phantom{-}1.3\phantom{0}$ & $\phantom{0}{-}0.9\phantom{0}$ & $\phantom{0}0.14\phantom{0}$ & $\phantom{0}0.18\phantom{0}$ & $\phantom{0}0.37\phantom{0}$ & $\phantom{0}0.20\phantom{0}$ & $\phantom{0}0.19\phantom{0}$ & $\phantom{0}0.19\phantom{0}$ & $\phantom{0}91.6\phantom{0}$ & $\phantom{0}87.4\phantom{0}$ & $\phantom{0}23.0\phantom{0}$ & $\phantom{0}92.3\phantom{0}$ & $\phantom{0}93.1\phantom{0}$ & $\phantom{0}91.7\phantom{0}$ \\
 & \nopagebreak $\;J=200$  & $\phantom{0}{-}0.2\phantom{0}$ & $\phantom{0}{-}2.5\phantom{0}$ & ${-}32.7\phantom{0}$ & $\phantom{0}\phantom{-}1.6\phantom{0}$ & $\phantom{0}\phantom{-}1.7\phantom{0}$ & $\phantom{0}\phantom{-}0.8\phantom{0}$ & $\phantom{0}0.10\phantom{0}$ & $\phantom{0}0.13\phantom{0}$ & $\phantom{0}0.34\phantom{0}$ & $\phantom{0}0.14\phantom{0}$ & $\phantom{0}0.14\phantom{0}$ & $\phantom{0}0.14\phantom{0}$ & $\phantom{0}93.6\phantom{0}$ & $\phantom{0}90.7\phantom{0}$ & $\phantom{0}\phantom{0}8.0\phantom{0}$ & $\phantom{0}94.2\phantom{0}$ & $\phantom{0}94.9\phantom{0}$ & $\phantom{0}94.3\phantom{0}$ \\
 & \nopagebreak $\;J=500$  & $\phantom{0}{-}0.2\phantom{0}$ & $\phantom{0}{-}2.8\phantom{0}$ & ${-}33.0\phantom{0}$ & $\phantom{0}\phantom{-}0.3\phantom{0}$ & $\phantom{0}\phantom{-}0.4\phantom{0}$ & $\phantom{0}\phantom{-}0.0\phantom{0}$ & $\phantom{0}0.06\phantom{0}$ & $\phantom{0}0.09\phantom{0}$ & $\phantom{0}0.34\phantom{0}$ & $\phantom{0}0.09\phantom{0}$ & $\phantom{0}0.09\phantom{0}$ & $\phantom{0}0.09\phantom{0}$ & $\phantom{0}93.9\phantom{0}$ & $\phantom{0}90.5\phantom{0}$ & $\phantom{0}\phantom{0}0.1\phantom{0}$ & $\phantom{0}93.4\phantom{0}$ & $\phantom{0}92.5\phantom{0}$ & $\phantom{0}93.0\phantom{0}$ \\
 & \nopagebreak $\;J=1000$  & $\phantom{0}{-}0.0\phantom{0}$ & $\phantom{0}{-}2.5\phantom{0}$ & ${-}32.8\phantom{0}$ & $\phantom{0}\phantom{-}0.4\phantom{0}$ & $\phantom{0}\phantom{-}0.3\phantom{0}$ & $\phantom{0}\phantom{-}0.1\phantom{0}$ & $\phantom{0}0.04\phantom{0}$ & $\phantom{0}0.06\phantom{0}$ & $\phantom{0}0.33\phantom{0}$ & $\phantom{0}0.06\phantom{0}$ & $\phantom{0}0.06\phantom{0}$ & $\phantom{0}0.06\phantom{0}$ & $\phantom{0}95.4\phantom{0}$ & $\phantom{0}91.6\phantom{0}$ & $\phantom{0}\phantom{0}0.0\phantom{0}$ & $\phantom{0}95.5\phantom{0}$ & $\phantom{0}95.8\phantom{0}$ & $\phantom{0}96.2\phantom{0}$ \\
\multicolumn{4}{l}{$n=20$} \\  & \nopagebreak $\;J=30$  & $\phantom{0}{-}3.1\phantom{0}$ & $\phantom{0}{-}8.3\phantom{0}$ & ${-}42.2\phantom{0}$ & $\phantom{0}\phantom{-}9.2\phantom{0}$ & $\phantom{0}\phantom{-}9.0\phantom{0}$ & $\phantom{0}\phantom{-}1.4\phantom{0}$ & $\phantom{0}0.26\phantom{0}$ & $\phantom{0}0.33\phantom{0}$ & $\phantom{0}0.48\phantom{0}$ & $\phantom{0}0.42\phantom{0}$ & $\phantom{0}0.41\phantom{0}$ & $\phantom{0}0.37\phantom{0}$ & $\phantom{0}87.5\phantom{0}$ & $\phantom{0}80.9\phantom{0}$ & $\phantom{0}37.2\phantom{0}$ & $\phantom{0}92.1\phantom{0}$ & $\phantom{0}92.3\phantom{0}$ & $\phantom{0}89.3\phantom{0}$ \\
 & \nopagebreak $\;J=50$  & $\phantom{0}{-}1.4\phantom{0}$ & $\phantom{0}{-}5.2\phantom{0}$ & ${-}40.3\phantom{0}$ & $\phantom{0}\phantom{-}5.4\phantom{0}$ & $\phantom{0}\phantom{-}5.3\phantom{0}$ & $\phantom{0}\phantom{-}1.5\phantom{0}$ & $\phantom{0}0.20\phantom{0}$ & $\phantom{0}0.25\phantom{0}$ & $\phantom{0}0.44\phantom{0}$ & $\phantom{0}0.29\phantom{0}$ & $\phantom{0}0.29\phantom{0}$ & $\phantom{0}0.27\phantom{0}$ & $\phantom{0}90.1\phantom{0}$ & $\phantom{0}85.1\phantom{0}$ & $\phantom{0}31.7\phantom{0}$ & $\phantom{0}93.5\phantom{0}$ & $\phantom{0}93.5\phantom{0}$ & $\phantom{0}91.5\phantom{0}$ \\
 & \nopagebreak $\;J=100$  & $\phantom{0}{-}0.7\phantom{0}$ & $\phantom{0}{-}4.5\phantom{0}$ & ${-}39.9\phantom{0}$ & $\phantom{0}\phantom{-}2.0\phantom{0}$ & $\phantom{0}\phantom{-}2.1\phantom{0}$ & $\phantom{0}\phantom{-}0.2\phantom{0}$ & $\phantom{0}0.14\phantom{0}$ & $\phantom{0}0.18\phantom{0}$ & $\phantom{0}0.42\phantom{0}$ & $\phantom{0}0.20\phantom{0}$ & $\phantom{0}0.20\phantom{0}$ & $\phantom{0}0.19\phantom{0}$ & $\phantom{0}92.2\phantom{0}$ & $\phantom{0}87.9\phantom{0}$ & $\phantom{0}12.4\phantom{0}$ & $\phantom{0}93.9\phantom{0}$ & $\phantom{0}93.3\phantom{0}$ & $\phantom{0}93.0\phantom{0}$ \\
 & \nopagebreak $\;J=200$  & $\phantom{0}{-}0.6\phantom{0}$ & $\phantom{0}{-}4.0\phantom{0}$ & ${-}39.5\phantom{0}$ & $\phantom{0}\phantom{-}0.8\phantom{0}$ & $\phantom{0}\phantom{-}0.7\phantom{0}$ & $\phantom{0}{-}0.0\phantom{0}$ & $\phantom{0}0.10\phantom{0}$ & $\phantom{0}0.13\phantom{0}$ & $\phantom{0}0.40\phantom{0}$ & $\phantom{0}0.13\phantom{0}$ & $\phantom{0}0.13\phantom{0}$ & $\phantom{0}0.13\phantom{0}$ & $\phantom{0}93.7\phantom{0}$ & $\phantom{0}89.9\phantom{0}$ & $\phantom{0}\phantom{0}1.8\phantom{0}$ & $\phantom{0}94.3\phantom{0}$ & $\phantom{0}93.7\phantom{0}$ & $\phantom{0}94.8\phantom{0}$ \\
 & \nopagebreak $\;J=500$  & $\phantom{0}{-}0.2\phantom{0}$ & $\phantom{0}{-}3.5\phantom{0}$ & ${-}39.3\phantom{0}$ & $\phantom{0}\phantom{-}0.4\phantom{0}$ & $\phantom{0}\phantom{-}0.4\phantom{0}$ & $\phantom{0}\phantom{-}0.1\phantom{0}$ & $\phantom{0}0.06\phantom{0}$ & $\phantom{0}0.08\phantom{0}$ & $\phantom{0}0.40\phantom{0}$ & $\phantom{0}0.08\phantom{0}$ & $\phantom{0}0.08\phantom{0}$ & $\phantom{0}0.08\phantom{0}$ & $\phantom{0}95.1\phantom{0}$ & $\phantom{0}90.4\phantom{0}$ & $\phantom{0}\phantom{0}0.0\phantom{0}$ & $\phantom{0}93.8\phantom{0}$ & $\phantom{0}95.2\phantom{0}$ & $\phantom{0}95.2\phantom{0}$ \\
 & \nopagebreak $\;J=1000$  & $\phantom{0}{-}0.1\phantom{0}$ & $\phantom{0}{-}3.4\phantom{0}$ & ${-}39.2\phantom{0}$ & $\phantom{0}\phantom{-}0.2\phantom{0}$ & $\phantom{0}\phantom{-}0.2\phantom{0}$ & $\phantom{0}\phantom{-}0.1\phantom{0}$ & $\phantom{0}0.05\phantom{0}$ & $\phantom{0}0.07\phantom{0}$ & $\phantom{0}0.39\phantom{0}$ & $\phantom{0}0.06\phantom{0}$ & $\phantom{0}0.06\phantom{0}$ & $\phantom{0}0.06\phantom{0}$ & $\phantom{0}94.2\phantom{0}$ & $\phantom{0}88.0\phantom{0}$ & $\phantom{0}\phantom{0}0.0\phantom{0}$ & $\phantom{0}94.6\phantom{0}$ & $\phantom{0}95.5\phantom{0}$ & $\phantom{0}94.6\phantom{0}$ \\
[0.5ex]\hline\\[-1.6ex] 
\end{tabular}
\begin{tablenotes}[para,flushleft]{\footnotesize \textit{Note.} $n$ = cluster size; $J$ = number of clusters; CD = complete data sets; LD = listwise deletion; FCS-SL = single-level FCS; FCS-MAN = two-level FCS with manifest cluster means; FCS-LAT = two-level FCS with latent cluster means; JM = joint modeling.}\end{tablenotes}
\end{threeparttable}
\end{sidewaystable}
\begin{sidewaystable}
\begin{threeparttable}
\setlength{\tabcolsep}{1.2pt}
\renewcommand{\arraystretch}{0.95}
\footnotesize
\caption{\small Study 1: Bias (in \%), RMSE, and Coverage of the 95\% Confidence Interval for the Variance of $z$ ($\hat\sigma_z^2$) With 40\% Missing Data (MAR, $\lambda=1$)}
\begin{tabular}{llcccccccccccccccccc}
\hline\\[-1.8ex]
& & \multicolumn{6}{c}{Bias (\%)} & \multicolumn{6}{c}{RMSE} & \multicolumn{6}{c}{Coverage (\%)} \\ \cmidrule(r){3-8}\cmidrule(r){9-14}\cmidrule(r){15-20}
 &  & CD & LD & \makecell{FCS-\\SL} & \makecell{FCS-\\MAN} & \makecell{FCS-\\LAT} & JM & CD & LD & \makecell{FCS-\\SL} & \makecell{FCS-\\MAN} & \makecell{FCS-\\LAT} & JM & CD & LD & \makecell{FCS-\\SL} & \makecell{FCS-\\MAN} & \makecell{FCS-\\LAT} & \multicolumn{1}{c}{JM} \\ 
[0.4ex]\hline\\[-1.8ex]
& & \multicolumn{18}{c}{Small intraclass correlation $(\rho_{Iy}=.10)$} \\[0.6ex]\hline\\[-1.8ex]
\multicolumn{4}{l}{$n=5$} \\  & \nopagebreak $\;J=30$  & $\phantom{0}{-}3.4\phantom{0}$ & ${-}10.6\phantom{0}$ & ${-}38.3\phantom{0}$ & $\phantom{-}25.9\phantom{0}$ & $\phantom{-}21.3\phantom{0}$ & $\phantom{0}{-}1.4\phantom{0}$ & $\phantom{0}0.26\phantom{0}$ & $\phantom{0}0.33\phantom{0}$ & $\phantom{0}0.44\phantom{0}$ & $\phantom{0}0.61\phantom{0}$ & $\phantom{0}0.53\phantom{0}$ & $\phantom{0}0.35\phantom{0}$ & $\phantom{0}87.7\phantom{0}$ & $\phantom{0}79.9\phantom{0}$ & $\phantom{0}42.3\phantom{0}$ & $\phantom{0}95.5\phantom{0}$ & $\phantom{0}95.3\phantom{0}$ & $\phantom{0}89.9\phantom{0}$ \\
 & \nopagebreak $\;J=50$  & $\phantom{0}{-}2.1\phantom{0}$ & $\phantom{0}{-}8.6\phantom{0}$ & ${-}37.4\phantom{0}$ & $\phantom{-}13.2\phantom{0}$ & $\phantom{-}13.4\phantom{0}$ & $\phantom{0}{-}1.8\phantom{0}$ & $\phantom{0}0.20\phantom{0}$ & $\phantom{0}0.26\phantom{0}$ & $\phantom{0}0.41\phantom{0}$ & $\phantom{0}0.38\phantom{0}$ & $\phantom{0}0.37\phantom{0}$ & $\phantom{0}0.27\phantom{0}$ & $\phantom{0}89.7\phantom{0}$ & $\phantom{0}81.9\phantom{0}$ & $\phantom{0}34.5\phantom{0}$ & $\phantom{0}94.6\phantom{0}$ & $\phantom{0}95.5\phantom{0}$ & $\phantom{0}89.3\phantom{0}$ \\
 & \nopagebreak $\;J=100$  & $\phantom{0}{-}1.1\phantom{0}$ & $\phantom{0}{-}7.1\phantom{0}$ & ${-}36.5\phantom{0}$ & $\phantom{0}\phantom{-}6.1\phantom{0}$ & $\phantom{0}\phantom{-}7.6\phantom{0}$ & $\phantom{0}{-}2.4\phantom{0}$ & $\phantom{0}0.14\phantom{0}$ & $\phantom{0}0.18\phantom{0}$ & $\phantom{0}0.38\phantom{0}$ & $\phantom{0}0.24\phantom{0}$ & $\phantom{0}0.24\phantom{0}$ & $\phantom{0}0.19\phantom{0}$ & $\phantom{0}92.7\phantom{0}$ & $\phantom{0}85.9\phantom{0}$ & $\phantom{0}17.9\phantom{0}$ & $\phantom{0}95.0\phantom{0}$ & $\phantom{0}94.7\phantom{0}$ & $\phantom{0}91.9\phantom{0}$ \\
 & \nopagebreak $\;J=200$  & $\phantom{0}{-}0.7\phantom{0}$ & $\phantom{0}{-}5.7\phantom{0}$ & ${-}35.6\phantom{0}$ & $\phantom{0}\phantom{-}3.1\phantom{0}$ & $\phantom{0}\phantom{-}4.6\phantom{0}$ & $\phantom{0}{-}1.7\phantom{0}$ & $\phantom{0}0.10\phantom{0}$ & $\phantom{0}0.14\phantom{0}$ & $\phantom{0}0.37\phantom{0}$ & $\phantom{0}0.15\phantom{0}$ & $\phantom{0}0.16\phantom{0}$ & $\phantom{0}0.14\phantom{0}$ & $\phantom{0}93.8\phantom{0}$ & $\phantom{0}85.6\phantom{0}$ & $\phantom{0}\phantom{0}3.8\phantom{0}$ & $\phantom{0}94.9\phantom{0}$ & $\phantom{0}95.2\phantom{0}$ & $\phantom{0}92.2\phantom{0}$ \\
 & \nopagebreak $\;J=500$  & $\phantom{0}{-}0.3\phantom{0}$ & $\phantom{0}{-}5.4\phantom{0}$ & ${-}35.5\phantom{0}$ & $\phantom{0}\phantom{-}1.3\phantom{0}$ & $\phantom{0}\phantom{-}2.0\phantom{0}$ & $\phantom{0}{-}1.3\phantom{0}$ & $\phantom{0}0.06\phantom{0}$ & $\phantom{0}0.09\phantom{0}$ & $\phantom{0}0.36\phantom{0}$ & $\phantom{0}0.10\phantom{0}$ & $\phantom{0}0.10\phantom{0}$ & $\phantom{0}0.09\phantom{0}$ & $\phantom{0}94.2\phantom{0}$ & $\phantom{0}85.5\phantom{0}$ & $\phantom{0}\phantom{0}0.0\phantom{0}$ & $\phantom{0}94.0\phantom{0}$ & $\phantom{0}93.8\phantom{0}$ & $\phantom{0}92.1\phantom{0}$ \\
 & \nopagebreak $\;J=1000$  & $\phantom{0}{-}0.1\phantom{0}$ & $\phantom{0}{-}5.4\phantom{0}$ & ${-}35.6\phantom{0}$ & $\phantom{0}\phantom{-}0.3\phantom{0}$ & $\phantom{0}\phantom{-}0.5\phantom{0}$ & $\phantom{0}{-}1.1\phantom{0}$ & $\phantom{0}0.04\phantom{0}$ & $\phantom{0}0.08\phantom{0}$ & $\phantom{0}0.36\phantom{0}$ & $\phantom{0}0.06\phantom{0}$ & $\phantom{0}0.06\phantom{0}$ & $\phantom{0}0.06\phantom{0}$ & $\phantom{0}95.7\phantom{0}$ & $\phantom{0}79.6\phantom{0}$ & $\phantom{0}\phantom{0}0.0\phantom{0}$ & $\phantom{0}94.1\phantom{0}$ & $\phantom{0}94.9\phantom{0}$ & $\phantom{0}93.6\phantom{0}$ \\
\multicolumn{4}{l}{$n=20$} \\  & \nopagebreak $\;J=30$  & $\phantom{0}{-}3.1\phantom{0}$ & ${-}15.0\phantom{0}$ & ${-}47.0\phantom{0}$ & $\phantom{-}21.7\phantom{0}$ & $\phantom{-}26.5\phantom{0}$ & $\phantom{0}{-}5.4\phantom{0}$ & $\phantom{0}0.25\phantom{0}$ & $\phantom{0}0.34\phantom{0}$ & $\phantom{0}0.51\phantom{0}$ & $\phantom{0}0.56\phantom{0}$ & $\phantom{0}0.61\phantom{0}$ & $\phantom{0}0.35\phantom{0}$ & $\phantom{0}87.0\phantom{0}$ & $\phantom{0}74.7\phantom{0}$ & $\phantom{0}29.0\phantom{0}$ & $\phantom{0}94.4\phantom{0}$ & $\phantom{0}95.0\phantom{0}$ & $\phantom{0}86.7\phantom{0}$ \\
 & \nopagebreak $\;J=50$  & $\phantom{0}{-}2.1\phantom{0}$ & ${-}12.9\phantom{0}$ & ${-}45.8\phantom{0}$ & $\phantom{-}11.3\phantom{0}$ & $\phantom{-}15.6\phantom{0}$ & $\phantom{0}{-}5.5\phantom{0}$ & $\phantom{0}0.20\phantom{0}$ & $\phantom{0}0.26\phantom{0}$ & $\phantom{0}0.48\phantom{0}$ & $\phantom{0}0.37\phantom{0}$ & $\phantom{0}0.40\phantom{0}$ & $\phantom{0}0.26\phantom{0}$ & $\phantom{0}89.5\phantom{0}$ & $\phantom{0}80.5\phantom{0}$ & $\phantom{0}20.1\phantom{0}$ & $\phantom{0}94.9\phantom{0}$ & $\phantom{0}94.8\phantom{0}$ & $\phantom{0}89.4\phantom{0}$ \\
 & \nopagebreak $\;J=100$  & $\phantom{0}{-}1.3\phantom{0}$ & ${-}11.7\phantom{0}$ & ${-}45.1\phantom{0}$ & $\phantom{0}\phantom{-}4.9\phantom{0}$ & $\phantom{0}\phantom{-}6.5\phantom{0}$ & $\phantom{0}{-}4.8\phantom{0}$ & $\phantom{0}0.15\phantom{0}$ & $\phantom{0}0.20\phantom{0}$ & $\phantom{0}0.46\phantom{0}$ & $\phantom{0}0.24\phantom{0}$ & $\phantom{0}0.25\phantom{0}$ & $\phantom{0}0.19\phantom{0}$ & $\phantom{0}90.5\phantom{0}$ & $\phantom{0}79.3\phantom{0}$ & $\phantom{0}\phantom{0}5.2\phantom{0}$ & $\phantom{0}94.1\phantom{0}$ & $\phantom{0}94.0\phantom{0}$ & $\phantom{0}89.2\phantom{0}$ \\
 & \nopagebreak $\;J=200$  & $\phantom{0}{-}0.6\phantom{0}$ & ${-}11.0\phantom{0}$ & ${-}44.8\phantom{0}$ & $\phantom{0}\phantom{-}2.3\phantom{0}$ & $\phantom{0}\phantom{-}2.6\phantom{0}$ & $\phantom{0}{-}3.6\phantom{0}$ & $\phantom{0}0.10\phantom{0}$ & $\phantom{0}0.16\phantom{0}$ & $\phantom{0}0.45\phantom{0}$ & $\phantom{0}0.16\phantom{0}$ & $\phantom{0}0.16\phantom{0}$ & $\phantom{0}0.14\phantom{0}$ & $\phantom{0}94.3\phantom{0}$ & $\phantom{0}75.8\phantom{0}$ & $\phantom{0}\phantom{0}0.2\phantom{0}$ & $\phantom{0}94.4\phantom{0}$ & $\phantom{0}95.2\phantom{0}$ & $\phantom{0}91.3\phantom{0}$ \\
 & \nopagebreak $\;J=500$  & $\phantom{0}\phantom{-}0.2\phantom{0}$ & $\phantom{0}{-}9.8\phantom{0}$ & ${-}43.9\phantom{0}$ & $\phantom{0}\phantom{-}1.4\phantom{0}$ & $\phantom{0}\phantom{-}1.5\phantom{0}$ & $\phantom{0}{-}1.4\phantom{0}$ & $\phantom{0}0.06\phantom{0}$ & $\phantom{0}0.12\phantom{0}$ & $\phantom{0}0.44\phantom{0}$ & $\phantom{0}0.10\phantom{0}$ & $\phantom{0}0.10\phantom{0}$ & $\phantom{0}0.09\phantom{0}$ & $\phantom{0}95.5\phantom{0}$ & $\phantom{0}69.3\phantom{0}$ & $\phantom{0}\phantom{0}0.0\phantom{0}$ & $\phantom{0}95.0\phantom{0}$ & $\phantom{0}94.4\phantom{0}$ & $\phantom{0}93.5\phantom{0}$ \\
 & \nopagebreak $\;J=1000$  & $\phantom{0}{-}0.1\phantom{0}$ & ${-}10.1\phantom{0}$ & ${-}44.2\phantom{0}$ & $\phantom{0}\phantom{-}0.3\phantom{0}$ & $\phantom{0}\phantom{-}0.3\phantom{0}$ & $\phantom{0}{-}1.0\phantom{0}$ & $\phantom{0}0.04\phantom{0}$ & $\phantom{0}0.11\phantom{0}$ & $\phantom{0}0.44\phantom{0}$ & $\phantom{0}0.07\phantom{0}$ & $\phantom{0}0.07\phantom{0}$ & $\phantom{0}0.07\phantom{0}$ & $\phantom{0}95.3\phantom{0}$ & $\phantom{0}49.8\phantom{0}$ & $\phantom{0}\phantom{0}0.0\phantom{0}$ & $\phantom{0}94.8\phantom{0}$ & $\phantom{0}94.7\phantom{0}$ & $\phantom{0}94.4\phantom{0}$ \\
[0.5ex]\hline\\[-1.6ex] 
& & \multicolumn{18}{c}{Moderate intraclass correlation $(\rho_{Iy}=.30)$} \\[0.6ex]\hline\\[-1.8ex]
\multicolumn{4}{l}{$n=5$} \\  & \nopagebreak $\;J=30$  & $\phantom{0}{-}3.0\phantom{0}$ & ${-}14.9\phantom{0}$ & ${-}40.4\phantom{0}$ & $\phantom{-}24.5\phantom{0}$ & $\phantom{-}30.0\phantom{0}$ & $\phantom{0}{-}0.2\phantom{0}$ & $\phantom{0}0.26\phantom{0}$ & $\phantom{0}0.32\phantom{0}$ & $\phantom{0}0.46\phantom{0}$ & $\phantom{0}0.61\phantom{0}$ & $\phantom{0}0.64\phantom{0}$ & $\phantom{0}0.36\phantom{0}$ & $\phantom{0}86.7\phantom{0}$ & $\phantom{0}75.8\phantom{0}$ & $\phantom{0}38.9\phantom{0}$ & $\phantom{0}95.2\phantom{0}$ & $\phantom{0}95.0\phantom{0}$ & $\phantom{0}89.8\phantom{0}$ \\
 & \nopagebreak $\;J=50$  & $\phantom{0}{-}2.6\phantom{0}$ & ${-}13.9\phantom{0}$ & ${-}39.9\phantom{0}$ & $\phantom{-}11.1\phantom{0}$ & $\phantom{-}15.7\phantom{0}$ & $\phantom{0}{-}1.9\phantom{0}$ & $\phantom{0}0.20\phantom{0}$ & $\phantom{0}0.26\phantom{0}$ & $\phantom{0}0.43\phantom{0}$ & $\phantom{0}0.37\phantom{0}$ & $\phantom{0}0.41\phantom{0}$ & $\phantom{0}0.27\phantom{0}$ & $\phantom{0}89.5\phantom{0}$ & $\phantom{0}77.6\phantom{0}$ & $\phantom{0}29.0\phantom{0}$ & $\phantom{0}94.4\phantom{0}$ & $\phantom{0}95.5\phantom{0}$ & $\phantom{0}90.6\phantom{0}$ \\
 & \nopagebreak $\;J=100$  & $\phantom{0}{-}1.2\phantom{0}$ & ${-}11.4\phantom{0}$ & ${-}38.8\phantom{0}$ & $\phantom{0}\phantom{-}5.9\phantom{0}$ & $\phantom{0}\phantom{-}7.4\phantom{0}$ & $\phantom{0}{-}0.1\phantom{0}$ & $\phantom{0}0.15\phantom{0}$ & $\phantom{0}0.20\phantom{0}$ & $\phantom{0}0.41\phantom{0}$ & $\phantom{0}0.25\phantom{0}$ & $\phantom{0}0.27\phantom{0}$ & $\phantom{0}0.21\phantom{0}$ & $\phantom{0}90.5\phantom{0}$ & $\phantom{0}79.1\phantom{0}$ & $\phantom{0}13.9\phantom{0}$ & $\phantom{0}93.7\phantom{0}$ & $\phantom{0}94.1\phantom{0}$ & $\phantom{0}92.3\phantom{0}$ \\
 & \nopagebreak $\;J=200$  & $\phantom{0}{-}0.5\phantom{0}$ & ${-}10.8\phantom{0}$ & ${-}38.5\phantom{0}$ & $\phantom{0}\phantom{-}2.4\phantom{0}$ & $\phantom{0}\phantom{-}2.6\phantom{0}$ & $\phantom{0}{-}0.3\phantom{0}$ & $\phantom{0}0.10\phantom{0}$ & $\phantom{0}0.16\phantom{0}$ & $\phantom{0}0.39\phantom{0}$ & $\phantom{0}0.16\phantom{0}$ & $\phantom{0}0.17\phantom{0}$ & $\phantom{0}0.15\phantom{0}$ & $\phantom{0}94.7\phantom{0}$ & $\phantom{0}76.8\phantom{0}$ & $\phantom{0}\phantom{0}1.5\phantom{0}$ & $\phantom{0}93.9\phantom{0}$ & $\phantom{0}93.4\phantom{0}$ & $\phantom{0}92.8\phantom{0}$ \\
 & \nopagebreak $\;J=500$  & $\phantom{0}{-}0.4\phantom{0}$ & ${-}10.5\phantom{0}$ & ${-}38.4\phantom{0}$ & $\phantom{0}\phantom{-}0.7\phantom{0}$ & $\phantom{0}\phantom{-}0.8\phantom{0}$ & $\phantom{0}{-}0.3\phantom{0}$ & $\phantom{0}0.06\phantom{0}$ & $\phantom{0}0.13\phantom{0}$ & $\phantom{0}0.39\phantom{0}$ & $\phantom{0}0.10\phantom{0}$ & $\phantom{0}0.10\phantom{0}$ & $\phantom{0}0.10\phantom{0}$ & $\phantom{0}94.6\phantom{0}$ & $\phantom{0}66.1\phantom{0}$ & $\phantom{0}\phantom{0}0.0\phantom{0}$ & $\phantom{0}93.2\phantom{0}$ & $\phantom{0}93.1\phantom{0}$ & $\phantom{0}92.5\phantom{0}$ \\
 & \nopagebreak $\;J=1000$  & $\phantom{0}{-}0.3\phantom{0}$ & ${-}10.1\phantom{0}$ & ${-}38.2\phantom{0}$ & $\phantom{0}\phantom{-}0.6\phantom{0}$ & $\phantom{0}\phantom{-}0.6\phantom{0}$ & $\phantom{0}\phantom{-}0.1\phantom{0}$ & $\phantom{0}0.05\phantom{0}$ & $\phantom{0}0.11\phantom{0}$ & $\phantom{0}0.38\phantom{0}$ & $\phantom{0}0.07\phantom{0}$ & $\phantom{0}0.07\phantom{0}$ & $\phantom{0}0.07\phantom{0}$ & $\phantom{0}93.7\phantom{0}$ & $\phantom{0}49.6\phantom{0}$ & $\phantom{0}\phantom{0}0.0\phantom{0}$ & $\phantom{0}93.5\phantom{0}$ & $\phantom{0}93.2\phantom{0}$ & $\phantom{0}94.5\phantom{0}$ \\
\multicolumn{4}{l}{$n=20$} \\  & \nopagebreak $\;J=30$  & $\phantom{0}{-}2.0\phantom{0}$ & ${-}17.1\phantom{0}$ & ${-}47.7\phantom{0}$ & $\phantom{-}21.2\phantom{0}$ & $\phantom{-}23.9\phantom{0}$ & $\phantom{0}\phantom{-}2.1\phantom{0}$ & $\phantom{0}0.26\phantom{0}$ & $\phantom{0}0.33\phantom{0}$ & $\phantom{0}0.51\phantom{0}$ & $\phantom{0}0.56\phantom{0}$ & $\phantom{0}0.59\phantom{0}$ & $\phantom{0}0.39\phantom{0}$ & $\phantom{0}86.7\phantom{0}$ & $\phantom{0}74.0\phantom{0}$ & $\phantom{0}28.2\phantom{0}$ & $\phantom{0}94.0\phantom{0}$ & $\phantom{0}94.1\phantom{0}$ & $\phantom{0}89.9\phantom{0}$ \\
 & \nopagebreak $\;J=50$  & $\phantom{0}{-}2.5\phantom{0}$ & ${-}16.3\phantom{0}$ & ${-}47.3\phantom{0}$ & $\phantom{-}10.3\phantom{0}$ & $\phantom{-}10.5\phantom{0}$ & $\phantom{0}{-}0.2\phantom{0}$ & $\phantom{0}0.20\phantom{0}$ & $\phantom{0}0.27\phantom{0}$ & $\phantom{0}0.50\phantom{0}$ & $\phantom{0}0.38\phantom{0}$ & $\phantom{0}0.37\phantom{0}$ & $\phantom{0}0.30\phantom{0}$ & $\phantom{0}90.1\phantom{0}$ & $\phantom{0}75.3\phantom{0}$ & $\phantom{0}16.7\phantom{0}$ & $\phantom{0}93.9\phantom{0}$ & $\phantom{0}93.1\phantom{0}$ & $\phantom{0}90.7\phantom{0}$ \\
 & \nopagebreak $\;J=100$  & $\phantom{0}{-}1.3\phantom{0}$ & ${-}15.1\phantom{0}$ & ${-}46.9\phantom{0}$ & $\phantom{0}\phantom{-}4.5\phantom{0}$ & $\phantom{0}\phantom{-}4.4\phantom{0}$ & $\phantom{0}{-}0.3\phantom{0}$ & $\phantom{0}0.14\phantom{0}$ & $\phantom{0}0.22\phantom{0}$ & $\phantom{0}0.48\phantom{0}$ & $\phantom{0}0.24\phantom{0}$ & $\phantom{0}0.25\phantom{0}$ & $\phantom{0}0.22\phantom{0}$ & $\phantom{0}92.7\phantom{0}$ & $\phantom{0}72.9\phantom{0}$ & $\phantom{0}\phantom{0}4.1\phantom{0}$ & $\phantom{0}94.1\phantom{0}$ & $\phantom{0}95.4\phantom{0}$ & $\phantom{0}93.1\phantom{0}$ \\
 & \nopagebreak $\;J=200$  & $\phantom{0}{-}0.7\phantom{0}$ & ${-}13.7\phantom{0}$ & ${-}45.9\phantom{0}$ & $\phantom{0}\phantom{-}1.9\phantom{0}$ & $\phantom{0}\phantom{-}2.2\phantom{0}$ & $\phantom{0}{-}0.2\phantom{0}$ & $\phantom{0}0.10\phantom{0}$ & $\phantom{0}0.18\phantom{0}$ & $\phantom{0}0.47\phantom{0}$ & $\phantom{0}0.17\phantom{0}$ & $\phantom{0}0.17\phantom{0}$ & $\phantom{0}0.16\phantom{0}$ & $\phantom{0}93.0\phantom{0}$ & $\phantom{0}69.6\phantom{0}$ & $\phantom{0}\phantom{0}0.1\phantom{0}$ & $\phantom{0}93.2\phantom{0}$ & $\phantom{0}94.4\phantom{0}$ & $\phantom{0}94.2\phantom{0}$ \\
 & \nopagebreak $\;J=500$  & $\phantom{0}{-}0.1\phantom{0}$ & ${-}13.0\phantom{0}$ & ${-}45.7\phantom{0}$ & $\phantom{0}\phantom{-}1.5\phantom{0}$ & $\phantom{0}\phantom{-}1.5\phantom{0}$ & $\phantom{0}\phantom{-}0.6\phantom{0}$ & $\phantom{0}0.06\phantom{0}$ & $\phantom{0}0.15\phantom{0}$ & $\phantom{0}0.46\phantom{0}$ & $\phantom{0}0.11\phantom{0}$ & $\phantom{0}0.10\phantom{0}$ & $\phantom{0}0.10\phantom{0}$ & $\phantom{0}93.8\phantom{0}$ & $\phantom{0}54.2\phantom{0}$ & $\phantom{0}\phantom{0}0.0\phantom{0}$ & $\phantom{0}94.3\phantom{0}$ & $\phantom{0}93.9\phantom{0}$ & $\phantom{0}94.7\phantom{0}$ \\
 & \nopagebreak $\;J=1000$  & $\phantom{0}{-}0.1\phantom{0}$ & ${-}13.0\phantom{0}$ & ${-}45.6\phantom{0}$ & $\phantom{0}\phantom{-}0.7\phantom{0}$ & $\phantom{0}\phantom{-}0.6\phantom{0}$ & $\phantom{0}\phantom{-}0.2\phantom{0}$ & $\phantom{0}0.05\phantom{0}$ & $\phantom{0}0.14\phantom{0}$ & $\phantom{0}0.46\phantom{0}$ & $\phantom{0}0.07\phantom{0}$ & $\phantom{0}0.07\phantom{0}$ & $\phantom{0}0.07\phantom{0}$ & $\phantom{0}94.4\phantom{0}$ & $\phantom{0}26.5\phantom{0}$ & $\phantom{0}\phantom{0}0.0\phantom{0}$ & $\phantom{0}93.4\phantom{0}$ & $\phantom{0}93.7\phantom{0}$ & $\phantom{0}94.8\phantom{0}$ \\
[0.5ex]\hline\\[-1.6ex] 
\end{tabular}
\begin{tablenotes}[para,flushleft]{\footnotesize \textit{Note.} $n$ = cluster size; $J$ = number of clusters; CD = complete data sets; LD = listwise deletion; FCS-SL = single-level FCS; FCS-MAN = two-level FCS with manifest cluster means; FCS-LAT = two-level FCS with latent cluster means; JM = joint modeling.}\end{tablenotes}
\end{threeparttable}
\end{sidewaystable}
\begin{sidewaystable}
\begin{threeparttable}
\setlength{\tabcolsep}{1.2pt}
\renewcommand{\arraystretch}{0.95}
\footnotesize
\caption{\small Study 1: Bias (in \%), RMSE, and Coverage of the 95\% Confidence Interval for the Covariance of $y$ With $z$ ($\hat\sigma_{yz}$) With 20\% Missing Data (MCAR, $\lambda=0$)}
\begin{tabular}{llcccccccccccccccccc}
\hline\\[-1.8ex]
& & \multicolumn{6}{c}{Bias (\%)} & \multicolumn{6}{c}{RMSE} & \multicolumn{6}{c}{Coverage (\%)} \\ \cmidrule(r){3-8}\cmidrule(r){9-14}\cmidrule(r){15-20}
 &  & CD & LD & \makecell{FCS-\\SL} & \makecell{FCS-\\MAN} & \makecell{FCS-\\LAT} & JM & CD & LD & \makecell{FCS-\\SL} & \makecell{FCS-\\MAN} & \makecell{FCS-\\LAT} & JM & CD & LD & \makecell{FCS-\\SL} & \makecell{FCS-\\MAN} & \makecell{FCS-\\LAT} & \multicolumn{1}{c}{JM} \\ 
[0.4ex]\hline\\[-1.8ex]
& & \multicolumn{18}{c}{Small intraclass correlation $(\rho_{Iy}=.10)$} \\[0.6ex]\hline\\[-1.8ex]
\multicolumn{4}{l}{$n=5$} \\  & \nopagebreak $\;J=30$  & $\phantom{0}{-}3.7\phantom{0}$ & $\phantom{0}{-}4.5\phantom{0}$ & ${-}18.3\phantom{0}$ & $\phantom{0}{-}3.9\phantom{0}$ & $\phantom{0}{-}0.7\phantom{0}$ & ${-}11.8\phantom{0}$ & $\phantom{0}0.10\phantom{0}$ & $\phantom{0}0.11\phantom{0}$ & $\phantom{0}0.10\phantom{0}$ & $\phantom{0}0.11\phantom{0}$ & $\phantom{0}0.11\phantom{0}$ & $\phantom{0}0.10\phantom{0}$ & $\phantom{0}91.8\phantom{0}$ & $\phantom{0}91.8\phantom{0}$ & $\phantom{0}87.9\phantom{0}$ & $\phantom{0}93.4\phantom{0}$ & $\phantom{0}93.4\phantom{0}$ & $\phantom{0}93.8\phantom{0}$ \\
 & \nopagebreak $\;J=50$  & $\phantom{0}{-}3.1\phantom{0}$ & $\phantom{0}{-}4.3\phantom{0}$ & ${-}17.8\phantom{0}$ & $\phantom{0}{-}3.5\phantom{0}$ & $\phantom{0}{-}1.0\phantom{0}$ & $\phantom{0}{-}10.0\phantom{0}$ & $\phantom{0}0.08\phantom{0}$ & $\phantom{0}0.08\phantom{0}$ & $\phantom{0}0.08\phantom{0}$ & $\phantom{0}0.09\phantom{0}$ & $\phantom{0}0.09\phantom{0}$ & $\phantom{0}0.08\phantom{0}$ & $\phantom{0}93.3\phantom{0}$ & $\phantom{0}93.1\phantom{0}$ & $\phantom{0}89.2\phantom{0}$ & $\phantom{0}93.4\phantom{0}$ & $\phantom{0}94.1\phantom{0}$ & $\phantom{0}93.8\phantom{0}$ \\
 & \nopagebreak $\;J=100$  & $\phantom{0}{-}0.5\phantom{0}$ & $\phantom{0}{-}0.3\phantom{0}$ & ${-}14.7\phantom{0}$ & $\phantom{0}{-}0.3\phantom{0}$ & $\phantom{0}\phantom{-}1.6\phantom{0}$ & $\phantom{0}{-}4.7\phantom{0}$ & $\phantom{0}0.06\phantom{0}$ & $\phantom{0}0.06\phantom{0}$ & $\phantom{0}0.06\phantom{0}$ & $\phantom{0}0.06\phantom{0}$ & $\phantom{0}0.06\phantom{0}$ & $\phantom{0}0.06\phantom{0}$ & $\phantom{0}93.7\phantom{0}$ & $\phantom{0}93.7\phantom{0}$ & $\phantom{0}89.3\phantom{0}$ & $\phantom{0}94.1\phantom{0}$ & $\phantom{0}94.1\phantom{0}$ & $\phantom{0}94.6\phantom{0}$ \\
 & \nopagebreak $\;J=200$  & $\phantom{0}\phantom{-}0.2\phantom{0}$ & $\phantom{0}{-}0.1\phantom{0}$ & ${-}14.4\phantom{0}$ & $\phantom{0}\phantom{-}0.1\phantom{0}$ & $\phantom{0}\phantom{-}1.0\phantom{0}$ & $\phantom{0}{-}2.6\phantom{0}$ & $\phantom{0}0.04\phantom{0}$ & $\phantom{0}0.04\phantom{0}$ & $\phantom{0}0.04\phantom{0}$ & $\phantom{0}0.04\phantom{0}$ & $\phantom{0}0.04\phantom{0}$ & $\phantom{0}0.04\phantom{0}$ & $\phantom{0}94.5\phantom{0}$ & $\phantom{0}93.8\phantom{0}$ & $\phantom{0}87.7\phantom{0}$ & $\phantom{0}94.9\phantom{0}$ & $\phantom{0}94.3\phantom{0}$ & $\phantom{0}94.8\phantom{0}$ \\
 & \nopagebreak $\;J=500$  & $\phantom{0}{-}0.1\phantom{0}$ & $\phantom{0}{-}0.2\phantom{0}$ & ${-}14.5\phantom{0}$ & $\phantom{0}\phantom{-}0.1\phantom{0}$ & $\phantom{0}\phantom{-}0.4\phantom{0}$ & $\phantom{0}{-}1.4\phantom{0}$ & $\phantom{0}0.03\phantom{0}$ & $\phantom{0}0.03\phantom{0}$ & $\phantom{0}0.03\phantom{0}$ & $\phantom{0}0.03\phantom{0}$ & $\phantom{0}0.03\phantom{0}$ & $\phantom{0}0.03\phantom{0}$ & $\phantom{0}94.3\phantom{0}$ & $\phantom{0}95.1\phantom{0}$ & $\phantom{0}79.9\phantom{0}$ & $\phantom{0}94.6\phantom{0}$ & $\phantom{0}95.1\phantom{0}$ & $\phantom{0}94.8\phantom{0}$ \\
 & \nopagebreak $\;J=1000$  & $\phantom{0}\phantom{-}0.2\phantom{0}$ & $\phantom{0}{-}0.0\phantom{0}$ & ${-}14.5\phantom{0}$ & $\phantom{0}{-}0.1\phantom{0}$ & $\phantom{0}\phantom{-}0.1\phantom{0}$ & $\phantom{0}{-}0.8\phantom{0}$ & $\phantom{0}0.02\phantom{0}$ & $\phantom{0}0.02\phantom{0}$ & $\phantom{0}0.03\phantom{0}$ & $\phantom{0}0.02\phantom{0}$ & $\phantom{0}0.02\phantom{0}$ & $\phantom{0}0.02\phantom{0}$ & $\phantom{0}93.6\phantom{0}$ & $\phantom{0}94.3\phantom{0}$ & $\phantom{0}68.8\phantom{0}$ & $\phantom{0}94.4\phantom{0}$ & $\phantom{0}94.5\phantom{0}$ & $\phantom{0}94.4\phantom{0}$ \\
\multicolumn{4}{l}{$n=20$} \\  & \nopagebreak $\;J=30$  & $\phantom{0}{-}4.0\phantom{0}$ & $\phantom{0}{-}4.3\phantom{0}$ & ${-}20.8\phantom{0}$ & $\phantom{0}{-}3.7\phantom{0}$ & $\phantom{0}{-}2.8\phantom{0}$ & ${-}11.1\phantom{0}$ & $\phantom{0}0.07\phantom{0}$ & $\phantom{0}0.08\phantom{0}$ & $\phantom{0}0.08\phantom{0}$ & $\phantom{0}0.08\phantom{0}$ & $\phantom{0}0.08\phantom{0}$ & $\phantom{0}0.08\phantom{0}$ & $\phantom{0}90.8\phantom{0}$ & $\phantom{0}90.0\phantom{0}$ & $\phantom{0}80.7\phantom{0}$ & $\phantom{0}92.0\phantom{0}$ & $\phantom{0}91.9\phantom{0}$ & $\phantom{0}91.0\phantom{0}$ \\
 & \nopagebreak $\;J=50$  & $\phantom{0}{-}1.0\phantom{0}$ & $\phantom{0}{-}1.7\phantom{0}$ & ${-}18.5\phantom{0}$ & $\phantom{0}{-}1.3\phantom{0}$ & $\phantom{0}{-}0.8\phantom{0}$ & $\phantom{0}{-}7.1\phantom{0}$ & $\phantom{0}0.06\phantom{0}$ & $\phantom{0}0.07\phantom{0}$ & $\phantom{0}0.06\phantom{0}$ & $\phantom{0}0.06\phantom{0}$ & $\phantom{0}0.06\phantom{0}$ & $\phantom{0}0.06\phantom{0}$ & $\phantom{0}92.5\phantom{0}$ & $\phantom{0}91.1\phantom{0}$ & $\phantom{0}82.0\phantom{0}$ & $\phantom{0}93.1\phantom{0}$ & $\phantom{0}92.9\phantom{0}$ & $\phantom{0}91.5\phantom{0}$ \\
 & \nopagebreak $\;J=100$  & $\phantom{0}{-}0.5\phantom{0}$ & $\phantom{0}{-}0.7\phantom{0}$ & ${-}17.8\phantom{0}$ & $\phantom{0}{-}0.6\phantom{0}$ & $\phantom{0}{-}0.3\phantom{0}$ & $\phantom{0}{-}4.1\phantom{0}$ & $\phantom{0}0.04\phantom{0}$ & $\phantom{0}0.05\phantom{0}$ & $\phantom{0}0.05\phantom{0}$ & $\phantom{0}0.04\phantom{0}$ & $\phantom{0}0.04\phantom{0}$ & $\phantom{0}0.04\phantom{0}$ & $\phantom{0}93.7\phantom{0}$ & $\phantom{0}93.7\phantom{0}$ & $\phantom{0}81.9\phantom{0}$ & $\phantom{0}94.5\phantom{0}$ & $\phantom{0}94.1\phantom{0}$ & $\phantom{0}93.2\phantom{0}$ \\
 & \nopagebreak $\;J=200$  & $\phantom{0}{-}0.4\phantom{0}$ & $\phantom{0}{-}0.8\phantom{0}$ & ${-}17.7\phantom{0}$ & $\phantom{0}{-}0.4\phantom{0}$ & $\phantom{0}{-}0.2\phantom{0}$ & $\phantom{0}{-}2.3\phantom{0}$ & $\phantom{0}0.03\phantom{0}$ & $\phantom{0}0.03\phantom{0}$ & $\phantom{0}0.04\phantom{0}$ & $\phantom{0}0.03\phantom{0}$ & $\phantom{0}0.03\phantom{0}$ & $\phantom{0}0.03\phantom{0}$ & $\phantom{0}94.8\phantom{0}$ & $\phantom{0}93.9\phantom{0}$ & $\phantom{0}76.3\phantom{0}$ & $\phantom{0}94.5\phantom{0}$ & $\phantom{0}93.9\phantom{0}$ & $\phantom{0}94.5\phantom{0}$ \\
 & \nopagebreak $\;J=500$  & $\phantom{0}{-}0.6\phantom{0}$ & $\phantom{0}{-}0.7\phantom{0}$ & ${-}17.7\phantom{0}$ & $\phantom{0}{-}0.8\phantom{0}$ & $\phantom{0}{-}0.8\phantom{0}$ & $\phantom{0}{-}1.4\phantom{0}$ & $\phantom{0}0.02\phantom{0}$ & $\phantom{0}0.02\phantom{0}$ & $\phantom{0}0.03\phantom{0}$ & $\phantom{0}0.02\phantom{0}$ & $\phantom{0}0.02\phantom{0}$ & $\phantom{0}0.02\phantom{0}$ & $\phantom{0}94.1\phantom{0}$ & $\phantom{0}93.6\phantom{0}$ & $\phantom{0}58.2\phantom{0}$ & $\phantom{0}94.0\phantom{0}$ & $\phantom{0}94.0\phantom{0}$ & $\phantom{0}93.2\phantom{0}$ \\
 & \nopagebreak $\;J=1000$  & $\phantom{0}{-}0.2\phantom{0}$ & $\phantom{0}{-}0.2\phantom{0}$ & ${-}17.3\phantom{0}$ & $\phantom{0}\phantom{-}0.0\phantom{0}$ & $\phantom{0}\phantom{-}0.0\phantom{0}$ & $\phantom{0}{-}0.5\phantom{0}$ & $\phantom{0}0.01\phantom{0}$ & $\phantom{0}0.01\phantom{0}$ & $\phantom{0}0.03\phantom{0}$ & $\phantom{0}0.01\phantom{0}$ & $\phantom{0}0.01\phantom{0}$ & $\phantom{0}0.01\phantom{0}$ & $\phantom{0}94.6\phantom{0}$ & $\phantom{0}94.0\phantom{0}$ & $\phantom{0}37.5\phantom{0}$ & $\phantom{0}95.0\phantom{0}$ & $\phantom{0}95.0\phantom{0}$ & $\phantom{0}95.0\phantom{0}$ \\
[0.5ex]\hline\\[-1.6ex] 
& & \multicolumn{18}{c}{Moderate intraclass correlation $(\rho_{Iy}=.30)$} \\[0.6ex]\hline\\[-1.8ex]
\multicolumn{4}{l}{$n=5$} \\  & \nopagebreak $\;J=30$  & $\phantom{0}{-}4.2\phantom{0}$ & $\phantom{0}{-}6.6\phantom{0}$ & ${-}16.4\phantom{0}$ & $\phantom{0}{-}5.0\phantom{0}$ & $\phantom{0}{-}3.5\phantom{0}$ & $\phantom{0}{-}9.5\phantom{0}$ & $\phantom{0}0.12\phantom{0}$ & $\phantom{0}0.14\phantom{0}$ & $\phantom{0}0.13\phantom{0}$ & $\phantom{0}0.14\phantom{0}$ & $\phantom{0}0.14\phantom{0}$ & $\phantom{0}0.14\phantom{0}$ & $\phantom{0}90.7\phantom{0}$ & $\phantom{0}89.7\phantom{0}$ & $\phantom{0}84.5\phantom{0}$ & $\phantom{0}92.2\phantom{0}$ & $\phantom{0}92.1\phantom{0}$ & $\phantom{0}91.5\phantom{0}$ \\
 & \nopagebreak $\;J=50$  & $\phantom{0}{-}3.5\phantom{0}$ & $\phantom{0}{-}3.8\phantom{0}$ & ${-}14.5\phantom{0}$ & $\phantom{0}{-}3.5\phantom{0}$ & $\phantom{0}{-}2.8\phantom{0}$ & $\phantom{0}{-}5.9\phantom{0}$ & $\phantom{0}0.10\phantom{0}$ & $\phantom{0}0.11\phantom{0}$ & $\phantom{0}0.11\phantom{0}$ & $\phantom{0}0.11\phantom{0}$ & $\phantom{0}0.11\phantom{0}$ & $\phantom{0}0.11\phantom{0}$ & $\phantom{0}91.1\phantom{0}$ & $\phantom{0}91.4\phantom{0}$ & $\phantom{0}86.6\phantom{0}$ & $\phantom{0}92.9\phantom{0}$ & $\phantom{0}92.4\phantom{0}$ & $\phantom{0}92.2\phantom{0}$ \\
 & \nopagebreak $\;J=100$  & $\phantom{0}\phantom{-}0.0\phantom{0}$ & $\phantom{0}{-}0.1\phantom{0}$ & ${-}11.2\phantom{0}$ & $\phantom{0}\phantom{-}0.4\phantom{0}$ & $\phantom{0}\phantom{-}0.5\phantom{0}$ & $\phantom{0}{-}1.1\phantom{0}$ & $\phantom{0}0.07\phantom{0}$ & $\phantom{0}0.08\phantom{0}$ & $\phantom{0}0.08\phantom{0}$ & $\phantom{0}0.08\phantom{0}$ & $\phantom{0}0.08\phantom{0}$ & $\phantom{0}0.08\phantom{0}$ & $\phantom{0}94.9\phantom{0}$ & $\phantom{0}94.3\phantom{0}$ & $\phantom{0}88.2\phantom{0}$ & $\phantom{0}95.1\phantom{0}$ & $\phantom{0}94.6\phantom{0}$ & $\phantom{0}94.7\phantom{0}$ \\
 & \nopagebreak $\;J=200$  & $\phantom{0}\phantom{-}0.3\phantom{0}$ & $\phantom{0}\phantom{-}0.2\phantom{0}$ & ${-}11.0\phantom{0}$ & $\phantom{0}\phantom{-}0.4\phantom{0}$ & $\phantom{0}\phantom{-}0.3\phantom{0}$ & $\phantom{0}{-}0.5\phantom{0}$ & $\phantom{0}0.05\phantom{0}$ & $\phantom{0}0.06\phantom{0}$ & $\phantom{0}0.06\phantom{0}$ & $\phantom{0}0.06\phantom{0}$ & $\phantom{0}0.06\phantom{0}$ & $\phantom{0}0.06\phantom{0}$ & $\phantom{0}93.6\phantom{0}$ & $\phantom{0}93.4\phantom{0}$ & $\phantom{0}84.7\phantom{0}$ & $\phantom{0}94.3\phantom{0}$ & $\phantom{0}94.0\phantom{0}$ & $\phantom{0}94.0\phantom{0}$ \\
 & \nopagebreak $\;J=500$  & $\phantom{0}\phantom{-}0.1\phantom{0}$ & $\phantom{0}\phantom{-}0.3\phantom{0}$ & ${-}11.0\phantom{0}$ & $\phantom{0}\phantom{-}0.3\phantom{0}$ & $\phantom{0}\phantom{-}0.2\phantom{0}$ & $\phantom{0}{-}0.1\phantom{0}$ & $\phantom{0}0.03\phantom{0}$ & $\phantom{0}0.04\phantom{0}$ & $\phantom{0}0.04\phantom{0}$ & $\phantom{0}0.04\phantom{0}$ & $\phantom{0}0.04\phantom{0}$ & $\phantom{0}0.04\phantom{0}$ & $\phantom{0}94.9\phantom{0}$ & $\phantom{0}95.7\phantom{0}$ & $\phantom{0}78.0\phantom{0}$ & $\phantom{0}95.7\phantom{0}$ & $\phantom{0}95.6\phantom{0}$ & $\phantom{0}95.9\phantom{0}$ \\
 & \nopagebreak $\;J=1000$  & $\phantom{0}{-}0.7\phantom{0}$ & $\phantom{0}{-}0.5\phantom{0}$ & ${-}11.7\phantom{0}$ & $\phantom{0}{-}0.6\phantom{0}$ & $\phantom{0}{-}0.6\phantom{0}$ & $\phantom{0}{-}0.8\phantom{0}$ & $\phantom{0}0.02\phantom{0}$ & $\phantom{0}0.02\phantom{0}$ & $\phantom{0}0.04\phantom{0}$ & $\phantom{0}0.02\phantom{0}$ & $\phantom{0}0.02\phantom{0}$ & $\phantom{0}0.02\phantom{0}$ & $\phantom{0}95.6\phantom{0}$ & $\phantom{0}94.2\phantom{0}$ & $\phantom{0}64.3\phantom{0}$ & $\phantom{0}94.7\phantom{0}$ & $\phantom{0}94.6\phantom{0}$ & $\phantom{0}94.8\phantom{0}$ \\
\multicolumn{4}{l}{$n=20$} \\  & \nopagebreak $\;J=30$  & $\phantom{0}{-}3.9\phantom{0}$ & $\phantom{0}{-}5.2\phantom{0}$ & ${-}17.7\phantom{0}$ & $\phantom{0}{-}4.4\phantom{0}$ & $\phantom{0}{-}4.1\phantom{0}$ & $\phantom{0}{-}8.1\phantom{0}$ & $\phantom{0}0.11\phantom{0}$ & $\phantom{0}0.12\phantom{0}$ & $\phantom{0}0.12\phantom{0}$ & $\phantom{0}0.12\phantom{0}$ & $\phantom{0}0.12\phantom{0}$ & $\phantom{0}0.12\phantom{0}$ & $\phantom{0}89.4\phantom{0}$ & $\phantom{0}88.0\phantom{0}$ & $\phantom{0}80.6\phantom{0}$ & $\phantom{0}90.7\phantom{0}$ & $\phantom{0}91.1\phantom{0}$ & $\phantom{0}89.5\phantom{0}$ \\
 & \nopagebreak $\;J=50$  & $\phantom{0}{-}1.3\phantom{0}$ & $\phantom{0}{-}0.8\phantom{0}$ & ${-}14.1\phantom{0}$ & $\phantom{0}{-}0.6\phantom{0}$ & $\phantom{0}{-}0.8\phantom{0}$ & $\phantom{0}{-}2.9\phantom{0}$ & $\phantom{0}0.09\phantom{0}$ & $\phantom{0}0.10\phantom{0}$ & $\phantom{0}0.10\phantom{0}$ & $\phantom{0}0.10\phantom{0}$ & $\phantom{0}0.10\phantom{0}$ & $\phantom{0}0.10\phantom{0}$ & $\phantom{0}90.7\phantom{0}$ & $\phantom{0}91.9\phantom{0}$ & $\phantom{0}81.6\phantom{0}$ & $\phantom{0}92.5\phantom{0}$ & $\phantom{0}92.3\phantom{0}$ & $\phantom{0}92.3\phantom{0}$ \\
 & \nopagebreak $\;J=100$  & $\phantom{0}{-}0.5\phantom{0}$ & $\phantom{0}{-}0.8\phantom{0}$ & ${-}14.0\phantom{0}$ & $\phantom{0}{-}0.5\phantom{0}$ & $\phantom{0}{-}0.7\phantom{0}$ & $\phantom{0}{-}1.7\phantom{0}$ & $\phantom{0}0.06\phantom{0}$ & $\phantom{0}0.07\phantom{0}$ & $\phantom{0}0.07\phantom{0}$ & $\phantom{0}0.07\phantom{0}$ & $\phantom{0}0.07\phantom{0}$ & $\phantom{0}0.07\phantom{0}$ & $\phantom{0}93.6\phantom{0}$ & $\phantom{0}93.9\phantom{0}$ & $\phantom{0}84.3\phantom{0}$ & $\phantom{0}94.5\phantom{0}$ & $\phantom{0}94.6\phantom{0}$ & $\phantom{0}94.5\phantom{0}$ \\
 & \nopagebreak $\;J=200$  & $\phantom{0}{-}0.4\phantom{0}$ & $\phantom{0}{-}0.1\phantom{0}$ & ${-}13.5\phantom{0}$ & $\phantom{0}{-}0.5\phantom{0}$ & $\phantom{0}{-}0.4\phantom{0}$ & $\phantom{0}{-}0.9\phantom{0}$ & $\phantom{0}0.05\phantom{0}$ & $\phantom{0}0.05\phantom{0}$ & $\phantom{0}0.06\phantom{0}$ & $\phantom{0}0.05\phantom{0}$ & $\phantom{0}0.05\phantom{0}$ & $\phantom{0}0.05\phantom{0}$ & $\phantom{0}93.1\phantom{0}$ & $\phantom{0}93.1\phantom{0}$ & $\phantom{0}79.0\phantom{0}$ & $\phantom{0}93.8\phantom{0}$ & $\phantom{0}92.9\phantom{0}$ & $\phantom{0}94.0\phantom{0}$ \\
 & \nopagebreak $\;J=500$  & $\phantom{0}{-}0.5\phantom{0}$ & $\phantom{0}{-}0.5\phantom{0}$ & ${-}13.7\phantom{0}$ & $\phantom{0}{-}0.5\phantom{0}$ & $\phantom{0}{-}0.4\phantom{0}$ & $\phantom{0}{-}0.6\phantom{0}$ & $\phantom{0}0.03\phantom{0}$ & $\phantom{0}0.03\phantom{0}$ & $\phantom{0}0.05\phantom{0}$ & $\phantom{0}0.03\phantom{0}$ & $\phantom{0}0.03\phantom{0}$ & $\phantom{0}0.03\phantom{0}$ & $\phantom{0}94.7\phantom{0}$ & $\phantom{0}93.9\phantom{0}$ & $\phantom{0}67.9\phantom{0}$ & $\phantom{0}94.8\phantom{0}$ & $\phantom{0}94.8\phantom{0}$ & $\phantom{0}94.6\phantom{0}$ \\
 & \nopagebreak $\;J=1000$  & $\phantom{0}\phantom{-}0.3\phantom{0}$ & $\phantom{0}\phantom{-}0.4\phantom{0}$ & ${-}13.0\phantom{0}$ & $\phantom{0}\phantom{-}0.3\phantom{0}$ & $\phantom{0}\phantom{-}0.3\phantom{0}$ & $\phantom{0}\phantom{-}0.1\phantom{0}$ & $\phantom{0}0.02\phantom{0}$ & $\phantom{0}0.02\phantom{0}$ & $\phantom{0}0.04\phantom{0}$ & $\phantom{0}0.02\phantom{0}$ & $\phantom{0}0.02\phantom{0}$ & $\phantom{0}0.02\phantom{0}$ & $\phantom{0}96.1\phantom{0}$ & $\phantom{0}95.2\phantom{0}$ & $\phantom{0}52.0\phantom{0}$ & $\phantom{0}95.2\phantom{0}$ & $\phantom{0}94.8\phantom{0}$ & $\phantom{0}95.2\phantom{0}$ \\
[0.5ex]\hline\\[-1.6ex] 
\end{tabular}
\begin{tablenotes}[para,flushleft]{\footnotesize \textit{Note.} $n$ = cluster size; $J$ = number of clusters; CD = complete data sets; LD = listwise deletion; FCS-SL = single-level FCS; FCS-MAN = two-level FCS with manifest cluster means; FCS-LAT = two-level FCS with latent cluster means; JM = joint modeling.}\end{tablenotes}
\end{threeparttable}
\end{sidewaystable}
\begin{sidewaystable}
\begin{threeparttable}
\setlength{\tabcolsep}{1.2pt}
\renewcommand{\arraystretch}{0.95}
\footnotesize
\caption{\small Study 1: Bias (in \%), RMSE, and Coverage of the 95\% Confidence Interval for the Covariance of $y$ With $z$ ($\hat\sigma_{yz}$) With 20\% Missing Data (MAR, $\lambda=0.5$)}
\begin{tabular}{llcccccccccccccccccc}
\hline\\[-1.8ex]
& & \multicolumn{6}{c}{Bias (\%)} & \multicolumn{6}{c}{RMSE} & \multicolumn{6}{c}{Coverage (\%)} \\ \cmidrule(r){3-8}\cmidrule(r){9-14}\cmidrule(r){15-20}
 &  & CD & LD & \makecell{FCS-\\SL} & \makecell{FCS-\\MAN} & \makecell{FCS-\\LAT} & JM & CD & LD & \makecell{FCS-\\SL} & \makecell{FCS-\\MAN} & \makecell{FCS-\\LAT} & JM & CD & LD & \makecell{FCS-\\SL} & \makecell{FCS-\\MAN} & \makecell{FCS-\\LAT} & \multicolumn{1}{c}{JM} \\ 
[0.4ex]\hline\\[-1.8ex]
& & \multicolumn{18}{c}{Small intraclass correlation $(\rho_{Iy}=.10)$} \\[0.6ex]\hline\\[-1.8ex]
\multicolumn{4}{l}{$n=5$} \\  & \nopagebreak $\;J=30$  & $\phantom{0}{-}3.9\phantom{0}$ & ${-}16.3\phantom{0}$ & ${-}26.3\phantom{0}$ & $\phantom{0}{-}6.1\phantom{0}$ & $\phantom{0}{-}2.2\phantom{0}$ & ${-}17.5\phantom{0}$ & $\phantom{0}0.10\phantom{0}$ & $\phantom{0}0.11\phantom{0}$ & $\phantom{0}0.10\phantom{0}$ & $\phantom{0}0.12\phantom{0}$ & $\phantom{0}0.12\phantom{0}$ & $\phantom{0}0.11\phantom{0}$ & $\phantom{0}92.2\phantom{0}$ & $\phantom{0}90.2\phantom{0}$ & $\phantom{0}86.3\phantom{0}$ & $\phantom{0}93.6\phantom{0}$ & $\phantom{0}93.2\phantom{0}$ & $\phantom{0}93.7\phantom{0}$ \\
 & \nopagebreak $\;J=50$  & $\phantom{0}{-}1.4\phantom{0}$ & ${-}11.5\phantom{0}$ & ${-}22.0\phantom{0}$ & $\phantom{0}{-}0.9\phantom{0}$ & $\phantom{0}\phantom{-}2.6\phantom{0}$ & ${-}10.8\phantom{0}$ & $\phantom{0}0.08\phantom{0}$ & $\phantom{0}0.08\phantom{0}$ & $\phantom{0}0.08\phantom{0}$ & $\phantom{0}0.09\phantom{0}$ & $\phantom{0}0.09\phantom{0}$ & $\phantom{0}0.08\phantom{0}$ & $\phantom{0}94.1\phantom{0}$ & $\phantom{0}91.5\phantom{0}$ & $\phantom{0}87.0\phantom{0}$ & $\phantom{0}95.3\phantom{0}$ & $\phantom{0}94.8\phantom{0}$ & $\phantom{0}94.1\phantom{0}$ \\
 & \nopagebreak $\;J=100$  & $\phantom{0}{-}3.0\phantom{0}$ & ${-}13.5\phantom{0}$ & ${-}23.6\phantom{0}$ & $\phantom{0}{-}3.0\phantom{0}$ & $\phantom{0}{-}0.2\phantom{0}$ & $\phantom{0}{-}9.4\phantom{0}$ & $\phantom{0}0.06\phantom{0}$ & $\phantom{0}0.06\phantom{0}$ & $\phantom{0}0.06\phantom{0}$ & $\phantom{0}0.07\phantom{0}$ & $\phantom{0}0.07\phantom{0}$ & $\phantom{0}0.06\phantom{0}$ & $\phantom{0}93.8\phantom{0}$ & $\phantom{0}91.2\phantom{0}$ & $\phantom{0}84.5\phantom{0}$ & $\phantom{0}93.5\phantom{0}$ & $\phantom{0}94.1\phantom{0}$ & $\phantom{0}94.2\phantom{0}$ \\
 & \nopagebreak $\;J=200$  & $\phantom{0}{-}0.1\phantom{0}$ & ${-}10.9\phantom{0}$ & ${-}21.2\phantom{0}$ & $\phantom{0}{-}0.2\phantom{0}$ & $\phantom{0}\phantom{-}1.6\phantom{0}$ & $\phantom{0}{-}4.5\phantom{0}$ & $\phantom{0}0.04\phantom{0}$ & $\phantom{0}0.04\phantom{0}$ & $\phantom{0}0.05\phantom{0}$ & $\phantom{0}0.04\phantom{0}$ & $\phantom{0}0.04\phantom{0}$ & $\phantom{0}0.04\phantom{0}$ & $\phantom{0}95.4\phantom{0}$ & $\phantom{0}92.2\phantom{0}$ & $\phantom{0}82.3\phantom{0}$ & $\phantom{0}96.0\phantom{0}$ & $\phantom{0}95.4\phantom{0}$ & $\phantom{0}95.6\phantom{0}$ \\
 & \nopagebreak $\;J=500$  & $\phantom{0}{-}0.7\phantom{0}$ & ${-}10.9\phantom{0}$ & ${-}21.5\phantom{0}$ & $\phantom{0}{-}0.8\phantom{0}$ & $\phantom{0}{-}0.2\phantom{0}$ & $\phantom{0}{-}2.8\phantom{0}$ & $\phantom{0}0.03\phantom{0}$ & $\phantom{0}0.03\phantom{0}$ & $\phantom{0}0.04\phantom{0}$ & $\phantom{0}0.03\phantom{0}$ & $\phantom{0}0.03\phantom{0}$ & $\phantom{0}0.03\phantom{0}$ & $\phantom{0}94.2\phantom{0}$ & $\phantom{0}86.6\phantom{0}$ & $\phantom{0}67.0\phantom{0}$ & $\phantom{0}94.7\phantom{0}$ & $\phantom{0}94.2\phantom{0}$ & $\phantom{0}94.6\phantom{0}$ \\
 & \nopagebreak $\;J=1000$  & $\phantom{0}{-}0.3\phantom{0}$ & ${-}10.5\phantom{0}$ & ${-}21.1\phantom{0}$ & $\phantom{0}{-}0.0\phantom{0}$ & $\phantom{0}\phantom{-}0.2\phantom{0}$ & $\phantom{0}{-}1.2\phantom{0}$ & $\phantom{0}0.02\phantom{0}$ & $\phantom{0}0.02\phantom{0}$ & $\phantom{0}0.04\phantom{0}$ & $\phantom{0}0.02\phantom{0}$ & $\phantom{0}0.02\phantom{0}$ & $\phantom{0}0.02\phantom{0}$ & $\phantom{0}95.4\phantom{0}$ & $\phantom{0}84.0\phantom{0}$ & $\phantom{0}46.2\phantom{0}$ & $\phantom{0}95.7\phantom{0}$ & $\phantom{0}95.5\phantom{0}$ & $\phantom{0}95.6\phantom{0}$ \\
\multicolumn{4}{l}{$n=20$} \\  & \nopagebreak $\;J=30$  & $\phantom{0}{-}2.6\phantom{0}$ & ${-}13.3\phantom{0}$ & ${-}26.9\phantom{0}$ & $\phantom{0}{-}1.9\phantom{0}$ & $\phantom{0}{-}0.5\phantom{0}$ & ${-}14.1\phantom{0}$ & $\phantom{0}0.07\phantom{0}$ & $\phantom{0}0.08\phantom{0}$ & $\phantom{0}0.08\phantom{0}$ & $\phantom{0}0.09\phantom{0}$ & $\phantom{0}0.09\phantom{0}$ & $\phantom{0}0.08\phantom{0}$ & $\phantom{0}90.5\phantom{0}$ & $\phantom{0}86.1\phantom{0}$ & $\phantom{0}77.4\phantom{0}$ & $\phantom{0}92.8\phantom{0}$ & $\phantom{0}92.5\phantom{0}$ & $\phantom{0}91.9\phantom{0}$ \\
 & \nopagebreak $\;J=50$  & $\phantom{0}\phantom{-}0.3\phantom{0}$ & ${-}10.9\phantom{0}$ & ${-}25.0\phantom{0}$ & $\phantom{0}{-}0.3\phantom{0}$ & $\phantom{0}\phantom{-}0.8\phantom{0}$ & $\phantom{0}{-}8.8\phantom{0}$ & $\phantom{0}0.06\phantom{0}$ & $\phantom{0}0.06\phantom{0}$ & $\phantom{0}0.06\phantom{0}$ & $\phantom{0}0.07\phantom{0}$ & $\phantom{0}0.07\phantom{0}$ & $\phantom{0}0.06\phantom{0}$ & $\phantom{0}93.3\phantom{0}$ & $\phantom{0}89.1\phantom{0}$ & $\phantom{0}77.8\phantom{0}$ & $\phantom{0}94.1\phantom{0}$ & $\phantom{0}93.7\phantom{0}$ & $\phantom{0}92.9\phantom{0}$ \\
 & \nopagebreak $\;J=100$  & $\phantom{0}{-}2.1\phantom{0}$ & ${-}12.3\phantom{0}$ & ${-}26.1\phantom{0}$ & $\phantom{0}{-}2.0\phantom{0}$ & $\phantom{0}{-}1.7\phantom{0}$ & $\phantom{0}{-}6.8\phantom{0}$ & $\phantom{0}0.04\phantom{0}$ & $\phantom{0}0.05\phantom{0}$ & $\phantom{0}0.06\phantom{0}$ & $\phantom{0}0.05\phantom{0}$ & $\phantom{0}0.05\phantom{0}$ & $\phantom{0}0.05\phantom{0}$ & $\phantom{0}92.7\phantom{0}$ & $\phantom{0}88.4\phantom{0}$ & $\phantom{0}71.3\phantom{0}$ & $\phantom{0}92.9\phantom{0}$ & $\phantom{0}93.3\phantom{0}$ & $\phantom{0}93.1\phantom{0}$ \\
 & \nopagebreak $\;J=200$  & $\phantom{0}{-}0.6\phantom{0}$ & ${-}11.3\phantom{0}$ & ${-}25.4\phantom{0}$ & $\phantom{0}{-}1.2\phantom{0}$ & $\phantom{0}{-}0.8\phantom{0}$ & $\phantom{0}{-}3.8\phantom{0}$ & $\phantom{0}0.03\phantom{0}$ & $\phantom{0}0.04\phantom{0}$ & $\phantom{0}0.05\phantom{0}$ & $\phantom{0}0.03\phantom{0}$ & $\phantom{0}0.03\phantom{0}$ & $\phantom{0}0.03\phantom{0}$ & $\phantom{0}93.3\phantom{0}$ & $\phantom{0}87.3\phantom{0}$ & $\phantom{0}58.8\phantom{0}$ & $\phantom{0}94.5\phantom{0}$ & $\phantom{0}94.2\phantom{0}$ & $\phantom{0}94.8\phantom{0}$ \\
 & \nopagebreak $\;J=500$  & $\phantom{0}\phantom{-}0.2\phantom{0}$ & ${-}10.1\phantom{0}$ & ${-}24.4\phantom{0}$ & $\phantom{0}\phantom{-}0.4\phantom{0}$ & $\phantom{0}\phantom{-}0.5\phantom{0}$ & $\phantom{0}{-}0.8\phantom{0}$ & $\phantom{0}0.02\phantom{0}$ & $\phantom{0}0.02\phantom{0}$ & $\phantom{0}0.04\phantom{0}$ & $\phantom{0}0.02\phantom{0}$ & $\phantom{0}0.02\phantom{0}$ & $\phantom{0}0.02\phantom{0}$ & $\phantom{0}95.3\phantom{0}$ & $\phantom{0}85.7\phantom{0}$ & $\phantom{0}33.6\phantom{0}$ & $\phantom{0}95.9\phantom{0}$ & $\phantom{0}96.1\phantom{0}$ & $\phantom{0}95.8\phantom{0}$ \\
 & \nopagebreak $\;J=1000$  & $\phantom{0}\phantom{-}0.3\phantom{0}$ & ${-}10.5\phantom{0}$ & ${-}24.6\phantom{0}$ & $\phantom{0}{-}0.0\phantom{0}$ & $\phantom{0}\phantom{-}0.1\phantom{0}$ & $\phantom{0}{-}0.7\phantom{0}$ & $\phantom{0}0.01\phantom{0}$ & $\phantom{0}0.02\phantom{0}$ & $\phantom{0}0.04\phantom{0}$ & $\phantom{0}0.02\phantom{0}$ & $\phantom{0}0.02\phantom{0}$ & $\phantom{0}0.02\phantom{0}$ & $\phantom{0}93.2\phantom{0}$ & $\phantom{0}75.6\phantom{0}$ & $\phantom{0}\phantom{0}8.7\phantom{0}$ & $\phantom{0}95.1\phantom{0}$ & $\phantom{0}94.1\phantom{0}$ & $\phantom{0}94.7\phantom{0}$ \\
[0.5ex]\hline\\[-1.6ex] 
& & \multicolumn{18}{c}{Moderate intraclass correlation $(\rho_{Iy}=.30)$} \\[0.6ex]\hline\\[-1.8ex]
\multicolumn{4}{l}{$n=5$} \\  & \nopagebreak $\;J=30$  & $\phantom{0}{-}4.5\phantom{0}$ & ${-}15.0\phantom{0}$ & ${-}21.4\phantom{0}$ & $\phantom{0}{-}4.5\phantom{0}$ & $\phantom{0}{-}2.2\phantom{0}$ & ${-}11.1\phantom{0}$ & $\phantom{0}0.13\phantom{0}$ & $\phantom{0}0.14\phantom{0}$ & $\phantom{0}0.13\phantom{0}$ & $\phantom{0}0.15\phantom{0}$ & $\phantom{0}0.15\phantom{0}$ & $\phantom{0}0.14\phantom{0}$ & $\phantom{0}90.1\phantom{0}$ & $\phantom{0}85.6\phantom{0}$ & $\phantom{0}82.3\phantom{0}$ & $\phantom{0}92.7\phantom{0}$ & $\phantom{0}93.1\phantom{0}$ & $\phantom{0}91.7\phantom{0}$ \\
 & \nopagebreak $\;J=50$  & $\phantom{0}{-}2.9\phantom{0}$ & ${-}13.1\phantom{0}$ & ${-}19.2\phantom{0}$ & $\phantom{0}{-}2.7\phantom{0}$ & $\phantom{0}{-}1.8\phantom{0}$ & $\phantom{0}{-}7.2\phantom{0}$ & $\phantom{0}0.10\phantom{0}$ & $\phantom{0}0.11\phantom{0}$ & $\phantom{0}0.11\phantom{0}$ & $\phantom{0}0.11\phantom{0}$ & $\phantom{0}0.12\phantom{0}$ & $\phantom{0}0.11\phantom{0}$ & $\phantom{0}91.7\phantom{0}$ & $\phantom{0}87.5\phantom{0}$ & $\phantom{0}83.6\phantom{0}$ & $\phantom{0}93.1\phantom{0}$ & $\phantom{0}93.1\phantom{0}$ & $\phantom{0}92.2\phantom{0}$ \\
 & \nopagebreak $\;J=100$  & $\phantom{0}{-}1.8\phantom{0}$ & ${-}13.2\phantom{0}$ & ${-}19.3\phantom{0}$ & $\phantom{0}{-}3.0\phantom{0}$ & $\phantom{0}{-}2.4\phantom{0}$ & $\phantom{0}{-}5.0\phantom{0}$ & $\phantom{0}0.07\phantom{0}$ & $\phantom{0}0.08\phantom{0}$ & $\phantom{0}0.09\phantom{0}$ & $\phantom{0}0.08\phantom{0}$ & $\phantom{0}0.08\phantom{0}$ & $\phantom{0}0.08\phantom{0}$ & $\phantom{0}92.3\phantom{0}$ & $\phantom{0}88.0\phantom{0}$ & $\phantom{0}80.5\phantom{0}$ & $\phantom{0}93.1\phantom{0}$ & $\phantom{0}93.1\phantom{0}$ & $\phantom{0}93.3\phantom{0}$ \\
 & \nopagebreak $\;J=200$  & $\phantom{0}{-}0.2\phantom{0}$ & ${-}10.8\phantom{0}$ & ${-}17.1\phantom{0}$ & $\phantom{0}{-}0.2\phantom{0}$ & $\phantom{0}{-}0.0\phantom{0}$ & $\phantom{0}{-}1.2\phantom{0}$ & $\phantom{0}0.05\phantom{0}$ & $\phantom{0}0.06\phantom{0}$ & $\phantom{0}0.07\phantom{0}$ & $\phantom{0}0.06\phantom{0}$ & $\phantom{0}0.06\phantom{0}$ & $\phantom{0}0.06\phantom{0}$ & $\phantom{0}93.7\phantom{0}$ & $\phantom{0}88.1\phantom{0}$ & $\phantom{0}76.5\phantom{0}$ & $\phantom{0}92.8\phantom{0}$ & $\phantom{0}93.2\phantom{0}$ & $\phantom{0}92.6\phantom{0}$ \\
 & \nopagebreak $\;J=500$  & $\phantom{0}{-}0.7\phantom{0}$ & ${-}11.2\phantom{0}$ & ${-}17.3\phantom{0}$ & $\phantom{0}{-}1.0\phantom{0}$ & $\phantom{0}{-}0.9\phantom{0}$ & $\phantom{0}{-}1.3\phantom{0}$ & $\phantom{0}0.03\phantom{0}$ & $\phantom{0}0.05\phantom{0}$ & $\phantom{0}0.06\phantom{0}$ & $\phantom{0}0.04\phantom{0}$ & $\phantom{0}0.04\phantom{0}$ & $\phantom{0}0.04\phantom{0}$ & $\phantom{0}93.1\phantom{0}$ & $\phantom{0}83.0\phantom{0}$ & $\phantom{0}60.2\phantom{0}$ & $\phantom{0}94.0\phantom{0}$ & $\phantom{0}94.6\phantom{0}$ & $\phantom{0}94.6\phantom{0}$ \\
 & \nopagebreak $\;J=1000$  & $\phantom{0}\phantom{-}0.3\phantom{0}$ & ${-}10.1\phantom{0}$ & ${-}16.3\phantom{0}$ & $\phantom{0}\phantom{-}0.3\phantom{0}$ & $\phantom{0}\phantom{-}0.4\phantom{0}$ & $\phantom{0}\phantom{-}0.0\phantom{0}$ & $\phantom{0}0.02\phantom{0}$ & $\phantom{0}0.04\phantom{0}$ & $\phantom{0}0.05\phantom{0}$ & $\phantom{0}0.03\phantom{0}$ & $\phantom{0}0.03\phantom{0}$ & $\phantom{0}0.03\phantom{0}$ & $\phantom{0}95.3\phantom{0}$ & $\phantom{0}78.6\phantom{0}$ & $\phantom{0}42.8\phantom{0}$ & $\phantom{0}94.9\phantom{0}$ & $\phantom{0}95.2\phantom{0}$ & $\phantom{0}94.3\phantom{0}$ \\
\multicolumn{4}{l}{$n=20$} \\  & \nopagebreak $\;J=30$  & $\phantom{0}{-}5.2\phantom{0}$ & ${-}15.3\phantom{0}$ & ${-}24.3\phantom{0}$ & $\phantom{0}{-}5.0\phantom{0}$ & $\phantom{0}{-}5.4\phantom{0}$ & ${-}10.3\phantom{0}$ & $\phantom{0}0.12\phantom{0}$ & $\phantom{0}0.13\phantom{0}$ & $\phantom{0}0.13\phantom{0}$ & $\phantom{0}0.13\phantom{0}$ & $\phantom{0}0.14\phantom{0}$ & $\phantom{0}0.13\phantom{0}$ & $\phantom{0}87.8\phantom{0}$ & $\phantom{0}82.6\phantom{0}$ & $\phantom{0}74.0\phantom{0}$ & $\phantom{0}90.2\phantom{0}$ & $\phantom{0}89.6\phantom{0}$ & $\phantom{0}88.2\phantom{0}$ \\
 & \nopagebreak $\;J=50$  & $\phantom{0}{-}2.3\phantom{0}$ & ${-}13.1\phantom{0}$ & ${-}21.8\phantom{0}$ & $\phantom{0}{-}2.5\phantom{0}$ & $\phantom{0}{-}2.5\phantom{0}$ & $\phantom{0}{-}6.1\phantom{0}$ & $\phantom{0}0.09\phantom{0}$ & $\phantom{0}0.10\phantom{0}$ & $\phantom{0}0.10\phantom{0}$ & $\phantom{0}0.10\phantom{0}$ & $\phantom{0}0.10\phantom{0}$ & $\phantom{0}0.10\phantom{0}$ & $\phantom{0}92.0\phantom{0}$ & $\phantom{0}86.5\phantom{0}$ & $\phantom{0}77.0\phantom{0}$ & $\phantom{0}93.3\phantom{0}$ & $\phantom{0}93.1\phantom{0}$ & $\phantom{0}92.1\phantom{0}$ \\
 & \nopagebreak $\;J=100$  & $\phantom{0}{-}1.2\phantom{0}$ & ${-}11.7\phantom{0}$ & ${-}20.8\phantom{0}$ & $\phantom{0}{-}1.4\phantom{0}$ & $\phantom{0}{-}1.3\phantom{0}$ & $\phantom{0}{-}3.1\phantom{0}$ & $\phantom{0}0.06\phantom{0}$ & $\phantom{0}0.07\phantom{0}$ & $\phantom{0}0.08\phantom{0}$ & $\phantom{0}0.07\phantom{0}$ & $\phantom{0}0.07\phantom{0}$ & $\phantom{0}0.07\phantom{0}$ & $\phantom{0}92.6\phantom{0}$ & $\phantom{0}87.0\phantom{0}$ & $\phantom{0}73.3\phantom{0}$ & $\phantom{0}93.9\phantom{0}$ & $\phantom{0}94.1\phantom{0}$ & $\phantom{0}93.0\phantom{0}$ \\
 & \nopagebreak $\;J=200$  & $\phantom{0}{-}0.3\phantom{0}$ & ${-}10.9\phantom{0}$ & ${-}19.9\phantom{0}$ & $\phantom{0}{-}0.2\phantom{0}$ & $\phantom{0}{-}0.1\phantom{0}$ & $\phantom{0}{-}1.0\phantom{0}$ & $\phantom{0}0.04\phantom{0}$ & $\phantom{0}0.05\phantom{0}$ & $\phantom{0}0.07\phantom{0}$ & $\phantom{0}0.05\phantom{0}$ & $\phantom{0}0.05\phantom{0}$ & $\phantom{0}0.05\phantom{0}$ & $\phantom{0}96.2\phantom{0}$ & $\phantom{0}87.7\phantom{0}$ & $\phantom{0}68.3\phantom{0}$ & $\phantom{0}95.4\phantom{0}$ & $\phantom{0}95.2\phantom{0}$ & $\phantom{0}96.3\phantom{0}$ \\
 & \nopagebreak $\;J=500$  & $\phantom{0}{-}0.3\phantom{0}$ & ${-}10.4\phantom{0}$ & ${-}19.5\phantom{0}$ & $\phantom{0}\phantom{-}0.1\phantom{0}$ & $\phantom{0}\phantom{-}0.1\phantom{0}$ & $\phantom{0}{-}0.3\phantom{0}$ & $\phantom{0}0.03\phantom{0}$ & $\phantom{0}0.04\phantom{0}$ & $\phantom{0}0.06\phantom{0}$ & $\phantom{0}0.03\phantom{0}$ & $\phantom{0}0.03\phantom{0}$ & $\phantom{0}0.03\phantom{0}$ & $\phantom{0}95.1\phantom{0}$ & $\phantom{0}82.0\phantom{0}$ & $\phantom{0}43.6\phantom{0}$ & $\phantom{0}96.1\phantom{0}$ & $\phantom{0}96.2\phantom{0}$ & $\phantom{0}95.8\phantom{0}$ \\
 & \nopagebreak $\;J=1000$  & $\phantom{0}{-}0.1\phantom{0}$ & ${-}10.4\phantom{0}$ & ${-}19.5\phantom{0}$ & $\phantom{0}\phantom{-}0.1\phantom{0}$ & $\phantom{0}\phantom{-}0.2\phantom{0}$ & $\phantom{0}{-}0.0\phantom{0}$ & $\phantom{0}0.02\phantom{0}$ & $\phantom{0}0.04\phantom{0}$ & $\phantom{0}0.06\phantom{0}$ & $\phantom{0}0.02\phantom{0}$ & $\phantom{0}0.02\phantom{0}$ & $\phantom{0}0.02\phantom{0}$ & $\phantom{0}94.7\phantom{0}$ & $\phantom{0}72.2\phantom{0}$ & $\phantom{0}15.5\phantom{0}$ & $\phantom{0}94.7\phantom{0}$ & $\phantom{0}94.4\phantom{0}$ & $\phantom{0}94.7\phantom{0}$ \\
[0.5ex]\hline\\[-1.6ex] 
\end{tabular}
\begin{tablenotes}[para,flushleft]{\footnotesize \textit{Note.} $n$ = cluster size; $J$ = number of clusters; CD = complete data sets; LD = listwise deletion; FCS-SL = single-level FCS; FCS-MAN = two-level FCS with manifest cluster means; FCS-LAT = two-level FCS with latent cluster means; JM = joint modeling.}\end{tablenotes}
\end{threeparttable}
\end{sidewaystable}
\begin{sidewaystable}
\begin{threeparttable}
\setlength{\tabcolsep}{1.2pt}
\renewcommand{\arraystretch}{0.95}
\footnotesize
\caption{\small Study 1: Bias (in \%), RMSE, and Coverage of the 95\% Confidence Interval for the Covariance of $y$ With $z$ ($\hat\sigma_{yz}$) With 20\% Missing Data (MAR, $\lambda=1$)}
\begin{tabular}{llcccccccccccccccccc}
\hline\\[-1.8ex]
& & \multicolumn{6}{c}{Bias (\%)} & \multicolumn{6}{c}{RMSE} & \multicolumn{6}{c}{Coverage (\%)} \\ \cmidrule(r){3-8}\cmidrule(r){9-14}\cmidrule(r){15-20}
 &  & CD & LD & \makecell{FCS-\\SL} & \makecell{FCS-\\MAN} & \makecell{FCS-\\LAT} & JM & CD & LD & \makecell{FCS-\\SL} & \makecell{FCS-\\MAN} & \makecell{FCS-\\LAT} & JM & CD & LD & \makecell{FCS-\\SL} & \makecell{FCS-\\MAN} & \makecell{FCS-\\LAT} & \multicolumn{1}{c}{JM} \\ 
[0.4ex]\hline\\[-1.8ex]
& & \multicolumn{18}{c}{Small intraclass correlation $(\rho_{Iy}=.10)$} \\[0.6ex]\hline\\[-1.8ex]
\multicolumn{4}{l}{$n=5$} \\  & \nopagebreak $\;J=30$  & $\phantom{0}{-}4.0\phantom{0}$ & ${-}45.5\phantom{0}$ & ${-}47.2\phantom{0}$ & $\phantom{0}{-}5.0\phantom{0}$ & $\phantom{0}\phantom{-}2.1\phantom{0}$ & ${-}32.4\phantom{0}$ & $\phantom{0}0.10\phantom{0}$ & $\phantom{0}0.11\phantom{0}$ & $\phantom{0}0.11\phantom{0}$ & $\phantom{0}0.15\phantom{0}$ & $\phantom{0}0.16\phantom{0}$ & $\phantom{0}0.12\phantom{0}$ & $\phantom{0}91.9\phantom{0}$ & $\phantom{0}86.2\phantom{0}$ & $\phantom{0}78.6\phantom{0}$ & $\phantom{0}94.8\phantom{0}$ & $\phantom{0}92.8\phantom{0}$ & $\phantom{0}93.9\phantom{0}$ \\
 & \nopagebreak $\;J=50$  & $\phantom{0}{-}2.7\phantom{0}$ & ${-}43.4\phantom{0}$ & ${-}45.0\phantom{0}$ & $\phantom{0}{-}2.3\phantom{0}$ & $\phantom{0}\phantom{-}7.2\phantom{0}$ & ${-}25.0\phantom{0}$ & $\phantom{0}0.08\phantom{0}$ & $\phantom{0}0.09\phantom{0}$ & $\phantom{0}0.09\phantom{0}$ & $\phantom{0}0.11\phantom{0}$ & $\phantom{0}0.12\phantom{0}$ & $\phantom{0}0.10\phantom{0}$ & $\phantom{0}92.5\phantom{0}$ & $\phantom{0}83.5\phantom{0}$ & $\phantom{0}76.0\phantom{0}$ & $\phantom{0}94.3\phantom{0}$ & $\phantom{0}93.7\phantom{0}$ & $\phantom{0}94.0\phantom{0}$ \\
 & \nopagebreak $\;J=100$  & $\phantom{0}{-}1.5\phantom{0}$ & ${-}42.4\phantom{0}$ & ${-}44.0\phantom{0}$ & $\phantom{0}{-}1.3\phantom{0}$ & $\phantom{0}\phantom{-}6.3\phantom{0}$ & ${-}17.6\phantom{0}$ & $\phantom{0}0.05\phantom{0}$ & $\phantom{0}0.08\phantom{0}$ & $\phantom{0}0.08\phantom{0}$ & $\phantom{0}0.08\phantom{0}$ & $\phantom{0}0.08\phantom{0}$ & $\phantom{0}0.07\phantom{0}$ & $\phantom{0}94.3\phantom{0}$ & $\phantom{0}75.0\phantom{0}$ & $\phantom{0}65.9\phantom{0}$ & $\phantom{0}94.0\phantom{0}$ & $\phantom{0}93.7\phantom{0}$ & $\phantom{0}93.7\phantom{0}$ \\
 & \nopagebreak $\;J=200$  & $\phantom{0}{-}0.1\phantom{0}$ & ${-}42.1\phantom{0}$ & ${-}43.8\phantom{0}$ & $\phantom{0}{-}0.4\phantom{0}$ & $\phantom{0}\phantom{-}4.0\phantom{0}$ & ${-}11.6\phantom{0}$ & $\phantom{0}0.04\phantom{0}$ & $\phantom{0}0.07\phantom{0}$ & $\phantom{0}0.08\phantom{0}$ & $\phantom{0}0.05\phantom{0}$ & $\phantom{0}0.06\phantom{0}$ & $\phantom{0}0.05\phantom{0}$ & $\phantom{0}95.6\phantom{0}$ & $\phantom{0}55.7\phantom{0}$ & $\phantom{0}44.4\phantom{0}$ & $\phantom{0}94.9\phantom{0}$ & $\phantom{0}94.0\phantom{0}$ & $\phantom{0}94.5\phantom{0}$ \\
 & \nopagebreak $\;J=500$  & $\phantom{0}\phantom{-}0.7\phantom{0}$ & ${-}41.1\phantom{0}$ & ${-}43.0\phantom{0}$ & $\phantom{0}\phantom{-}0.5\phantom{0}$ & $\phantom{0}\phantom{-}2.3\phantom{0}$ & $\phantom{0}{-}5.1\phantom{0}$ & $\phantom{0}0.02\phantom{0}$ & $\phantom{0}0.07\phantom{0}$ & $\phantom{0}0.07\phantom{0}$ & $\phantom{0}0.03\phantom{0}$ & $\phantom{0}0.03\phantom{0}$ & $\phantom{0}0.03\phantom{0}$ & $\phantom{0}95.4\phantom{0}$ & $\phantom{0}18.5\phantom{0}$ & $\phantom{0}10.7\phantom{0}$ & $\phantom{0}95.5\phantom{0}$ & $\phantom{0}94.3\phantom{0}$ & $\phantom{0}94.7\phantom{0}$ \\
 & \nopagebreak $\;J=1000$  & $\phantom{0}{-}0.3\phantom{0}$ & ${-}41.8\phantom{0}$ & ${-}43.7\phantom{0}$ & $\phantom{0}{-}0.4\phantom{0}$ & $\phantom{0}\phantom{-}0.4\phantom{0}$ & $\phantom{0}{-}3.7\phantom{0}$ & $\phantom{0}0.02\phantom{0}$ & $\phantom{0}0.07\phantom{0}$ & $\phantom{0}0.07\phantom{0}$ & $\phantom{0}0.03\phantom{0}$ & $\phantom{0}0.03\phantom{0}$ & $\phantom{0}0.03\phantom{0}$ & $\phantom{0}94.4\phantom{0}$ & $\phantom{0}\phantom{0}2.4\phantom{0}$ & $\phantom{0}\phantom{0}1.1\phantom{0}$ & $\phantom{0}92.8\phantom{0}$ & $\phantom{0}92.1\phantom{0}$ & $\phantom{0}93.1\phantom{0}$ \\
\multicolumn{4}{l}{$n=20$} \\  & \nopagebreak $\;J=30$  & $\phantom{0}{-}2.3\phantom{0}$ & ${-}42.5\phantom{0}$ & ${-}49.3\phantom{0}$ & $\phantom{0}{-}1.6\phantom{0}$ & $\phantom{0}\phantom{-}1.6\phantom{0}$ & ${-}28.3\phantom{0}$ & $\phantom{0}0.08\phantom{0}$ & $\phantom{0}0.09\phantom{0}$ & $\phantom{0}0.10\phantom{0}$ & $\phantom{0}0.11\phantom{0}$ & $\phantom{0}0.11\phantom{0}$ & $\phantom{0}0.09\phantom{0}$ & $\phantom{0}91.2\phantom{0}$ & $\phantom{0}65.3\phantom{0}$ & $\phantom{0}57.4\phantom{0}$ & $\phantom{0}93.7\phantom{0}$ & $\phantom{0}92.3\phantom{0}$ & $\phantom{0}91.5\phantom{0}$ \\
 & \nopagebreak $\;J=50$  & $\phantom{0}{-}1.1\phantom{0}$ & ${-}42.0\phantom{0}$ & ${-}48.8\phantom{0}$ & $\phantom{0}\phantom{-}0.3\phantom{0}$ & $\phantom{0}\phantom{-}3.1\phantom{0}$ & ${-}20.3\phantom{0}$ & $\phantom{0}0.06\phantom{0}$ & $\phantom{0}0.08\phantom{0}$ & $\phantom{0}0.09\phantom{0}$ & $\phantom{0}0.08\phantom{0}$ & $\phantom{0}0.09\phantom{0}$ & $\phantom{0}0.07\phantom{0}$ & $\phantom{0}91.2\phantom{0}$ & $\phantom{0}59.0\phantom{0}$ & $\phantom{0}48.8\phantom{0}$ & $\phantom{0}93.3\phantom{0}$ & $\phantom{0}92.6\phantom{0}$ & $\phantom{0}90.3\phantom{0}$ \\
 & \nopagebreak $\;J=100$  & $\phantom{0}{-}0.2\phantom{0}$ & ${-}41.6\phantom{0}$ & ${-}48.5\phantom{0}$ & $\phantom{0}\phantom{-}0.5\phantom{0}$ & $\phantom{0}\phantom{-}1.2\phantom{0}$ & ${-}12.4\phantom{0}$ & $\phantom{0}0.04\phantom{0}$ & $\phantom{0}0.07\phantom{0}$ & $\phantom{0}0.08\phantom{0}$ & $\phantom{0}0.06\phantom{0}$ & $\phantom{0}0.06\phantom{0}$ & $\phantom{0}0.05\phantom{0}$ & $\phantom{0}94.3\phantom{0}$ & $\phantom{0}44.7\phantom{0}$ & $\phantom{0}29.9\phantom{0}$ & $\phantom{0}93.8\phantom{0}$ & $\phantom{0}92.9\phantom{0}$ & $\phantom{0}93.2\phantom{0}$ \\
 & \nopagebreak $\;J=200$  & $\phantom{0}{-}0.5\phantom{0}$ & ${-}42.6\phantom{0}$ & ${-}49.4\phantom{0}$ & $\phantom{0}{-}1.0\phantom{0}$ & $\phantom{0}{-}0.5\phantom{0}$ & $\phantom{0}{-}8.9\phantom{0}$ & $\phantom{0}0.03\phantom{0}$ & $\phantom{0}0.07\phantom{0}$ & $\phantom{0}0.08\phantom{0}$ & $\phantom{0}0.04\phantom{0}$ & $\phantom{0}0.04\phantom{0}$ & $\phantom{0}0.04\phantom{0}$ & $\phantom{0}93.3\phantom{0}$ & $\phantom{0}17.9\phantom{0}$ & $\phantom{0}\phantom{0}6.2\phantom{0}$ & $\phantom{0}93.9\phantom{0}$ & $\phantom{0}94.4\phantom{0}$ & $\phantom{0}93.0\phantom{0}$ \\
 & \nopagebreak $\;J=500$  & $\phantom{0}{-}0.8\phantom{0}$ & ${-}42.2\phantom{0}$ & ${-}49.0\phantom{0}$ & $\phantom{0}{-}0.8\phantom{0}$ & $\phantom{0}{-}0.8\phantom{0}$ & $\phantom{0}{-}4.4\phantom{0}$ & $\phantom{0}0.02\phantom{0}$ & $\phantom{0}0.07\phantom{0}$ & $\phantom{0}0.08\phantom{0}$ & $\phantom{0}0.02\phantom{0}$ & $\phantom{0}0.03\phantom{0}$ & $\phantom{0}0.02\phantom{0}$ & $\phantom{0}94.4\phantom{0}$ & $\phantom{0}\phantom{0}1.2\phantom{0}$ & $\phantom{0}\phantom{0}0.0\phantom{0}$ & $\phantom{0}95.8\phantom{0}$ & $\phantom{0}94.5\phantom{0}$ & $\phantom{0}93.5\phantom{0}$ \\
 & \nopagebreak $\;J=1000$  & $\phantom{0}{-}0.2\phantom{0}$ & ${-}41.8\phantom{0}$ & ${-}48.7\phantom{0}$ & $\phantom{0}{-}0.3\phantom{0}$ & $\phantom{0}{-}0.2\phantom{0}$ & $\phantom{0}{-}2.1\phantom{0}$ & $\phantom{0}0.01\phantom{0}$ & $\phantom{0}0.07\phantom{0}$ & $\phantom{0}0.08\phantom{0}$ & $\phantom{0}0.02\phantom{0}$ & $\phantom{0}0.02\phantom{0}$ & $\phantom{0}0.02\phantom{0}$ & $\phantom{0}95.4\phantom{0}$ & $\phantom{0}\phantom{0}0.0\phantom{0}$ & $\phantom{0}\phantom{0}0.0\phantom{0}$ & $\phantom{0}93.5\phantom{0}$ & $\phantom{0}93.5\phantom{0}$ & $\phantom{0}94.7\phantom{0}$ \\
[0.5ex]\hline\\[-1.6ex] 
& & \multicolumn{18}{c}{Moderate intraclass correlation $(\rho_{Iy}=.30)$} \\[0.6ex]\hline\\[-1.8ex]
\multicolumn{4}{l}{$n=5$} \\  & \nopagebreak $\;J=30$  & $\phantom{0}{-}3.1\phantom{0}$ & ${-}44.0\phantom{0}$ & ${-}39.4\phantom{0}$ & $\phantom{0}{-}3.8\phantom{0}$ & $\phantom{0}\phantom{-}2.2\phantom{0}$ & ${-}19.6\phantom{0}$ & $\phantom{0}0.13\phantom{0}$ & $\phantom{0}0.16\phantom{0}$ & $\phantom{0}0.16\phantom{0}$ & $\phantom{0}0.18\phantom{0}$ & $\phantom{0}0.19\phantom{0}$ & $\phantom{0}0.16\phantom{0}$ & $\phantom{0}89.7\phantom{0}$ & $\phantom{0}64.6\phantom{0}$ & $\phantom{0}70.2\phantom{0}$ & $\phantom{0}93.5\phantom{0}$ & $\phantom{0}92.7\phantom{0}$ & $\phantom{0}90.7\phantom{0}$ \\
 & \nopagebreak $\;J=50$  & $\phantom{0}{-}2.4\phantom{0}$ & ${-}43.1\phantom{0}$ & ${-}38.5\phantom{0}$ & $\phantom{0}{-}2.4\phantom{0}$ & $\phantom{0}\phantom{-}0.6\phantom{0}$ & ${-}13.8\phantom{0}$ & $\phantom{0}0.10\phantom{0}$ & $\phantom{0}0.14\phantom{0}$ & $\phantom{0}0.14\phantom{0}$ & $\phantom{0}0.14\phantom{0}$ & $\phantom{0}0.14\phantom{0}$ & $\phantom{0}0.13\phantom{0}$ & $\phantom{0}92.2\phantom{0}$ & $\phantom{0}60.1\phantom{0}$ & $\phantom{0}67.2\phantom{0}$ & $\phantom{0}93.3\phantom{0}$ & $\phantom{0}92.9\phantom{0}$ & $\phantom{0}92.8\phantom{0}$ \\
 & \nopagebreak $\;J=100$  & $\phantom{0}{-}0.7\phantom{0}$ & ${-}42.4\phantom{0}$ & ${-}37.3\phantom{0}$ & $\phantom{0}{-}0.9\phantom{0}$ & $\phantom{0}\phantom{-}0.3\phantom{0}$ & $\phantom{0}{-}6.7\phantom{0}$ & $\phantom{0}0.07\phantom{0}$ & $\phantom{0}0.13\phantom{0}$ & $\phantom{0}0.12\phantom{0}$ & $\phantom{0}0.10\phantom{0}$ & $\phantom{0}0.10\phantom{0}$ & $\phantom{0}0.10\phantom{0}$ & $\phantom{0}93.1\phantom{0}$ & $\phantom{0}45.0\phantom{0}$ & $\phantom{0}55.9\phantom{0}$ & $\phantom{0}93.3\phantom{0}$ & $\phantom{0}92.7\phantom{0}$ & $\phantom{0}93.3\phantom{0}$ \\
 & \nopagebreak $\;J=200$  & $\phantom{0}{-}1.1\phantom{0}$ & ${-}42.5\phantom{0}$ & ${-}37.5\phantom{0}$ & $\phantom{0}{-}1.6\phantom{0}$ & $\phantom{0}{-}1.2\phantom{0}$ & $\phantom{0}{-}4.2\phantom{0}$ & $\phantom{0}0.05\phantom{0}$ & $\phantom{0}0.12\phantom{0}$ & $\phantom{0}0.11\phantom{0}$ & $\phantom{0}0.07\phantom{0}$ & $\phantom{0}0.07\phantom{0}$ & $\phantom{0}0.07\phantom{0}$ & $\phantom{0}93.9\phantom{0}$ & $\phantom{0}19.5\phantom{0}$ & $\phantom{0}33.1\phantom{0}$ & $\phantom{0}93.5\phantom{0}$ & $\phantom{0}92.3\phantom{0}$ & $\phantom{0}93.2\phantom{0}$ \\
 & \nopagebreak $\;J=500$  & $\phantom{0}{-}0.2\phantom{0}$ & ${-}42.0\phantom{0}$ & ${-}37.0\phantom{0}$ & $\phantom{0}{-}0.6\phantom{0}$ & $\phantom{0}{-}0.2\phantom{0}$ & $\phantom{0}{-}1.5\phantom{0}$ & $\phantom{0}0.03\phantom{0}$ & $\phantom{0}0.12\phantom{0}$ & $\phantom{0}0.11\phantom{0}$ & $\phantom{0}0.04\phantom{0}$ & $\phantom{0}0.05\phantom{0}$ & $\phantom{0}0.04\phantom{0}$ & $\phantom{0}95.3\phantom{0}$ & $\phantom{0}\phantom{0}1.0\phantom{0}$ & $\phantom{0}\phantom{0}5.6\phantom{0}$ & $\phantom{0}94.0\phantom{0}$ & $\phantom{0}94.2\phantom{0}$ & $\phantom{0}94.8\phantom{0}$ \\
 & \nopagebreak $\;J=1000$  & $\phantom{0}\phantom{-}0.1\phantom{0}$ & ${-}41.8\phantom{0}$ & ${-}36.7\phantom{0}$ & $\phantom{0}{-}0.3\phantom{0}$ & $\phantom{0}{-}0.0\phantom{0}$ & $\phantom{0}{-}0.7\phantom{0}$ & $\phantom{0}0.02\phantom{0}$ & $\phantom{0}0.12\phantom{0}$ & $\phantom{0}0.10\phantom{0}$ & $\phantom{0}0.03\phantom{0}$ & $\phantom{0}0.03\phantom{0}$ & $\phantom{0}0.03\phantom{0}$ & $\phantom{0}94.3\phantom{0}$ & $\phantom{0}\phantom{0}0.0\phantom{0}$ & $\phantom{0}\phantom{0}0.1\phantom{0}$ & $\phantom{0}94.7\phantom{0}$ & $\phantom{0}93.6\phantom{0}$ & $\phantom{0}94.8\phantom{0}$ \\
\multicolumn{4}{l}{$n=20$} \\  & \nopagebreak $\;J=30$  & $\phantom{0}{-}6.1\phantom{0}$ & ${-}45.3\phantom{0}$ & ${-}45.0\phantom{0}$ & $\phantom{0}{-}4.7\phantom{0}$ & $\phantom{0}{-}4.4\phantom{0}$ & ${-}17.8\phantom{0}$ & $\phantom{0}0.11\phantom{0}$ & $\phantom{0}0.15\phantom{0}$ & $\phantom{0}0.15\phantom{0}$ & $\phantom{0}0.16\phantom{0}$ & $\phantom{0}0.16\phantom{0}$ & $\phantom{0}0.14\phantom{0}$ & $\phantom{0}89.7\phantom{0}$ & $\phantom{0}56.9\phantom{0}$ & $\phantom{0}55.4\phantom{0}$ & $\phantom{0}93.7\phantom{0}$ & $\phantom{0}93.1\phantom{0}$ & $\phantom{0}91.3\phantom{0}$ \\
 & \nopagebreak $\;J=50$  & $\phantom{0}{-}1.7\phantom{0}$ & ${-}42.5\phantom{0}$ & ${-}42.1\phantom{0}$ & $\phantom{0}{-}1.3\phantom{0}$ & $\phantom{0}{-}1.1\phantom{0}$ & ${-}10.0\phantom{0}$ & $\phantom{0}0.09\phantom{0}$ & $\phantom{0}0.14\phantom{0}$ & $\phantom{0}0.14\phantom{0}$ & $\phantom{0}0.13\phantom{0}$ & $\phantom{0}0.13\phantom{0}$ & $\phantom{0}0.12\phantom{0}$ & $\phantom{0}91.7\phantom{0}$ & $\phantom{0}53.0\phantom{0}$ & $\phantom{0}51.2\phantom{0}$ & $\phantom{0}92.9\phantom{0}$ & $\phantom{0}93.2\phantom{0}$ & $\phantom{0}92.3\phantom{0}$ \\
 & \nopagebreak $\;J=100$  & $\phantom{0}{-}1.1\phantom{0}$ & ${-}42.2\phantom{0}$ & ${-}41.8\phantom{0}$ & $\phantom{0}{-}0.8\phantom{0}$ & $\phantom{0}{-}0.5\phantom{0}$ & $\phantom{0}{-}5.2\phantom{0}$ & $\phantom{0}0.06\phantom{0}$ & $\phantom{0}0.13\phantom{0}$ & $\phantom{0}0.13\phantom{0}$ & $\phantom{0}0.09\phantom{0}$ & $\phantom{0}0.09\phantom{0}$ & $\phantom{0}0.08\phantom{0}$ & $\phantom{0}94.1\phantom{0}$ & $\phantom{0}37.0\phantom{0}$ & $\phantom{0}35.4\phantom{0}$ & $\phantom{0}93.3\phantom{0}$ & $\phantom{0}93.6\phantom{0}$ & $\phantom{0}93.5\phantom{0}$ \\
 & \nopagebreak $\;J=200$  & $\phantom{0}\phantom{-}0.2\phantom{0}$ & ${-}41.9\phantom{0}$ & ${-}41.4\phantom{0}$ & $\phantom{0}\phantom{-}0.8\phantom{0}$ & $\phantom{0}\phantom{-}0.4\phantom{0}$ & $\phantom{0}{-}2.0\phantom{0}$ & $\phantom{0}0.04\phantom{0}$ & $\phantom{0}0.12\phantom{0}$ & $\phantom{0}0.12\phantom{0}$ & $\phantom{0}0.06\phantom{0}$ & $\phantom{0}0.06\phantom{0}$ & $\phantom{0}0.06\phantom{0}$ & $\phantom{0}94.9\phantom{0}$ & $\phantom{0}12.5\phantom{0}$ & $\phantom{0}12.3\phantom{0}$ & $\phantom{0}95.2\phantom{0}$ & $\phantom{0}95.3\phantom{0}$ & $\phantom{0}95.1\phantom{0}$ \\
 & \nopagebreak $\;J=500$  & $\phantom{0}{-}0.3\phantom{0}$ & ${-}42.1\phantom{0}$ & ${-}41.7\phantom{0}$ & $\phantom{0}{-}0.7\phantom{0}$ & $\phantom{0}{-}0.6\phantom{0}$ & $\phantom{0}{-}1.5\phantom{0}$ & $\phantom{0}0.03\phantom{0}$ & $\phantom{0}0.12\phantom{0}$ & $\phantom{0}0.12\phantom{0}$ & $\phantom{0}0.04\phantom{0}$ & $\phantom{0}0.04\phantom{0}$ & $\phantom{0}0.04\phantom{0}$ & $\phantom{0}95.1\phantom{0}$ & $\phantom{0}\phantom{0}0.2\phantom{0}$ & $\phantom{0}\phantom{0}0.2\phantom{0}$ & $\phantom{0}93.9\phantom{0}$ & $\phantom{0}94.1\phantom{0}$ & $\phantom{0}94.7\phantom{0}$ \\
 & \nopagebreak $\;J=1000$  & $\phantom{0}{-}0.1\phantom{0}$ & ${-}41.6\phantom{0}$ & ${-}41.2\phantom{0}$ & $\phantom{0}{-}0.1\phantom{0}$ & $\phantom{0}{-}0.1\phantom{0}$ & $\phantom{0}{-}0.5\phantom{0}$ & $\phantom{0}0.02\phantom{0}$ & $\phantom{0}0.12\phantom{0}$ & $\phantom{0}0.11\phantom{0}$ & $\phantom{0}0.03\phantom{0}$ & $\phantom{0}0.03\phantom{0}$ & $\phantom{0}0.03\phantom{0}$ & $\phantom{0}95.6\phantom{0}$ & $\phantom{0}\phantom{0}0.0\phantom{0}$ & $\phantom{0}\phantom{0}0.0\phantom{0}$ & $\phantom{0}94.3\phantom{0}$ & $\phantom{0}95.0\phantom{0}$ & $\phantom{0}94.2\phantom{0}$ \\
[0.5ex]\hline\\[-1.6ex] 
\end{tabular}
\begin{tablenotes}[para,flushleft]{\footnotesize \textit{Note.} $n$ = cluster size; $J$ = number of clusters; CD = complete data sets; LD = listwise deletion; FCS-SL = single-level FCS; FCS-MAN = two-level FCS with manifest cluster means; FCS-LAT = two-level FCS with latent cluster means; JM = joint modeling.}\end{tablenotes}
\end{threeparttable}
\end{sidewaystable}
\begin{sidewaystable}
\begin{threeparttable}
\setlength{\tabcolsep}{1.2pt}
\renewcommand{\arraystretch}{0.95}
\footnotesize
\caption{\small Study 1: Bias (in \%), RMSE, and Coverage of the 95\% Confidence Interval for the Covariance of $y$ With $z$ ($\hat\sigma_{yz}$) With 40\% Missing Data (MCAR, $\lambda=0$)}
\begin{tabular}{llcccccccccccccccccc}
\hline\\[-1.8ex]
& & \multicolumn{6}{c}{Bias (\%)} & \multicolumn{6}{c}{RMSE} & \multicolumn{6}{c}{Coverage (\%)} \\ \cmidrule(r){3-8}\cmidrule(r){9-14}\cmidrule(r){15-20}
 &  & CD & LD & \makecell{FCS-\\SL} & \makecell{FCS-\\MAN} & \makecell{FCS-\\LAT} & JM & CD & LD & \makecell{FCS-\\SL} & \makecell{FCS-\\MAN} & \makecell{FCS-\\LAT} & JM & CD & LD & \makecell{FCS-\\SL} & \makecell{FCS-\\MAN} & \makecell{FCS-\\LAT} & \multicolumn{1}{c}{JM} \\ 
[0.4ex]\hline\\[-1.8ex]
& & \multicolumn{18}{c}{Small intraclass correlation $(\rho_{Iy}=.10)$} \\[0.6ex]\hline\\[-1.8ex]
\multicolumn{4}{l}{$n=5$} \\  & \nopagebreak $\;J=30$  & $\phantom{0}{-}4.6\phantom{0}$ & $\phantom{0}{-}6.7\phantom{0}$ & ${-}33.4\phantom{0}$ & $\phantom{0}{-}5.9\phantom{0}$ & $\phantom{0}{-}1.7\phantom{0}$ & ${-}24.5\phantom{0}$ & $\phantom{0}0.10\phantom{0}$ & $\phantom{0}0.13\phantom{0}$ & $\phantom{0}0.11\phantom{0}$ & $\phantom{0}0.13\phantom{0}$ & $\phantom{0}0.14\phantom{0}$ & $\phantom{0}0.11\phantom{0}$ & $\phantom{0}91.2\phantom{0}$ & $\phantom{0}89.5\phantom{0}$ & $\phantom{0}81.9\phantom{0}$ & $\phantom{0}94.5\phantom{0}$ & $\phantom{0}93.1\phantom{0}$ & $\phantom{0}93.7\phantom{0}$ \\
 & \nopagebreak $\;J=50$  & $\phantom{0}{-}1.4\phantom{0}$ & $\phantom{0}{-}3.2\phantom{0}$ & ${-}30.8\phantom{0}$ & $\phantom{0}{-}1.6\phantom{0}$ & $\phantom{0}\phantom{-}3.8\phantom{0}$ & ${-}16.7\phantom{0}$ & $\phantom{0}0.08\phantom{0}$ & $\phantom{0}0.10\phantom{0}$ & $\phantom{0}0.09\phantom{0}$ & $\phantom{0}0.10\phantom{0}$ & $\phantom{0}0.11\phantom{0}$ & $\phantom{0}0.09\phantom{0}$ & $\phantom{0}93.5\phantom{0}$ & $\phantom{0}91.7\phantom{0}$ & $\phantom{0}80.8\phantom{0}$ & $\phantom{0}95.4\phantom{0}$ & $\phantom{0}93.8\phantom{0}$ & $\phantom{0}93.4\phantom{0}$ \\
 & \nopagebreak $\;J=100$  & $\phantom{0}{-}0.4\phantom{0}$ & $\phantom{0}{-}2.4\phantom{0}$ & ${-}30.4\phantom{0}$ & $\phantom{0}{-}1.7\phantom{0}$ & $\phantom{0}\phantom{-}2.9\phantom{0}$ & ${-}12.2\phantom{0}$ & $\phantom{0}0.05\phantom{0}$ & $\phantom{0}0.07\phantom{0}$ & $\phantom{0}0.07\phantom{0}$ & $\phantom{0}0.07\phantom{0}$ & $\phantom{0}0.07\phantom{0}$ & $\phantom{0}0.07\phantom{0}$ & $\phantom{0}94.1\phantom{0}$ & $\phantom{0}93.2\phantom{0}$ & $\phantom{0}76.2\phantom{0}$ & $\phantom{0}94.5\phantom{0}$ & $\phantom{0}93.3\phantom{0}$ & $\phantom{0}94.1\phantom{0}$ \\
 & \nopagebreak $\;J=200$  & $\phantom{0}{-}0.2\phantom{0}$ & $\phantom{0}{-}1.5\phantom{0}$ & ${-}30.0\phantom{0}$ & $\phantom{0}{-}0.3\phantom{0}$ & $\phantom{0}\phantom{-}2.5\phantom{0}$ & $\phantom{0}{-}7.7\phantom{0}$ & $\phantom{0}0.04\phantom{0}$ & $\phantom{0}0.05\phantom{0}$ & $\phantom{0}0.06\phantom{0}$ & $\phantom{0}0.05\phantom{0}$ & $\phantom{0}0.05\phantom{0}$ & $\phantom{0}0.05\phantom{0}$ & $\phantom{0}95.2\phantom{0}$ & $\phantom{0}94.3\phantom{0}$ & $\phantom{0}68.0\phantom{0}$ & $\phantom{0}94.9\phantom{0}$ & $\phantom{0}94.7\phantom{0}$ & $\phantom{0}94.4\phantom{0}$ \\
 & \nopagebreak $\;J=500$  & $\phantom{0}{-}0.9\phantom{0}$ & $\phantom{0}{-}1.5\phantom{0}$ & ${-}29.7\phantom{0}$ & $\phantom{0}{-}1.2\phantom{0}$ & $\phantom{0}{-}0.4\phantom{0}$ & $\phantom{0}{-}4.8\phantom{0}$ & $\phantom{0}0.02\phantom{0}$ & $\phantom{0}0.03\phantom{0}$ & $\phantom{0}0.05\phantom{0}$ & $\phantom{0}0.03\phantom{0}$ & $\phantom{0}0.03\phantom{0}$ & $\phantom{0}0.03\phantom{0}$ & $\phantom{0}96.0\phantom{0}$ & $\phantom{0}93.4\phantom{0}$ & $\phantom{0}42.5\phantom{0}$ & $\phantom{0}93.5\phantom{0}$ & $\phantom{0}93.7\phantom{0}$ & $\phantom{0}93.5\phantom{0}$ \\
 & \nopagebreak $\;J=1000$  & $\phantom{0}{-}0.5\phantom{0}$ & $\phantom{0}{-}0.6\phantom{0}$ & ${-}29.2\phantom{0}$ & $\phantom{0}{-}0.4\phantom{0}$ & $\phantom{0}\phantom{-}0.0\phantom{0}$ & $\phantom{0}{-}2.5\phantom{0}$ & $\phantom{0}0.02\phantom{0}$ & $\phantom{0}0.02\phantom{0}$ & $\phantom{0}0.05\phantom{0}$ & $\phantom{0}0.02\phantom{0}$ & $\phantom{0}0.02\phantom{0}$ & $\phantom{0}0.02\phantom{0}$ & $\phantom{0}95.7\phantom{0}$ & $\phantom{0}95.0\phantom{0}$ & $\phantom{0}17.1\phantom{0}$ & $\phantom{0}95.3\phantom{0}$ & $\phantom{0}94.9\phantom{0}$ & $\phantom{0}95.2\phantom{0}$ \\
\multicolumn{4}{l}{$n=20$} \\  & \nopagebreak $\;J=30$  & $\phantom{0}{-}2.2\phantom{0}$ & $\phantom{0}{-}6.0\phantom{0}$ & ${-}38.1\phantom{0}$ & $\phantom{0}{-}2.9\phantom{0}$ & $\phantom{0}\phantom{-}0.7\phantom{0}$ & ${-}22.1\phantom{0}$ & $\phantom{0}0.07\phantom{0}$ & $\phantom{0}0.10\phantom{0}$ & $\phantom{0}0.09\phantom{0}$ & $\phantom{0}0.10\phantom{0}$ & $\phantom{0}0.10\phantom{0}$ & $\phantom{0}0.09\phantom{0}$ & $\phantom{0}89.7\phantom{0}$ & $\phantom{0}86.3\phantom{0}$ & $\phantom{0}65.8\phantom{0}$ & $\phantom{0}93.1\phantom{0}$ & $\phantom{0}92.9\phantom{0}$ & $\phantom{0}89.9\phantom{0}$ \\
 & \nopagebreak $\;J=50$  & $\phantom{0}{-}0.8\phantom{0}$ & $\phantom{0}{-}2.2\phantom{0}$ & ${-}35.6\phantom{0}$ & $\phantom{0}{-}0.8\phantom{0}$ & $\phantom{0}\phantom{-}0.7\phantom{0}$ & ${-}15.2\phantom{0}$ & $\phantom{0}0.06\phantom{0}$ & $\phantom{0}0.08\phantom{0}$ & $\phantom{0}0.08\phantom{0}$ & $\phantom{0}0.07\phantom{0}$ & $\phantom{0}0.08\phantom{0}$ & $\phantom{0}0.07\phantom{0}$ & $\phantom{0}91.5\phantom{0}$ & $\phantom{0}89.6\phantom{0}$ & $\phantom{0}63.5\phantom{0}$ & $\phantom{0}93.1\phantom{0}$ & $\phantom{0}92.4\phantom{0}$ & $\phantom{0}91.3\phantom{0}$ \\
 & \nopagebreak $\;J=100$  & $\phantom{0}\phantom{-}0.4\phantom{0}$ & $\phantom{0}{-}0.9\phantom{0}$ & ${-}34.8\phantom{0}$ & $\phantom{0}\phantom{-}0.2\phantom{0}$ & $\phantom{0}\phantom{-}0.8\phantom{0}$ & $\phantom{0}{-}8.5\phantom{0}$ & $\phantom{0}0.04\phantom{0}$ & $\phantom{0}0.05\phantom{0}$ & $\phantom{0}0.07\phantom{0}$ & $\phantom{0}0.05\phantom{0}$ & $\phantom{0}0.05\phantom{0}$ & $\phantom{0}0.05\phantom{0}$ & $\phantom{0}94.3\phantom{0}$ & $\phantom{0}92.6\phantom{0}$ & $\phantom{0}54.9\phantom{0}$ & $\phantom{0}94.3\phantom{0}$ & $\phantom{0}93.8\phantom{0}$ & $\phantom{0}93.3\phantom{0}$ \\
 & \nopagebreak $\;J=200$  & $\phantom{0}{-}0.7\phantom{0}$ & $\phantom{0}{-}0.5\phantom{0}$ & ${-}34.6\phantom{0}$ & $\phantom{0}{-}0.0\phantom{0}$ & $\phantom{0}\phantom{-}0.5\phantom{0}$ & $\phantom{0}{-}4.9\phantom{0}$ & $\phantom{0}0.03\phantom{0}$ & $\phantom{0}0.04\phantom{0}$ & $\phantom{0}0.06\phantom{0}$ & $\phantom{0}0.04\phantom{0}$ & $\phantom{0}0.04\phantom{0}$ & $\phantom{0}0.04\phantom{0}$ & $\phantom{0}94.7\phantom{0}$ & $\phantom{0}93.8\phantom{0}$ & $\phantom{0}36.1\phantom{0}$ & $\phantom{0}94.4\phantom{0}$ & $\phantom{0}93.6\phantom{0}$ & $\phantom{0}94.1\phantom{0}$ \\
 & \nopagebreak $\;J=500$  & $\phantom{0}\phantom{-}0.1\phantom{0}$ & $\phantom{0}\phantom{-}0.3\phantom{0}$ & ${-}34.0\phantom{0}$ & $\phantom{0}\phantom{-}0.4\phantom{0}$ & $\phantom{0}\phantom{-}0.6\phantom{0}$ & $\phantom{0}{-}1.7\phantom{0}$ & $\phantom{0}0.02\phantom{0}$ & $\phantom{0}0.02\phantom{0}$ & $\phantom{0}0.06\phantom{0}$ & $\phantom{0}0.02\phantom{0}$ & $\phantom{0}0.02\phantom{0}$ & $\phantom{0}0.02\phantom{0}$ & $\phantom{0}95.0\phantom{0}$ & $\phantom{0}95.6\phantom{0}$ & $\phantom{0}\phantom{0}8.1\phantom{0}$ & $\phantom{0}95.4\phantom{0}$ & $\phantom{0}95.9\phantom{0}$ & $\phantom{0}95.2\phantom{0}$ \\
 & \nopagebreak $\;J=1000$  & $\phantom{0}\phantom{-}0.1\phantom{0}$ & $\phantom{0}\phantom{-}0.0\phantom{0}$ & ${-}34.2\phantom{0}$ & $\phantom{0}\phantom{-}0.1\phantom{0}$ & $\phantom{0}\phantom{-}0.1\phantom{0}$ & $\phantom{0}{-}1.1\phantom{0}$ & $\phantom{0}0.01\phantom{0}$ & $\phantom{0}0.02\phantom{0}$ & $\phantom{0}0.06\phantom{0}$ & $\phantom{0}0.02\phantom{0}$ & $\phantom{0}0.02\phantom{0}$ & $\phantom{0}0.02\phantom{0}$ & $\phantom{0}95.7\phantom{0}$ & $\phantom{0}94.7\phantom{0}$ & $\phantom{0}\phantom{0}0.3\phantom{0}$ & $\phantom{0}95.2\phantom{0}$ & $\phantom{0}95.4\phantom{0}$ & $\phantom{0}95.4\phantom{0}$ \\
[0.5ex]\hline\\[-1.6ex] 
& & \multicolumn{18}{c}{Moderate intraclass correlation $(\rho_{Iy}=.30)$} \\[0.6ex]\hline\\[-1.8ex]
\multicolumn{4}{l}{$n=5$} \\  & \nopagebreak $\;J=30$  & $\phantom{0}{-}2.3\phantom{0}$ & $\phantom{0}{-}3.1\phantom{0}$ & ${-}25.3\phantom{0}$ & $\phantom{0}{-}2.4\phantom{0}$ & $\phantom{0}\phantom{-}0.6\phantom{0}$ & ${-}13.2\phantom{0}$ & $\phantom{0}0.13\phantom{0}$ & $\phantom{0}0.17\phantom{0}$ & $\phantom{0}0.15\phantom{0}$ & $\phantom{0}0.17\phantom{0}$ & $\phantom{0}0.17\phantom{0}$ & $\phantom{0}0.16\phantom{0}$ & $\phantom{0}91.0\phantom{0}$ & $\phantom{0}88.3\phantom{0}$ & $\phantom{0}77.3\phantom{0}$ & $\phantom{0}93.1\phantom{0}$ & $\phantom{0}93.7\phantom{0}$ & $\phantom{0}92.1\phantom{0}$ \\
 & \nopagebreak $\;J=50$  & $\phantom{0}{-}2.2\phantom{0}$ & $\phantom{0}{-}3.6\phantom{0}$ & ${-}25.0\phantom{0}$ & $\phantom{0}{-}2.5\phantom{0}$ & $\phantom{0}{-}0.6\phantom{0}$ & $\phantom{0}{-}9.8\phantom{0}$ & $\phantom{0}0.10\phantom{0}$ & $\phantom{0}0.13\phantom{0}$ & $\phantom{0}0.12\phantom{0}$ & $\phantom{0}0.13\phantom{0}$ & $\phantom{0}0.13\phantom{0}$ & $\phantom{0}0.12\phantom{0}$ & $\phantom{0}91.7\phantom{0}$ & $\phantom{0}90.3\phantom{0}$ & $\phantom{0}75.8\phantom{0}$ & $\phantom{0}92.9\phantom{0}$ & $\phantom{0}93.1\phantom{0}$ & $\phantom{0}92.9\phantom{0}$ \\
 & \nopagebreak $\;J=100$  & $\phantom{0}{-}1.1\phantom{0}$ & $\phantom{0}{-}1.9\phantom{0}$ & ${-}23.9\phantom{0}$ & $\phantom{0}{-}1.0\phantom{0}$ & $\phantom{0}{-}0.4\phantom{0}$ & $\phantom{0}{-}4.6\phantom{0}$ & $\phantom{0}0.07\phantom{0}$ & $\phantom{0}0.09\phantom{0}$ & $\phantom{0}0.10\phantom{0}$ & $\phantom{0}0.09\phantom{0}$ & $\phantom{0}0.09\phantom{0}$ & $\phantom{0}0.09\phantom{0}$ & $\phantom{0}92.7\phantom{0}$ & $\phantom{0}91.9\phantom{0}$ & $\phantom{0}71.7\phantom{0}$ & $\phantom{0}92.9\phantom{0}$ & $\phantom{0}93.9\phantom{0}$ & $\phantom{0}92.7\phantom{0}$ \\
 & \nopagebreak $\;J=200$  & $\phantom{0}{-}1.3\phantom{0}$ & $\phantom{0}{-}0.9\phantom{0}$ & ${-}22.9\phantom{0}$ & $\phantom{0}{-}0.4\phantom{0}$ & $\phantom{0}{-}0.1\phantom{0}$ & $\phantom{0}{-}1.8\phantom{0}$ & $\phantom{0}0.05\phantom{0}$ & $\phantom{0}0.06\phantom{0}$ & $\phantom{0}0.08\phantom{0}$ & $\phantom{0}0.06\phantom{0}$ & $\phantom{0}0.06\phantom{0}$ & $\phantom{0}0.06\phantom{0}$ & $\phantom{0}94.0\phantom{0}$ & $\phantom{0}93.8\phantom{0}$ & $\phantom{0}64.5\phantom{0}$ & $\phantom{0}94.5\phantom{0}$ & $\phantom{0}94.7\phantom{0}$ & $\phantom{0}94.8\phantom{0}$ \\
 & \nopagebreak $\;J=500$  & $\phantom{0}{-}0.0\phantom{0}$ & $\phantom{0}\phantom{-}0.0\phantom{0}$ & ${-}22.3\phantom{0}$ & $\phantom{0}\phantom{-}0.0\phantom{0}$ & $\phantom{0}\phantom{-}0.0\phantom{0}$ & $\phantom{0}{-}0.8\phantom{0}$ & $\phantom{0}0.03\phantom{0}$ & $\phantom{0}0.04\phantom{0}$ & $\phantom{0}0.07\phantom{0}$ & $\phantom{0}0.04\phantom{0}$ & $\phantom{0}0.04\phantom{0}$ & $\phantom{0}0.04\phantom{0}$ & $\phantom{0}95.9\phantom{0}$ & $\phantom{0}94.7\phantom{0}$ & $\phantom{0}44.1\phantom{0}$ & $\phantom{0}95.2\phantom{0}$ & $\phantom{0}94.4\phantom{0}$ & $\phantom{0}94.8\phantom{0}$ \\
 & \nopagebreak $\;J=1000$  & $\phantom{0}{-}0.1\phantom{0}$ & $\phantom{0}{-}0.3\phantom{0}$ & ${-}22.6\phantom{0}$ & $\phantom{0}{-}0.2\phantom{0}$ & $\phantom{0}{-}0.2\phantom{0}$ & $\phantom{0}{-}0.5\phantom{0}$ & $\phantom{0}0.02\phantom{0}$ & $\phantom{0}0.03\phantom{0}$ & $\phantom{0}0.07\phantom{0}$ & $\phantom{0}0.03\phantom{0}$ & $\phantom{0}0.03\phantom{0}$ & $\phantom{0}0.03\phantom{0}$ & $\phantom{0}94.8\phantom{0}$ & $\phantom{0}95.3\phantom{0}$ & $\phantom{0}18.0\phantom{0}$ & $\phantom{0}95.0\phantom{0}$ & $\phantom{0}94.8\phantom{0}$ & $\phantom{0}93.8\phantom{0}$ \\
\multicolumn{4}{l}{$n=20$} \\  & \nopagebreak $\;J=30$  & $\phantom{0}{-}5.7\phantom{0}$ & $\phantom{0}{-}7.7\phantom{0}$ & ${-}32.4\phantom{0}$ & $\phantom{0}{-}5.2\phantom{0}$ & $\phantom{0}{-}4.4\phantom{0}$ & ${-}14.9\phantom{0}$ & $\phantom{0}0.11\phantom{0}$ & $\phantom{0}0.15\phantom{0}$ & $\phantom{0}0.14\phantom{0}$ & $\phantom{0}0.14\phantom{0}$ & $\phantom{0}0.14\phantom{0}$ & $\phantom{0}0.14\phantom{0}$ & $\phantom{0}88.3\phantom{0}$ & $\phantom{0}85.7\phantom{0}$ & $\phantom{0}66.0\phantom{0}$ & $\phantom{0}92.4\phantom{0}$ & $\phantom{0}92.5\phantom{0}$ & $\phantom{0}90.8\phantom{0}$ \\
 & \nopagebreak $\;J=50$  & $\phantom{0}{-}1.1\phantom{0}$ & $\phantom{0}{-}2.6\phantom{0}$ & ${-}28.1\phantom{0}$ & $\phantom{0}{-}0.4\phantom{0}$ & $\phantom{0}\phantom{-}0.5\phantom{0}$ & $\phantom{0}{-}6.2\phantom{0}$ & $\phantom{0}0.09\phantom{0}$ & $\phantom{0}0.11\phantom{0}$ & $\phantom{0}0.11\phantom{0}$ & $\phantom{0}0.11\phantom{0}$ & $\phantom{0}0.11\phantom{0}$ & $\phantom{0}0.10\phantom{0}$ & $\phantom{0}91.5\phantom{0}$ & $\phantom{0}90.8\phantom{0}$ & $\phantom{0}66.1\phantom{0}$ & $\phantom{0}94.2\phantom{0}$ & $\phantom{0}93.6\phantom{0}$ & $\phantom{0}92.9\phantom{0}$ \\
 & \nopagebreak $\;J=100$  & $\phantom{0}{-}0.1\phantom{0}$ & $\phantom{0}{-}0.3\phantom{0}$ & ${-}27.1\phantom{0}$ & $\phantom{0}{-}0.2\phantom{0}$ & $\phantom{0}{-}0.0\phantom{0}$ & $\phantom{0}{-}3.0\phantom{0}$ & $\phantom{0}0.06\phantom{0}$ & $\phantom{0}0.08\phantom{0}$ & $\phantom{0}0.09\phantom{0}$ & $\phantom{0}0.08\phantom{0}$ & $\phantom{0}0.08\phantom{0}$ & $\phantom{0}0.07\phantom{0}$ & $\phantom{0}94.2\phantom{0}$ & $\phantom{0}92.9\phantom{0}$ & $\phantom{0}61.5\phantom{0}$ & $\phantom{0}94.7\phantom{0}$ & $\phantom{0}94.4\phantom{0}$ & $\phantom{0}94.1\phantom{0}$ \\
 & \nopagebreak $\;J=200$  & $\phantom{0}{-}1.2\phantom{0}$ & $\phantom{0}{-}1.4\phantom{0}$ & ${-}27.5\phantom{0}$ & $\phantom{0}{-}0.5\phantom{0}$ & $\phantom{0}{-}0.4\phantom{0}$ & $\phantom{0}{-}2.2\phantom{0}$ & $\phantom{0}0.04\phantom{0}$ & $\phantom{0}0.06\phantom{0}$ & $\phantom{0}0.09\phantom{0}$ & $\phantom{0}0.06\phantom{0}$ & $\phantom{0}0.05\phantom{0}$ & $\phantom{0}0.05\phantom{0}$ & $\phantom{0}95.2\phantom{0}$ & $\phantom{0}93.6\phantom{0}$ & $\phantom{0}44.7\phantom{0}$ & $\phantom{0}94.9\phantom{0}$ & $\phantom{0}94.0\phantom{0}$ & $\phantom{0}94.9\phantom{0}$ \\
 & \nopagebreak $\;J=500$  & $\phantom{0}\phantom{-}0.1\phantom{0}$ & $\phantom{0}\phantom{-}0.0\phantom{0}$ & ${-}26.5\phantom{0}$ & $\phantom{0}\phantom{-}0.1\phantom{0}$ & $\phantom{0}\phantom{-}0.1\phantom{0}$ & $\phantom{0}{-}0.5\phantom{0}$ & $\phantom{0}0.03\phantom{0}$ & $\phantom{0}0.04\phantom{0}$ & $\phantom{0}0.08\phantom{0}$ & $\phantom{0}0.03\phantom{0}$ & $\phantom{0}0.03\phantom{0}$ & $\phantom{0}0.03\phantom{0}$ & $\phantom{0}96.1\phantom{0}$ & $\phantom{0}95.2\phantom{0}$ & $\phantom{0}19.0\phantom{0}$ & $\phantom{0}95.6\phantom{0}$ & $\phantom{0}96.0\phantom{0}$ & $\phantom{0}95.7\phantom{0}$ \\
 & \nopagebreak $\;J=1000$  & $\phantom{0}{-}0.2\phantom{0}$ & $\phantom{0}{-}0.4\phantom{0}$ & ${-}27.0\phantom{0}$ & $\phantom{0}{-}0.3\phantom{0}$ & $\phantom{0}{-}0.3\phantom{0}$ & $\phantom{0}{-}0.7\phantom{0}$ & $\phantom{0}0.02\phantom{0}$ & $\phantom{0}0.03\phantom{0}$ & $\phantom{0}0.08\phantom{0}$ & $\phantom{0}0.02\phantom{0}$ & $\phantom{0}0.02\phantom{0}$ & $\phantom{0}0.02\phantom{0}$ & $\phantom{0}95.8\phantom{0}$ & $\phantom{0}95.4\phantom{0}$ & $\phantom{0}\phantom{0}2.7\phantom{0}$ & $\phantom{0}95.8\phantom{0}$ & $\phantom{0}95.4\phantom{0}$ & $\phantom{0}94.9\phantom{0}$ \\
[0.5ex]\hline\\[-1.6ex] 
\end{tabular}
\begin{tablenotes}[para,flushleft]{\footnotesize \textit{Note.} $n$ = cluster size; $J$ = number of clusters; CD = complete data sets; LD = listwise deletion; FCS-SL = single-level FCS; FCS-MAN = two-level FCS with manifest cluster means; FCS-LAT = two-level FCS with latent cluster means; JM = joint modeling.}\end{tablenotes}
\end{threeparttable}
\end{sidewaystable}
\begin{sidewaystable}
\begin{threeparttable}
\setlength{\tabcolsep}{1.2pt}
\renewcommand{\arraystretch}{0.95}
\footnotesize
\caption{\small Study 1: Bias (in \%), RMSE, and Coverage of the 95\% Confidence Interval for the Covariance of $y$ With $z$ ($\hat\sigma_{yz}$) With 40\% Missing Data (MAR, $\lambda=0.5$)}
\begin{tabular}{llcccccccccccccccccc}
\hline\\[-1.8ex]
& & \multicolumn{6}{c}{Bias (\%)} & \multicolumn{6}{c}{RMSE} & \multicolumn{6}{c}{Coverage (\%)} \\ \cmidrule(r){3-8}\cmidrule(r){9-14}\cmidrule(r){15-20}
 &  & CD & LD & \makecell{FCS-\\SL} & \makecell{FCS-\\MAN} & \makecell{FCS-\\LAT} & JM & CD & LD & \makecell{FCS-\\SL} & \makecell{FCS-\\MAN} & \makecell{FCS-\\LAT} & JM & CD & LD & \makecell{FCS-\\SL} & \makecell{FCS-\\MAN} & \makecell{FCS-\\LAT} & \multicolumn{1}{c}{JM} \\ 
[0.4ex]\hline\\[-1.8ex]
& & \multicolumn{18}{c}{Small intraclass correlation $(\rho_{Iy}=.10)$} \\[0.6ex]\hline\\[-1.8ex]
\multicolumn{4}{l}{$n=5$} \\  & \nopagebreak $\;J=30$  & $\phantom{0}{-}4.0\phantom{0}$ & ${-}17.1\phantom{0}$ & ${-}38.3\phantom{0}$ & $\phantom{0}{-}0.1\phantom{0}$ & $\phantom{0}\phantom{-}5.3\phantom{0}$ & ${-}27.4\phantom{0}$ & $\phantom{0}0.10\phantom{0}$ & $\phantom{0}0.12\phantom{0}$ & $\phantom{0}0.11\phantom{0}$ & $\phantom{0}0.15\phantom{0}$ & $\phantom{0}0.15\phantom{0}$ & $\phantom{0}0.11\phantom{0}$ & $\phantom{0}92.5\phantom{0}$ & $\phantom{0}90.8\phantom{0}$ & $\phantom{0}80.1\phantom{0}$ & $\phantom{0}94.3\phantom{0}$ & $\phantom{0}93.9\phantom{0}$ & $\phantom{0}94.7\phantom{0}$ \\
 & \nopagebreak $\;J=50$  & $\phantom{0}{-}1.4\phantom{0}$ & ${-}16.5\phantom{0}$ & ${-}38.3\phantom{0}$ & $\phantom{0}{-}2.0\phantom{0}$ & $\phantom{0}\phantom{-}4.1\phantom{0}$ & ${-}24.1\phantom{0}$ & $\phantom{0}0.08\phantom{0}$ & $\phantom{0}0.10\phantom{0}$ & $\phantom{0}0.09\phantom{0}$ & $\phantom{0}0.11\phantom{0}$ & $\phantom{0}0.11\phantom{0}$ & $\phantom{0}0.10\phantom{0}$ & $\phantom{0}93.3\phantom{0}$ & $\phantom{0}90.3\phantom{0}$ & $\phantom{0}77.5\phantom{0}$ & $\phantom{0}95.3\phantom{0}$ & $\phantom{0}94.1\phantom{0}$ & $\phantom{0}94.8\phantom{0}$ \\
 & \nopagebreak $\;J=100$  & $\phantom{0}\phantom{-}0.7\phantom{0}$ & ${-}13.5\phantom{0}$ & ${-}35.7\phantom{0}$ & $\phantom{0}\phantom{-}2.0\phantom{0}$ & $\phantom{0}\phantom{-}7.3\phantom{0}$ & ${-}13.3\phantom{0}$ & $\phantom{0}0.05\phantom{0}$ & $\phantom{0}0.07\phantom{0}$ & $\phantom{0}0.07\phantom{0}$ & $\phantom{0}0.08\phantom{0}$ & $\phantom{0}0.08\phantom{0}$ & $\phantom{0}0.07\phantom{0}$ & $\phantom{0}95.2\phantom{0}$ & $\phantom{0}90.8\phantom{0}$ & $\phantom{0}73.4\phantom{0}$ & $\phantom{0}94.7\phantom{0}$ & $\phantom{0}94.3\phantom{0}$ & $\phantom{0}94.9\phantom{0}$ \\
 & \nopagebreak $\;J=200$  & $\phantom{0}{-}0.8\phantom{0}$ & ${-}15.2\phantom{0}$ & ${-}37.1\phantom{0}$ & $\phantom{0}{-}0.4\phantom{0}$ & $\phantom{0}\phantom{-}3.1\phantom{0}$ & $\phantom{0}{-}9.9\phantom{0}$ & $\phantom{0}0.04\phantom{0}$ & $\phantom{0}0.05\phantom{0}$ & $\phantom{0}0.07\phantom{0}$ & $\phantom{0}0.05\phantom{0}$ & $\phantom{0}0.06\phantom{0}$ & $\phantom{0}0.05\phantom{0}$ & $\phantom{0}94.6\phantom{0}$ & $\phantom{0}88.8\phantom{0}$ & $\phantom{0}56.3\phantom{0}$ & $\phantom{0}93.5\phantom{0}$ & $\phantom{0}93.0\phantom{0}$ & $\phantom{0}93.2\phantom{0}$ \\
 & \nopagebreak $\;J=500$  & $\phantom{0}{-}0.4\phantom{0}$ & ${-}15.5\phantom{0}$ & ${-}37.3\phantom{0}$ & $\phantom{0}{-}0.4\phantom{0}$ & $\phantom{0}\phantom{-}0.8\phantom{0}$ & $\phantom{0}{-}5.8\phantom{0}$ & $\phantom{0}0.03\phantom{0}$ & $\phantom{0}0.04\phantom{0}$ & $\phantom{0}0.06\phantom{0}$ & $\phantom{0}0.03\phantom{0}$ & $\phantom{0}0.03\phantom{0}$ & $\phantom{0}0.03\phantom{0}$ & $\phantom{0}95.1\phantom{0}$ & $\phantom{0}85.0\phantom{0}$ & $\phantom{0}24.2\phantom{0}$ & $\phantom{0}94.7\phantom{0}$ & $\phantom{0}93.0\phantom{0}$ & $\phantom{0}93.5\phantom{0}$ \\
 & \nopagebreak $\;J=1000$  & $\phantom{0}\phantom{-}0.2\phantom{0}$ & ${-}14.4\phantom{0}$ & ${-}36.6\phantom{0}$ & $\phantom{0}\phantom{-}0.1\phantom{0}$ & $\phantom{0}\phantom{-}0.7\phantom{0}$ & $\phantom{0}{-}2.7\phantom{0}$ & $\phantom{0}0.02\phantom{0}$ & $\phantom{0}0.03\phantom{0}$ & $\phantom{0}0.06\phantom{0}$ & $\phantom{0}0.02\phantom{0}$ & $\phantom{0}0.02\phantom{0}$ & $\phantom{0}0.02\phantom{0}$ & $\phantom{0}94.5\phantom{0}$ & $\phantom{0}79.9\phantom{0}$ & $\phantom{0}\phantom{0}4.1\phantom{0}$ & $\phantom{0}94.4\phantom{0}$ & $\phantom{0}94.2\phantom{0}$ & $\phantom{0}94.9\phantom{0}$ \\
\multicolumn{4}{l}{$n=20$} \\  & \nopagebreak $\;J=30$  & $\phantom{0}{-}1.9\phantom{0}$ & ${-}17.6\phantom{0}$ & ${-}44.7\phantom{0}$ & $\phantom{0}{-}1.7\phantom{0}$ & $\phantom{0}\phantom{-}2.9\phantom{0}$ & ${-}25.9\phantom{0}$ & $\phantom{0}0.07\phantom{0}$ & $\phantom{0}0.09\phantom{0}$ & $\phantom{0}0.09\phantom{0}$ & $\phantom{0}0.11\phantom{0}$ & $\phantom{0}0.11\phantom{0}$ & $\phantom{0}0.09\phantom{0}$ & $\phantom{0}89.3\phantom{0}$ & $\phantom{0}83.3\phantom{0}$ & $\phantom{0}60.1\phantom{0}$ & $\phantom{0}93.3\phantom{0}$ & $\phantom{0}93.6\phantom{0}$ & $\phantom{0}91.3\phantom{0}$ \\
 & \nopagebreak $\;J=50$  & $\phantom{0}{-}1.1\phantom{0}$ & ${-}16.7\phantom{0}$ & ${-}43.9\phantom{0}$ & $\phantom{0}{-}1.5\phantom{0}$ & $\phantom{0}\phantom{-}0.5\phantom{0}$ & ${-}20.1\phantom{0}$ & $\phantom{0}0.06\phantom{0}$ & $\phantom{0}0.07\phantom{0}$ & $\phantom{0}0.08\phantom{0}$ & $\phantom{0}0.08\phantom{0}$ & $\phantom{0}0.08\phantom{0}$ & $\phantom{0}0.07\phantom{0}$ & $\phantom{0}92.4\phantom{0}$ & $\phantom{0}85.4\phantom{0}$ & $\phantom{0}55.9\phantom{0}$ & $\phantom{0}94.6\phantom{0}$ & $\phantom{0}93.5\phantom{0}$ & $\phantom{0}91.5\phantom{0}$ \\
 & \nopagebreak $\;J=100$  & $\phantom{0}{-}1.4\phantom{0}$ & ${-}15.1\phantom{0}$ & ${-}43.0\phantom{0}$ & $\phantom{0}{-}0.6\phantom{0}$ & $\phantom{0}\phantom{-}0.1\phantom{0}$ & ${-}11.9\phantom{0}$ & $\phantom{0}0.04\phantom{0}$ & $\phantom{0}0.05\phantom{0}$ & $\phantom{0}0.08\phantom{0}$ & $\phantom{0}0.05\phantom{0}$ & $\phantom{0}0.05\phantom{0}$ & $\phantom{0}0.05\phantom{0}$ & $\phantom{0}92.7\phantom{0}$ & $\phantom{0}87.0\phantom{0}$ & $\phantom{0}40.7\phantom{0}$ & $\phantom{0}94.9\phantom{0}$ & $\phantom{0}94.4\phantom{0}$ & $\phantom{0}92.8\phantom{0}$ \\
 & \nopagebreak $\;J=200$  & $\phantom{0}\phantom{-}0.4\phantom{0}$ & ${-}13.8\phantom{0}$ & ${-}42.1\phantom{0}$ & $\phantom{0}\phantom{-}0.8\phantom{0}$ & $\phantom{0}\phantom{-}1.0\phantom{0}$ & $\phantom{0}{-}6.2\phantom{0}$ & $\phantom{0}0.03\phantom{0}$ & $\phantom{0}0.04\phantom{0}$ & $\phantom{0}0.07\phantom{0}$ & $\phantom{0}0.04\phantom{0}$ & $\phantom{0}0.04\phantom{0}$ & $\phantom{0}0.04\phantom{0}$ & $\phantom{0}94.0\phantom{0}$ & $\phantom{0}86.7\phantom{0}$ & $\phantom{0}18.7\phantom{0}$ & $\phantom{0}95.7\phantom{0}$ & $\phantom{0}95.1\phantom{0}$ & $\phantom{0}95.1\phantom{0}$ \\
 & \nopagebreak $\;J=500$  & $\phantom{0}\phantom{-}0.1\phantom{0}$ & ${-}14.6\phantom{0}$ & ${-}42.7\phantom{0}$ & $\phantom{0}{-}0.1\phantom{0}$ & $\phantom{0}\phantom{-}0.1\phantom{0}$ & $\phantom{0}{-}2.8\phantom{0}$ & $\phantom{0}0.02\phantom{0}$ & $\phantom{0}0.03\phantom{0}$ & $\phantom{0}0.07\phantom{0}$ & $\phantom{0}0.02\phantom{0}$ & $\phantom{0}0.03\phantom{0}$ & $\phantom{0}0.03\phantom{0}$ & $\phantom{0}93.7\phantom{0}$ & $\phantom{0}78.3\phantom{0}$ & $\phantom{0}\phantom{0}1.1\phantom{0}$ & $\phantom{0}94.3\phantom{0}$ & $\phantom{0}93.2\phantom{0}$ & $\phantom{0}93.6\phantom{0}$ \\
 & \nopagebreak $\;J=1000$  & $\phantom{0}{-}0.3\phantom{0}$ & ${-}14.5\phantom{0}$ & ${-}42.6\phantom{0}$ & $\phantom{0}{-}0.1\phantom{0}$ & $\phantom{0}{-}0.0\phantom{0}$ & $\phantom{0}{-}1.5\phantom{0}$ & $\phantom{0}0.01\phantom{0}$ & $\phantom{0}0.03\phantom{0}$ & $\phantom{0}0.07\phantom{0}$ & $\phantom{0}0.02\phantom{0}$ & $\phantom{0}0.02\phantom{0}$ & $\phantom{0}0.02\phantom{0}$ & $\phantom{0}95.1\phantom{0}$ & $\phantom{0}67.7\phantom{0}$ & $\phantom{0}\phantom{0}0.0\phantom{0}$ & $\phantom{0}93.8\phantom{0}$ & $\phantom{0}94.3\phantom{0}$ & $\phantom{0}94.3\phantom{0}$ \\
[0.5ex]\hline\\[-1.6ex] 
& & \multicolumn{18}{c}{Moderate intraclass correlation $(\rho_{Iy}=.30)$} \\[0.6ex]\hline\\[-1.8ex]
\multicolumn{4}{l}{$n=5$} \\  & \nopagebreak $\;J=30$  & $\phantom{0}{-}2.8\phantom{0}$ & ${-}20.0\phantom{0}$ & ${-}33.4\phantom{0}$ & $\phantom{0}{-}3.0\phantom{0}$ & $\phantom{0}\phantom{-}1.3\phantom{0}$ & ${-}18.0\phantom{0}$ & $\phantom{0}0.13\phantom{0}$ & $\phantom{0}0.16\phantom{0}$ & $\phantom{0}0.15\phantom{0}$ & $\phantom{0}0.18\phantom{0}$ & $\phantom{0}0.19\phantom{0}$ & $\phantom{0}0.16\phantom{0}$ & $\phantom{0}89.2\phantom{0}$ & $\phantom{0}81.2\phantom{0}$ & $\phantom{0}72.1\phantom{0}$ & $\phantom{0}92.6\phantom{0}$ & $\phantom{0}91.8\phantom{0}$ & $\phantom{0}90.3\phantom{0}$ \\
 & \nopagebreak $\;J=50$  & $\phantom{0}{-}3.5\phantom{0}$ & ${-}19.1\phantom{0}$ & ${-}32.9\phantom{0}$ & $\phantom{0}{-}3.1\phantom{0}$ & $\phantom{0}{-}0.8\phantom{0}$ & ${-}12.9\phantom{0}$ & $\phantom{0}0.10\phantom{0}$ & $\phantom{0}0.13\phantom{0}$ & $\phantom{0}0.13\phantom{0}$ & $\phantom{0}0.14\phantom{0}$ & $\phantom{0}0.14\phantom{0}$ & $\phantom{0}0.13\phantom{0}$ & $\phantom{0}91.9\phantom{0}$ & $\phantom{0}84.6\phantom{0}$ & $\phantom{0}69.3\phantom{0}$ & $\phantom{0}93.1\phantom{0}$ & $\phantom{0}93.4\phantom{0}$ & $\phantom{0}91.5\phantom{0}$ \\
 & \nopagebreak $\;J=100$  & $\phantom{0}{-}2.3\phantom{0}$ & ${-}17.6\phantom{0}$ & ${-}31.7\phantom{0}$ & $\phantom{0}{-}2.6\phantom{0}$ & $\phantom{0}{-}1.5\phantom{0}$ & $\phantom{0}{-}7.7\phantom{0}$ & $\phantom{0}0.07\phantom{0}$ & $\phantom{0}0.10\phantom{0}$ & $\phantom{0}0.11\phantom{0}$ & $\phantom{0}0.10\phantom{0}$ & $\phantom{0}0.10\phantom{0}$ & $\phantom{0}0.09\phantom{0}$ & $\phantom{0}93.5\phantom{0}$ & $\phantom{0}84.1\phantom{0}$ & $\phantom{0}63.2\phantom{0}$ & $\phantom{0}93.9\phantom{0}$ & $\phantom{0}94.7\phantom{0}$ & $\phantom{0}92.9\phantom{0}$ \\
 & \nopagebreak $\;J=200$  & $\phantom{0}{-}0.4\phantom{0}$ & ${-}14.2\phantom{0}$ & ${-}29.1\phantom{0}$ & $\phantom{0}\phantom{-}0.3\phantom{0}$ & $\phantom{0}\phantom{-}0.2\phantom{0}$ & $\phantom{0}{-}2.4\phantom{0}$ & $\phantom{0}0.05\phantom{0}$ & $\phantom{0}0.07\phantom{0}$ & $\phantom{0}0.09\phantom{0}$ & $\phantom{0}0.07\phantom{0}$ & $\phantom{0}0.07\phantom{0}$ & $\phantom{0}0.06\phantom{0}$ & $\phantom{0}93.3\phantom{0}$ & $\phantom{0}86.6\phantom{0}$ & $\phantom{0}52.5\phantom{0}$ & $\phantom{0}95.1\phantom{0}$ & $\phantom{0}94.8\phantom{0}$ & $\phantom{0}95.3\phantom{0}$ \\
 & \nopagebreak $\;J=500$  & $\phantom{0}{-}0.2\phantom{0}$ & ${-}14.6\phantom{0}$ & ${-}29.2\phantom{0}$ & $\phantom{0}\phantom{-}0.0\phantom{0}$ & $\phantom{0}\phantom{-}0.2\phantom{0}$ & $\phantom{0}{-}1.0\phantom{0}$ & $\phantom{0}0.03\phantom{0}$ & $\phantom{0}0.06\phantom{0}$ & $\phantom{0}0.09\phantom{0}$ & $\phantom{0}0.04\phantom{0}$ & $\phantom{0}0.04\phantom{0}$ & $\phantom{0}0.04\phantom{0}$ & $\phantom{0}94.4\phantom{0}$ & $\phantom{0}76.2\phantom{0}$ & $\phantom{0}24.2\phantom{0}$ & $\phantom{0}94.8\phantom{0}$ & $\phantom{0}94.2\phantom{0}$ & $\phantom{0}94.7\phantom{0}$ \\
 & \nopagebreak $\;J=1000$  & $\phantom{0}{-}0.0\phantom{0}$ & ${-}14.1\phantom{0}$ & ${-}28.8\phantom{0}$ & $\phantom{0}\phantom{-}0.4\phantom{0}$ & $\phantom{0}\phantom{-}0.2\phantom{0}$ & $\phantom{0}{-}0.3\phantom{0}$ & $\phantom{0}0.02\phantom{0}$ & $\phantom{0}0.05\phantom{0}$ & $\phantom{0}0.08\phantom{0}$ & $\phantom{0}0.03\phantom{0}$ & $\phantom{0}0.03\phantom{0}$ & $\phantom{0}0.03\phantom{0}$ & $\phantom{0}95.3\phantom{0}$ & $\phantom{0}68.2\phantom{0}$ & $\phantom{0}\phantom{0}4.4\phantom{0}$ & $\phantom{0}94.8\phantom{0}$ & $\phantom{0}93.9\phantom{0}$ & $\phantom{0}95.0\phantom{0}$ \\
\multicolumn{4}{l}{$n=20$} \\  & \nopagebreak $\;J=30$  & $\phantom{0}{-}3.4\phantom{0}$ & ${-}19.0\phantom{0}$ & ${-}37.9\phantom{0}$ & $\phantom{0}{-}3.3\phantom{0}$ & $\phantom{0}{-}2.4\phantom{0}$ & ${-}15.3\phantom{0}$ & $\phantom{0}0.12\phantom{0}$ & $\phantom{0}0.15\phantom{0}$ & $\phantom{0}0.15\phantom{0}$ & $\phantom{0}0.16\phantom{0}$ & $\phantom{0}0.16\phantom{0}$ & $\phantom{0}0.15\phantom{0}$ & $\phantom{0}88.9\phantom{0}$ & $\phantom{0}79.9\phantom{0}$ & $\phantom{0}60.3\phantom{0}$ & $\phantom{0}92.1\phantom{0}$ & $\phantom{0}92.3\phantom{0}$ & $\phantom{0}89.9\phantom{0}$ \\
 & \nopagebreak $\;J=50$  & $\phantom{0}{-}1.5\phantom{0}$ & ${-}15.8\phantom{0}$ & ${-}35.3\phantom{0}$ & $\phantom{0}{-}0.8\phantom{0}$ & $\phantom{0}{-}0.8\phantom{0}$ & $\phantom{0}{-}8.5\phantom{0}$ & $\phantom{0}0.09\phantom{0}$ & $\phantom{0}0.11\phantom{0}$ & $\phantom{0}0.13\phantom{0}$ & $\phantom{0}0.12\phantom{0}$ & $\phantom{0}0.12\phantom{0}$ & $\phantom{0}0.11\phantom{0}$ & $\phantom{0}91.9\phantom{0}$ & $\phantom{0}84.6\phantom{0}$ & $\phantom{0}58.2\phantom{0}$ & $\phantom{0}94.3\phantom{0}$ & $\phantom{0}94.4\phantom{0}$ & $\phantom{0}93.3\phantom{0}$ \\
 & \nopagebreak $\;J=100$  & $\phantom{0}{-}0.5\phantom{0}$ & ${-}15.3\phantom{0}$ & ${-}34.8\phantom{0}$ & $\phantom{0}{-}0.6\phantom{0}$ & $\phantom{0}{-}0.4\phantom{0}$ & $\phantom{0}{-}4.7\phantom{0}$ & $\phantom{0}0.06\phantom{0}$ & $\phantom{0}0.09\phantom{0}$ & $\phantom{0}0.11\phantom{0}$ & $\phantom{0}0.09\phantom{0}$ & $\phantom{0}0.08\phantom{0}$ & $\phantom{0}0.08\phantom{0}$ & $\phantom{0}93.2\phantom{0}$ & $\phantom{0}83.6\phantom{0}$ & $\phantom{0}46.7\phantom{0}$ & $\phantom{0}93.5\phantom{0}$ & $\phantom{0}93.9\phantom{0}$ & $\phantom{0}93.2\phantom{0}$ \\
 & \nopagebreak $\;J=200$  & $\phantom{0}{-}0.5\phantom{0}$ & ${-}15.2\phantom{0}$ & ${-}34.7\phantom{0}$ & $\phantom{0}{-}0.7\phantom{0}$ & $\phantom{0}{-}0.7\phantom{0}$ & $\phantom{0}{-}2.7\phantom{0}$ & $\phantom{0}0.04\phantom{0}$ & $\phantom{0}0.07\phantom{0}$ & $\phantom{0}0.10\phantom{0}$ & $\phantom{0}0.06\phantom{0}$ & $\phantom{0}0.06\phantom{0}$ & $\phantom{0}0.06\phantom{0}$ & $\phantom{0}95.3\phantom{0}$ & $\phantom{0}83.4\phantom{0}$ & $\phantom{0}27.4\phantom{0}$ & $\phantom{0}94.8\phantom{0}$ & $\phantom{0}94.3\phantom{0}$ & $\phantom{0}94.5\phantom{0}$ \\
 & \nopagebreak $\;J=500$  & $\phantom{0}{-}0.3\phantom{0}$ & ${-}15.1\phantom{0}$ & ${-}34.4\phantom{0}$ & $\phantom{0}{-}0.3\phantom{0}$ & $\phantom{0}{-}0.4\phantom{0}$ & $\phantom{0}{-}1.2\phantom{0}$ & $\phantom{0}0.03\phantom{0}$ & $\phantom{0}0.05\phantom{0}$ & $\phantom{0}0.10\phantom{0}$ & $\phantom{0}0.04\phantom{0}$ & $\phantom{0}0.04\phantom{0}$ & $\phantom{0}0.04\phantom{0}$ & $\phantom{0}94.8\phantom{0}$ & $\phantom{0}73.1\phantom{0}$ & $\phantom{0}\phantom{0}4.5\phantom{0}$ & $\phantom{0}95.8\phantom{0}$ & $\phantom{0}94.4\phantom{0}$ & $\phantom{0}94.8\phantom{0}$ \\
 & \nopagebreak $\;J=1000$  & $\phantom{0}{-}0.0\phantom{0}$ & ${-}14.3\phantom{0}$ & ${-}33.9\phantom{0}$ & $\phantom{0}\phantom{-}0.2\phantom{0}$ & $\phantom{0}\phantom{-}0.1\phantom{0}$ & $\phantom{0}{-}0.2\phantom{0}$ & $\phantom{0}0.02\phantom{0}$ & $\phantom{0}0.05\phantom{0}$ & $\phantom{0}0.09\phantom{0}$ & $\phantom{0}0.03\phantom{0}$ & $\phantom{0}0.03\phantom{0}$ & $\phantom{0}0.03\phantom{0}$ & $\phantom{0}93.8\phantom{0}$ & $\phantom{0}60.0\phantom{0}$ & $\phantom{0}\phantom{0}0.1\phantom{0}$ & $\phantom{0}94.6\phantom{0}$ & $\phantom{0}95.5\phantom{0}$ & $\phantom{0}95.3\phantom{0}$ \\
[0.5ex]\hline\\[-1.6ex] 
\end{tabular}
\begin{tablenotes}[para,flushleft]{\footnotesize \textit{Note.} $n$ = cluster size; $J$ = number of clusters; CD = complete data sets; LD = listwise deletion; FCS-SL = single-level FCS; FCS-MAN = two-level FCS with manifest cluster means; FCS-LAT = two-level FCS with latent cluster means; JM = joint modeling.}\end{tablenotes}
\end{threeparttable}
\end{sidewaystable}
\begin{sidewaystable}
\begin{threeparttable}
\setlength{\tabcolsep}{1.2pt}
\renewcommand{\arraystretch}{0.95}
\footnotesize
\caption{\small Study 1: Bias (in \%), RMSE, and Coverage of the 95\% Confidence Interval for the Covariance of $y$ With $z$ ($\hat\sigma_{yz}$) With 40\% Missing Data (MAR, $\lambda=1$)}
\begin{tabular}{llcccccccccccccccccc}
\hline\\[-1.8ex]
& & \multicolumn{6}{c}{Bias (\%)} & \multicolumn{6}{c}{RMSE} & \multicolumn{6}{c}{Coverage (\%)} \\ \cmidrule(r){3-8}\cmidrule(r){9-14}\cmidrule(r){15-20}
 &  & CD & LD & \makecell{FCS-\\SL} & \makecell{FCS-\\MAN} & \makecell{FCS-\\LAT} & JM & CD & LD & \makecell{FCS-\\SL} & \makecell{FCS-\\MAN} & \makecell{FCS-\\LAT} & JM & CD & LD & \makecell{FCS-\\SL} & \makecell{FCS-\\MAN} & \makecell{FCS-\\LAT} & \multicolumn{1}{c}{JM} \\ 
[0.4ex]\hline\\[-1.8ex]
& & \multicolumn{18}{c}{Small intraclass correlation $(\rho_{Iy}=.10)$} \\[0.6ex]\hline\\[-1.8ex]
\multicolumn{4}{l}{$n=5$} \\  & \nopagebreak $\;J=30$  & $\phantom{0}{-}3.5\phantom{0}$ & ${-}58.1\phantom{0}$ & ${-}64.5\phantom{0}$ & $\phantom{0}\phantom{-}1.1\phantom{0}$ & $\phantom{0}{-}1.3\phantom{0}$ & ${-}51.1\phantom{0}$ & $\phantom{0}0.10\phantom{0}$ & $\phantom{0}0.12\phantom{0}$ & $\phantom{0}0.12\phantom{0}$ & $\phantom{0}0.21\phantom{0}$ & $\phantom{0}0.21\phantom{0}$ & $\phantom{0}0.13\phantom{0}$ & $\phantom{0}92.3\phantom{0}$ & $\phantom{0}88.6\phantom{0}$ & $\phantom{0}70.5\phantom{0}$ & $\phantom{0}95.3\phantom{0}$ & $\phantom{0}93.1\phantom{0}$ & $\phantom{0}94.7\phantom{0}$ \\
 & \nopagebreak $\;J=50$  & $\phantom{0}\phantom{-}0.7\phantom{0}$ & ${-}58.4\phantom{0}$ & ${-}64.2\phantom{0}$ & $\phantom{0}{-}0.2\phantom{0}$ & $\phantom{0}\phantom{-}8.9\phantom{0}$ & ${-}43.3\phantom{0}$ & $\phantom{0}0.08\phantom{0}$ & $\phantom{0}0.11\phantom{0}$ & $\phantom{0}0.12\phantom{0}$ & $\phantom{0}0.16\phantom{0}$ & $\phantom{0}0.16\phantom{0}$ & $\phantom{0}0.11\phantom{0}$ & $\phantom{0}92.3\phantom{0}$ & $\phantom{0}85.1\phantom{0}$ & $\phantom{0}60.9\phantom{0}$ & $\phantom{0}95.2\phantom{0}$ & $\phantom{0}92.2\phantom{0}$ & $\phantom{0}94.1\phantom{0}$ \\
 & \nopagebreak $\;J=100$  & $\phantom{0}{-}0.3\phantom{0}$ & ${-}58.0\phantom{0}$ & ${-}64.2\phantom{0}$ & $\phantom{0}\phantom{-}1.9\phantom{0}$ & $\phantom{-}13.1\phantom{0}$ & ${-}34.2\phantom{0}$ & $\phantom{0}0.06\phantom{0}$ & $\phantom{0}0.10\phantom{0}$ & $\phantom{0}0.11\phantom{0}$ & $\phantom{0}0.11\phantom{0}$ & $\phantom{0}0.11\phantom{0}$ & $\phantom{0}0.09\phantom{0}$ & $\phantom{0}94.1\phantom{0}$ & $\phantom{0}71.1\phantom{0}$ & $\phantom{0}38.7\phantom{0}$ & $\phantom{0}94.1\phantom{0}$ & $\phantom{0}93.4\phantom{0}$ & $\phantom{0}92.9\phantom{0}$ \\
 & \nopagebreak $\;J=200$  & $\phantom{0}{-}0.2\phantom{0}$ & ${-}58.2\phantom{0}$ & ${-}64.5\phantom{0}$ & $\phantom{0}{-}0.3\phantom{0}$ & $\phantom{0}\phantom{-}9.0\phantom{0}$ & ${-}24.5\phantom{0}$ & $\phantom{0}0.04\phantom{0}$ & $\phantom{0}0.10\phantom{0}$ & $\phantom{0}0.11\phantom{0}$ & $\phantom{0}0.07\phantom{0}$ & $\phantom{0}0.08\phantom{0}$ & $\phantom{0}0.07\phantom{0}$ & $\phantom{0}94.6\phantom{0}$ & $\phantom{0}42.3\phantom{0}$ & $\phantom{0}12.6\phantom{0}$ & $\phantom{0}93.6\phantom{0}$ & $\phantom{0}91.5\phantom{0}$ & $\phantom{0}91.5\phantom{0}$ \\
 & \nopagebreak $\;J=500$  & $\phantom{0}\phantom{-}0.0\phantom{0}$ & ${-}57.3\phantom{0}$ & ${-}63.7\phantom{0}$ & $\phantom{0}\phantom{-}1.0\phantom{0}$ & $\phantom{0}\phantom{-}5.1\phantom{0}$ & ${-}12.9\phantom{0}$ & $\phantom{0}0.02\phantom{0}$ & $\phantom{0}0.09\phantom{0}$ & $\phantom{0}0.10\phantom{0}$ & $\phantom{0}0.05\phantom{0}$ & $\phantom{0}0.05\phantom{0}$ & $\phantom{0}0.05\phantom{0}$ & $\phantom{0}95.3\phantom{0}$ & $\phantom{0}\phantom{0}6.3\phantom{0}$ & $\phantom{0}\phantom{0}0.3\phantom{0}$ & $\phantom{0}93.0\phantom{0}$ & $\phantom{0}90.8\phantom{0}$ & $\phantom{0}91.8\phantom{0}$ \\
 & \nopagebreak $\;J=1000$  & $\phantom{0}{-}0.7\phantom{0}$ & ${-}58.4\phantom{0}$ & ${-}64.8\phantom{0}$ & $\phantom{0}{-}1.4\phantom{0}$ & $\phantom{0}\phantom{-}0.0\phantom{0}$ & $\phantom{0}{-}9.6\phantom{0}$ & $\phantom{0}0.02\phantom{0}$ & $\phantom{0}0.09\phantom{0}$ & $\phantom{0}0.10\phantom{0}$ & $\phantom{0}0.03\phantom{0}$ & $\phantom{0}0.03\phantom{0}$ & $\phantom{0}0.03\phantom{0}$ & $\phantom{0}95.8\phantom{0}$ & $\phantom{0}\phantom{0}0.1\phantom{0}$ & $\phantom{0}\phantom{0}0.0\phantom{0}$ & $\phantom{0}93.9\phantom{0}$ & $\phantom{0}92.4\phantom{0}$ & $\phantom{0}92.9\phantom{0}$ \\
\multicolumn{4}{l}{$n=20$} \\  & \nopagebreak $\;J=30$  & $\phantom{0}{-}4.5\phantom{0}$ & ${-}60.0\phantom{0}$ & ${-}71.3\phantom{0}$ & $\phantom{0}{-}5.7\phantom{0}$ & $\phantom{0}\phantom{-}3.5\phantom{0}$ & ${-}51.0\phantom{0}$ & $\phantom{0}0.07\phantom{0}$ & $\phantom{0}0.11\phantom{0}$ & $\phantom{0}0.12\phantom{0}$ & $\phantom{0}0.14\phantom{0}$ & $\phantom{0}0.15\phantom{0}$ & $\phantom{0}0.11\phantom{0}$ & $\phantom{0}91.0\phantom{0}$ & $\phantom{0}52.2\phantom{0}$ & $\phantom{0}33.4\phantom{0}$ & $\phantom{0}93.6\phantom{0}$ & $\phantom{0}92.8\phantom{0}$ & $\phantom{0}89.6\phantom{0}$ \\
 & \nopagebreak $\;J=50$  & $\phantom{0}{-}1.4\phantom{0}$ & ${-}57.9\phantom{0}$ & ${-}69.9\phantom{0}$ & $\phantom{0}{-}2.0\phantom{0}$ & $\phantom{0}\phantom{-}6.3\phantom{0}$ & ${-}40.0\phantom{0}$ & $\phantom{0}0.06\phantom{0}$ & $\phantom{0}0.10\phantom{0}$ & $\phantom{0}0.12\phantom{0}$ & $\phantom{0}0.11\phantom{0}$ & $\phantom{0}0.12\phantom{0}$ & $\phantom{0}0.09\phantom{0}$ & $\phantom{0}93.3\phantom{0}$ & $\phantom{0}42.2\phantom{0}$ & $\phantom{0}21.7\phantom{0}$ & $\phantom{0}94.3\phantom{0}$ & $\phantom{0}92.5\phantom{0}$ & $\phantom{0}89.1\phantom{0}$ \\
 & \nopagebreak $\;J=100$  & $\phantom{0}{-}1.6\phantom{0}$ & ${-}58.9\phantom{0}$ & ${-}70.5\phantom{0}$ & $\phantom{0}{-}1.9\phantom{0}$ & $\phantom{0}\phantom{-}1.3\phantom{0}$ & ${-}29.9\phantom{0}$ & $\phantom{0}0.04\phantom{0}$ & $\phantom{0}0.10\phantom{0}$ & $\phantom{0}0.11\phantom{0}$ & $\phantom{0}0.07\phantom{0}$ & $\phantom{0}0.08\phantom{0}$ & $\phantom{0}0.07\phantom{0}$ & $\phantom{0}93.1\phantom{0}$ & $\phantom{0}20.7\phantom{0}$ & $\phantom{0}\phantom{0}4.5\phantom{0}$ & $\phantom{0}93.7\phantom{0}$ & $\phantom{0}92.7\phantom{0}$ & $\phantom{0}89.1\phantom{0}$ \\
 & \nopagebreak $\;J=200$  & $\phantom{0}{-}0.4\phantom{0}$ & ${-}58.1\phantom{0}$ & ${-}70.0\phantom{0}$ & $\phantom{0}{-}0.6\phantom{0}$ & $\phantom{0}\phantom{-}0.4\phantom{0}$ & ${-}18.4\phantom{0}$ & $\phantom{0}0.03\phantom{0}$ & $\phantom{0}0.09\phantom{0}$ & $\phantom{0}0.11\phantom{0}$ & $\phantom{0}0.05\phantom{0}$ & $\phantom{0}0.05\phantom{0}$ & $\phantom{0}0.05\phantom{0}$ & $\phantom{0}94.4\phantom{0}$ & $\phantom{0}\phantom{0}3.9\phantom{0}$ & $\phantom{0}\phantom{0}0.1\phantom{0}$ & $\phantom{0}93.8\phantom{0}$ & $\phantom{0}93.8\phantom{0}$ & $\phantom{0}90.3\phantom{0}$ \\
 & \nopagebreak $\;J=500$  & $\phantom{0}{-}0.4\phantom{0}$ & ${-}57.9\phantom{0}$ & ${-}69.8\phantom{0}$ & $\phantom{0}{-}0.4\phantom{0}$ & $\phantom{0}{-}0.3\phantom{0}$ & $\phantom{0}{-}9.2\phantom{0}$ & $\phantom{0}0.02\phantom{0}$ & $\phantom{0}0.09\phantom{0}$ & $\phantom{0}0.11\phantom{0}$ & $\phantom{0}0.03\phantom{0}$ & $\phantom{0}0.03\phantom{0}$ & $\phantom{0}0.03\phantom{0}$ & $\phantom{0}94.4\phantom{0}$ & $\phantom{0}\phantom{0}0.0\phantom{0}$ & $\phantom{0}\phantom{0}0.0\phantom{0}$ & $\phantom{0}94.5\phantom{0}$ & $\phantom{0}92.8\phantom{0}$ & $\phantom{0}92.9\phantom{0}$ \\
 & \nopagebreak $\;J=1000$  & $\phantom{0}{-}0.2\phantom{0}$ & ${-}58.2\phantom{0}$ & ${-}69.9\phantom{0}$ & $\phantom{0}{-}1.0\phantom{0}$ & $\phantom{0}{-}0.9\phantom{0}$ & $\phantom{0}{-}5.4\phantom{0}$ & $\phantom{0}0.01\phantom{0}$ & $\phantom{0}0.09\phantom{0}$ & $\phantom{0}0.11\phantom{0}$ & $\phantom{0}0.02\phantom{0}$ & $\phantom{0}0.02\phantom{0}$ & $\phantom{0}0.02\phantom{0}$ & $\phantom{0}94.2\phantom{0}$ & $\phantom{0}\phantom{0}0.0\phantom{0}$ & $\phantom{0}\phantom{0}0.0\phantom{0}$ & $\phantom{0}94.0\phantom{0}$ & $\phantom{0}93.0\phantom{0}$ & $\phantom{0}93.0\phantom{0}$ \\
[0.5ex]\hline\\[-1.6ex] 
& & \multicolumn{18}{c}{Moderate intraclass correlation $(\rho_{Iy}=.30)$} \\[0.6ex]\hline\\[-1.8ex]
\multicolumn{4}{l}{$n=5$} \\  & \nopagebreak $\;J=30$  & $\phantom{0}{-}2.8\phantom{0}$ & ${-}60.4\phantom{0}$ & ${-}59.1\phantom{0}$ & $\phantom{0}{-}4.8\phantom{0}$ & $\phantom{0}\phantom{-}6.7\phantom{0}$ & ${-}37.9\phantom{0}$ & $\phantom{0}0.13\phantom{0}$ & $\phantom{0}0.19\phantom{0}$ & $\phantom{0}0.19\phantom{0}$ & $\phantom{0}0.26\phantom{0}$ & $\phantom{0}0.28\phantom{0}$ & $\phantom{0}0.20\phantom{0}$ & $\phantom{0}89.5\phantom{0}$ & $\phantom{0}50.7\phantom{0}$ & $\phantom{0}56.7\phantom{0}$ & $\phantom{0}93.6\phantom{0}$ & $\phantom{0}92.4\phantom{0}$ & $\phantom{0}90.9\phantom{0}$ \\
 & \nopagebreak $\;J=50$  & $\phantom{0}{-}2.6\phantom{0}$ & ${-}60.2\phantom{0}$ & ${-}58.6\phantom{0}$ & $\phantom{0}{-}3.5\phantom{0}$ & $\phantom{0}\phantom{-}6.1\phantom{0}$ & ${-}28.3\phantom{0}$ & $\phantom{0}0.10\phantom{0}$ & $\phantom{0}0.18\phantom{0}$ & $\phantom{0}0.18\phantom{0}$ & $\phantom{0}0.19\phantom{0}$ & $\phantom{0}0.21\phantom{0}$ & $\phantom{0}0.17\phantom{0}$ & $\phantom{0}91.9\phantom{0}$ & $\phantom{0}38.4\phantom{0}$ & $\phantom{0}44.5\phantom{0}$ & $\phantom{0}94.2\phantom{0}$ & $\phantom{0}91.7\phantom{0}$ & $\phantom{0}91.8\phantom{0}$ \\
 & \nopagebreak $\;J=100$  & $\phantom{0}{-}0.6\phantom{0}$ & ${-}58.0\phantom{0}$ & ${-}56.4\phantom{0}$ & $\phantom{0}\phantom{-}0.5\phantom{0}$ & $\phantom{0}\phantom{-}3.9\phantom{0}$ & ${-}13.8\phantom{0}$ & $\phantom{0}0.07\phantom{0}$ & $\phantom{0}0.17\phantom{0}$ & $\phantom{0}0.17\phantom{0}$ & $\phantom{0}0.13\phantom{0}$ & $\phantom{0}0.14\phantom{0}$ & $\phantom{0}0.12\phantom{0}$ & $\phantom{0}92.7\phantom{0}$ & $\phantom{0}21.7\phantom{0}$ & $\phantom{0}29.3\phantom{0}$ & $\phantom{0}93.9\phantom{0}$ & $\phantom{0}92.6\phantom{0}$ & $\phantom{0}93.0\phantom{0}$ \\
 & \nopagebreak $\;J=200$  & $\phantom{0}{-}0.3\phantom{0}$ & ${-}58.4\phantom{0}$ & ${-}56.7\phantom{0}$ & $\phantom{0}{-}0.6\phantom{0}$ & $\phantom{0}\phantom{-}0.5\phantom{0}$ & $\phantom{0}{-}7.8\phantom{0}$ & $\phantom{0}0.05\phantom{0}$ & $\phantom{0}0.16\phantom{0}$ & $\phantom{0}0.16\phantom{0}$ & $\phantom{0}0.09\phantom{0}$ & $\phantom{0}0.09\phantom{0}$ & $\phantom{0}0.09\phantom{0}$ & $\phantom{0}94.3\phantom{0}$ & $\phantom{0}\phantom{0}4.3\phantom{0}$ & $\phantom{0}\phantom{0}7.5\phantom{0}$ & $\phantom{0}94.3\phantom{0}$ & $\phantom{0}93.8\phantom{0}$ & $\phantom{0}93.2\phantom{0}$ \\
 & \nopagebreak $\;J=500$  & $\phantom{0}{-}0.2\phantom{0}$ & ${-}57.9\phantom{0}$ & ${-}56.1\phantom{0}$ & $\phantom{0}{-}0.3\phantom{0}$ & $\phantom{0}\phantom{-}0.3\phantom{0}$ & $\phantom{0}{-}3.0\phantom{0}$ & $\phantom{0}0.03\phantom{0}$ & $\phantom{0}0.16\phantom{0}$ & $\phantom{0}0.16\phantom{0}$ & $\phantom{0}0.06\phantom{0}$ & $\phantom{0}0.06\phantom{0}$ & $\phantom{0}0.06\phantom{0}$ & $\phantom{0}95.3\phantom{0}$ & $\phantom{0}\phantom{0}0.0\phantom{0}$ & $\phantom{0}\phantom{0}0.0\phantom{0}$ & $\phantom{0}93.6\phantom{0}$ & $\phantom{0}93.4\phantom{0}$ & $\phantom{0}94.6\phantom{0}$ \\
 & \nopagebreak $\;J=1000$  & $\phantom{0}\phantom{-}0.1\phantom{0}$ & ${-}57.5\phantom{0}$ & ${-}55.8\phantom{0}$ & $\phantom{0}\phantom{-}0.8\phantom{0}$ & $\phantom{0}\phantom{-}0.9\phantom{0}$ & $\phantom{0}{-}0.6\phantom{0}$ & $\phantom{0}0.02\phantom{0}$ & $\phantom{0}0.16\phantom{0}$ & $\phantom{0}0.15\phantom{0}$ & $\phantom{0}0.04\phantom{0}$ & $\phantom{0}0.04\phantom{0}$ & $\phantom{0}0.04\phantom{0}$ & $\phantom{0}94.8\phantom{0}$ & $\phantom{0}\phantom{0}0.0\phantom{0}$ & $\phantom{0}\phantom{0}0.0\phantom{0}$ & $\phantom{0}93.5\phantom{0}$ & $\phantom{0}93.3\phantom{0}$ & $\phantom{0}94.4\phantom{0}$ \\
\multicolumn{4}{l}{$n=20$} \\  & \nopagebreak $\;J=30$  & $\phantom{0}{-}3.2\phantom{0}$ & ${-}60.1\phantom{0}$ & ${-}63.5\phantom{0}$ & $\phantom{0}{-}4.1\phantom{0}$ & $\phantom{0}{-}1.8\phantom{0}$ & ${-}30.8\phantom{0}$ & $\phantom{0}0.12\phantom{0}$ & $\phantom{0}0.19\phantom{0}$ & $\phantom{0}0.19\phantom{0}$ & $\phantom{0}0.21\phantom{0}$ & $\phantom{0}0.21\phantom{0}$ & $\phantom{0}0.18\phantom{0}$ & $\phantom{0}89.5\phantom{0}$ & $\phantom{0}38.6\phantom{0}$ & $\phantom{0}35.3\phantom{0}$ & $\phantom{0}93.8\phantom{0}$ & $\phantom{0}93.9\phantom{0}$ & $\phantom{0}91.8\phantom{0}$ \\
 & \nopagebreak $\;J=50$  & $\phantom{0}{-}3.5\phantom{0}$ & ${-}59.7\phantom{0}$ & ${-}63.4\phantom{0}$ & $\phantom{0}{-}3.2\phantom{0}$ & $\phantom{0}{-}2.0\phantom{0}$ & ${-}22.3\phantom{0}$ & $\phantom{0}0.09\phantom{0}$ & $\phantom{0}0.18\phantom{0}$ & $\phantom{0}0.18\phantom{0}$ & $\phantom{0}0.16\phantom{0}$ & $\phantom{0}0.16\phantom{0}$ & $\phantom{0}0.14\phantom{0}$ & $\phantom{0}91.7\phantom{0}$ & $\phantom{0}28.4\phantom{0}$ & $\phantom{0}23.3\phantom{0}$ & $\phantom{0}93.5\phantom{0}$ & $\phantom{0}93.4\phantom{0}$ & $\phantom{0}91.1\phantom{0}$ \\
 & \nopagebreak $\;J=100$  & $\phantom{0}{-}1.1\phantom{0}$ & ${-}59.2\phantom{0}$ & ${-}62.9\phantom{0}$ & $\phantom{0}{-}1.4\phantom{0}$ & $\phantom{0}{-}1.4\phantom{0}$ & ${-}11.8\phantom{0}$ & $\phantom{0}0.06\phantom{0}$ & $\phantom{0}0.17\phantom{0}$ & $\phantom{0}0.18\phantom{0}$ & $\phantom{0}0.11\phantom{0}$ & $\phantom{0}0.11\phantom{0}$ & $\phantom{0}0.11\phantom{0}$ & $\phantom{0}93.8\phantom{0}$ & $\phantom{0}12.6\phantom{0}$ & $\phantom{0}\phantom{0}7.0\phantom{0}$ & $\phantom{0}93.5\phantom{0}$ & $\phantom{0}93.8\phantom{0}$ & $\phantom{0}93.3\phantom{0}$ \\
 & \nopagebreak $\;J=200$  & $\phantom{0}{-}1.1\phantom{0}$ & ${-}58.5\phantom{0}$ & ${-}62.2\phantom{0}$ & $\phantom{0}{-}1.7\phantom{0}$ & $\phantom{0}{-}1.5\phantom{0}$ & $\phantom{0}{-}7.1\phantom{0}$ & $\phantom{0}0.05\phantom{0}$ & $\phantom{0}0.16\phantom{0}$ & $\phantom{0}0.17\phantom{0}$ & $\phantom{0}0.08\phantom{0}$ & $\phantom{0}0.08\phantom{0}$ & $\phantom{0}0.08\phantom{0}$ & $\phantom{0}91.9\phantom{0}$ & $\phantom{0}\phantom{0}1.1\phantom{0}$ & $\phantom{0}\phantom{0}0.1\phantom{0}$ & $\phantom{0}93.7\phantom{0}$ & $\phantom{0}94.3\phantom{0}$ & $\phantom{0}94.2\phantom{0}$ \\
 & \nopagebreak $\;J=500$  & $\phantom{0}\phantom{-}0.3\phantom{0}$ & ${-}57.4\phantom{0}$ & ${-}61.2\phantom{0}$ & $\phantom{0}\phantom{-}1.2\phantom{0}$ & $\phantom{0}\phantom{-}1.2\phantom{0}$ & $\phantom{0}{-}1.2\phantom{0}$ & $\phantom{0}0.03\phantom{0}$ & $\phantom{0}0.16\phantom{0}$ & $\phantom{0}0.17\phantom{0}$ & $\phantom{0}0.05\phantom{0}$ & $\phantom{0}0.05\phantom{0}$ & $\phantom{0}0.05\phantom{0}$ & $\phantom{0}94.5\phantom{0}$ & $\phantom{0}\phantom{0}0.0\phantom{0}$ & $\phantom{0}\phantom{0}0.0\phantom{0}$ & $\phantom{0}94.4\phantom{0}$ & $\phantom{0}93.8\phantom{0}$ & $\phantom{0}95.8\phantom{0}$ \\
 & \nopagebreak $\;J=1000$  & $\phantom{0}\phantom{-}0.0\phantom{0}$ & ${-}57.5\phantom{0}$ & ${-}61.3\phantom{0}$ & $\phantom{0}\phantom{-}0.6\phantom{0}$ & $\phantom{0}\phantom{-}0.4\phantom{0}$ & $\phantom{0}{-}0.7\phantom{0}$ & $\phantom{0}0.02\phantom{0}$ & $\phantom{0}0.16\phantom{0}$ & $\phantom{0}0.17\phantom{0}$ & $\phantom{0}0.04\phantom{0}$ & $\phantom{0}0.04\phantom{0}$ & $\phantom{0}0.04\phantom{0}$ & $\phantom{0}96.2\phantom{0}$ & $\phantom{0}\phantom{0}0.0\phantom{0}$ & $\phantom{0}\phantom{0}0.0\phantom{0}$ & $\phantom{0}92.3\phantom{0}$ & $\phantom{0}92.1\phantom{0}$ & $\phantom{0}94.3\phantom{0}$ \\
[0.5ex]\hline\\[-1.6ex] 
\end{tabular}
\begin{tablenotes}[para,flushleft]{\footnotesize \textit{Note.} $n$ = cluster size; $J$ = number of clusters; CD = complete data sets; LD = listwise deletion; FCS-SL = single-level FCS; FCS-MAN = two-level FCS with manifest cluster means; FCS-LAT = two-level FCS with latent cluster means; JM = joint modeling.}\end{tablenotes}
\end{threeparttable}
\end{sidewaystable}
\begin{sidewaystable}
\begin{threeparttable}
\setlength{\tabcolsep}{1.2pt}
\renewcommand{\arraystretch}{0.95}
\footnotesize
\caption{\small Study 1: Bias (in \%), RMSE, and Coverage of the 95\% Confidence Interval for the Regression Coefficient of $y$ on $z$ ($\hat\beta_{yz}$) With 20\% Missing Data (MCAR, $\lambda=0$)}
\begin{tabular}{llcccccccccccccccccc}
\hline\\[-1.8ex]
& & \multicolumn{6}{c}{Bias (\%)} & \multicolumn{6}{c}{RMSE} & \multicolumn{6}{c}{Coverage (\%)} \\ \cmidrule(r){3-8}\cmidrule(r){9-14}\cmidrule(r){15-20}
 &  & CD & LD & \makecell{FCS-\\SL} & \makecell{FCS-\\MAN} & \makecell{FCS-\\LAT} & JM & CD & LD & \makecell{FCS-\\SL} & \makecell{FCS-\\MAN} & \makecell{FCS-\\LAT} & JM & CD & LD & \makecell{FCS-\\SL} & \makecell{FCS-\\MAN} & \makecell{FCS-\\LAT} & \multicolumn{1}{c}{JM} \\ 
[0.4ex]\hline\\[-1.8ex]
& & \multicolumn{18}{c}{Small intraclass correlation $(\rho_{Iy}=.10)$} \\[0.6ex]\hline\\[-1.8ex]
\multicolumn{4}{l}{$n=5$} \\  & \nopagebreak $\;J=30$  & ${-}1.8\phantom{0}$ & ${-}1.0\phantom{0}$ & $\phantom{-}0.4\phantom{0}$ & ${-}4.2\phantom{0}$ & ${-}1.4\phantom{0}$ & ${-}9.5\phantom{0}$ & $\phantom{0}0.10\phantom{0}$ & $\phantom{0}0.11\phantom{0}$ & $\phantom{0}0.11\phantom{0}$ & $\phantom{0}0.11\phantom{0}$ & $\phantom{0}0.11\phantom{0}$ & $\phantom{0}0.10\phantom{0}$ & $\phantom{0}90.5\phantom{0}$ & $\phantom{0}89.0\phantom{0}$ & $\phantom{0}89.4\phantom{0}$ & $\phantom{0}90.7\phantom{0}$ & $\phantom{0}90.8\phantom{0}$ & $\phantom{0}93.2\phantom{0}$ \\
 & \nopagebreak $\;J=50$  & ${-}0.8\phantom{0}$ & ${-}1.5\phantom{0}$ & $\phantom{-}0.2\phantom{0}$ & ${-}3.1\phantom{0}$ & ${-}0.9\phantom{0}$ & ${-}8.1\phantom{0}$ & $\phantom{0}0.07\phantom{0}$ & $\phantom{0}0.08\phantom{0}$ & $\phantom{0}0.08\phantom{0}$ & $\phantom{0}0.08\phantom{0}$ & $\phantom{0}0.08\phantom{0}$ & $\phantom{0}0.08\phantom{0}$ & $\phantom{0}92.1\phantom{0}$ & $\phantom{0}92.1\phantom{0}$ & $\phantom{0}92.1\phantom{0}$ & $\phantom{0}93.0\phantom{0}$ & $\phantom{0}92.1\phantom{0}$ & $\phantom{0}94.6\phantom{0}$ \\
 & \nopagebreak $\;J=100$  & $\phantom{-}0.9\phantom{0}$ & $\phantom{-}1.3\phantom{0}$ & $\phantom{-}3.0\phantom{0}$ & $\phantom{-}0.3\phantom{0}$ & $\phantom{-}1.9\phantom{0}$ & ${-}3.2\phantom{0}$ & $\phantom{0}0.05\phantom{0}$ & $\phantom{0}0.06\phantom{0}$ & $\phantom{0}0.06\phantom{0}$ & $\phantom{0}0.06\phantom{0}$ & $\phantom{0}0.06\phantom{0}$ & $\phantom{0}0.06\phantom{0}$ & $\phantom{0}93.7\phantom{0}$ & $\phantom{0}93.3\phantom{0}$ & $\phantom{0}92.7\phantom{0}$ & $\phantom{0}94.3\phantom{0}$ & $\phantom{0}93.4\phantom{0}$ & $\phantom{0}94.9\phantom{0}$ \\
 & \nopagebreak $\;J=200$  & $\phantom{-}0.5\phantom{0}$ & $\phantom{-}0.4\phantom{0}$ & $\phantom{-}2.2\phantom{0}$ & ${-}0.1\phantom{0}$ & $\phantom{-}0.9\phantom{0}$ & ${-}2.4\phantom{0}$ & $\phantom{0}0.04\phantom{0}$ & $\phantom{0}0.04\phantom{0}$ & $\phantom{0}0.04\phantom{0}$ & $\phantom{0}0.04\phantom{0}$ & $\phantom{0}0.04\phantom{0}$ & $\phantom{0}0.04\phantom{0}$ & $\phantom{0}93.3\phantom{0}$ & $\phantom{0}94.1\phantom{0}$ & $\phantom{0}93.9\phantom{0}$ & $\phantom{0}93.7\phantom{0}$ & $\phantom{0}94.3\phantom{0}$ & $\phantom{0}94.8\phantom{0}$ \\
 & \nopagebreak $\;J=500$  & ${-}0.1\phantom{0}$ & $\phantom{-}0.0\phantom{0}$ & $\phantom{-}1.9\phantom{0}$ & $\phantom{-}0.0\phantom{0}$ & $\phantom{-}0.3\phantom{0}$ & ${-}1.3\phantom{0}$ & $\phantom{0}0.02\phantom{0}$ & $\phantom{0}0.03\phantom{0}$ & $\phantom{0}0.03\phantom{0}$ & $\phantom{0}0.03\phantom{0}$ & $\phantom{0}0.03\phantom{0}$ & $\phantom{0}0.03\phantom{0}$ & $\phantom{0}93.8\phantom{0}$ & $\phantom{0}94.5\phantom{0}$ & $\phantom{0}93.5\phantom{0}$ & $\phantom{0}94.1\phantom{0}$ & $\phantom{0}93.7\phantom{0}$ & $\phantom{0}94.8\phantom{0}$ \\
 & \nopagebreak $\;J=1000$  & $\phantom{-}0.0\phantom{0}$ & ${-}0.2\phantom{0}$ & $\phantom{-}1.6\phantom{0}$ & ${-}0.3\phantom{0}$ & ${-}0.2\phantom{0}$ & ${-}1.0\phantom{0}$ & $\phantom{0}0.02\phantom{0}$ & $\phantom{0}0.02\phantom{0}$ & $\phantom{0}0.02\phantom{0}$ & $\phantom{0}0.02\phantom{0}$ & $\phantom{0}0.02\phantom{0}$ & $\phantom{0}0.02\phantom{0}$ & $\phantom{0}94.1\phantom{0}$ & $\phantom{0}94.6\phantom{0}$ & $\phantom{0}93.6\phantom{0}$ & $\phantom{0}94.0\phantom{0}$ & $\phantom{0}94.0\phantom{0}$ & $\phantom{0}94.9\phantom{0}$ \\
\multicolumn{4}{l}{$n=20$} \\  & \nopagebreak $\;J=30$  & ${-}1.0\phantom{0}$ & ${-}0.4\phantom{0}$ & $\phantom{-}1.9\phantom{0}$ & ${-}3.1\phantom{0}$ & ${-}2.7\phantom{0}$ & ${-}8.2\phantom{0}$ & $\phantom{0}0.07\phantom{0}$ & $\phantom{0}0.08\phantom{0}$ & $\phantom{0}0.08\phantom{0}$ & $\phantom{0}0.07\phantom{0}$ & $\phantom{0}0.07\phantom{0}$ & $\phantom{0}0.07\phantom{0}$ & $\phantom{0}91.1\phantom{0}$ & $\phantom{0}89.6\phantom{0}$ & $\phantom{0}89.8\phantom{0}$ & $\phantom{0}91.6\phantom{0}$ & $\phantom{0}91.0\phantom{0}$ & $\phantom{0}94.4\phantom{0}$ \\
 & \nopagebreak $\;J=50$  & $\phantom{-}0.6\phantom{0}$ & $\phantom{-}0.5\phantom{0}$ & $\phantom{-}2.9\phantom{0}$ & ${-}1.0\phantom{0}$ & ${-}0.8\phantom{0}$ & ${-}5.3\phantom{0}$ & $\phantom{0}0.05\phantom{0}$ & $\phantom{0}0.06\phantom{0}$ & $\phantom{0}0.06\phantom{0}$ & $\phantom{0}0.05\phantom{0}$ & $\phantom{0}0.05\phantom{0}$ & $\phantom{0}0.05\phantom{0}$ & $\phantom{0}94.2\phantom{0}$ & $\phantom{0}92.3\phantom{0}$ & $\phantom{0}92.0\phantom{0}$ & $\phantom{0}93.6\phantom{0}$ & $\phantom{0}94.1\phantom{0}$ & $\phantom{0}94.3\phantom{0}$ \\
 & \nopagebreak $\;J=100$  & $\phantom{-}0.1\phantom{0}$ & $\phantom{-}0.1\phantom{0}$ & $\phantom{-}2.4\phantom{0}$ & ${-}0.7\phantom{0}$ & ${-}0.6\phantom{0}$ & ${-}3.3\phantom{0}$ & $\phantom{0}0.04\phantom{0}$ & $\phantom{0}0.04\phantom{0}$ & $\phantom{0}0.04\phantom{0}$ & $\phantom{0}0.04\phantom{0}$ & $\phantom{0}0.04\phantom{0}$ & $\phantom{0}0.04\phantom{0}$ & $\phantom{0}94.0\phantom{0}$ & $\phantom{0}94.1\phantom{0}$ & $\phantom{0}93.5\phantom{0}$ & $\phantom{0}93.9\phantom{0}$ & $\phantom{0}94.3\phantom{0}$ & $\phantom{0}94.8\phantom{0}$ \\
 & \nopagebreak $\;J=200$  & $\phantom{-}0.1\phantom{0}$ & ${-}0.2\phantom{0}$ & $\phantom{-}2.2\phantom{0}$ & ${-}0.3\phantom{0}$ & ${-}0.1\phantom{0}$ & ${-}1.6\phantom{0}$ & $\phantom{0}0.03\phantom{0}$ & $\phantom{0}0.03\phantom{0}$ & $\phantom{0}0.03\phantom{0}$ & $\phantom{0}0.03\phantom{0}$ & $\phantom{0}0.03\phantom{0}$ & $\phantom{0}0.03\phantom{0}$ & $\phantom{0}94.1\phantom{0}$ & $\phantom{0}93.6\phantom{0}$ & $\phantom{0}93.2\phantom{0}$ & $\phantom{0}93.8\phantom{0}$ & $\phantom{0}93.2\phantom{0}$ & $\phantom{0}93.8\phantom{0}$ \\
 & \nopagebreak $\;J=500$  & ${-}0.3\phantom{0}$ & ${-}0.4\phantom{0}$ & $\phantom{-}1.9\phantom{0}$ & ${-}0.7\phantom{0}$ & ${-}0.6\phantom{0}$ & ${-}1.1\phantom{0}$ & $\phantom{0}0.02\phantom{0}$ & $\phantom{0}0.02\phantom{0}$ & $\phantom{0}0.02\phantom{0}$ & $\phantom{0}0.02\phantom{0}$ & $\phantom{0}0.02\phantom{0}$ & $\phantom{0}0.02\phantom{0}$ & $\phantom{0}94.2\phantom{0}$ & $\phantom{0}93.5\phantom{0}$ & $\phantom{0}92.0\phantom{0}$ & $\phantom{0}93.7\phantom{0}$ & $\phantom{0}94.2\phantom{0}$ & $\phantom{0}93.7\phantom{0}$ \\
 & \nopagebreak $\;J=1000$  & ${-}0.1\phantom{0}$ & ${-}0.2\phantom{0}$ & $\phantom{-}2.1\phantom{0}$ & ${-}0.1\phantom{0}$ & ${-}0.1\phantom{0}$ & ${-}0.5\phantom{0}$ & $\phantom{0}0.01\phantom{0}$ & $\phantom{0}0.01\phantom{0}$ & $\phantom{0}0.01\phantom{0}$ & $\phantom{0}0.01\phantom{0}$ & $\phantom{0}0.01\phantom{0}$ & $\phantom{0}0.01\phantom{0}$ & $\phantom{0}94.8\phantom{0}$ & $\phantom{0}94.9\phantom{0}$ & $\phantom{0}93.9\phantom{0}$ & $\phantom{0}94.4\phantom{0}$ & $\phantom{0}94.7\phantom{0}$ & $\phantom{0}94.9\phantom{0}$ \\
[0.5ex]\hline\\[-1.6ex] 
& & \multicolumn{18}{c}{Moderate intraclass correlation $(\rho_{Iy}=.30)$} \\[0.6ex]\hline\\[-1.8ex]
\multicolumn{4}{l}{$n=5$} \\  & \nopagebreak $\;J=30$  & ${-}0.7\phantom{0}$ & ${-}2.1\phantom{0}$ & $\phantom{-}3.1\phantom{0}$ & ${-}4.2\phantom{0}$ & ${-}3.2\phantom{0}$ & ${-}6.7\phantom{0}$ & $\phantom{0}0.11\phantom{0}$ & $\phantom{0}0.13\phantom{0}$ & $\phantom{0}0.14\phantom{0}$ & $\phantom{0}0.13\phantom{0}$ & $\phantom{0}0.13\phantom{0}$ & $\phantom{0}0.13\phantom{0}$ & $\phantom{0}91.9\phantom{0}$ & $\phantom{0}90.5\phantom{0}$ & $\phantom{0}90.2\phantom{0}$ & $\phantom{0}91.8\phantom{0}$ & $\phantom{0}91.5\phantom{0}$ & $\phantom{0}93.0\phantom{0}$ \\
 & \nopagebreak $\;J=50$  & ${-}0.4\phantom{0}$ & ${-}0.5\phantom{0}$ & $\phantom{-}4.5\phantom{0}$ & ${-}2.1\phantom{0}$ & ${-}1.5\phantom{0}$ & ${-}3.6\phantom{0}$ & $\phantom{0}0.09\phantom{0}$ & $\phantom{0}0.10\phantom{0}$ & $\phantom{0}0.10\phantom{0}$ & $\phantom{0}0.10\phantom{0}$ & $\phantom{0}0.10\phantom{0}$ & $\phantom{0}0.10\phantom{0}$ & $\phantom{0}92.5\phantom{0}$ & $\phantom{0}91.7\phantom{0}$ & $\phantom{0}91.7\phantom{0}$ & $\phantom{0}93.1\phantom{0}$ & $\phantom{0}92.7\phantom{0}$ & $\phantom{0}94.1\phantom{0}$ \\
 & \nopagebreak $\;J=100$  & $\phantom{-}1.0\phantom{0}$ & $\phantom{-}1.0\phantom{0}$ & $\phantom{-}6.3\phantom{0}$ & $\phantom{-}0.6\phantom{0}$ & $\phantom{-}0.6\phantom{0}$ & ${-}0.4\phantom{0}$ & $\phantom{0}0.06\phantom{0}$ & $\phantom{0}0.07\phantom{0}$ & $\phantom{0}0.07\phantom{0}$ & $\phantom{0}0.07\phantom{0}$ & $\phantom{0}0.07\phantom{0}$ & $\phantom{0}0.06\phantom{0}$ & $\phantom{0}94.1\phantom{0}$ & $\phantom{0}94.1\phantom{0}$ & $\phantom{0}92.6\phantom{0}$ & $\phantom{0}94.2\phantom{0}$ & $\phantom{0}94.0\phantom{0}$ & $\phantom{0}94.3\phantom{0}$ \\
 & \nopagebreak $\;J=200$  & $\phantom{-}0.6\phantom{0}$ & $\phantom{-}0.5\phantom{0}$ & $\phantom{-}5.8\phantom{0}$ & $\phantom{-}0.2\phantom{0}$ & $\phantom{-}0.2\phantom{0}$ & ${-}0.4\phantom{0}$ & $\phantom{0}0.04\phantom{0}$ & $\phantom{0}0.05\phantom{0}$ & $\phantom{0}0.05\phantom{0}$ & $\phantom{0}0.05\phantom{0}$ & $\phantom{0}0.05\phantom{0}$ & $\phantom{0}0.05\phantom{0}$ & $\phantom{0}93.5\phantom{0}$ & $\phantom{0}93.9\phantom{0}$ & $\phantom{0}91.1\phantom{0}$ & $\phantom{0}94.3\phantom{0}$ & $\phantom{0}94.4\phantom{0}$ & $\phantom{0}94.1\phantom{0}$ \\
 & \nopagebreak $\;J=500$  & $\phantom{-}0.4\phantom{0}$ & $\phantom{-}0.5\phantom{0}$ & $\phantom{-}5.7\phantom{0}$ & $\phantom{-}0.3\phantom{0}$ & $\phantom{-}0.3\phantom{0}$ & $\phantom{-}0.1\phantom{0}$ & $\phantom{0}0.03\phantom{0}$ & $\phantom{0}0.03\phantom{0}$ & $\phantom{0}0.04\phantom{0}$ & $\phantom{0}0.03\phantom{0}$ & $\phantom{0}0.03\phantom{0}$ & $\phantom{0}0.03\phantom{0}$ & $\phantom{0}94.0\phantom{0}$ & $\phantom{0}93.8\phantom{0}$ & $\phantom{0}90.5\phantom{0}$ & $\phantom{0}94.4\phantom{0}$ & $\phantom{0}94.5\phantom{0}$ & $\phantom{0}94.2\phantom{0}$ \\
 & \nopagebreak $\;J=1000$  & ${-}0.5\phantom{0}$ & ${-}0.4\phantom{0}$ & $\phantom{-}4.8\phantom{0}$ & ${-}0.6\phantom{0}$ & ${-}0.5\phantom{0}$ & ${-}0.7\phantom{0}$ & $\phantom{0}0.02\phantom{0}$ & $\phantom{0}0.02\phantom{0}$ & $\phantom{0}0.03\phantom{0}$ & $\phantom{0}0.02\phantom{0}$ & $\phantom{0}0.02\phantom{0}$ & $\phantom{0}0.02\phantom{0}$ & $\phantom{0}95.9\phantom{0}$ & $\phantom{0}95.2\phantom{0}$ & $\phantom{0}91.2\phantom{0}$ & $\phantom{0}95.8\phantom{0}$ & $\phantom{0}95.6\phantom{0}$ & $\phantom{0}95.9\phantom{0}$ \\
\multicolumn{4}{l}{$n=20$} \\  & \nopagebreak $\;J=30$  & ${-}0.8\phantom{0}$ & ${-}1.1\phantom{0}$ & $\phantom{-}5.2\phantom{0}$ & ${-}3.2\phantom{0}$ & ${-}3.0\phantom{0}$ & ${-}5.4\phantom{0}$ & $\phantom{0}0.10\phantom{0}$ & $\phantom{0}0.11\phantom{0}$ & $\phantom{0}0.12\phantom{0}$ & $\phantom{0}0.11\phantom{0}$ & $\phantom{0}0.11\phantom{0}$ & $\phantom{0}0.11\phantom{0}$ & $\phantom{0}90.0\phantom{0}$ & $\phantom{0}89.7\phantom{0}$ & $\phantom{0}88.4\phantom{0}$ & $\phantom{0}91.7\phantom{0}$ & $\phantom{0}91.4\phantom{0}$ & $\phantom{0}92.5\phantom{0}$ \\
 & \nopagebreak $\;J=50$  & ${-}0.1\phantom{0}$ & ${-}0.0\phantom{0}$ & $\phantom{-}6.2\phantom{0}$ & ${-}1.4\phantom{0}$ & ${-}1.7\phantom{0}$ & ${-}2.7\phantom{0}$ & $\phantom{0}0.08\phantom{0}$ & $\phantom{0}0.08\phantom{0}$ & $\phantom{0}0.09\phantom{0}$ & $\phantom{0}0.08\phantom{0}$ & $\phantom{0}0.08\phantom{0}$ & $\phantom{0}0.08\phantom{0}$ & $\phantom{0}92.0\phantom{0}$ & $\phantom{0}90.7\phantom{0}$ & $\phantom{0}89.7\phantom{0}$ & $\phantom{0}92.5\phantom{0}$ & $\phantom{0}91.9\phantom{0}$ & $\phantom{0}93.5\phantom{0}$ \\
 & \nopagebreak $\;J=100$  & $\phantom{-}0.3\phantom{0}$ & $\phantom{-}0.2\phantom{0}$ & $\phantom{-}6.6\phantom{0}$ & ${-}0.3\phantom{0}$ & ${-}0.5\phantom{0}$ & ${-}1.1\phantom{0}$ & $\phantom{0}0.05\phantom{0}$ & $\phantom{0}0.06\phantom{0}$ & $\phantom{0}0.06\phantom{0}$ & $\phantom{0}0.06\phantom{0}$ & $\phantom{0}0.06\phantom{0}$ & $\phantom{0}0.06\phantom{0}$ & $\phantom{0}93.5\phantom{0}$ & $\phantom{0}93.5\phantom{0}$ & $\phantom{0}91.3\phantom{0}$ & $\phantom{0}93.5\phantom{0}$ & $\phantom{0}93.5\phantom{0}$ & $\phantom{0}93.8\phantom{0}$ \\
 & \nopagebreak $\;J=200$  & $\phantom{-}0.1\phantom{0}$ & $\phantom{-}0.2\phantom{0}$ & $\phantom{-}6.5\phantom{0}$ & ${-}0.5\phantom{0}$ & ${-}0.4\phantom{0}$ & ${-}0.7\phantom{0}$ & $\phantom{0}0.04\phantom{0}$ & $\phantom{0}0.04\phantom{0}$ & $\phantom{0}0.05\phantom{0}$ & $\phantom{0}0.04\phantom{0}$ & $\phantom{0}0.04\phantom{0}$ & $\phantom{0}0.04\phantom{0}$ & $\phantom{0}92.4\phantom{0}$ & $\phantom{0}93.0\phantom{0}$ & $\phantom{0}89.8\phantom{0}$ & $\phantom{0}93.6\phantom{0}$ & $\phantom{0}93.2\phantom{0}$ & $\phantom{0}93.6\phantom{0}$ \\
 & \nopagebreak $\;J=500$  & ${-}0.1\phantom{0}$ & ${-}0.1\phantom{0}$ & $\phantom{-}6.4\phantom{0}$ & ${-}0.2\phantom{0}$ & ${-}0.2\phantom{0}$ & ${-}0.3\phantom{0}$ & $\phantom{0}0.02\phantom{0}$ & $\phantom{0}0.03\phantom{0}$ & $\phantom{0}0.03\phantom{0}$ & $\phantom{0}0.02\phantom{0}$ & $\phantom{0}0.02\phantom{0}$ & $\phantom{0}0.02\phantom{0}$ & $\phantom{0}94.9\phantom{0}$ & $\phantom{0}94.4\phantom{0}$ & $\phantom{0}88.0\phantom{0}$ & $\phantom{0}94.8\phantom{0}$ & $\phantom{0}94.7\phantom{0}$ & $\phantom{0}95.8\phantom{0}$ \\
 & \nopagebreak $\;J=1000$  & $\phantom{-}0.2\phantom{0}$ & $\phantom{-}0.2\phantom{0}$ & $\phantom{-}6.7\phantom{0}$ & $\phantom{-}0.1\phantom{0}$ & $\phantom{-}0.1\phantom{0}$ & $\phantom{-}0.0\phantom{0}$ & $\phantom{0}0.02\phantom{0}$ & $\phantom{0}0.02\phantom{0}$ & $\phantom{0}0.03\phantom{0}$ & $\phantom{0}0.02\phantom{0}$ & $\phantom{0}0.02\phantom{0}$ & $\phantom{0}0.02\phantom{0}$ & $\phantom{0}96.0\phantom{0}$ & $\phantom{0}95.9\phantom{0}$ & $\phantom{0}82.6\phantom{0}$ & $\phantom{0}95.6\phantom{0}$ & $\phantom{0}95.4\phantom{0}$ & $\phantom{0}95.6\phantom{0}$ \\
[0.5ex]\hline\\[-1.6ex] 
\end{tabular}
\begin{tablenotes}[para,flushleft]{\footnotesize \textit{Note.} $n$ = cluster size; $J$ = number of clusters; CD = complete data sets; LD = listwise deletion; FCS-SL = single-level FCS; FCS-MAN = two-level FCS with manifest cluster means; FCS-LAT = two-level FCS with latent cluster means; JM = joint modeling.}\end{tablenotes}
\end{threeparttable}
\end{sidewaystable}
\begin{sidewaystable}
\begin{threeparttable}
\setlength{\tabcolsep}{1.2pt}
\renewcommand{\arraystretch}{0.95}
\footnotesize
\caption{\small Study 1: Bias (in \%), RMSE, and Coverage of the 95\% Confidence Interval for the Regression Coefficient of $y$ on $z$ ($\hat\beta_{yz}$) With 20\% Missing Data (MAR, $\lambda=0.5$)}
\begin{tabular}{llcccccccccccccccccc}
\hline\\[-1.8ex]
& & \multicolumn{6}{c}{Bias (\%)} & \multicolumn{6}{c}{RMSE} & \multicolumn{6}{c}{Coverage (\%)} \\ \cmidrule(r){3-8}\cmidrule(r){9-14}\cmidrule(r){15-20}
 &  & CD & LD & \makecell{FCS-\\SL} & \makecell{FCS-\\MAN} & \makecell{FCS-\\LAT} & JM & CD & LD & \makecell{FCS-\\SL} & \makecell{FCS-\\MAN} & \makecell{FCS-\\LAT} & JM & CD & LD & \makecell{FCS-\\SL} & \makecell{FCS-\\MAN} & \makecell{FCS-\\LAT} & \multicolumn{1}{c}{JM} \\ 
[0.4ex]\hline\\[-1.8ex]
& & \multicolumn{18}{c}{Small intraclass correlation $(\rho_{Iy}=.10)$} \\[0.6ex]\hline\\[-1.8ex]
\multicolumn{4}{l}{$n=5$} \\  & \nopagebreak $\;J=30$  & $\phantom{0}{-}0.6\phantom{0}$ & ${-}11.4\phantom{0}$ & $\phantom{0}{-}7.6\phantom{0}$ & $\phantom{0}{-}7.2\phantom{0}$ & $\phantom{0}{-}4.1\phantom{0}$ & ${-}14.5\phantom{0}$ & $\phantom{0}0.10\phantom{0}$ & $\phantom{0}0.11\phantom{0}$ & $\phantom{0}0.11\phantom{0}$ & $\phantom{0}0.11\phantom{0}$ & $\phantom{0}0.11\phantom{0}$ & $\phantom{0}0.10\phantom{0}$ & $\phantom{0}89.1\phantom{0}$ & $\phantom{0}89.4\phantom{0}$ & $\phantom{0}90.4\phantom{0}$ & $\phantom{0}90.4\phantom{0}$ & $\phantom{0}89.6\phantom{0}$ & $\phantom{0}93.6\phantom{0}$ \\
 & \nopagebreak $\;J=50$  & $\phantom{0}\phantom{-}1.2\phantom{0}$ & $\phantom{0}{-}7.4\phantom{0}$ & $\phantom{0}{-}3.1\phantom{0}$ & $\phantom{0}{-}0.8\phantom{0}$ & $\phantom{0}\phantom{-}2.0\phantom{0}$ & $\phantom{0}{-}8.1\phantom{0}$ & $\phantom{0}0.07\phantom{0}$ & $\phantom{0}0.08\phantom{0}$ & $\phantom{0}0.08\phantom{0}$ & $\phantom{0}0.08\phantom{0}$ & $\phantom{0}0.09\phantom{0}$ & $\phantom{0}0.08\phantom{0}$ & $\phantom{0}92.6\phantom{0}$ & $\phantom{0}91.8\phantom{0}$ & $\phantom{0}93.1\phantom{0}$ & $\phantom{0}93.4\phantom{0}$ & $\phantom{0}91.6\phantom{0}$ & $\phantom{0}95.0\phantom{0}$ \\
 & \nopagebreak $\;J=100$  & $\phantom{0}{-}1.3\phantom{0}$ & ${-}10.9\phantom{0}$ & $\phantom{0}{-}6.6\phantom{0}$ & $\phantom{0}{-}2.9\phantom{0}$ & $\phantom{0}{-}0.6\phantom{0}$ & $\phantom{0}{-}8.0\phantom{0}$ & $\phantom{0}0.05\phantom{0}$ & $\phantom{0}0.06\phantom{0}$ & $\phantom{0}0.06\phantom{0}$ & $\phantom{0}0.06\phantom{0}$ & $\phantom{0}0.06\phantom{0}$ & $\phantom{0}0.06\phantom{0}$ & $\phantom{0}94.6\phantom{0}$ & $\phantom{0}92.7\phantom{0}$ & $\phantom{0}93.9\phantom{0}$ & $\phantom{0}94.3\phantom{0}$ & $\phantom{0}93.9\phantom{0}$ & $\phantom{0}95.3\phantom{0}$ \\
 & \nopagebreak $\;J=200$  & $\phantom{0}\phantom{-}0.7\phantom{0}$ & $\phantom{0}{-}9.0\phantom{0}$ & $\phantom{0}{-}4.6\phantom{0}$ & $\phantom{0}{-}0.1\phantom{0}$ & $\phantom{0}\phantom{-}1.6\phantom{0}$ & $\phantom{0}{-}3.6\phantom{0}$ & $\phantom{0}0.04\phantom{0}$ & $\phantom{0}0.04\phantom{0}$ & $\phantom{0}0.04\phantom{0}$ & $\phantom{0}0.04\phantom{0}$ & $\phantom{0}0.04\phantom{0}$ & $\phantom{0}0.04\phantom{0}$ & $\phantom{0}95.0\phantom{0}$ & $\phantom{0}93.7\phantom{0}$ & $\phantom{0}94.6\phantom{0}$ & $\phantom{0}95.1\phantom{0}$ & $\phantom{0}94.1\phantom{0}$ & $\phantom{0}95.2\phantom{0}$ \\
 & \nopagebreak $\;J=500$  & $\phantom{0}{-}0.7\phantom{0}$ & $\phantom{0}{-}9.9\phantom{0}$ & $\phantom{0}{-}5.7\phantom{0}$ & $\phantom{0}{-}1.0\phantom{0}$ & $\phantom{0}{-}0.4\phantom{0}$ & $\phantom{0}{-}2.7\phantom{0}$ & $\phantom{0}0.02\phantom{0}$ & $\phantom{0}0.03\phantom{0}$ & $\phantom{0}0.03\phantom{0}$ & $\phantom{0}0.03\phantom{0}$ & $\phantom{0}0.03\phantom{0}$ & $\phantom{0}0.03\phantom{0}$ & $\phantom{0}94.1\phantom{0}$ & $\phantom{0}88.5\phantom{0}$ & $\phantom{0}93.3\phantom{0}$ & $\phantom{0}94.8\phantom{0}$ & $\phantom{0}94.5\phantom{0}$ & $\phantom{0}95.0\phantom{0}$ \\
 & \nopagebreak $\;J=1000$  & $\phantom{0}{-}0.1\phantom{0}$ & $\phantom{0}{-}9.4\phantom{0}$ & $\phantom{0}{-}5.1\phantom{0}$ & $\phantom{0}\phantom{-}0.0\phantom{0}$ & $\phantom{0}\phantom{-}0.2\phantom{0}$ & $\phantom{0}{-}0.9\phantom{0}$ & $\phantom{0}0.02\phantom{0}$ & $\phantom{0}0.02\phantom{0}$ & $\phantom{0}0.02\phantom{0}$ & $\phantom{0}0.02\phantom{0}$ & $\phantom{0}0.02\phantom{0}$ & $\phantom{0}0.02\phantom{0}$ & $\phantom{0}95.2\phantom{0}$ & $\phantom{0}86.2\phantom{0}$ & $\phantom{0}92.8\phantom{0}$ & $\phantom{0}94.6\phantom{0}$ & $\phantom{0}94.2\phantom{0}$ & $\phantom{0}94.7\phantom{0}$ \\
\multicolumn{4}{l}{$n=20$} \\  & \nopagebreak $\;J=30$  & $\phantom{0}\phantom{-}0.1\phantom{0}$ & $\phantom{0}{-}8.1\phantom{0}$ & $\phantom{0}{-}4.4\phantom{0}$ & $\phantom{0}{-}3.3\phantom{0}$ & $\phantom{0}{-}2.6\phantom{0}$ & ${-}11.0\phantom{0}$ & $\phantom{0}0.07\phantom{0}$ & $\phantom{0}0.07\phantom{0}$ & $\phantom{0}0.08\phantom{0}$ & $\phantom{0}0.07\phantom{0}$ & $\phantom{0}0.07\phantom{0}$ & $\phantom{0}0.07\phantom{0}$ & $\phantom{0}91.6\phantom{0}$ & $\phantom{0}89.9\phantom{0}$ & $\phantom{0}90.7\phantom{0}$ & $\phantom{0}92.3\phantom{0}$ & $\phantom{0}92.0\phantom{0}$ & $\phantom{0}94.5\phantom{0}$ \\
 & \nopagebreak $\;J=50$  & $\phantom{0}\phantom{-}2.1\phantom{0}$ & $\phantom{0}{-}7.3\phantom{0}$ & $\phantom{0}{-}3.7\phantom{0}$ & $\phantom{0}{-}1.0\phantom{0}$ & $\phantom{0}{-}0.5\phantom{0}$ & $\phantom{0}{-}6.9\phantom{0}$ & $\phantom{0}0.05\phantom{0}$ & $\phantom{0}0.05\phantom{0}$ & $\phantom{0}0.06\phantom{0}$ & $\phantom{0}0.06\phantom{0}$ & $\phantom{0}0.06\phantom{0}$ & $\phantom{0}0.05\phantom{0}$ & $\phantom{0}92.5\phantom{0}$ & $\phantom{0}92.1\phantom{0}$ & $\phantom{0}93.1\phantom{0}$ & $\phantom{0}92.9\phantom{0}$ & $\phantom{0}93.1\phantom{0}$ & $\phantom{0}95.1\phantom{0}$ \\
 & \nopagebreak $\;J=100$  & $\phantom{0}{-}1.0\phantom{0}$ & $\phantom{0}{-}9.7\phantom{0}$ & $\phantom{0}{-}6.2\phantom{0}$ & $\phantom{0}{-}2.4\phantom{0}$ & $\phantom{0}{-}2.1\phantom{0}$ & $\phantom{0}{-}5.7\phantom{0}$ & $\phantom{0}0.03\phantom{0}$ & $\phantom{0}0.04\phantom{0}$ & $\phantom{0}0.04\phantom{0}$ & $\phantom{0}0.04\phantom{0}$ & $\phantom{0}0.04\phantom{0}$ & $\phantom{0}0.04\phantom{0}$ & $\phantom{0}93.5\phantom{0}$ & $\phantom{0}91.5\phantom{0}$ & $\phantom{0}92.6\phantom{0}$ & $\phantom{0}94.0\phantom{0}$ & $\phantom{0}93.6\phantom{0}$ & $\phantom{0}94.3\phantom{0}$ \\
 & \nopagebreak $\;J=200$  & $\phantom{0}\phantom{-}0.1\phantom{0}$ & $\phantom{0}{-}8.9\phantom{0}$ & $\phantom{0}{-}5.4\phantom{0}$ & $\phantom{0}{-}0.9\phantom{0}$ & $\phantom{0}{-}0.6\phantom{0}$ & $\phantom{0}{-}2.8\phantom{0}$ & $\phantom{0}0.03\phantom{0}$ & $\phantom{0}0.03\phantom{0}$ & $\phantom{0}0.03\phantom{0}$ & $\phantom{0}0.03\phantom{0}$ & $\phantom{0}0.03\phantom{0}$ & $\phantom{0}0.03\phantom{0}$ & $\phantom{0}94.4\phantom{0}$ & $\phantom{0}90.6\phantom{0}$ & $\phantom{0}92.6\phantom{0}$ & $\phantom{0}94.4\phantom{0}$ & $\phantom{0}93.8\phantom{0}$ & $\phantom{0}95.0\phantom{0}$ \\
 & \nopagebreak $\;J=500$  & $\phantom{0}\phantom{-}0.5\phantom{0}$ & $\phantom{0}{-}8.3\phantom{0}$ & $\phantom{0}{-}4.7\phantom{0}$ & $\phantom{0}\phantom{-}0.3\phantom{0}$ & $\phantom{0}\phantom{-}0.4\phantom{0}$ & $\phantom{0}{-}0.6\phantom{0}$ & $\phantom{0}0.02\phantom{0}$ & $\phantom{0}0.02\phantom{0}$ & $\phantom{0}0.02\phantom{0}$ & $\phantom{0}0.02\phantom{0}$ & $\phantom{0}0.02\phantom{0}$ & $\phantom{0}0.02\phantom{0}$ & $\phantom{0}94.6\phantom{0}$ & $\phantom{0}87.7\phantom{0}$ & $\phantom{0}91.3\phantom{0}$ & $\phantom{0}95.3\phantom{0}$ & $\phantom{0}95.8\phantom{0}$ & $\phantom{0}95.9\phantom{0}$ \\
 & \nopagebreak $\;J=1000$  & $\phantom{0}\phantom{-}0.2\phantom{0}$ & $\phantom{0}{-}8.9\phantom{0}$ & $\phantom{0}{-}5.4\phantom{0}$ & $\phantom{0}{-}0.2\phantom{0}$ & $\phantom{0}{-}0.2\phantom{0}$ & $\phantom{0}{-}0.7\phantom{0}$ & $\phantom{0}0.01\phantom{0}$ & $\phantom{0}0.02\phantom{0}$ & $\phantom{0}0.02\phantom{0}$ & $\phantom{0}0.01\phantom{0}$ & $\phantom{0}0.01\phantom{0}$ & $\phantom{0}0.01\phantom{0}$ & $\phantom{0}94.5\phantom{0}$ & $\phantom{0}77.4\phantom{0}$ & $\phantom{0}88.0\phantom{0}$ & $\phantom{0}94.0\phantom{0}$ & $\phantom{0}93.9\phantom{0}$ & $\phantom{0}94.5\phantom{0}$ \\
[0.5ex]\hline\\[-1.6ex] 
& & \multicolumn{18}{c}{Moderate intraclass correlation $(\rho_{Iy}=.30)$} \\[0.6ex]\hline\\[-1.8ex]
\multicolumn{4}{l}{$n=5$} \\  & \nopagebreak $\;J=30$  & ${-}1.0\phantom{0}$ & ${-}9.3\phantom{0}$ & ${-}1.2\phantom{0}$ & ${-}5.1\phantom{0}$ & ${-}3.7\phantom{0}$ & ${-}8.5\phantom{0}$ & $\phantom{0}0.11\phantom{0}$ & $\phantom{0}0.13\phantom{0}$ & $\phantom{0}0.13\phantom{0}$ & $\phantom{0}0.13\phantom{0}$ & $\phantom{0}0.13\phantom{0}$ & $\phantom{0}0.13\phantom{0}$ & $\phantom{0}90.9\phantom{0}$ & $\phantom{0}89.7\phantom{0}$ & $\phantom{0}90.5\phantom{0}$ & $\phantom{0}91.7\phantom{0}$ & $\phantom{0}91.0\phantom{0}$ & $\phantom{0}93.1\phantom{0}$ \\
 & \nopagebreak $\;J=50$  & ${-}0.2\phantom{0}$ & ${-}9.0\phantom{0}$ & ${-}0.3\phantom{0}$ & ${-}2.7\phantom{0}$ & ${-}2.0\phantom{0}$ & ${-}5.7\phantom{0}$ & $\phantom{0}0.09\phantom{0}$ & $\phantom{0}0.10\phantom{0}$ & $\phantom{0}0.10\phantom{0}$ & $\phantom{0}0.10\phantom{0}$ & $\phantom{0}0.10\phantom{0}$ & $\phantom{0}0.10\phantom{0}$ & $\phantom{0}92.3\phantom{0}$ & $\phantom{0}91.1\phantom{0}$ & $\phantom{0}91.5\phantom{0}$ & $\phantom{0}93.3\phantom{0}$ & $\phantom{0}92.5\phantom{0}$ & $\phantom{0}94.3\phantom{0}$ \\
 & \nopagebreak $\;J=100$  & ${-}0.3\phantom{0}$ & ${-}9.7\phantom{0}$ & ${-}0.8\phantom{0}$ & ${-}2.2\phantom{0}$ & ${-}1.8\phantom{0}$ & ${-}3.6\phantom{0}$ & $\phantom{0}0.06\phantom{0}$ & $\phantom{0}0.07\phantom{0}$ & $\phantom{0}0.07\phantom{0}$ & $\phantom{0}0.07\phantom{0}$ & $\phantom{0}0.07\phantom{0}$ & $\phantom{0}0.07\phantom{0}$ & $\phantom{0}93.3\phantom{0}$ & $\phantom{0}90.8\phantom{0}$ & $\phantom{0}93.3\phantom{0}$ & $\phantom{0}93.8\phantom{0}$ & $\phantom{0}94.1\phantom{0}$ & $\phantom{0}94.7\phantom{0}$ \\
 & \nopagebreak $\;J=200$  & $\phantom{-}0.6\phantom{0}$ & ${-}8.4\phantom{0}$ & $\phantom{-}0.8\phantom{0}$ & ${-}0.1\phantom{0}$ & $\phantom{-}0.0\phantom{0}$ & ${-}0.8\phantom{0}$ & $\phantom{0}0.04\phantom{0}$ & $\phantom{0}0.05\phantom{0}$ & $\phantom{0}0.05\phantom{0}$ & $\phantom{0}0.05\phantom{0}$ & $\phantom{0}0.05\phantom{0}$ & $\phantom{0}0.05\phantom{0}$ & $\phantom{0}94.6\phantom{0}$ & $\phantom{0}91.3\phantom{0}$ & $\phantom{0}93.6\phantom{0}$ & $\phantom{0}94.3\phantom{0}$ & $\phantom{0}94.3\phantom{0}$ & $\phantom{0}95.1\phantom{0}$ \\
 & \nopagebreak $\;J=500$  & ${-}0.4\phantom{0}$ & ${-}9.2\phantom{0}$ & ${-}0.2\phantom{0}$ & ${-}0.9\phantom{0}$ & ${-}0.8\phantom{0}$ & ${-}1.1\phantom{0}$ & $\phantom{0}0.03\phantom{0}$ & $\phantom{0}0.04\phantom{0}$ & $\phantom{0}0.03\phantom{0}$ & $\phantom{0}0.03\phantom{0}$ & $\phantom{0}0.03\phantom{0}$ & $\phantom{0}0.03\phantom{0}$ & $\phantom{0}93.9\phantom{0}$ & $\phantom{0}85.7\phantom{0}$ & $\phantom{0}93.2\phantom{0}$ & $\phantom{0}93.8\phantom{0}$ & $\phantom{0}93.7\phantom{0}$ & $\phantom{0}93.6\phantom{0}$ \\
 & \nopagebreak $\;J=1000$  & $\phantom{-}0.2\phantom{0}$ & ${-}8.6\phantom{0}$ & $\phantom{-}0.6\phantom{0}$ & $\phantom{-}0.0\phantom{0}$ & $\phantom{-}0.1\phantom{0}$ & ${-}0.1\phantom{0}$ & $\phantom{0}0.02\phantom{0}$ & $\phantom{0}0.03\phantom{0}$ & $\phantom{0}0.02\phantom{0}$ & $\phantom{0}0.02\phantom{0}$ & $\phantom{0}0.02\phantom{0}$ & $\phantom{0}0.02\phantom{0}$ & $\phantom{0}95.4\phantom{0}$ & $\phantom{0}79.6\phantom{0}$ & $\phantom{0}95.1\phantom{0}$ & $\phantom{0}95.6\phantom{0}$ & $\phantom{0}95.3\phantom{0}$ & $\phantom{0}95.3\phantom{0}$ \\
\multicolumn{4}{l}{$n=20$} \\  & \nopagebreak $\;J=30$  & ${-}0.6\phantom{0}$ & ${-}8.7\phantom{0}$ & $\phantom{-}0.5\phantom{0}$ & ${-}4.6\phantom{0}$ & ${-}4.8\phantom{0}$ & ${-}7.5\phantom{0}$ & $\phantom{0}0.10\phantom{0}$ & $\phantom{0}0.11\phantom{0}$ & $\phantom{0}0.12\phantom{0}$ & $\phantom{0}0.11\phantom{0}$ & $\phantom{0}0.11\phantom{0}$ & $\phantom{0}0.11\phantom{0}$ & $\phantom{0}90.8\phantom{0}$ & $\phantom{0}89.1\phantom{0}$ & $\phantom{0}88.5\phantom{0}$ & $\phantom{0}91.8\phantom{0}$ & $\phantom{0}91.7\phantom{0}$ & $\phantom{0}93.4\phantom{0}$ \\
 & \nopagebreak $\;J=50$  & ${-}0.2\phantom{0}$ & ${-}8.4\phantom{0}$ & $\phantom{-}0.6\phantom{0}$ & ${-}2.4\phantom{0}$ & ${-}2.6\phantom{0}$ & ${-}4.5\phantom{0}$ & $\phantom{0}0.07\phantom{0}$ & $\phantom{0}0.08\phantom{0}$ & $\phantom{0}0.09\phantom{0}$ & $\phantom{0}0.08\phantom{0}$ & $\phantom{0}0.08\phantom{0}$ & $\phantom{0}0.08\phantom{0}$ & $\phantom{0}93.1\phantom{0}$ & $\phantom{0}91.4\phantom{0}$ & $\phantom{0}91.3\phantom{0}$ & $\phantom{0}94.5\phantom{0}$ & $\phantom{0}94.1\phantom{0}$ & $\phantom{0}94.5\phantom{0}$ \\
 & \nopagebreak $\;J=100$  & ${-}0.4\phantom{0}$ & ${-}8.9\phantom{0}$ & $\phantom{-}0.3\phantom{0}$ & ${-}1.9\phantom{0}$ & ${-}1.8\phantom{0}$ & ${-}2.8\phantom{0}$ & $\phantom{0}0.05\phantom{0}$ & $\phantom{0}0.06\phantom{0}$ & $\phantom{0}0.06\phantom{0}$ & $\phantom{0}0.06\phantom{0}$ & $\phantom{0}0.06\phantom{0}$ & $\phantom{0}0.06\phantom{0}$ & $\phantom{0}94.3\phantom{0}$ & $\phantom{0}90.6\phantom{0}$ & $\phantom{0}92.1\phantom{0}$ & $\phantom{0}94.3\phantom{0}$ & $\phantom{0}94.3\phantom{0}$ & $\phantom{0}93.8\phantom{0}$ \\
 & \nopagebreak $\;J=200$  & ${-}0.1\phantom{0}$ & ${-}8.3\phantom{0}$ & $\phantom{-}1.0\phantom{0}$ & ${-}0.4\phantom{0}$ & ${-}0.4\phantom{0}$ & ${-}0.8\phantom{0}$ & $\phantom{0}0.03\phantom{0}$ & $\phantom{0}0.04\phantom{0}$ & $\phantom{0}0.04\phantom{0}$ & $\phantom{0}0.04\phantom{0}$ & $\phantom{0}0.04\phantom{0}$ & $\phantom{0}0.04\phantom{0}$ & $\phantom{0}95.3\phantom{0}$ & $\phantom{0}91.0\phantom{0}$ & $\phantom{0}94.3\phantom{0}$ & $\phantom{0}95.9\phantom{0}$ & $\phantom{0}95.3\phantom{0}$ & $\phantom{0}96.0\phantom{0}$ \\
 & \nopagebreak $\;J=500$  & ${-}0.1\phantom{0}$ & ${-}8.2\phantom{0}$ & $\phantom{-}1.1\phantom{0}$ & ${-}0.1\phantom{0}$ & ${-}0.1\phantom{0}$ & ${-}0.3\phantom{0}$ & $\phantom{0}0.02\phantom{0}$ & $\phantom{0}0.03\phantom{0}$ & $\phantom{0}0.03\phantom{0}$ & $\phantom{0}0.03\phantom{0}$ & $\phantom{0}0.03\phantom{0}$ & $\phantom{0}0.03\phantom{0}$ & $\phantom{0}95.3\phantom{0}$ & $\phantom{0}85.0\phantom{0}$ & $\phantom{0}94.5\phantom{0}$ & $\phantom{0}95.7\phantom{0}$ & $\phantom{0}95.2\phantom{0}$ & $\phantom{0}95.8\phantom{0}$ \\
 & \nopagebreak $\;J=1000$  & $\phantom{-}0.1\phantom{0}$ & ${-}8.0\phantom{0}$ & $\phantom{-}1.4\phantom{0}$ & $\phantom{-}0.2\phantom{0}$ & $\phantom{-}0.3\phantom{0}$ & $\phantom{-}0.1\phantom{0}$ & $\phantom{0}0.02\phantom{0}$ & $\phantom{0}0.03\phantom{0}$ & $\phantom{0}0.02\phantom{0}$ & $\phantom{0}0.02\phantom{0}$ & $\phantom{0}0.02\phantom{0}$ & $\phantom{0}0.02\phantom{0}$ & $\phantom{0}94.2\phantom{0}$ & $\phantom{0}75.1\phantom{0}$ & $\phantom{0}93.1\phantom{0}$ & $\phantom{0}94.3\phantom{0}$ & $\phantom{0}95.0\phantom{0}$ & $\phantom{0}94.4\phantom{0}$ \\
[0.5ex]\hline\\[-1.6ex] 
\end{tabular}
\begin{tablenotes}[para,flushleft]{\footnotesize \textit{Note.} $n$ = cluster size; $J$ = number of clusters; CD = complete data sets; LD = listwise deletion; FCS-SL = single-level FCS; FCS-MAN = two-level FCS with manifest cluster means; FCS-LAT = two-level FCS with latent cluster means; JM = joint modeling.}\end{tablenotes}
\end{threeparttable}
\end{sidewaystable}
\begin{sidewaystable}
\begin{threeparttable}
\setlength{\tabcolsep}{1.2pt}
\renewcommand{\arraystretch}{0.95}
\footnotesize
\caption{\small Study 1: Bias (in \%), RMSE, and Coverage of the 95\% Confidence Interval for the Regression Coefficient of $y$ on $z$ ($\hat\beta_{yz}$) With 20\% Missing Data (MAR, $\lambda=1$)}
\begin{tabular}{llcccccccccccccccccc}
\hline\\[-1.8ex]
& & \multicolumn{6}{c}{Bias (\%)} & \multicolumn{6}{c}{RMSE} & \multicolumn{6}{c}{Coverage (\%)} \\ \cmidrule(r){3-8}\cmidrule(r){9-14}\cmidrule(r){15-20}
 &  & CD & LD & \makecell{FCS-\\SL} & \makecell{FCS-\\MAN} & \makecell{FCS-\\LAT} & JM & CD & LD & \makecell{FCS-\\SL} & \makecell{FCS-\\MAN} & \makecell{FCS-\\LAT} & JM & CD & LD & \makecell{FCS-\\SL} & \makecell{FCS-\\MAN} & \makecell{FCS-\\LAT} & \multicolumn{1}{c}{JM} \\ 
[0.4ex]\hline\\[-1.8ex]
& & \multicolumn{18}{c}{Small intraclass correlation $(\rho_{Iy}=.10)$} \\[0.6ex]\hline\\[-1.8ex]
\multicolumn{4}{l}{$n=5$} \\  & \nopagebreak $\;J=30$  & $\phantom{0}{-}1.8\phantom{0}$ & ${-}42.2\phantom{0}$ & ${-}34.0\phantom{0}$ & ${-}17.2\phantom{0}$ & ${-}12.2\phantom{0}$ & ${-}33.3\phantom{0}$ & $\phantom{0}0.10\phantom{0}$ & $\phantom{0}0.11\phantom{0}$ & $\phantom{0}0.12\phantom{0}$ & $\phantom{0}0.13\phantom{0}$ & $\phantom{0}0.14\phantom{0}$ & $\phantom{0}0.12\phantom{0}$ & $\phantom{0}92.1\phantom{0}$ & $\phantom{0}78.5\phantom{0}$ & $\phantom{0}89.3\phantom{0}$ & $\phantom{0}90.2\phantom{0}$ & $\phantom{0}87.1\phantom{0}$ & $\phantom{0}94.9\phantom{0}$ \\
 & \nopagebreak $\;J=50$  & $\phantom{0}{-}0.8\phantom{0}$ & ${-}39.9\phantom{0}$ & ${-}30.7\phantom{0}$ & $\phantom{0}{-}9.5\phantom{0}$ & $\phantom{0}{-}2.4\phantom{0}$ & ${-}24.7\phantom{0}$ & $\phantom{0}0.07\phantom{0}$ & $\phantom{0}0.09\phantom{0}$ & $\phantom{0}0.09\phantom{0}$ & $\phantom{0}0.10\phantom{0}$ & $\phantom{0}0.10\phantom{0}$ & $\phantom{0}0.09\phantom{0}$ & $\phantom{0}92.1\phantom{0}$ & $\phantom{0}77.3\phantom{0}$ & $\phantom{0}89.1\phantom{0}$ & $\phantom{0}91.5\phantom{0}$ & $\phantom{0}89.0\phantom{0}$ & $\phantom{0}95.2\phantom{0}$ \\
 & \nopagebreak $\;J=100$  & $\phantom{0}{-}0.8\phantom{0}$ & ${-}39.7\phantom{0}$ & ${-}30.5\phantom{0}$ & $\phantom{0}{-}5.5\phantom{0}$ & $\phantom{0}\phantom{-}0.4\phantom{0}$ & ${-}17.7\phantom{0}$ & $\phantom{0}0.05\phantom{0}$ & $\phantom{0}0.08\phantom{0}$ & $\phantom{0}0.07\phantom{0}$ & $\phantom{0}0.07\phantom{0}$ & $\phantom{0}0.07\phantom{0}$ & $\phantom{0}0.07\phantom{0}$ & $\phantom{0}94.3\phantom{0}$ & $\phantom{0}69.6\phantom{0}$ & $\phantom{0}86.2\phantom{0}$ & $\phantom{0}93.1\phantom{0}$ & $\phantom{0}92.3\phantom{0}$ & $\phantom{0}94.4\phantom{0}$ \\
 & \nopagebreak $\;J=200$  & $\phantom{0}\phantom{-}0.5\phantom{0}$ & ${-}39.2\phantom{0}$ & ${-}29.9\phantom{0}$ & $\phantom{0}{-}2.0\phantom{0}$ & $\phantom{0}\phantom{-}1.5\phantom{0}$ & ${-}11.0\phantom{0}$ & $\phantom{0}0.04\phantom{0}$ & $\phantom{0}0.07\phantom{0}$ & $\phantom{0}0.06\phantom{0}$ & $\phantom{0}0.05\phantom{0}$ & $\phantom{0}0.05\phantom{0}$ & $\phantom{0}0.05\phantom{0}$ & $\phantom{0}95.1\phantom{0}$ & $\phantom{0}48.1\phantom{0}$ & $\phantom{0}79.0\phantom{0}$ & $\phantom{0}93.5\phantom{0}$ & $\phantom{0}92.1\phantom{0}$ & $\phantom{0}94.9\phantom{0}$ \\
 & \nopagebreak $\;J=500$  & $\phantom{0}\phantom{-}0.8\phantom{0}$ & ${-}38.8\phantom{0}$ & ${-}29.6\phantom{0}$ & $\phantom{0}{-}0.4\phantom{0}$ & $\phantom{0}\phantom{-}1.1\phantom{0}$ & $\phantom{0}{-}5.0\phantom{0}$ & $\phantom{0}0.02\phantom{0}$ & $\phantom{0}0.06\phantom{0}$ & $\phantom{0}0.05\phantom{0}$ & $\phantom{0}0.03\phantom{0}$ & $\phantom{0}0.03\phantom{0}$ & $\phantom{0}0.03\phantom{0}$ & $\phantom{0}95.6\phantom{0}$ & $\phantom{0}12.9\phantom{0}$ & $\phantom{0}54.2\phantom{0}$ & $\phantom{0}95.6\phantom{0}$ & $\phantom{0}93.8\phantom{0}$ & $\phantom{0}95.4\phantom{0}$ \\
 & \nopagebreak $\;J=1000$  & $\phantom{0}{-}0.1\phantom{0}$ & ${-}39.5\phantom{0}$ & ${-}30.3\phantom{0}$ & $\phantom{0}{-}0.7\phantom{0}$ & $\phantom{0}{-}0.0\phantom{0}$ & $\phantom{0}{-}3.4\phantom{0}$ & $\phantom{0}0.02\phantom{0}$ & $\phantom{0}0.06\phantom{0}$ & $\phantom{0}0.05\phantom{0}$ & $\phantom{0}0.02\phantom{0}$ & $\phantom{0}0.02\phantom{0}$ & $\phantom{0}0.02\phantom{0}$ & $\phantom{0}94.3\phantom{0}$ & $\phantom{0}\phantom{0}1.8\phantom{0}$ & $\phantom{0}22.6\phantom{0}$ & $\phantom{0}93.6\phantom{0}$ & $\phantom{0}92.7\phantom{0}$ & $\phantom{0}94.1\phantom{0}$ \\
\multicolumn{4}{l}{$n=20$} \\  & \nopagebreak $\;J=30$  & $\phantom{0}{-}0.9\phantom{0}$ & ${-}37.2\phantom{0}$ & ${-}31.6\phantom{0}$ & ${-}12.7\phantom{0}$ & ${-}11.8\phantom{0}$ & ${-}27.2\phantom{0}$ & $\phantom{0}0.07\phantom{0}$ & $\phantom{0}0.09\phantom{0}$ & $\phantom{0}0.09\phantom{0}$ & $\phantom{0}0.09\phantom{0}$ & $\phantom{0}0.09\phantom{0}$ & $\phantom{0}0.09\phantom{0}$ & $\phantom{0}91.3\phantom{0}$ & $\phantom{0}73.5\phantom{0}$ & $\phantom{0}83.2\phantom{0}$ & $\phantom{0}92.4\phantom{0}$ & $\phantom{0}90.6\phantom{0}$ & $\phantom{0}94.8\phantom{0}$ \\
 & \nopagebreak $\;J=50$  & $\phantom{0}\phantom{-}0.6\phantom{0}$ & ${-}36.1\phantom{0}$ & ${-}30.4\phantom{0}$ & $\phantom{0}{-}6.0\phantom{0}$ & $\phantom{0}{-}4.8\phantom{0}$ & ${-}18.5\phantom{0}$ & $\phantom{0}0.05\phantom{0}$ & $\phantom{0}0.07\phantom{0}$ & $\phantom{0}0.07\phantom{0}$ & $\phantom{0}0.07\phantom{0}$ & $\phantom{0}0.07\phantom{0}$ & $\phantom{0}0.07\phantom{0}$ & $\phantom{0}92.5\phantom{0}$ & $\phantom{0}69.5\phantom{0}$ & $\phantom{0}81.5\phantom{0}$ & $\phantom{0}92.9\phantom{0}$ & $\phantom{0}92.4\phantom{0}$ & $\phantom{0}94.7\phantom{0}$ \\
 & \nopagebreak $\;J=100$  & $\phantom{0}\phantom{-}0.7\phantom{0}$ & ${-}36.4\phantom{0}$ & ${-}30.7\phantom{0}$ & $\phantom{0}{-}2.7\phantom{0}$ & $\phantom{0}{-}2.3\phantom{0}$ & ${-}11.1\phantom{0}$ & $\phantom{0}0.04\phantom{0}$ & $\phantom{0}0.07\phantom{0}$ & $\phantom{0}0.06\phantom{0}$ & $\phantom{0}0.04\phantom{0}$ & $\phantom{0}0.05\phantom{0}$ & $\phantom{0}0.05\phantom{0}$ & $\phantom{0}93.9\phantom{0}$ & $\phantom{0}54.3\phantom{0}$ & $\phantom{0}73.6\phantom{0}$ & $\phantom{0}94.3\phantom{0}$ & $\phantom{0}92.9\phantom{0}$ & $\phantom{0}95.4\phantom{0}$ \\
 & \nopagebreak $\;J=200$  & $\phantom{0}\phantom{-}0.5\phantom{0}$ & ${-}37.2\phantom{0}$ & ${-}31.6\phantom{0}$ & $\phantom{0}{-}1.5\phantom{0}$ & $\phantom{0}{-}1.2\phantom{0}$ & $\phantom{0}{-}7.0\phantom{0}$ & $\phantom{0}0.02\phantom{0}$ & $\phantom{0}0.06\phantom{0}$ & $\phantom{0}0.06\phantom{0}$ & $\phantom{0}0.03\phantom{0}$ & $\phantom{0}0.03\phantom{0}$ & $\phantom{0}0.03\phantom{0}$ & $\phantom{0}95.2\phantom{0}$ & $\phantom{0}25.6\phantom{0}$ & $\phantom{0}52.8\phantom{0}$ & $\phantom{0}95.1\phantom{0}$ & $\phantom{0}94.4\phantom{0}$ & $\phantom{0}96.2\phantom{0}$ \\
 & \nopagebreak $\;J=500$  & $\phantom{0}{-}0.4\phantom{0}$ & ${-}37.4\phantom{0}$ & ${-}31.8\phantom{0}$ & $\phantom{0}{-}1.1\phantom{0}$ & $\phantom{0}{-}1.1\phantom{0}$ & $\phantom{0}{-}3.5\phantom{0}$ & $\phantom{0}0.02\phantom{0}$ & $\phantom{0}0.06\phantom{0}$ & $\phantom{0}0.05\phantom{0}$ & $\phantom{0}0.02\phantom{0}$ & $\phantom{0}0.02\phantom{0}$ & $\phantom{0}0.02\phantom{0}$ & $\phantom{0}93.8\phantom{0}$ & $\phantom{0}\phantom{0}1.8\phantom{0}$ & $\phantom{0}15.1\phantom{0}$ & $\phantom{0}94.6\phantom{0}$ & $\phantom{0}94.8\phantom{0}$ & $\phantom{0}94.8\phantom{0}$ \\
 & \nopagebreak $\;J=1000$  & $\phantom{0}{-}0.1\phantom{0}$ & ${-}37.3\phantom{0}$ & ${-}31.8\phantom{0}$ & $\phantom{0}{-}0.7\phantom{0}$ & $\phantom{0}{-}0.7\phantom{0}$ & $\phantom{0}{-}1.9\phantom{0}$ & $\phantom{0}0.01\phantom{0}$ & $\phantom{0}0.06\phantom{0}$ & $\phantom{0}0.05\phantom{0}$ & $\phantom{0}0.01\phantom{0}$ & $\phantom{0}0.01\phantom{0}$ & $\phantom{0}0.01\phantom{0}$ & $\phantom{0}95.6\phantom{0}$ & $\phantom{0}\phantom{0}0.0\phantom{0}$ & $\phantom{0}\phantom{0}0.7\phantom{0}$ & $\phantom{0}94.1\phantom{0}$ & $\phantom{0}94.1\phantom{0}$ & $\phantom{0}93.9\phantom{0}$ \\
[0.5ex]\hline\\[-1.6ex] 
& & \multicolumn{18}{c}{Moderate intraclass correlation $(\rho_{Iy}=.30)$} \\[0.6ex]\hline\\[-1.8ex]
\multicolumn{4}{l}{$n=5$} \\  & \nopagebreak $\;J=30$  & $\phantom{0}\phantom{-}0.2\phantom{0}$ & ${-}36.9\phantom{0}$ & ${-}20.5\phantom{0}$ & ${-}12.6\phantom{0}$ & $\phantom{0}{-}9.3\phantom{0}$ & ${-}19.5\phantom{0}$ & $\phantom{0}0.12\phantom{0}$ & $\phantom{0}0.15\phantom{0}$ & $\phantom{0}0.15\phantom{0}$ & $\phantom{0}0.15\phantom{0}$ & $\phantom{0}0.15\phantom{0}$ & $\phantom{0}0.15\phantom{0}$ & $\phantom{0}90.5\phantom{0}$ & $\phantom{0}74.2\phantom{0}$ & $\phantom{0}87.9\phantom{0}$ & $\phantom{0}91.2\phantom{0}$ & $\phantom{0}89.9\phantom{0}$ & $\phantom{0}93.2\phantom{0}$ \\
 & \nopagebreak $\;J=50$  & $\phantom{0}{-}0.3\phantom{0}$ & ${-}36.7\phantom{0}$ & ${-}20.0\phantom{0}$ & $\phantom{0}{-}7.4\phantom{0}$ & $\phantom{0}{-}6.0\phantom{0}$ & ${-}13.4\phantom{0}$ & $\phantom{0}0.09\phantom{0}$ & $\phantom{0}0.13\phantom{0}$ & $\phantom{0}0.12\phantom{0}$ & $\phantom{0}0.12\phantom{0}$ & $\phantom{0}0.12\phantom{0}$ & $\phantom{0}0.12\phantom{0}$ & $\phantom{0}92.3\phantom{0}$ & $\phantom{0}70.9\phantom{0}$ & $\phantom{0}89.8\phantom{0}$ & $\phantom{0}93.1\phantom{0}$ & $\phantom{0}92.0\phantom{0}$ & $\phantom{0}94.5\phantom{0}$ \\
 & \nopagebreak $\;J=100$  & $\phantom{0}\phantom{-}0.3\phantom{0}$ & ${-}37.3\phantom{0}$ & ${-}19.9\phantom{0}$ & $\phantom{0}{-}4.1\phantom{0}$ & $\phantom{0}{-}3.3\phantom{0}$ & $\phantom{0}{-}7.6\phantom{0}$ & $\phantom{0}0.06\phantom{0}$ & $\phantom{0}0.12\phantom{0}$ & $\phantom{0}0.09\phantom{0}$ & $\phantom{0}0.08\phantom{0}$ & $\phantom{0}0.08\phantom{0}$ & $\phantom{0}0.08\phantom{0}$ & $\phantom{0}94.2\phantom{0}$ & $\phantom{0}51.5\phantom{0}$ & $\phantom{0}86.1\phantom{0}$ & $\phantom{0}93.0\phantom{0}$ & $\phantom{0}92.1\phantom{0}$ & $\phantom{0}93.9\phantom{0}$ \\
 & \nopagebreak $\;J=200$  & $\phantom{0}{-}0.4\phantom{0}$ & ${-}37.7\phantom{0}$ & ${-}20.3\phantom{0}$ & $\phantom{0}{-}2.7\phantom{0}$ & $\phantom{0}{-}2.4\phantom{0}$ & $\phantom{0}{-}4.4\phantom{0}$ & $\phantom{0}0.04\phantom{0}$ & $\phantom{0}0.11\phantom{0}$ & $\phantom{0}0.08\phantom{0}$ & $\phantom{0}0.06\phantom{0}$ & $\phantom{0}0.06\phantom{0}$ & $\phantom{0}0.06\phantom{0}$ & $\phantom{0}93.7\phantom{0}$ & $\phantom{0}25.5\phantom{0}$ & $\phantom{0}79.9\phantom{0}$ & $\phantom{0}93.6\phantom{0}$ & $\phantom{0}92.5\phantom{0}$ & $\phantom{0}93.8\phantom{0}$ \\
 & \nopagebreak $\;J=500$  & $\phantom{0}{-}0.2\phantom{0}$ & ${-}37.6\phantom{0}$ & ${-}20.1\phantom{0}$ & $\phantom{0}{-}1.5\phantom{0}$ & $\phantom{0}{-}1.2\phantom{0}$ & $\phantom{0}{-}2.0\phantom{0}$ & $\phantom{0}0.03\phantom{0}$ & $\phantom{0}0.11\phantom{0}$ & $\phantom{0}0.06\phantom{0}$ & $\phantom{0}0.04\phantom{0}$ & $\phantom{0}0.04\phantom{0}$ & $\phantom{0}0.04\phantom{0}$ & $\phantom{0}94.9\phantom{0}$ & $\phantom{0}\phantom{0}1.6\phantom{0}$ & $\phantom{0}60.7\phantom{0}$ & $\phantom{0}94.6\phantom{0}$ & $\phantom{0}94.2\phantom{0}$ & $\phantom{0}95.1\phantom{0}$ \\
 & \nopagebreak $\;J=1000$  & $\phantom{0}\phantom{-}0.1\phantom{0}$ & ${-}37.3\phantom{0}$ & ${-}19.6\phantom{0}$ & $\phantom{0}{-}0.6\phantom{0}$ & $\phantom{0}{-}0.4\phantom{0}$ & $\phantom{0}{-}0.8\phantom{0}$ & $\phantom{0}0.02\phantom{0}$ & $\phantom{0}0.10\phantom{0}$ & $\phantom{0}0.06\phantom{0}$ & $\phantom{0}0.02\phantom{0}$ & $\phantom{0}0.02\phantom{0}$ & $\phantom{0}0.02\phantom{0}$ & $\phantom{0}94.0\phantom{0}$ & $\phantom{0}\phantom{0}0.0\phantom{0}$ & $\phantom{0}33.7\phantom{0}$ & $\phantom{0}94.8\phantom{0}$ & $\phantom{0}94.5\phantom{0}$ & $\phantom{0}95.3\phantom{0}$ \\
\multicolumn{4}{l}{$n=20$} \\  & \nopagebreak $\;J=30$  & $\phantom{0}{-}1.9\phantom{0}$ & ${-}36.7\phantom{0}$ & ${-}22.1\phantom{0}$ & ${-}11.8\phantom{0}$ & ${-}11.2\phantom{0}$ & ${-}17.7\phantom{0}$ & $\phantom{0}0.10\phantom{0}$ & $\phantom{0}0.14\phantom{0}$ & $\phantom{0}0.13\phantom{0}$ & $\phantom{0}0.12\phantom{0}$ & $\phantom{0}0.12\phantom{0}$ & $\phantom{0}0.13\phantom{0}$ & $\phantom{0}91.0\phantom{0}$ & $\phantom{0}71.5\phantom{0}$ & $\phantom{0}85.1\phantom{0}$ & $\phantom{0}92.5\phantom{0}$ & $\phantom{0}92.6\phantom{0}$ & $\phantom{0}94.1\phantom{0}$ \\
 & \nopagebreak $\;J=50$  & $\phantom{0}{-}0.7\phantom{0}$ & ${-}35.9\phantom{0}$ & ${-}21.2\phantom{0}$ & $\phantom{0}{-}7.5\phantom{0}$ & $\phantom{0}{-}7.5\phantom{0}$ & ${-}11.9\phantom{0}$ & $\phantom{0}0.07\phantom{0}$ & $\phantom{0}0.12\phantom{0}$ & $\phantom{0}0.11\phantom{0}$ & $\phantom{0}0.10\phantom{0}$ & $\phantom{0}0.10\phantom{0}$ & $\phantom{0}0.10\phantom{0}$ & $\phantom{0}92.9\phantom{0}$ & $\phantom{0}64.7\phantom{0}$ & $\phantom{0}84.5\phantom{0}$ & $\phantom{0}92.7\phantom{0}$ & $\phantom{0}93.5\phantom{0}$ & $\phantom{0}93.9\phantom{0}$ \\
 & \nopagebreak $\;J=100$  & $\phantom{0}\phantom{-}0.2\phantom{0}$ & ${-}35.5\phantom{0}$ & ${-}20.4\phantom{0}$ & $\phantom{0}{-}3.3\phantom{0}$ & $\phantom{0}{-}3.2\phantom{0}$ & $\phantom{0}{-}5.8\phantom{0}$ & $\phantom{0}0.05\phantom{0}$ & $\phantom{0}0.11\phantom{0}$ & $\phantom{0}0.08\phantom{0}$ & $\phantom{0}0.07\phantom{0}$ & $\phantom{0}0.07\phantom{0}$ & $\phantom{0}0.07\phantom{0}$ & $\phantom{0}94.4\phantom{0}$ & $\phantom{0}47.5\phantom{0}$ & $\phantom{0}79.9\phantom{0}$ & $\phantom{0}94.8\phantom{0}$ & $\phantom{0}94.2\phantom{0}$ & $\phantom{0}95.0\phantom{0}$ \\
 & \nopagebreak $\;J=200$  & $\phantom{0}\phantom{-}0.4\phantom{0}$ & ${-}35.7\phantom{0}$ & ${-}20.4\phantom{0}$ & $\phantom{0}{-}1.1\phantom{0}$ & $\phantom{0}{-}1.4\phantom{0}$ & $\phantom{0}{-}2.6\phantom{0}$ & $\phantom{0}0.04\phantom{0}$ & $\phantom{0}0.10\phantom{0}$ & $\phantom{0}0.07\phantom{0}$ & $\phantom{0}0.04\phantom{0}$ & $\phantom{0}0.04\phantom{0}$ & $\phantom{0}0.04\phantom{0}$ & $\phantom{0}94.4\phantom{0}$ & $\phantom{0}18.6\phantom{0}$ & $\phantom{0}71.2\phantom{0}$ & $\phantom{0}95.7\phantom{0}$ & $\phantom{0}95.7\phantom{0}$ & $\phantom{0}95.9\phantom{0}$ \\
 & \nopagebreak $\;J=500$  & $\phantom{0}{-}0.1\phantom{0}$ & ${-}35.9\phantom{0}$ & ${-}20.9\phantom{0}$ & $\phantom{0}{-}1.0\phantom{0}$ & $\phantom{0}{-}1.0\phantom{0}$ & $\phantom{0}{-}1.5\phantom{0}$ & $\phantom{0}0.02\phantom{0}$ & $\phantom{0}0.10\phantom{0}$ & $\phantom{0}0.06\phantom{0}$ & $\phantom{0}0.03\phantom{0}$ & $\phantom{0}0.03\phantom{0}$ & $\phantom{0}0.03\phantom{0}$ & $\phantom{0}94.1\phantom{0}$ & $\phantom{0}\phantom{0}0.8\phantom{0}$ & $\phantom{0}40.1\phantom{0}$ & $\phantom{0}92.9\phantom{0}$ & $\phantom{0}94.4\phantom{0}$ & $\phantom{0}93.7\phantom{0}$ \\
 & \nopagebreak $\;J=1000$  & $\phantom{0}\phantom{-}0.1\phantom{0}$ & ${-}35.4\phantom{0}$ & ${-}20.3\phantom{0}$ & $\phantom{0}{-}0.2\phantom{0}$ & $\phantom{0}{-}0.2\phantom{0}$ & $\phantom{0}{-}0.5\phantom{0}$ & $\phantom{0}0.02\phantom{0}$ & $\phantom{0}0.10\phantom{0}$ & $\phantom{0}0.06\phantom{0}$ & $\phantom{0}0.02\phantom{0}$ & $\phantom{0}0.02\phantom{0}$ & $\phantom{0}0.02\phantom{0}$ & $\phantom{0}94.5\phantom{0}$ & $\phantom{0}\phantom{0}0.0\phantom{0}$ & $\phantom{0}15.2\phantom{0}$ & $\phantom{0}92.9\phantom{0}$ & $\phantom{0}93.7\phantom{0}$ & $\phantom{0}94.6\phantom{0}$ \\
[0.5ex]\hline\\[-1.6ex] 
\end{tabular}
\begin{tablenotes}[para,flushleft]{\footnotesize \textit{Note.} $n$ = cluster size; $J$ = number of clusters; CD = complete data sets; LD = listwise deletion; FCS-SL = single-level FCS; FCS-MAN = two-level FCS with manifest cluster means; FCS-LAT = two-level FCS with latent cluster means; JM = joint modeling.}\end{tablenotes}
\end{threeparttable}
\end{sidewaystable}
\begin{sidewaystable}
\begin{threeparttable}
\setlength{\tabcolsep}{1.2pt}
\renewcommand{\arraystretch}{0.95}
\footnotesize
\caption{\small Study 1: Bias (in \%), RMSE, and Coverage of the 95\% Confidence Interval for the Regression Coefficient of $y$ on $z$ ($\hat\beta_{yz}$) With 40\% Missing Data (MCAR, $\lambda=0$)}
\begin{tabular}{llcccccccccccccccccc}
\hline\\[-1.8ex]
& & \multicolumn{6}{c}{Bias (\%)} & \multicolumn{6}{c}{RMSE} & \multicolumn{6}{c}{Coverage (\%)} \\ \cmidrule(r){3-8}\cmidrule(r){9-14}\cmidrule(r){15-20}
 &  & CD & LD & \makecell{FCS-\\SL} & \makecell{FCS-\\MAN} & \makecell{FCS-\\LAT} & JM & CD & LD & \makecell{FCS-\\SL} & \makecell{FCS-\\MAN} & \makecell{FCS-\\LAT} & JM & CD & LD & \makecell{FCS-\\SL} & \makecell{FCS-\\MAN} & \makecell{FCS-\\LAT} & \multicolumn{1}{c}{JM} \\ 
[0.4ex]\hline\\[-1.8ex]
& & \multicolumn{18}{c}{Small intraclass correlation $(\rho_{Iy}=.10)$} \\[0.6ex]\hline\\[-1.8ex]
\multicolumn{4}{l}{$n=5$} \\  & \nopagebreak $\;J=30$  & $\phantom{0}{-}1.7\phantom{0}$ & $\phantom{0}{-}1.2\phantom{0}$ & $\phantom{0}\phantom{-}2.3\phantom{0}$ & $\phantom{0}{-}9.5\phantom{0}$ & $\phantom{0}{-}5.8\phantom{0}$ & ${-}21.7\phantom{0}$ & $\phantom{0}0.10\phantom{0}$ & $\phantom{0}0.13\phantom{0}$ & $\phantom{0}0.14\phantom{0}$ & $\phantom{0}0.12\phantom{0}$ & $\phantom{0}0.13\phantom{0}$ & $\phantom{0}0.11\phantom{0}$ & $\phantom{0}89.9\phantom{0}$ & $\phantom{0}87.5\phantom{0}$ & $\phantom{0}89.4\phantom{0}$ & $\phantom{0}91.6\phantom{0}$ & $\phantom{0}90.1\phantom{0}$ & $\phantom{0}94.3\phantom{0}$ \\
 & \nopagebreak $\;J=50$  & $\phantom{0}\phantom{-}0.2\phantom{0}$ & $\phantom{0}\phantom{-}0.0\phantom{0}$ & $\phantom{0}\phantom{-}4.1\phantom{0}$ & $\phantom{0}{-}3.9\phantom{0}$ & $\phantom{0}\phantom{-}0.2\phantom{0}$ & ${-}15.3\phantom{0}$ & $\phantom{0}0.08\phantom{0}$ & $\phantom{0}0.10\phantom{0}$ & $\phantom{0}0.10\phantom{0}$ & $\phantom{0}0.10\phantom{0}$ & $\phantom{0}0.10\phantom{0}$ & $\phantom{0}0.09\phantom{0}$ & $\phantom{0}92.8\phantom{0}$ & $\phantom{0}90.5\phantom{0}$ & $\phantom{0}91.0\phantom{0}$ & $\phantom{0}92.9\phantom{0}$ & $\phantom{0}91.5\phantom{0}$ & $\phantom{0}93.9\phantom{0}$ \\
 & \nopagebreak $\;J=100$  & $\phantom{0}\phantom{-}0.3\phantom{0}$ & $\phantom{0}{-}0.8\phantom{0}$ & $\phantom{0}\phantom{-}3.6\phantom{0}$ & $\phantom{0}{-}2.9\phantom{0}$ & $\phantom{0}\phantom{-}1.1\phantom{0}$ & ${-}11.4\phantom{0}$ & $\phantom{0}0.05\phantom{0}$ & $\phantom{0}0.07\phantom{0}$ & $\phantom{0}0.07\phantom{0}$ & $\phantom{0}0.07\phantom{0}$ & $\phantom{0}0.07\phantom{0}$ & $\phantom{0}0.06\phantom{0}$ & $\phantom{0}93.3\phantom{0}$ & $\phantom{0}92.4\phantom{0}$ & $\phantom{0}91.7\phantom{0}$ & $\phantom{0}93.1\phantom{0}$ & $\phantom{0}91.9\phantom{0}$ & $\phantom{0}94.3\phantom{0}$ \\
 & \nopagebreak $\;J=200$  & $\phantom{0}\phantom{-}0.3\phantom{0}$ & $\phantom{0}{-}0.6\phantom{0}$ & $\phantom{0}\phantom{-}3.9\phantom{0}$ & $\phantom{0}{-}0.9\phantom{0}$ & $\phantom{0}\phantom{-}1.6\phantom{0}$ & $\phantom{0}{-}7.2\phantom{0}$ & $\phantom{0}0.04\phantom{0}$ & $\phantom{0}0.05\phantom{0}$ & $\phantom{0}0.05\phantom{0}$ & $\phantom{0}0.04\phantom{0}$ & $\phantom{0}0.04\phantom{0}$ & $\phantom{0}0.04\phantom{0}$ & $\phantom{0}94.2\phantom{0}$ & $\phantom{0}94.0\phantom{0}$ & $\phantom{0}94.0\phantom{0}$ & $\phantom{0}93.9\phantom{0}$ & $\phantom{0}94.3\phantom{0}$ & $\phantom{0}95.5\phantom{0}$ \\
 & \nopagebreak $\;J=500$  & $\phantom{0}{-}0.8\phantom{0}$ & $\phantom{0}{-}1.3\phantom{0}$ & $\phantom{0}\phantom{-}3.4\phantom{0}$ & $\phantom{0}{-}1.7\phantom{0}$ & $\phantom{0}{-}1.0\phantom{0}$ & $\phantom{0}{-}4.7\phantom{0}$ & $\phantom{0}0.02\phantom{0}$ & $\phantom{0}0.03\phantom{0}$ & $\phantom{0}0.03\phantom{0}$ & $\phantom{0}0.03\phantom{0}$ & $\phantom{0}0.03\phantom{0}$ & $\phantom{0}0.03\phantom{0}$ & $\phantom{0}94.1\phantom{0}$ & $\phantom{0}93.8\phantom{0}$ & $\phantom{0}92.6\phantom{0}$ & $\phantom{0}94.2\phantom{0}$ & $\phantom{0}93.5\phantom{0}$ & $\phantom{0}93.9\phantom{0}$ \\
 & \nopagebreak $\;J=1000$  & $\phantom{0}{-}0.3\phantom{0}$ & $\phantom{0}{-}0.3\phantom{0}$ & $\phantom{0}\phantom{-}4.2\phantom{0}$ & $\phantom{0}{-}0.4\phantom{0}$ & $\phantom{0}{-}0.1\phantom{0}$ & $\phantom{0}{-}2.3\phantom{0}$ & $\phantom{0}0.02\phantom{0}$ & $\phantom{0}0.02\phantom{0}$ & $\phantom{0}0.02\phantom{0}$ & $\phantom{0}0.02\phantom{0}$ & $\phantom{0}0.02\phantom{0}$ & $\phantom{0}0.02\phantom{0}$ & $\phantom{0}95.8\phantom{0}$ & $\phantom{0}95.4\phantom{0}$ & $\phantom{0}93.0\phantom{0}$ & $\phantom{0}94.3\phantom{0}$ & $\phantom{0}94.5\phantom{0}$ & $\phantom{0}95.3\phantom{0}$ \\
\multicolumn{4}{l}{$n=20$} \\  & \nopagebreak $\;J=30$  & $\phantom{0}\phantom{-}1.4\phantom{0}$ & $\phantom{0}\phantom{-}1.3\phantom{0}$ & $\phantom{0}\phantom{-}8.0\phantom{0}$ & $\phantom{0}{-}4.6\phantom{0}$ & $\phantom{0}{-}2.7\phantom{0}$ & ${-}17.6\phantom{0}$ & $\phantom{0}0.07\phantom{0}$ & $\phantom{0}0.09\phantom{0}$ & $\phantom{0}0.10\phantom{0}$ & $\phantom{0}0.09\phantom{0}$ & $\phantom{0}0.09\phantom{0}$ & $\phantom{0}0.08\phantom{0}$ & $\phantom{0}90.9\phantom{0}$ & $\phantom{0}88.9\phantom{0}$ & $\phantom{0}87.9\phantom{0}$ & $\phantom{0}91.7\phantom{0}$ & $\phantom{0}91.1\phantom{0}$ & $\phantom{0}95.1\phantom{0}$ \\
 & \nopagebreak $\;J=50$  & $\phantom{0}\phantom{-}1.0\phantom{0}$ & $\phantom{0}\phantom{-}1.7\phantom{0}$ & $\phantom{0}\phantom{-}7.7\phantom{0}$ & $\phantom{0}{-}2.2\phantom{0}$ & $\phantom{0}{-}1.2\phantom{0}$ & ${-}12.5\phantom{0}$ & $\phantom{0}0.05\phantom{0}$ & $\phantom{0}0.07\phantom{0}$ & $\phantom{0}0.07\phantom{0}$ & $\phantom{0}0.07\phantom{0}$ & $\phantom{0}0.07\phantom{0}$ & $\phantom{0}0.06\phantom{0}$ & $\phantom{0}91.8\phantom{0}$ & $\phantom{0}89.7\phantom{0}$ & $\phantom{0}89.1\phantom{0}$ & $\phantom{0}92.2\phantom{0}$ & $\phantom{0}91.4\phantom{0}$ & $\phantom{0}94.7\phantom{0}$ \\
 & \nopagebreak $\;J=100$  & $\phantom{0}\phantom{-}0.7\phantom{0}$ & $\phantom{0}\phantom{-}0.4\phantom{0}$ & $\phantom{0}\phantom{-}6.5\phantom{0}$ & $\phantom{0}{-}1.2\phantom{0}$ & $\phantom{0}{-}0.9\phantom{0}$ & $\phantom{0}{-}7.5\phantom{0}$ & $\phantom{0}0.03\phantom{0}$ & $\phantom{0}0.05\phantom{0}$ & $\phantom{0}0.05\phantom{0}$ & $\phantom{0}0.04\phantom{0}$ & $\phantom{0}0.04\phantom{0}$ & $\phantom{0}0.04\phantom{0}$ & $\phantom{0}93.9\phantom{0}$ & $\phantom{0}91.8\phantom{0}$ & $\phantom{0}90.6\phantom{0}$ & $\phantom{0}92.9\phantom{0}$ & $\phantom{0}92.8\phantom{0}$ & $\phantom{0}95.7\phantom{0}$ \\
 & \nopagebreak $\;J=200$  & $\phantom{0}\phantom{-}0.4\phantom{0}$ & $\phantom{0}\phantom{-}0.6\phantom{0}$ & $\phantom{0}\phantom{-}6.7\phantom{0}$ & $\phantom{0}{-}0.2\phantom{0}$ & $\phantom{0}\phantom{-}0.1\phantom{0}$ & $\phantom{0}{-}4.0\phantom{0}$ & $\phantom{0}0.02\phantom{0}$ & $\phantom{0}0.03\phantom{0}$ & $\phantom{0}0.04\phantom{0}$ & $\phantom{0}0.03\phantom{0}$ & $\phantom{0}0.03\phantom{0}$ & $\phantom{0}0.03\phantom{0}$ & $\phantom{0}94.1\phantom{0}$ & $\phantom{0}94.1\phantom{0}$ & $\phantom{0}91.9\phantom{0}$ & $\phantom{0}94.1\phantom{0}$ & $\phantom{0}94.5\phantom{0}$ & $\phantom{0}95.6\phantom{0}$ \\
 & \nopagebreak $\;J=500$  & $\phantom{0}\phantom{-}0.4\phantom{0}$ & $\phantom{0}\phantom{-}0.8\phantom{0}$ & $\phantom{0}\phantom{-}6.8\phantom{0}$ & $\phantom{0}\phantom{-}0.4\phantom{0}$ & $\phantom{0}\phantom{-}0.5\phantom{0}$ & $\phantom{0}{-}1.2\phantom{0}$ & $\phantom{0}0.01\phantom{0}$ & $\phantom{0}0.02\phantom{0}$ & $\phantom{0}0.02\phantom{0}$ & $\phantom{0}0.02\phantom{0}$ & $\phantom{0}0.02\phantom{0}$ & $\phantom{0}0.02\phantom{0}$ & $\phantom{0}96.0\phantom{0}$ & $\phantom{0}95.2\phantom{0}$ & $\phantom{0}90.5\phantom{0}$ & $\phantom{0}95.3\phantom{0}$ & $\phantom{0}95.1\phantom{0}$ & $\phantom{0}95.5\phantom{0}$ \\
 & \nopagebreak $\;J=1000$  & $\phantom{0}\phantom{-}0.0\phantom{0}$ & $\phantom{0}\phantom{-}0.1\phantom{0}$ & $\phantom{0}\phantom{-}6.1\phantom{0}$ & $\phantom{0}{-}0.1\phantom{0}$ & $\phantom{0}{-}0.1\phantom{0}$ & $\phantom{0}{-}1.0\phantom{0}$ & $\phantom{0}0.01\phantom{0}$ & $\phantom{0}0.01\phantom{0}$ & $\phantom{0}0.02\phantom{0}$ & $\phantom{0}0.01\phantom{0}$ & $\phantom{0}0.01\phantom{0}$ & $\phantom{0}0.01\phantom{0}$ & $\phantom{0}96.2\phantom{0}$ & $\phantom{0}94.5\phantom{0}$ & $\phantom{0}89.1\phantom{0}$ & $\phantom{0}94.5\phantom{0}$ & $\phantom{0}95.4\phantom{0}$ & $\phantom{0}95.4\phantom{0}$ \\
[0.5ex]\hline\\[-1.6ex] 
& & \multicolumn{18}{c}{Moderate intraclass correlation $(\rho_{Iy}=.30)$} \\[0.6ex]\hline\\[-1.8ex]
\multicolumn{4}{l}{$n=5$} \\  & \nopagebreak $\;J=30$  & $\phantom{0}\phantom{-}0.6\phantom{0}$ & $\phantom{0}\phantom{-}1.6\phantom{0}$ & $\phantom{-}12.8\phantom{0}$ & $\phantom{0}{-}6.0\phantom{0}$ & $\phantom{0}{-}3.9\phantom{0}$ & ${-}12.0\phantom{0}$ & $\phantom{0}0.12\phantom{0}$ & $\phantom{0}0.16\phantom{0}$ & $\phantom{0}0.18\phantom{0}$ & $\phantom{0}0.15\phantom{0}$ & $\phantom{0}0.15\phantom{0}$ & $\phantom{0}0.15\phantom{0}$ & $\phantom{0}90.9\phantom{0}$ & $\phantom{0}87.7\phantom{0}$ & $\phantom{0}86.3\phantom{0}$ & $\phantom{0}92.6\phantom{0}$ & $\phantom{0}91.5\phantom{0}$ & $\phantom{0}94.5\phantom{0}$ \\
 & \nopagebreak $\;J=50$  & $\phantom{0}\phantom{-}0.4\phantom{0}$ & $\phantom{0}{-}0.4\phantom{0}$ & $\phantom{-}11.9\phantom{0}$ & $\phantom{0}{-}4.4\phantom{0}$ & $\phantom{0}{-}2.9\phantom{0}$ & $\phantom{0}{-}8.8\phantom{0}$ & $\phantom{0}0.09\phantom{0}$ & $\phantom{0}0.12\phantom{0}$ & $\phantom{0}0.13\phantom{0}$ & $\phantom{0}0.11\phantom{0}$ & $\phantom{0}0.11\phantom{0}$ & $\phantom{0}0.11\phantom{0}$ & $\phantom{0}92.9\phantom{0}$ & $\phantom{0}91.5\phantom{0}$ & $\phantom{0}89.3\phantom{0}$ & $\phantom{0}93.0\phantom{0}$ & $\phantom{0}92.9\phantom{0}$ & $\phantom{0}94.4\phantom{0}$ \\
 & \nopagebreak $\;J=100$  & $\phantom{0}\phantom{-}0.1\phantom{0}$ & $\phantom{0}\phantom{-}0.1\phantom{0}$ & $\phantom{-}12.7\phantom{0}$ & $\phantom{0}{-}1.6\phantom{0}$ & $\phantom{0}{-}1.1\phantom{0}$ & $\phantom{0}{-}4.0\phantom{0}$ & $\phantom{0}0.06\phantom{0}$ & $\phantom{0}0.08\phantom{0}$ & $\phantom{0}0.10\phantom{0}$ & $\phantom{0}0.08\phantom{0}$ & $\phantom{0}0.08\phantom{0}$ & $\phantom{0}0.08\phantom{0}$ & $\phantom{0}93.2\phantom{0}$ & $\phantom{0}93.2\phantom{0}$ & $\phantom{0}88.4\phantom{0}$ & $\phantom{0}92.8\phantom{0}$ & $\phantom{0}93.4\phantom{0}$ & $\phantom{0}95.2\phantom{0}$ \\
 & \nopagebreak $\;J=200$  & $\phantom{0}\phantom{-}0.1\phantom{0}$ & $\phantom{0}\phantom{-}0.5\phantom{0}$ & $\phantom{-}13.3\phantom{0}$ & $\phantom{0}{-}0.3\phantom{0}$ & $\phantom{0}{-}0.1\phantom{0}$ & $\phantom{0}{-}1.0\phantom{0}$ & $\phantom{0}0.04\phantom{0}$ & $\phantom{0}0.06\phantom{0}$ & $\phantom{0}0.07\phantom{0}$ & $\phantom{0}0.05\phantom{0}$ & $\phantom{0}0.05\phantom{0}$ & $\phantom{0}0.05\phantom{0}$ & $\phantom{0}95.1\phantom{0}$ & $\phantom{0}94.5\phantom{0}$ & $\phantom{0}87.3\phantom{0}$ & $\phantom{0}94.4\phantom{0}$ & $\phantom{0}94.2\phantom{0}$ & $\phantom{0}94.2\phantom{0}$ \\
 & \nopagebreak $\;J=500$  & $\phantom{0}\phantom{-}0.0\phantom{0}$ & $\phantom{0}\phantom{-}0.0\phantom{0}$ & $\phantom{-}12.8\phantom{0}$ & $\phantom{0}{-}0.4\phantom{0}$ & $\phantom{0}{-}0.4\phantom{0}$ & $\phantom{0}{-}0.9\phantom{0}$ & $\phantom{0}0.03\phantom{0}$ & $\phantom{0}0.03\phantom{0}$ & $\phantom{0}0.05\phantom{0}$ & $\phantom{0}0.03\phantom{0}$ & $\phantom{0}0.03\phantom{0}$ & $\phantom{0}0.03\phantom{0}$ & $\phantom{0}94.8\phantom{0}$ & $\phantom{0}95.1\phantom{0}$ & $\phantom{0}79.3\phantom{0}$ & $\phantom{0}94.6\phantom{0}$ & $\phantom{0}95.3\phantom{0}$ & $\phantom{0}94.6\phantom{0}$ \\
 & \nopagebreak $\;J=1000$  & $\phantom{0}{-}0.1\phantom{0}$ & $\phantom{0}{-}0.1\phantom{0}$ & $\phantom{-}12.8\phantom{0}$ & $\phantom{0}{-}0.3\phantom{0}$ & $\phantom{0}{-}0.2\phantom{0}$ & $\phantom{0}{-}0.5\phantom{0}$ & $\phantom{0}0.02\phantom{0}$ & $\phantom{0}0.02\phantom{0}$ & $\phantom{0}0.04\phantom{0}$ & $\phantom{0}0.02\phantom{0}$ & $\phantom{0}0.02\phantom{0}$ & $\phantom{0}0.02\phantom{0}$ & $\phantom{0}94.9\phantom{0}$ & $\phantom{0}94.9\phantom{0}$ & $\phantom{0}69.4\phantom{0}$ & $\phantom{0}94.1\phantom{0}$ & $\phantom{0}93.9\phantom{0}$ & $\phantom{0}95.3\phantom{0}$ \\
\multicolumn{4}{l}{$n=20$} \\  & \nopagebreak $\;J=30$  & $\phantom{0}{-}1.6\phantom{0}$ & $\phantom{0}{-}1.4\phantom{0}$ & $\phantom{-}15.1\phantom{0}$ & $\phantom{0}{-}7.1\phantom{0}$ & $\phantom{0}{-}6.6\phantom{0}$ & ${-}12.4\phantom{0}$ & $\phantom{0}0.10\phantom{0}$ & $\phantom{0}0.13\phantom{0}$ & $\phantom{0}0.16\phantom{0}$ & $\phantom{0}0.13\phantom{0}$ & $\phantom{0}0.13\phantom{0}$ & $\phantom{0}0.12\phantom{0}$ & $\phantom{0}91.4\phantom{0}$ & $\phantom{0}89.0\phantom{0}$ & $\phantom{0}84.7\phantom{0}$ & $\phantom{0}92.8\phantom{0}$ & $\phantom{0}92.9\phantom{0}$ & $\phantom{0}94.9\phantom{0}$ \\
 & \nopagebreak $\;J=50$  & $\phantom{0}\phantom{-}1.3\phantom{0}$ & $\phantom{0}\phantom{-}1.0\phantom{0}$ & $\phantom{-}18.0\phantom{0}$ & $\phantom{0}{-}1.8\phantom{0}$ & $\phantom{0}{-}1.7\phantom{0}$ & $\phantom{0}{-}5.0\phantom{0}$ & $\phantom{0}0.08\phantom{0}$ & $\phantom{0}0.10\phantom{0}$ & $\phantom{0}0.12\phantom{0}$ & $\phantom{0}0.09\phantom{0}$ & $\phantom{0}0.09\phantom{0}$ & $\phantom{0}0.09\phantom{0}$ & $\phantom{0}91.0\phantom{0}$ & $\phantom{0}91.3\phantom{0}$ & $\phantom{0}85.5\phantom{0}$ & $\phantom{0}93.7\phantom{0}$ & $\phantom{0}92.8\phantom{0}$ & $\phantom{0}94.5\phantom{0}$ \\
 & \nopagebreak $\;J=100$  & $\phantom{0}\phantom{-}1.1\phantom{0}$ & $\phantom{0}\phantom{-}1.5\phantom{0}$ & $\phantom{-}18.5\phantom{0}$ & $\phantom{0}{-}0.7\phantom{0}$ & $\phantom{0}{-}0.4\phantom{0}$ & $\phantom{0}{-}2.4\phantom{0}$ & $\phantom{0}0.05\phantom{0}$ & $\phantom{0}0.07\phantom{0}$ & $\phantom{0}0.09\phantom{0}$ & $\phantom{0}0.06\phantom{0}$ & $\phantom{0}0.06\phantom{0}$ & $\phantom{0}0.06\phantom{0}$ & $\phantom{0}94.2\phantom{0}$ & $\phantom{0}93.5\phantom{0}$ & $\phantom{0}84.8\phantom{0}$ & $\phantom{0}94.8\phantom{0}$ & $\phantom{0}94.4\phantom{0}$ & $\phantom{0}94.9\phantom{0}$ \\
 & \nopagebreak $\;J=200$  & $\phantom{0}{-}0.7\phantom{0}$ & $\phantom{0}{-}0.6\phantom{0}$ & $\phantom{-}16.1\phantom{0}$ & $\phantom{0}{-}1.1\phantom{0}$ & $\phantom{0}{-}1.0\phantom{0}$ & $\phantom{0}{-}1.9\phantom{0}$ & $\phantom{0}0.04\phantom{0}$ & $\phantom{0}0.05\phantom{0}$ & $\phantom{0}0.07\phantom{0}$ & $\phantom{0}0.05\phantom{0}$ & $\phantom{0}0.05\phantom{0}$ & $\phantom{0}0.05\phantom{0}$ & $\phantom{0}93.8\phantom{0}$ & $\phantom{0}93.2\phantom{0}$ & $\phantom{0}80.8\phantom{0}$ & $\phantom{0}94.0\phantom{0}$ & $\phantom{0}93.8\phantom{0}$ & $\phantom{0}94.4\phantom{0}$ \\
 & \nopagebreak $\;J=500$  & $\phantom{0}\phantom{-}0.5\phantom{0}$ & $\phantom{0}\phantom{-}0.3\phantom{0}$ & $\phantom{-}17.1\phantom{0}$ & $\phantom{0}\phantom{-}0.1\phantom{0}$ & $\phantom{0}\phantom{-}0.1\phantom{0}$ & $\phantom{0}{-}0.4\phantom{0}$ & $\phantom{0}0.02\phantom{0}$ & $\phantom{0}0.03\phantom{0}$ & $\phantom{0}0.06\phantom{0}$ & $\phantom{0}0.03\phantom{0}$ & $\phantom{0}0.03\phantom{0}$ & $\phantom{0}0.03\phantom{0}$ & $\phantom{0}95.2\phantom{0}$ & $\phantom{0}96.2\phantom{0}$ & $\phantom{0}63.7\phantom{0}$ & $\phantom{0}96.2\phantom{0}$ & $\phantom{0}95.2\phantom{0}$ & $\phantom{0}96.2\phantom{0}$ \\
 & \nopagebreak $\;J=1000$  & $\phantom{0}{-}0.2\phantom{0}$ & $\phantom{0}{-}0.2\phantom{0}$ & $\phantom{-}16.6\phantom{0}$ & $\phantom{0}{-}0.3\phantom{0}$ & $\phantom{0}{-}0.3\phantom{0}$ & $\phantom{0}{-}0.6\phantom{0}$ & $\phantom{0}0.02\phantom{0}$ & $\phantom{0}0.02\phantom{0}$ & $\phantom{0}0.05\phantom{0}$ & $\phantom{0}0.02\phantom{0}$ & $\phantom{0}0.02\phantom{0}$ & $\phantom{0}0.02\phantom{0}$ & $\phantom{0}95.7\phantom{0}$ & $\phantom{0}95.2\phantom{0}$ & $\phantom{0}44.3\phantom{0}$ & $\phantom{0}95.2\phantom{0}$ & $\phantom{0}95.1\phantom{0}$ & $\phantom{0}95.6\phantom{0}$ \\
[0.5ex]\hline\\[-1.6ex] 
\end{tabular}
\begin{tablenotes}[para,flushleft]{\footnotesize \textit{Note.} $n$ = cluster size; $J$ = number of clusters; CD = complete data sets; LD = listwise deletion; FCS-SL = single-level FCS; FCS-MAN = two-level FCS with manifest cluster means; FCS-LAT = two-level FCS with latent cluster means; JM = joint modeling.}\end{tablenotes}
\end{threeparttable}
\end{sidewaystable}
\begin{sidewaystable}
\begin{threeparttable}
\setlength{\tabcolsep}{1.2pt}
\renewcommand{\arraystretch}{0.95}
\footnotesize
\caption{\small Study 1: Bias (in \%), RMSE, and Coverage of the 95\% Confidence Interval for the Regression Coefficient of $y$ on $z$ ($\hat\beta_{yz}$) With 40\% Missing Data (MAR, $\lambda=0.5$)}
\begin{tabular}{llcccccccccccccccccc}
\hline\\[-1.8ex]
& & \multicolumn{6}{c}{Bias (\%)} & \multicolumn{6}{c}{RMSE} & \multicolumn{6}{c}{Coverage (\%)} \\ \cmidrule(r){3-8}\cmidrule(r){9-14}\cmidrule(r){15-20}
 &  & CD & LD & \makecell{FCS-\\SL} & \makecell{FCS-\\MAN} & \makecell{FCS-\\LAT} & JM & CD & LD & \makecell{FCS-\\SL} & \makecell{FCS-\\MAN} & \makecell{FCS-\\LAT} & JM & CD & LD & \makecell{FCS-\\SL} & \makecell{FCS-\\MAN} & \makecell{FCS-\\LAT} & \multicolumn{1}{c}{JM} \\ 
[0.4ex]\hline\\[-1.8ex]
& & \multicolumn{18}{c}{Small intraclass correlation $(\rho_{Iy}=.10)$} \\[0.6ex]\hline\\[-1.8ex]
\multicolumn{4}{l}{$n=5$} \\  & \nopagebreak $\;J=30$  & $\phantom{0}\phantom{-}0.9\phantom{0}$ & ${-}10.7\phantom{0}$ & $\phantom{0}{-}3.6\phantom{0}$ & $\phantom{0}{-}8.6\phantom{0}$ & $\phantom{0}{-}2.6\phantom{0}$ & ${-}24.6\phantom{0}$ & $\phantom{0}0.10\phantom{0}$ & $\phantom{0}0.13\phantom{0}$ & $\phantom{0}0.14\phantom{0}$ & $\phantom{0}0.13\phantom{0}$ & $\phantom{0}0.13\phantom{0}$ & $\phantom{0}0.11\phantom{0}$ & $\phantom{0}90.7\phantom{0}$ & $\phantom{0}87.5\phantom{0}$ & $\phantom{0}90.7\phantom{0}$ & $\phantom{0}90.0\phantom{0}$ & $\phantom{0}88.7\phantom{0}$ & $\phantom{0}95.8\phantom{0}$ \\
 & \nopagebreak $\;J=50$  & $\phantom{0}{-}0.0\phantom{0}$ & ${-}12.1\phantom{0}$ & $\phantom{0}{-}5.0\phantom{0}$ & $\phantom{0}{-}5.7\phantom{0}$ & $\phantom{0}{-}0.9\phantom{0}$ & ${-}22.3\phantom{0}$ & $\phantom{0}0.07\phantom{0}$ & $\phantom{0}0.09\phantom{0}$ & $\phantom{0}0.10\phantom{0}$ & $\phantom{0}0.10\phantom{0}$ & $\phantom{0}0.10\phantom{0}$ & $\phantom{0}0.09\phantom{0}$ & $\phantom{0}92.8\phantom{0}$ & $\phantom{0}90.1\phantom{0}$ & $\phantom{0}92.8\phantom{0}$ & $\phantom{0}93.2\phantom{0}$ & $\phantom{0}90.7\phantom{0}$ & $\phantom{0}96.0\phantom{0}$ \\
 & \nopagebreak $\;J=100$  & $\phantom{0}\phantom{-}1.3\phantom{0}$ & ${-}11.6\phantom{0}$ & $\phantom{0}{-}3.8\phantom{0}$ & $\phantom{0}{-}1.4\phantom{0}$ & $\phantom{0}\phantom{-}3.0\phantom{0}$ & ${-}13.1\phantom{0}$ & $\phantom{0}0.05\phantom{0}$ & $\phantom{0}0.06\phantom{0}$ & $\phantom{0}0.07\phantom{0}$ & $\phantom{0}0.07\phantom{0}$ & $\phantom{0}0.07\phantom{0}$ & $\phantom{0}0.06\phantom{0}$ & $\phantom{0}94.6\phantom{0}$ & $\phantom{0}92.5\phantom{0}$ & $\phantom{0}95.0\phantom{0}$ & $\phantom{0}94.1\phantom{0}$ & $\phantom{0}92.5\phantom{0}$ & $\phantom{0}96.3\phantom{0}$ \\
 & \nopagebreak $\;J=200$  & $\phantom{0}{-}0.8\phantom{0}$ & ${-}14.0\phantom{0}$ & $\phantom{0}{-}6.5\phantom{0}$ & $\phantom{0}{-}2.4\phantom{0}$ & $\phantom{0}\phantom{-}0.7\phantom{0}$ & ${-}10.1\phantom{0}$ & $\phantom{0}0.04\phantom{0}$ & $\phantom{0}0.05\phantom{0}$ & $\phantom{0}0.05\phantom{0}$ & $\phantom{0}0.05\phantom{0}$ & $\phantom{0}0.05\phantom{0}$ & $\phantom{0}0.05\phantom{0}$ & $\phantom{0}95.3\phantom{0}$ & $\phantom{0}90.6\phantom{0}$ & $\phantom{0}94.1\phantom{0}$ & $\phantom{0}94.2\phantom{0}$ & $\phantom{0}92.8\phantom{0}$ & $\phantom{0}95.0\phantom{0}$ \\
 & \nopagebreak $\;J=500$  & $\phantom{0}{-}0.4\phantom{0}$ & ${-}14.1\phantom{0}$ & $\phantom{0}{-}6.6\phantom{0}$ & $\phantom{0}{-}1.0\phantom{0}$ & $\phantom{0}\phantom{-}0.0\phantom{0}$ & $\phantom{0}{-}5.6\phantom{0}$ & $\phantom{0}0.02\phantom{0}$ & $\phantom{0}0.04\phantom{0}$ & $\phantom{0}0.03\phantom{0}$ & $\phantom{0}0.03\phantom{0}$ & $\phantom{0}0.03\phantom{0}$ & $\phantom{0}0.03\phantom{0}$ & $\phantom{0}95.4\phantom{0}$ & $\phantom{0}85.2\phantom{0}$ & $\phantom{0}93.7\phantom{0}$ & $\phantom{0}95.0\phantom{0}$ & $\phantom{0}93.8\phantom{0}$ & $\phantom{0}95.1\phantom{0}$ \\
 & \nopagebreak $\;J=1000$  & $\phantom{0}\phantom{-}0.3\phantom{0}$ & ${-}13.1\phantom{0}$ & $\phantom{0}{-}5.4\phantom{0}$ & $\phantom{0}{-}0.1\phantom{0}$ & $\phantom{0}\phantom{-}0.5\phantom{0}$ & $\phantom{0}{-}2.5\phantom{0}$ & $\phantom{0}0.02\phantom{0}$ & $\phantom{0}0.03\phantom{0}$ & $\phantom{0}0.02\phantom{0}$ & $\phantom{0}0.02\phantom{0}$ & $\phantom{0}0.02\phantom{0}$ & $\phantom{0}0.02\phantom{0}$ & $\phantom{0}95.1\phantom{0}$ & $\phantom{0}82.9\phantom{0}$ & $\phantom{0}93.8\phantom{0}$ & $\phantom{0}94.9\phantom{0}$ & $\phantom{0}94.0\phantom{0}$ & $\phantom{0}95.6\phantom{0}$ \\
\multicolumn{4}{l}{$n=20$} \\  & \nopagebreak $\;J=30$  & $\phantom{0}\phantom{-}1.5\phantom{0}$ & ${-}10.2\phantom{0}$ & $\phantom{0}{-}2.9\phantom{0}$ & $\phantom{0}{-}7.7\phantom{0}$ & $\phantom{0}{-}5.3\phantom{0}$ & ${-}22.5\phantom{0}$ & $\phantom{0}0.07\phantom{0}$ & $\phantom{0}0.09\phantom{0}$ & $\phantom{0}0.09\phantom{0}$ & $\phantom{0}0.09\phantom{0}$ & $\phantom{0}0.09\phantom{0}$ & $\phantom{0}0.08\phantom{0}$ & $\phantom{0}90.8\phantom{0}$ & $\phantom{0}87.6\phantom{0}$ & $\phantom{0}90.4\phantom{0}$ & $\phantom{0}91.7\phantom{0}$ & $\phantom{0}90.8\phantom{0}$ & $\phantom{0}95.5\phantom{0}$ \\
 & \nopagebreak $\;J=50$  & $\phantom{0}\phantom{-}0.9\phantom{0}$ & ${-}10.6\phantom{0}$ & $\phantom{0}{-}3.3\phantom{0}$ & $\phantom{0}{-}4.0\phantom{0}$ & $\phantom{0}{-}3.2\phantom{0}$ & ${-}16.9\phantom{0}$ & $\phantom{0}0.05\phantom{0}$ & $\phantom{0}0.07\phantom{0}$ & $\phantom{0}0.07\phantom{0}$ & $\phantom{0}0.07\phantom{0}$ & $\phantom{0}0.07\phantom{0}$ & $\phantom{0}0.07\phantom{0}$ & $\phantom{0}92.9\phantom{0}$ & $\phantom{0}89.5\phantom{0}$ & $\phantom{0}91.6\phantom{0}$ & $\phantom{0}93.7\phantom{0}$ & $\phantom{0}92.3\phantom{0}$ & $\phantom{0}95.3\phantom{0}$ \\
 & \nopagebreak $\;J=100$  & $\phantom{0}{-}0.6\phantom{0}$ & ${-}11.9\phantom{0}$ & $\phantom{0}{-}4.8\phantom{0}$ & $\phantom{0}{-}2.9\phantom{0}$ & $\phantom{0}{-}2.1\phantom{0}$ & ${-}11.0\phantom{0}$ & $\phantom{0}0.04\phantom{0}$ & $\phantom{0}0.05\phantom{0}$ & $\phantom{0}0.05\phantom{0}$ & $\phantom{0}0.05\phantom{0}$ & $\phantom{0}0.05\phantom{0}$ & $\phantom{0}0.05\phantom{0}$ & $\phantom{0}93.1\phantom{0}$ & $\phantom{0}90.6\phantom{0}$ & $\phantom{0}93.6\phantom{0}$ & $\phantom{0}93.6\phantom{0}$ & $\phantom{0}94.4\phantom{0}$ & $\phantom{0}95.1\phantom{0}$ \\
 & \nopagebreak $\;J=200$  & $\phantom{0}\phantom{-}0.6\phantom{0}$ & ${-}11.3\phantom{0}$ & $\phantom{0}{-}4.0\phantom{0}$ & $\phantom{0}{-}0.5\phantom{0}$ & $\phantom{0}{-}0.6\phantom{0}$ & $\phantom{0}{-}5.7\phantom{0}$ & $\phantom{0}0.02\phantom{0}$ & $\phantom{0}0.03\phantom{0}$ & $\phantom{0}0.03\phantom{0}$ & $\phantom{0}0.03\phantom{0}$ & $\phantom{0}0.03\phantom{0}$ & $\phantom{0}0.03\phantom{0}$ & $\phantom{0}94.8\phantom{0}$ & $\phantom{0}90.6\phantom{0}$ & $\phantom{0}93.9\phantom{0}$ & $\phantom{0}95.4\phantom{0}$ & $\phantom{0}95.1\phantom{0}$ & $\phantom{0}96.7\phantom{0}$ \\
 & \nopagebreak $\;J=500$  & $\phantom{0}\phantom{-}0.1\phantom{0}$ & ${-}12.2\phantom{0}$ & $\phantom{0}{-}5.0\phantom{0}$ & $\phantom{0}{-}0.5\phantom{0}$ & $\phantom{0}{-}0.5\phantom{0}$ & $\phantom{0}{-}2.6\phantom{0}$ & $\phantom{0}0.02\phantom{0}$ & $\phantom{0}0.03\phantom{0}$ & $\phantom{0}0.02\phantom{0}$ & $\phantom{0}0.02\phantom{0}$ & $\phantom{0}0.02\phantom{0}$ & $\phantom{0}0.02\phantom{0}$ & $\phantom{0}93.7\phantom{0}$ & $\phantom{0}82.5\phantom{0}$ & $\phantom{0}92.9\phantom{0}$ & $\phantom{0}93.8\phantom{0}$ & $\phantom{0}93.4\phantom{0}$ & $\phantom{0}95.0\phantom{0}$ \\
 & \nopagebreak $\;J=1000$  & $\phantom{0}{-}0.2\phantom{0}$ & ${-}12.4\phantom{0}$ & $\phantom{0}{-}5.2\phantom{0}$ & $\phantom{0}{-}0.5\phantom{0}$ & $\phantom{0}{-}0.4\phantom{0}$ & $\phantom{0}{-}1.5\phantom{0}$ & $\phantom{0}0.01\phantom{0}$ & $\phantom{0}0.02\phantom{0}$ & $\phantom{0}0.02\phantom{0}$ & $\phantom{0}0.01\phantom{0}$ & $\phantom{0}0.01\phantom{0}$ & $\phantom{0}0.01\phantom{0}$ & $\phantom{0}94.7\phantom{0}$ & $\phantom{0}68.2\phantom{0}$ & $\phantom{0}90.8\phantom{0}$ & $\phantom{0}94.7\phantom{0}$ & $\phantom{0}93.5\phantom{0}$ & $\phantom{0}94.7\phantom{0}$ \\
[0.5ex]\hline\\[-1.6ex] 
& & \multicolumn{18}{c}{Moderate intraclass correlation $(\rho_{Iy}=.30)$} \\[0.6ex]\hline\\[-1.8ex]
\multicolumn{4}{l}{$n=5$} \\  & \nopagebreak $\;J=30$  & $\phantom{0}{-}1.3\phantom{0}$ & ${-}15.1\phantom{0}$ & $\phantom{0}\phantom{-}0.1\phantom{0}$ & ${-}11.6\phantom{0}$ & $\phantom{0}{-}9.4\phantom{0}$ & ${-}19.4\phantom{0}$ & $\phantom{0}0.12\phantom{0}$ & $\phantom{0}0.15\phantom{0}$ & $\phantom{0}0.17\phantom{0}$ & $\phantom{0}0.15\phantom{0}$ & $\phantom{0}0.15\phantom{0}$ & $\phantom{0}0.15\phantom{0}$ & $\phantom{0}90.4\phantom{0}$ & $\phantom{0}86.4\phantom{0}$ & $\phantom{0}89.5\phantom{0}$ & $\phantom{0}92.6\phantom{0}$ & $\phantom{0}90.9\phantom{0}$ & $\phantom{0}94.3\phantom{0}$ \\
 & \nopagebreak $\;J=50$  & $\phantom{0}{-}1.2\phantom{0}$ & ${-}13.4\phantom{0}$ & $\phantom{0}\phantom{-}3.4\phantom{0}$ & $\phantom{0}{-}6.5\phantom{0}$ & $\phantom{0}{-}4.9\phantom{0}$ & ${-}12.2\phantom{0}$ & $\phantom{0}0.09\phantom{0}$ & $\phantom{0}0.12\phantom{0}$ & $\phantom{0}0.13\phantom{0}$ & $\phantom{0}0.12\phantom{0}$ & $\phantom{0}0.12\phantom{0}$ & $\phantom{0}0.12\phantom{0}$ & $\phantom{0}91.9\phantom{0}$ & $\phantom{0}88.7\phantom{0}$ & $\phantom{0}90.1\phantom{0}$ & $\phantom{0}92.9\phantom{0}$ & $\phantom{0}91.9\phantom{0}$ & $\phantom{0}94.3\phantom{0}$ \\
 & \nopagebreak $\;J=100$  & $\phantom{0}{-}0.3\phantom{0}$ & ${-}13.3\phantom{0}$ & $\phantom{0}\phantom{-}3.7\phantom{0}$ & $\phantom{0}{-}3.8\phantom{0}$ & $\phantom{0}{-}2.9\phantom{0}$ & $\phantom{0}{-}7.1\phantom{0}$ & $\phantom{0}0.06\phantom{0}$ & $\phantom{0}0.09\phantom{0}$ & $\phantom{0}0.09\phantom{0}$ & $\phantom{0}0.08\phantom{0}$ & $\phantom{0}0.08\phantom{0}$ & $\phantom{0}0.08\phantom{0}$ & $\phantom{0}94.0\phantom{0}$ & $\phantom{0}90.3\phantom{0}$ & $\phantom{0}92.0\phantom{0}$ & $\phantom{0}93.3\phantom{0}$ & $\phantom{0}93.7\phantom{0}$ & $\phantom{0}94.7\phantom{0}$ \\
 & \nopagebreak $\;J=200$  & $\phantom{0}{-}0.3\phantom{0}$ & ${-}11.9\phantom{0}$ & $\phantom{0}\phantom{-}5.4\phantom{0}$ & $\phantom{0}{-}1.4\phantom{0}$ & $\phantom{0}{-}1.5\phantom{0}$ & $\phantom{0}{-}3.1\phantom{0}$ & $\phantom{0}0.04\phantom{0}$ & $\phantom{0}0.06\phantom{0}$ & $\phantom{0}0.06\phantom{0}$ & $\phantom{0}0.05\phantom{0}$ & $\phantom{0}0.05\phantom{0}$ & $\phantom{0}0.05\phantom{0}$ & $\phantom{0}94.5\phantom{0}$ & $\phantom{0}89.5\phantom{0}$ & $\phantom{0}92.8\phantom{0}$ & $\phantom{0}95.3\phantom{0}$ & $\phantom{0}95.2\phantom{0}$ & $\phantom{0}95.8\phantom{0}$ \\
 & \nopagebreak $\;J=500$  & $\phantom{0}{-}0.1\phantom{0}$ & ${-}12.1\phantom{0}$ & $\phantom{0}\phantom{-}5.7\phantom{0}$ & $\phantom{0}{-}0.3\phantom{0}$ & $\phantom{0}{-}0.2\phantom{0}$ & $\phantom{0}{-}1.1\phantom{0}$ & $\phantom{0}0.03\phantom{0}$ & $\phantom{0}0.05\phantom{0}$ & $\phantom{0}0.04\phantom{0}$ & $\phantom{0}0.04\phantom{0}$ & $\phantom{0}0.04\phantom{0}$ & $\phantom{0}0.04\phantom{0}$ & $\phantom{0}93.9\phantom{0}$ & $\phantom{0}81.3\phantom{0}$ & $\phantom{0}89.5\phantom{0}$ & $\phantom{0}94.1\phantom{0}$ & $\phantom{0}93.2\phantom{0}$ & $\phantom{0}94.2\phantom{0}$ \\
 & \nopagebreak $\;J=1000$  & $\phantom{0}\phantom{-}0.0\phantom{0}$ & ${-}11.9\phantom{0}$ & $\phantom{0}\phantom{-}5.8\phantom{0}$ & $\phantom{0}{-}0.1\phantom{0}$ & $\phantom{0}{-}0.1\phantom{0}$ & $\phantom{0}{-}0.4\phantom{0}$ & $\phantom{0}0.02\phantom{0}$ & $\phantom{0}0.04\phantom{0}$ & $\phantom{0}0.03\phantom{0}$ & $\phantom{0}0.02\phantom{0}$ & $\phantom{0}0.02\phantom{0}$ & $\phantom{0}0.02\phantom{0}$ & $\phantom{0}94.4\phantom{0}$ & $\phantom{0}72.1\phantom{0}$ & $\phantom{0}89.4\phantom{0}$ & $\phantom{0}94.8\phantom{0}$ & $\phantom{0}95.2\phantom{0}$ & $\phantom{0}96.1\phantom{0}$ \\
\multicolumn{4}{l}{$n=20$} \\  & \nopagebreak $\;J=30$  & $\phantom{0}{-}0.6\phantom{0}$ & ${-}12.0\phantom{0}$ & $\phantom{0}\phantom{-}7.7\phantom{0}$ & $\phantom{0}{-}9.6\phantom{0}$ & $\phantom{0}{-}8.8\phantom{0}$ & ${-}15.8\phantom{0}$ & $\phantom{0}0.10\phantom{0}$ & $\phantom{0}0.14\phantom{0}$ & $\phantom{0}0.16\phantom{0}$ & $\phantom{0}0.14\phantom{0}$ & $\phantom{0}0.13\phantom{0}$ & $\phantom{0}0.13\phantom{0}$ & $\phantom{0}91.1\phantom{0}$ & $\phantom{0}84.7\phantom{0}$ & $\phantom{0}85.7\phantom{0}$ & $\phantom{0}91.1\phantom{0}$ & $\phantom{0}90.9\phantom{0}$ & $\phantom{0}93.4\phantom{0}$ \\
 & \nopagebreak $\;J=50$  & $\phantom{0}{-}0.1\phantom{0}$ & ${-}10.9\phantom{0}$ & $\phantom{0}\phantom{-}8.8\phantom{0}$ & $\phantom{0}{-}5.1\phantom{0}$ & $\phantom{0}{-}4.9\phantom{0}$ & $\phantom{0}{-}9.2\phantom{0}$ & $\phantom{0}0.07\phantom{0}$ & $\phantom{0}0.10\phantom{0}$ & $\phantom{0}0.11\phantom{0}$ & $\phantom{0}0.10\phantom{0}$ & $\phantom{0}0.10\phantom{0}$ & $\phantom{0}0.09\phantom{0}$ & $\phantom{0}92.2\phantom{0}$ & $\phantom{0}89.0\phantom{0}$ & $\phantom{0}88.9\phantom{0}$ & $\phantom{0}93.0\phantom{0}$ & $\phantom{0}93.5\phantom{0}$ & $\phantom{0}95.3\phantom{0}$ \\
 & \nopagebreak $\;J=100$  & $\phantom{0}\phantom{-}0.2\phantom{0}$ & ${-}11.4\phantom{0}$ & $\phantom{0}\phantom{-}8.5\phantom{0}$ & $\phantom{0}{-}2.6\phantom{0}$ & $\phantom{0}{-}2.4\phantom{0}$ & $\phantom{0}{-}4.9\phantom{0}$ & $\phantom{0}0.05\phantom{0}$ & $\phantom{0}0.07\phantom{0}$ & $\phantom{0}0.08\phantom{0}$ & $\phantom{0}0.07\phantom{0}$ & $\phantom{0}0.07\phantom{0}$ & $\phantom{0}0.07\phantom{0}$ & $\phantom{0}92.0\phantom{0}$ & $\phantom{0}88.8\phantom{0}$ & $\phantom{0}88.7\phantom{0}$ & $\phantom{0}93.9\phantom{0}$ & $\phantom{0}93.7\phantom{0}$ & $\phantom{0}94.6\phantom{0}$ \\
 & \nopagebreak $\;J=200$  & $\phantom{0}\phantom{-}0.1\phantom{0}$ & ${-}11.7\phantom{0}$ & $\phantom{0}\phantom{-}8.1\phantom{0}$ & $\phantom{0}{-}1.4\phantom{0}$ & $\phantom{0}{-}1.4\phantom{0}$ & $\phantom{0}{-}2.7\phantom{0}$ & $\phantom{0}0.04\phantom{0}$ & $\phantom{0}0.06\phantom{0}$ & $\phantom{0}0.06\phantom{0}$ & $\phantom{0}0.05\phantom{0}$ & $\phantom{0}0.05\phantom{0}$ & $\phantom{0}0.05\phantom{0}$ & $\phantom{0}94.4\phantom{0}$ & $\phantom{0}86.2\phantom{0}$ & $\phantom{0}88.5\phantom{0}$ & $\phantom{0}93.6\phantom{0}$ & $\phantom{0}93.6\phantom{0}$ & $\phantom{0}93.3\phantom{0}$ \\
 & \nopagebreak $\;J=500$  & $\phantom{0}{-}0.1\phantom{0}$ & ${-}12.0\phantom{0}$ & $\phantom{0}\phantom{-}8.0\phantom{0}$ & $\phantom{0}{-}0.8\phantom{0}$ & $\phantom{0}{-}0.8\phantom{0}$ & $\phantom{0}{-}1.3\phantom{0}$ & $\phantom{0}0.02\phantom{0}$ & $\phantom{0}0.04\phantom{0}$ & $\phantom{0}0.04\phantom{0}$ & $\phantom{0}0.03\phantom{0}$ & $\phantom{0}0.03\phantom{0}$ & $\phantom{0}0.03\phantom{0}$ & $\phantom{0}94.1\phantom{0}$ & $\phantom{0}77.1\phantom{0}$ & $\phantom{0}86.6\phantom{0}$ & $\phantom{0}94.7\phantom{0}$ & $\phantom{0}95.1\phantom{0}$ & $\phantom{0}94.7\phantom{0}$ \\
 & \nopagebreak $\;J=1000$  & $\phantom{0}\phantom{-}0.0\phantom{0}$ & ${-}11.3\phantom{0}$ & $\phantom{0}\phantom{-}8.7\phantom{0}$ & $\phantom{0}\phantom{-}0.0\phantom{0}$ & $\phantom{0}{-}0.1\phantom{0}$ & $\phantom{0}{-}0.3\phantom{0}$ & $\phantom{0}0.02\phantom{0}$ & $\phantom{0}0.04\phantom{0}$ & $\phantom{0}0.03\phantom{0}$ & $\phantom{0}0.02\phantom{0}$ & $\phantom{0}0.02\phantom{0}$ & $\phantom{0}0.02\phantom{0}$ & $\phantom{0}95.4\phantom{0}$ & $\phantom{0}65.0\phantom{0}$ & $\phantom{0}78.4\phantom{0}$ & $\phantom{0}95.3\phantom{0}$ & $\phantom{0}94.7\phantom{0}$ & $\phantom{0}95.3\phantom{0}$ \\
[0.5ex]\hline\\[-1.6ex] 
\end{tabular}
\begin{tablenotes}[para,flushleft]{\footnotesize \textit{Note.} $n$ = cluster size; $J$ = number of clusters; CD = complete data sets; LD = listwise deletion; FCS-SL = single-level FCS; FCS-MAN = two-level FCS with manifest cluster means; FCS-LAT = two-level FCS with latent cluster means; JM = joint modeling.}\end{tablenotes}
\end{threeparttable}
\end{sidewaystable}
\begin{sidewaystable}
\begin{threeparttable}
\setlength{\tabcolsep}{1.2pt}
\renewcommand{\arraystretch}{0.95}
\footnotesize
\caption{\small Study 1: Bias (in \%), RMSE, and Coverage of the 95\% Confidence Interval for the Regression Coefficient of $y$ on $z$ ($\hat\beta_{yz}$) With 40\% Missing Data (MAR, $\lambda=1$)}
\begin{tabular}{llcccccccccccccccccc}
\hline\\[-1.8ex]
& & \multicolumn{6}{c}{Bias (\%)} & \multicolumn{6}{c}{RMSE} & \multicolumn{6}{c}{Coverage (\%)} \\ \cmidrule(r){3-8}\cmidrule(r){9-14}\cmidrule(r){15-20}
 &  & CD & LD & \makecell{FCS-\\SL} & \makecell{FCS-\\MAN} & \makecell{FCS-\\LAT} & JM & CD & LD & \makecell{FCS-\\SL} & \makecell{FCS-\\MAN} & \makecell{FCS-\\LAT} & JM & CD & LD & \makecell{FCS-\\SL} & \makecell{FCS-\\MAN} & \makecell{FCS-\\LAT} & \multicolumn{1}{c}{JM} \\ 
[0.4ex]\hline\\[-1.8ex]
& & \multicolumn{18}{c}{Small intraclass correlation $(\rho_{Iy}=.10)$} \\[0.6ex]\hline\\[-1.8ex]
\multicolumn{4}{l}{$n=5$} \\  & \nopagebreak $\;J=30$  & $\phantom{0}{-}0.2\phantom{0}$ & ${-}53.2\phantom{0}$ & ${-}42.3\phantom{0}$ & ${-}25.2\phantom{0}$ & ${-}20.9\phantom{0}$ & ${-}51.4\phantom{0}$ & $\phantom{0}0.10\phantom{0}$ & $\phantom{0}0.13\phantom{0}$ & $\phantom{0}0.13\phantom{0}$ & $\phantom{0}0.15\phantom{0}$ & $\phantom{0}0.16\phantom{0}$ & $\phantom{0}0.13\phantom{0}$ & $\phantom{0}90.1\phantom{0}$ & $\phantom{0}72.3\phantom{0}$ & $\phantom{0}92.7\phantom{0}$ & $\phantom{0}88.1\phantom{0}$ & $\phantom{0}86.8\phantom{0}$ & $\phantom{0}95.1\phantom{0}$ \\
 & \nopagebreak $\;J=50$  & $\phantom{0}\phantom{-}2.7\phantom{0}$ & ${-}54.5\phantom{0}$ & ${-}43.4\phantom{0}$ & ${-}18.2\phantom{0}$ & $\phantom{0}{-}9.0\phantom{0}$ & ${-}43.6\phantom{0}$ & $\phantom{0}0.08\phantom{0}$ & $\phantom{0}0.11\phantom{0}$ & $\phantom{0}0.11\phantom{0}$ & $\phantom{0}0.12\phantom{0}$ & $\phantom{0}0.13\phantom{0}$ & $\phantom{0}0.11\phantom{0}$ & $\phantom{0}90.9\phantom{0}$ & $\phantom{0}65.9\phantom{0}$ & $\phantom{0}92.1\phantom{0}$ & $\phantom{0}89.9\phantom{0}$ & $\phantom{0}86.7\phantom{0}$ & $\phantom{0}95.7\phantom{0}$ \\
 & \nopagebreak $\;J=100$  & $\phantom{0}\phantom{-}0.8\phantom{0}$ & ${-}54.8\phantom{0}$ & ${-}43.8\phantom{0}$ & $\phantom{0}{-}9.0\phantom{0}$ & $\phantom{0}\phantom{-}0.5\phantom{0}$ & ${-}34.1\phantom{0}$ & $\phantom{0}0.05\phantom{0}$ & $\phantom{0}0.10\phantom{0}$ & $\phantom{0}0.09\phantom{0}$ & $\phantom{0}0.09\phantom{0}$ & $\phantom{0}0.09\phantom{0}$ & $\phantom{0}0.09\phantom{0}$ & $\phantom{0}94.4\phantom{0}$ & $\phantom{0}47.6\phantom{0}$ & $\phantom{0}87.8\phantom{0}$ & $\phantom{0}91.4\phantom{0}$ & $\phantom{0}89.1\phantom{0}$ & $\phantom{0}95.7\phantom{0}$ \\
 & \nopagebreak $\;J=200$  & $\phantom{0}\phantom{-}0.7\phantom{0}$ & ${-}55.5\phantom{0}$ & ${-}44.7\phantom{0}$ & $\phantom{0}{-}5.7\phantom{0}$ & $\phantom{0}\phantom{-}1.8\phantom{0}$ & ${-}24.1\phantom{0}$ & $\phantom{0}0.04\phantom{0}$ & $\phantom{0}0.09\phantom{0}$ & $\phantom{0}0.08\phantom{0}$ & $\phantom{0}0.07\phantom{0}$ & $\phantom{0}0.07\phantom{0}$ & $\phantom{0}0.07\phantom{0}$ & $\phantom{0}93.8\phantom{0}$ & $\phantom{0}20.5\phantom{0}$ & $\phantom{0}73.8\phantom{0}$ & $\phantom{0}92.8\phantom{0}$ & $\phantom{0}89.7\phantom{0}$ & $\phantom{0}92.7\phantom{0}$ \\
 & \nopagebreak $\;J=500$  & $\phantom{0}\phantom{-}0.4\phantom{0}$ & ${-}54.9\phantom{0}$ & ${-}43.8\phantom{0}$ & $\phantom{0}{-}1.6\phantom{0}$ & $\phantom{0}\phantom{-}1.8\phantom{0}$ & ${-}12.6\phantom{0}$ & $\phantom{0}0.02\phantom{0}$ & $\phantom{0}0.09\phantom{0}$ & $\phantom{0}0.07\phantom{0}$ & $\phantom{0}0.04\phantom{0}$ & $\phantom{0}0.04\phantom{0}$ & $\phantom{0}0.04\phantom{0}$ & $\phantom{0}95.3\phantom{0}$ & $\phantom{0}\phantom{0}1.3\phantom{0}$ & $\phantom{0}41.9\phantom{0}$ & $\phantom{0}93.3\phantom{0}$ & $\phantom{0}90.7\phantom{0}$ & $\phantom{0}93.0\phantom{0}$ \\
 & \nopagebreak $\;J=1000$  & $\phantom{0}{-}0.6\phantom{0}$ & ${-}56.0\phantom{0}$ & ${-}45.4\phantom{0}$ & $\phantom{0}{-}2.3\phantom{0}$ & $\phantom{0}{-}1.0\phantom{0}$ & $\phantom{0}{-}9.0\phantom{0}$ & $\phantom{0}0.02\phantom{0}$ & $\phantom{0}0.09\phantom{0}$ & $\phantom{0}0.07\phantom{0}$ & $\phantom{0}0.03\phantom{0}$ & $\phantom{0}0.03\phantom{0}$ & $\phantom{0}0.03\phantom{0}$ & $\phantom{0}96.1\phantom{0}$ & $\phantom{0}\phantom{0}0.1\phantom{0}$ & $\phantom{0}\phantom{0}7.8\phantom{0}$ & $\phantom{0}92.9\phantom{0}$ & $\phantom{0}91.2\phantom{0}$ & $\phantom{0}94.1\phantom{0}$ \\
\multicolumn{4}{l}{$n=20$} \\  & \nopagebreak $\;J=30$  & $\phantom{0}{-}1.1\phantom{0}$ & ${-}52.9\phantom{0}$ & ${-}45.8\phantom{0}$ & ${-}26.1\phantom{0}$ & ${-}21.9\phantom{0}$ & ${-}48.7\phantom{0}$ & $\phantom{0}0.07\phantom{0}$ & $\phantom{0}0.11\phantom{0}$ & $\phantom{0}0.11\phantom{0}$ & $\phantom{0}0.11\phantom{0}$ & $\phantom{0}0.12\phantom{0}$ & $\phantom{0}0.11\phantom{0}$ & $\phantom{0}91.8\phantom{0}$ & $\phantom{0}59.7\phantom{0}$ & $\phantom{0}86.1\phantom{0}$ & $\phantom{0}91.4\phantom{0}$ & $\phantom{0}89.1\phantom{0}$ & $\phantom{0}94.5\phantom{0}$ \\
 & \nopagebreak $\;J=50$  & $\phantom{0}\phantom{-}0.9\phantom{0}$ & ${-}52.0\phantom{0}$ & ${-}44.5\phantom{0}$ & ${-}16.8\phantom{0}$ & ${-}13.5\phantom{0}$ & ${-}38.3\phantom{0}$ & $\phantom{0}0.05\phantom{0}$ & $\phantom{0}0.10\phantom{0}$ & $\phantom{0}0.09\phantom{0}$ & $\phantom{0}0.09\phantom{0}$ & $\phantom{0}0.09\phantom{0}$ & $\phantom{0}0.09\phantom{0}$ & $\phantom{0}92.8\phantom{0}$ & $\phantom{0}49.7\phantom{0}$ & $\phantom{0}82.2\phantom{0}$ & $\phantom{0}92.5\phantom{0}$ & $\phantom{0}90.7\phantom{0}$ & $\phantom{0}94.3\phantom{0}$ \\
 & \nopagebreak $\;J=100$  & $\phantom{0}{-}0.4\phantom{0}$ & ${-}53.6\phantom{0}$ & ${-}46.2\phantom{0}$ & $\phantom{0}{-}9.8\phantom{0}$ & $\phantom{0}{-}8.4\phantom{0}$ & ${-}28.1\phantom{0}$ & $\phantom{0}0.04\phantom{0}$ & $\phantom{0}0.09\phantom{0}$ & $\phantom{0}0.08\phantom{0}$ & $\phantom{0}0.06\phantom{0}$ & $\phantom{0}0.06\phantom{0}$ & $\phantom{0}0.07\phantom{0}$ & $\phantom{0}94.1\phantom{0}$ & $\phantom{0}26.7\phantom{0}$ & $\phantom{0}70.7\phantom{0}$ & $\phantom{0}94.3\phantom{0}$ & $\phantom{0}92.6\phantom{0}$ & $\phantom{0}93.9\phantom{0}$ \\
 & \nopagebreak $\;J=200$  & $\phantom{0}\phantom{-}0.2\phantom{0}$ & ${-}53.0\phantom{0}$ & ${-}45.5\phantom{0}$ & $\phantom{0}{-}4.7\phantom{0}$ & $\phantom{0}{-}4.1\phantom{0}$ & ${-}16.8\phantom{0}$ & $\phantom{0}0.02\phantom{0}$ & $\phantom{0}0.09\phantom{0}$ & $\phantom{0}0.08\phantom{0}$ & $\phantom{0}0.04\phantom{0}$ & $\phantom{0}0.04\phantom{0}$ & $\phantom{0}0.05\phantom{0}$ & $\phantom{0}96.0\phantom{0}$ & $\phantom{0}\phantom{0}6.5\phantom{0}$ & $\phantom{0}46.6\phantom{0}$ & $\phantom{0}93.5\phantom{0}$ & $\phantom{0}92.8\phantom{0}$ & $\phantom{0}94.1\phantom{0}$ \\
 & \nopagebreak $\;J=500$  & $\phantom{0}{-}0.6\phantom{0}$ & ${-}53.3\phantom{0}$ & ${-}46.0\phantom{0}$ & $\phantom{0}{-}2.6\phantom{0}$ & $\phantom{0}{-}2.5\phantom{0}$ & $\phantom{0}{-}8.6\phantom{0}$ & $\phantom{0}0.02\phantom{0}$ & $\phantom{0}0.09\phantom{0}$ & $\phantom{0}0.07\phantom{0}$ & $\phantom{0}0.03\phantom{0}$ & $\phantom{0}0.03\phantom{0}$ & $\phantom{0}0.03\phantom{0}$ & $\phantom{0}93.5\phantom{0}$ & $\phantom{0}\phantom{0}0.0\phantom{0}$ & $\phantom{0}\phantom{0}6.3\phantom{0}$ & $\phantom{0}92.6\phantom{0}$ & $\phantom{0}92.7\phantom{0}$ & $\phantom{0}94.1\phantom{0}$ \\
 & \nopagebreak $\;J=1000$  & $\phantom{0}{-}0.1\phantom{0}$ & ${-}53.5\phantom{0}$ & ${-}46.1\phantom{0}$ & $\phantom{0}{-}1.7\phantom{0}$ & $\phantom{0}{-}1.6\phantom{0}$ & $\phantom{0}{-}4.8\phantom{0}$ & $\phantom{0}0.01\phantom{0}$ & $\phantom{0}0.09\phantom{0}$ & $\phantom{0}0.07\phantom{0}$ & $\phantom{0}0.02\phantom{0}$ & $\phantom{0}0.02\phantom{0}$ & $\phantom{0}0.02\phantom{0}$ & $\phantom{0}94.2\phantom{0}$ & $\phantom{0}\phantom{0}0.0\phantom{0}$ & $\phantom{0}\phantom{0}0.1\phantom{0}$ & $\phantom{0}95.0\phantom{0}$ & $\phantom{0}94.1\phantom{0}$ & $\phantom{0}94.2\phantom{0}$ \\
[0.5ex]\hline\\[-1.6ex] 
& & \multicolumn{18}{c}{Moderate intraclass correlation $(\rho_{Iy}=.30)$} \\[0.6ex]\hline\\[-1.8ex]
\multicolumn{4}{l}{$n=5$} \\  & \nopagebreak $\;J=30$  & $\phantom{0}\phantom{-}0.5\phantom{0}$ & ${-}53.6\phantom{0}$ & ${-}32.7\phantom{0}$ & ${-}27.0\phantom{0}$ & ${-}21.5\phantom{0}$ & ${-}40.8\phantom{0}$ & $\phantom{0}0.12\phantom{0}$ & $\phantom{0}0.19\phantom{0}$ & $\phantom{0}0.20\phantom{0}$ & $\phantom{0}0.20\phantom{0}$ & $\phantom{0}0.20\phantom{0}$ & $\phantom{0}0.19\phantom{0}$ & $\phantom{0}90.1\phantom{0}$ & $\phantom{0}59.5\phantom{0}$ & $\phantom{0}88.9\phantom{0}$ & $\phantom{0}91.3\phantom{0}$ & $\phantom{0}89.6\phantom{0}$ & $\phantom{0}93.7\phantom{0}$ \\
 & \nopagebreak $\;J=50$  & $\phantom{0}{-}0.2\phantom{0}$ & ${-}54.0\phantom{0}$ & ${-}32.2\phantom{0}$ & ${-}18.0\phantom{0}$ & ${-}14.2\phantom{0}$ & ${-}30.1\phantom{0}$ & $\phantom{0}0.09\phantom{0}$ & $\phantom{0}0.17\phantom{0}$ & $\phantom{0}0.16\phantom{0}$ & $\phantom{0}0.15\phantom{0}$ & $\phantom{0}0.15\phantom{0}$ & $\phantom{0}0.15\phantom{0}$ & $\phantom{0}93.1\phantom{0}$ & $\phantom{0}49.5\phantom{0}$ & $\phantom{0}89.2\phantom{0}$ & $\phantom{0}92.3\phantom{0}$ & $\phantom{0}89.1\phantom{0}$ & $\phantom{0}94.9\phantom{0}$ \\
 & \nopagebreak $\;J=100$  & $\phantom{0}\phantom{-}0.5\phantom{0}$ & ${-}52.4\phantom{0}$ & ${-}28.9\phantom{0}$ & $\phantom{0}{-}8.2\phantom{0}$ & $\phantom{0}{-}6.8\phantom{0}$ & ${-}16.1\phantom{0}$ & $\phantom{0}0.06\phantom{0}$ & $\phantom{0}0.16\phantom{0}$ & $\phantom{0}0.12\phantom{0}$ & $\phantom{0}0.10\phantom{0}$ & $\phantom{0}0.10\phantom{0}$ & $\phantom{0}0.11\phantom{0}$ & $\phantom{0}93.1\phantom{0}$ & $\phantom{0}29.2\phantom{0}$ & $\phantom{0}85.9\phantom{0}$ & $\phantom{0}92.7\phantom{0}$ & $\phantom{0}91.1\phantom{0}$ & $\phantom{0}94.7\phantom{0}$ \\
 & \nopagebreak $\;J=200$  & $\phantom{0}\phantom{-}0.2\phantom{0}$ & ${-}53.4\phantom{0}$ & ${-}29.9\phantom{0}$ & $\phantom{0}{-}4.8\phantom{0}$ & $\phantom{0}{-}4.1\phantom{0}$ & $\phantom{0}{-}9.4\phantom{0}$ & $\phantom{0}0.04\phantom{0}$ & $\phantom{0}0.15\phantom{0}$ & $\phantom{0}0.10\phantom{0}$ & $\phantom{0}0.07\phantom{0}$ & $\phantom{0}0.07\phantom{0}$ & $\phantom{0}0.07\phantom{0}$ & $\phantom{0}95.0\phantom{0}$ & $\phantom{0}\phantom{0}6.0\phantom{0}$ & $\phantom{0}76.8\phantom{0}$ & $\phantom{0}94.4\phantom{0}$ & $\phantom{0}92.9\phantom{0}$ & $\phantom{0}95.2\phantom{0}$ \\
 & \nopagebreak $\;J=500$  & $\phantom{0}\phantom{-}0.1\phantom{0}$ & ${-}53.0\phantom{0}$ & ${-}29.0\phantom{0}$ & $\phantom{0}{-}1.9\phantom{0}$ & $\phantom{0}{-}1.3\phantom{0}$ & $\phantom{0}{-}3.5\phantom{0}$ & $\phantom{0}0.03\phantom{0}$ & $\phantom{0}0.15\phantom{0}$ & $\phantom{0}0.09\phantom{0}$ & $\phantom{0}0.04\phantom{0}$ & $\phantom{0}0.04\phantom{0}$ & $\phantom{0}0.04\phantom{0}$ & $\phantom{0}95.2\phantom{0}$ & $\phantom{0}\phantom{0}0.0\phantom{0}$ & $\phantom{0}53.2\phantom{0}$ & $\phantom{0}94.3\phantom{0}$ & $\phantom{0}93.0\phantom{0}$ & $\phantom{0}95.3\phantom{0}$ \\
 & \nopagebreak $\;J=1000$  & $\phantom{0}\phantom{-}0.4\phantom{0}$ & ${-}52.7\phantom{0}$ & ${-}28.6\phantom{0}$ & $\phantom{0}{-}0.3\phantom{0}$ & $\phantom{0}{-}0.1\phantom{0}$ & $\phantom{0}{-}1.2\phantom{0}$ & $\phantom{0}0.02\phantom{0}$ & $\phantom{0}0.15\phantom{0}$ & $\phantom{0}0.08\phantom{0}$ & $\phantom{0}0.03\phantom{0}$ & $\phantom{0}0.03\phantom{0}$ & $\phantom{0}0.03\phantom{0}$ & $\phantom{0}94.8\phantom{0}$ & $\phantom{0}\phantom{0}0.0\phantom{0}$ & $\phantom{0}24.9\phantom{0}$ & $\phantom{0}93.9\phantom{0}$ & $\phantom{0}92.9\phantom{0}$ & $\phantom{0}93.5\phantom{0}$ \\
\multicolumn{4}{l}{$n=20$} \\  & \nopagebreak $\;J=30$  & $\phantom{0}{-}1.2\phantom{0}$ & ${-}51.3\phantom{0}$ & ${-}29.7\phantom{0}$ & ${-}23.2\phantom{0}$ & ${-}22.9\phantom{0}$ & ${-}34.7\phantom{0}$ & $\phantom{0}0.10\phantom{0}$ & $\phantom{0}0.17\phantom{0}$ & $\phantom{0}0.17\phantom{0}$ & $\phantom{0}0.16\phantom{0}$ & $\phantom{0}0.16\phantom{0}$ & $\phantom{0}0.16\phantom{0}$ & $\phantom{0}90.5\phantom{0}$ & $\phantom{0}53.3\phantom{0}$ & $\phantom{0}85.1\phantom{0}$ & $\phantom{0}92.5\phantom{0}$ & $\phantom{0}92.9\phantom{0}$ & $\phantom{0}94.3\phantom{0}$ \\
 & \nopagebreak $\;J=50$  & $\phantom{0}{-}0.9\phantom{0}$ & ${-}52.1\phantom{0}$ & ${-}30.8\phantom{0}$ & ${-}16.3\phantom{0}$ & ${-}15.5\phantom{0}$ & ${-}25.6\phantom{0}$ & $\phantom{0}0.08\phantom{0}$ & $\phantom{0}0.16\phantom{0}$ & $\phantom{0}0.14\phantom{0}$ & $\phantom{0}0.12\phantom{0}$ & $\phantom{0}0.12\phantom{0}$ & $\phantom{0}0.13\phantom{0}$ & $\phantom{0}91.5\phantom{0}$ & $\phantom{0}42.2\phantom{0}$ & $\phantom{0}82.7\phantom{0}$ & $\phantom{0}93.9\phantom{0}$ & $\phantom{0}94.5\phantom{0}$ & $\phantom{0}95.0\phantom{0}$ \\
 & \nopagebreak $\;J=100$  & $\phantom{0}\phantom{-}0.1\phantom{0}$ & ${-}51.9\phantom{0}$ & ${-}30.1\phantom{0}$ & $\phantom{0}{-}8.4\phantom{0}$ & $\phantom{0}{-}8.3\phantom{0}$ & ${-}14.3\phantom{0}$ & $\phantom{0}0.05\phantom{0}$ & $\phantom{0}0.15\phantom{0}$ & $\phantom{0}0.11\phantom{0}$ & $\phantom{0}0.09\phantom{0}$ & $\phantom{0}0.09\phantom{0}$ & $\phantom{0}0.09\phantom{0}$ & $\phantom{0}94.7\phantom{0}$ & $\phantom{0}20.6\phantom{0}$ & $\phantom{0}78.4\phantom{0}$ & $\phantom{0}94.7\phantom{0}$ & $\phantom{0}94.8\phantom{0}$ & $\phantom{0}95.3\phantom{0}$ \\
 & \nopagebreak $\;J=200$  & $\phantom{0}{-}0.5\phantom{0}$ & ${-}51.8\phantom{0}$ & ${-}30.1\phantom{0}$ & $\phantom{0}{-}5.0\phantom{0}$ & $\phantom{0}{-}5.2\phantom{0}$ & $\phantom{0}{-}8.5\phantom{0}$ & $\phantom{0}0.04\phantom{0}$ & $\phantom{0}0.15\phantom{0}$ & $\phantom{0}0.10\phantom{0}$ & $\phantom{0}0.06\phantom{0}$ & $\phantom{0}0.06\phantom{0}$ & $\phantom{0}0.06\phantom{0}$ & $\phantom{0}93.5\phantom{0}$ & $\phantom{0}\phantom{0}3.3\phantom{0}$ & $\phantom{0}65.3\phantom{0}$ & $\phantom{0}94.7\phantom{0}$ & $\phantom{0}94.2\phantom{0}$ & $\phantom{0}96.7\phantom{0}$ \\
 & \nopagebreak $\;J=500$  & $\phantom{0}\phantom{-}0.4\phantom{0}$ & ${-}51.0\phantom{0}$ & ${-}28.6\phantom{0}$ & $\phantom{0}{-}1.0\phantom{0}$ & $\phantom{0}{-}1.0\phantom{0}$ & $\phantom{0}{-}2.6\phantom{0}$ & $\phantom{0}0.02\phantom{0}$ & $\phantom{0}0.14\phantom{0}$ & $\phantom{0}0.09\phantom{0}$ & $\phantom{0}0.03\phantom{0}$ & $\phantom{0}0.04\phantom{0}$ & $\phantom{0}0.04\phantom{0}$ & $\phantom{0}95.0\phantom{0}$ & $\phantom{0}\phantom{0}0.0\phantom{0}$ & $\phantom{0}38.9\phantom{0}$ & $\phantom{0}94.0\phantom{0}$ & $\phantom{0}92.6\phantom{0}$ & $\phantom{0}96.1\phantom{0}$ \\
 & \nopagebreak $\;J=1000$  & $\phantom{0}\phantom{-}0.1\phantom{0}$ & ${-}51.2\phantom{0}$ & ${-}29.0\phantom{0}$ & $\phantom{0}{-}0.5\phantom{0}$ & $\phantom{0}{-}0.6\phantom{0}$ & $\phantom{0}{-}1.3\phantom{0}$ & $\phantom{0}0.02\phantom{0}$ & $\phantom{0}0.14\phantom{0}$ & $\phantom{0}0.08\phantom{0}$ & $\phantom{0}0.03\phantom{0}$ & $\phantom{0}0.03\phantom{0}$ & $\phantom{0}0.03\phantom{0}$ & $\phantom{0}94.7\phantom{0}$ & $\phantom{0}\phantom{0}0.0\phantom{0}$ & $\phantom{0}12.2\phantom{0}$ & $\phantom{0}93.7\phantom{0}$ & $\phantom{0}93.1\phantom{0}$ & $\phantom{0}93.9\phantom{0}$ \\
[0.5ex]\hline\\[-1.6ex] 
\end{tabular}
\begin{tablenotes}[para,flushleft]{\footnotesize \textit{Note.} $n$ = cluster size; $J$ = number of clusters; CD = complete data sets; LD = listwise deletion; FCS-SL = single-level FCS; FCS-MAN = two-level FCS with manifest cluster means; FCS-LAT = two-level FCS with latent cluster means; JM = joint modeling.}\end{tablenotes}
\end{threeparttable}
\end{sidewaystable}
\begin{sidewaystable}
\begin{threeparttable}
\setlength{\tabcolsep}{1.2pt}
\renewcommand{\arraystretch}{0.95}
\footnotesize
\caption{\small Study 1: Bias (in \%), RMSE, and Coverage of the 95\% Confidence Interval for the Regression Coefficient of $z$ on $y$ ($\hat\beta_{zy}$) With 20\% Missing Data (MCAR, $\lambda=0$)}
\begin{tabular}{llcccccccccccccccccc}
\hline\\[-1.8ex]
& & \multicolumn{6}{c}{Bias (\%)} & \multicolumn{6}{c}{RMSE} & \multicolumn{6}{c}{Coverage (\%)} \\ \cmidrule(r){3-8}\cmidrule(r){9-14}\cmidrule(r){15-20}
 &  & CD & LD & \makecell{FCS-\\SL} & \makecell{FCS-\\MAN} & \makecell{FCS-\\LAT} & JM & CD & LD & \makecell{FCS-\\SL} & \makecell{FCS-\\MAN} & \makecell{FCS-\\LAT} & JM & CD & LD & \makecell{FCS-\\SL} & \makecell{FCS-\\MAN} & \makecell{FCS-\\LAT} & \multicolumn{1}{c}{JM} \\ 
[0.4ex]\hline\\[-1.8ex]
& & \multicolumn{18}{c}{Small intraclass correlation $(\rho_{Iy}=.10)$} \\[0.6ex]\hline\\[-1.8ex]
\multicolumn{4}{l}{$n=5$} \\  & \nopagebreak $\;J=30$  & $\phantom{-}45.0\phantom{0}$ & $\phantom{-}45.9\phantom{0}$ & $\phantom{-}25.9\phantom{0}$ & $\phantom{-}34.0\phantom{0}$ & $\phantom{-}39.3\phantom{0}$ & $\phantom{-}33.1\phantom{0}$ & $\phantom{0}2.64\phantom{0}$ & $\phantom{0}3.42\phantom{0}$ & $\phantom{0}2.29\phantom{0}$ & $\phantom{0}2.31\phantom{0}$ & $\phantom{0}2.40\phantom{0}$ & $\phantom{0}2.28\phantom{0}$ & $\phantom{0}88.3\phantom{0}$ & $\phantom{0}86.9\phantom{0}$ & $\phantom{0}86.6\phantom{0}$ & $\phantom{0}92.3\phantom{0}$ & $\phantom{0}92.3\phantom{0}$ & $\phantom{0}92.6\phantom{0}$ \\
 & \nopagebreak $\;J=50$  & $\phantom{-}35.7\phantom{0}$ & $\phantom{-}35.8\phantom{0}$ & $\phantom{-}12.3\phantom{0}$ & $\phantom{-}27.2\phantom{0}$ & $\phantom{-}29.9\phantom{0}$ & $\phantom{-}22.2\phantom{0}$ & $\phantom{0}2.28\phantom{0}$ & $\phantom{0}1.99\phantom{0}$ & $\phantom{0}1.55\phantom{0}$ & $\phantom{0}1.68\phantom{0}$ & $\phantom{0}1.73\phantom{0}$ & $\phantom{0}1.74\phantom{0}$ & $\phantom{0}92.1\phantom{0}$ & $\phantom{0}90.4\phantom{0}$ & $\phantom{0}89.2\phantom{0}$ & $\phantom{0}93.1\phantom{0}$ & $\phantom{0}93.5\phantom{0}$ & $\phantom{0}93.6\phantom{0}$ \\
 & \nopagebreak $\;J=100$  & $\phantom{-}18.9\phantom{0}$ & $\phantom{-}22.6\phantom{0}$ & $\phantom{0}\phantom{-}2.7\phantom{0}$ & $\phantom{-}18.4\phantom{0}$ & $\phantom{-}20.8\phantom{0}$ & $\phantom{-}13.9\phantom{0}$ & $\phantom{0}1.25\phantom{0}$ & $\phantom{0}1.29\phantom{0}$ & $\phantom{0}1.05\phantom{0}$ & $\phantom{0}1.16\phantom{0}$ & $\phantom{0}1.17\phantom{0}$ & $\phantom{0}1.13\phantom{0}$ & $\phantom{0}92.3\phantom{0}$ & $\phantom{0}92.7\phantom{0}$ & $\phantom{0}86.6\phantom{0}$ & $\phantom{0}93.1\phantom{0}$ & $\phantom{0}92.8\phantom{0}$ & $\phantom{0}92.4\phantom{0}$ \\
 & \nopagebreak $\;J=200$  & $\phantom{0}\phantom{-}9.7\phantom{0}$ & $\phantom{-}13.4\phantom{0}$ & $\phantom{0}{-}6.7\phantom{0}$ & $\phantom{0}\phantom{-}9.3\phantom{0}$ & $\phantom{-}10.5\phantom{0}$ & $\phantom{0}\phantom{-}6.1\phantom{0}$ & $\phantom{0}0.67\phantom{0}$ & $\phantom{0}0.88\phantom{0}$ & $\phantom{0}0.58\phantom{0}$ & $\phantom{0}0.69\phantom{0}$ & $\phantom{0}0.71\phantom{0}$ & $\phantom{0}0.65\phantom{0}$ & $\phantom{0}95.8\phantom{0}$ & $\phantom{0}95.3\phantom{0}$ & $\phantom{0}87.3\phantom{0}$ & $\phantom{0}95.9\phantom{0}$ & $\phantom{0}96.4\phantom{0}$ & $\phantom{0}95.8\phantom{0}$ \\
 & \nopagebreak $\;J=500$  & $\phantom{0}\phantom{-}3.4\phantom{0}$ & $\phantom{0}\phantom{-}4.7\phantom{0}$ & ${-}11.5\phantom{0}$ & $\phantom{0}\phantom{-}3.7\phantom{0}$ & $\phantom{0}\phantom{-}4.1\phantom{0}$ & $\phantom{0}\phantom{-}2.0\phantom{0}$ & $\phantom{0}0.34\phantom{0}$ & $\phantom{0}0.40\phantom{0}$ & $\phantom{0}0.35\phantom{0}$ & $\phantom{0}0.37\phantom{0}$ & $\phantom{0}0.37\phantom{0}$ & $\phantom{0}0.35\phantom{0}$ & $\phantom{0}96.2\phantom{0}$ & $\phantom{0}96.0\phantom{0}$ & $\phantom{0}83.9\phantom{0}$ & $\phantom{0}96.0\phantom{0}$ & $\phantom{0}96.1\phantom{0}$ & $\phantom{0}95.8\phantom{0}$ \\
 & \nopagebreak $\;J=1000$  & $\phantom{0}\phantom{-}1.7\phantom{0}$ & $\phantom{0}\phantom{-}1.7\phantom{0}$ & ${-}13.2\phantom{0}$ & $\phantom{0}\phantom{-}1.4\phantom{0}$ & $\phantom{0}\phantom{-}1.6\phantom{0}$ & $\phantom{0}\phantom{-}0.6\phantom{0}$ & $\phantom{0}0.23\phantom{0}$ & $\phantom{0}0.26\phantom{0}$ & $\phantom{0}0.30\phantom{0}$ & $\phantom{0}0.25\phantom{0}$ & $\phantom{0}0.25\phantom{0}$ & $\phantom{0}0.24\phantom{0}$ & $\phantom{0}95.2\phantom{0}$ & $\phantom{0}95.4\phantom{0}$ & $\phantom{0}76.0\phantom{0}$ & $\phantom{0}95.0\phantom{0}$ & $\phantom{0}94.7\phantom{0}$ & $\phantom{0}94.8\phantom{0}$ \\
\multicolumn{4}{l}{$n=20$} \\  & \nopagebreak $\;J=30$  & $\phantom{0}\phantom{-}8.1\phantom{0}$ & $\phantom{-}11.9\phantom{0}$ & ${-}11.2\phantom{0}$ & $\phantom{0}\phantom{-}7.9\phantom{0}$ & $\phantom{0}\phantom{-}9.3\phantom{0}$ & $\phantom{0}{-}0.8\phantom{0}$ & $\phantom{0}0.88\phantom{0}$ & $\phantom{0}1.25\phantom{0}$ & $\phantom{0}0.81\phantom{0}$ & $\phantom{0}0.94\phantom{0}$ & $\phantom{0}0.98\phantom{0}$ & $\phantom{0}0.87\phantom{0}$ & $\phantom{0}92.5\phantom{0}$ & $\phantom{0}91.5\phantom{0}$ & $\phantom{0}86.6\phantom{0}$ & $\phantom{0}93.5\phantom{0}$ & $\phantom{0}92.8\phantom{0}$ & $\phantom{0}94.4\phantom{0}$ \\
 & \nopagebreak $\;J=50$  & $\phantom{0}\phantom{-}5.3\phantom{0}$ & $\phantom{0}\phantom{-}6.5\phantom{0}$ & ${-}13.5\phantom{0}$ & $\phantom{0}\phantom{-}5.1\phantom{0}$ & $\phantom{0}\phantom{-}5.6\phantom{0}$ & $\phantom{0}{-}1.5\phantom{0}$ & $\phantom{0}0.59\phantom{0}$ & $\phantom{0}0.70\phantom{0}$ & $\phantom{0}0.59\phantom{0}$ & $\phantom{0}0.67\phantom{0}$ & $\phantom{0}0.67\phantom{0}$ & $\phantom{0}0.61\phantom{0}$ & $\phantom{0}93.7\phantom{0}$ & $\phantom{0}93.4\phantom{0}$ & $\phantom{0}87.1\phantom{0}$ & $\phantom{0}94.0\phantom{0}$ & $\phantom{0}94.3\phantom{0}$ & $\phantom{0}95.1\phantom{0}$ \\
 & \nopagebreak $\;J=100$  & $\phantom{0}\phantom{-}2.3\phantom{0}$ & $\phantom{0}\phantom{-}2.7\phantom{0}$ & ${-}15.5\phantom{0}$ & $\phantom{0}\phantom{-}2.2\phantom{0}$ & $\phantom{0}\phantom{-}2.5\phantom{0}$ & $\phantom{0}{-}1.5\phantom{0}$ & $\phantom{0}0.39\phantom{0}$ & $\phantom{0}0.44\phantom{0}$ & $\phantom{0}0.44\phantom{0}$ & $\phantom{0}0.44\phantom{0}$ & $\phantom{0}0.44\phantom{0}$ & $\phantom{0}0.41\phantom{0}$ & $\phantom{0}94.3\phantom{0}$ & $\phantom{0}93.7\phantom{0}$ & $\phantom{0}83.8\phantom{0}$ & $\phantom{0}94.3\phantom{0}$ & $\phantom{0}94.1\phantom{0}$ & $\phantom{0}95.4\phantom{0}$ \\
 & \nopagebreak $\;J=200$  & $\phantom{0}\phantom{-}1.0\phantom{0}$ & $\phantom{0}\phantom{-}1.4\phantom{0}$ & ${-}16.6\phantom{0}$ & $\phantom{0}\phantom{-}0.9\phantom{0}$ & $\phantom{0}\phantom{-}1.2\phantom{0}$ & $\phantom{0}{-}0.9\phantom{0}$ & $\phantom{0}0.26\phantom{0}$ & $\phantom{0}0.29\phantom{0}$ & $\phantom{0}0.36\phantom{0}$ & $\phantom{0}0.29\phantom{0}$ & $\phantom{0}0.29\phantom{0}$ & $\phantom{0}0.28\phantom{0}$ & $\phantom{0}94.9\phantom{0}$ & $\phantom{0}94.9\phantom{0}$ & $\phantom{0}77.0\phantom{0}$ & $\phantom{0}94.4\phantom{0}$ & $\phantom{0}94.2\phantom{0}$ & $\phantom{0}94.6\phantom{0}$ \\
 & \nopagebreak $\;J=500$  & $\phantom{0}{-}0.3\phantom{0}$ & $\phantom{0}{-}0.5\phantom{0}$ & ${-}17.5\phantom{0}$ & $\phantom{0}{-}0.5\phantom{0}$ & $\phantom{0}{-}0.5\phantom{0}$ & $\phantom{0}{-}1.2\phantom{0}$ & $\phantom{0}0.16\phantom{0}$ & $\phantom{0}0.18\phantom{0}$ & $\phantom{0}0.32\phantom{0}$ & $\phantom{0}0.19\phantom{0}$ & $\phantom{0}0.18\phantom{0}$ & $\phantom{0}0.18\phantom{0}$ & $\phantom{0}95.6\phantom{0}$ & $\phantom{0}94.2\phantom{0}$ & $\phantom{0}53.5\phantom{0}$ & $\phantom{0}94.1\phantom{0}$ & $\phantom{0}94.2\phantom{0}$ & $\phantom{0}94.1\phantom{0}$ \\
 & \nopagebreak $\;J=1000$  & $\phantom{0}\phantom{-}0.2\phantom{0}$ & $\phantom{0}\phantom{-}0.5\phantom{0}$ & ${-}17.0\phantom{0}$ & $\phantom{0}\phantom{-}0.4\phantom{0}$ & $\phantom{0}\phantom{-}0.4\phantom{0}$ & $\phantom{0}{-}0.1\phantom{0}$ & $\phantom{0}0.12\phantom{0}$ & $\phantom{0}0.13\phantom{0}$ & $\phantom{0}0.29\phantom{0}$ & $\phantom{0}0.13\phantom{0}$ & $\phantom{0}0.13\phantom{0}$ & $\phantom{0}0.13\phantom{0}$ & $\phantom{0}94.5\phantom{0}$ & $\phantom{0}94.5\phantom{0}$ & $\phantom{0}31.5\phantom{0}$ & $\phantom{0}94.6\phantom{0}$ & $\phantom{0}94.8\phantom{0}$ & $\phantom{0}94.7\phantom{0}$ \\
[0.5ex]\hline\\[-1.6ex] 
& & \multicolumn{18}{c}{Moderate intraclass correlation $(\rho_{Iy}=.30)$} \\[0.6ex]\hline\\[-1.8ex]
\multicolumn{4}{l}{$n=5$} \\  & \nopagebreak $\;J=30$  & $\phantom{0}\phantom{-}6.5\phantom{0}$ & $\phantom{-}10.8\phantom{0}$ & $\phantom{0}{-}7.2\phantom{0}$ & $\phantom{0}\phantom{-}6.4\phantom{0}$ & $\phantom{0}\phantom{-}8.2\phantom{0}$ & $\phantom{0}\phantom{-}1.0\phantom{0}$ & $\phantom{0}0.72\phantom{0}$ & $\phantom{0}0.66\phantom{0}$ & $\phantom{0}0.53\phantom{0}$ & $\phantom{0}0.61\phantom{0}$ & $\phantom{0}0.62\phantom{0}$ & $\phantom{0}0.54\phantom{0}$ & $\phantom{0}93.6\phantom{0}$ & $\phantom{0}92.9\phantom{0}$ & $\phantom{0}89.5\phantom{0}$ & $\phantom{0}94.7\phantom{0}$ & $\phantom{0}93.9\phantom{0}$ & $\phantom{0}94.8\phantom{0}$ \\
 & \nopagebreak $\;J=50$  & $\phantom{0}\phantom{-}3.1\phantom{0}$ & $\phantom{0}\phantom{-}4.9\phantom{0}$ & $\phantom{0}{-}9.2\phantom{0}$ & $\phantom{0}\phantom{-}3.1\phantom{0}$ & $\phantom{0}\phantom{-}3.8\phantom{0}$ & $\phantom{0}\phantom{-}0.0\phantom{0}$ & $\phantom{0}0.35\phantom{0}$ & $\phantom{0}0.42\phantom{0}$ & $\phantom{0}0.35\phantom{0}$ & $\phantom{0}0.39\phantom{0}$ & $\phantom{0}0.40\phantom{0}$ & $\phantom{0}0.37\phantom{0}$ & $\phantom{0}93.5\phantom{0}$ & $\phantom{0}93.4\phantom{0}$ & $\phantom{0}89.0\phantom{0}$ & $\phantom{0}93.7\phantom{0}$ & $\phantom{0}93.7\phantom{0}$ & $\phantom{0}94.6\phantom{0}$ \\
 & \nopagebreak $\;J=100$  & $\phantom{0}\phantom{-}2.6\phantom{0}$ & $\phantom{0}\phantom{-}3.3\phantom{0}$ & $\phantom{0}{-}9.1\phantom{0}$ & $\phantom{0}\phantom{-}2.9\phantom{0}$ & $\phantom{0}\phantom{-}3.1\phantom{0}$ & $\phantom{0}\phantom{-}1.4\phantom{0}$ & $\phantom{0}0.22\phantom{0}$ & $\phantom{0}0.25\phantom{0}$ & $\phantom{0}0.23\phantom{0}$ & $\phantom{0}0.24\phantom{0}$ & $\phantom{0}0.25\phantom{0}$ & $\phantom{0}0.24\phantom{0}$ & $\phantom{0}94.2\phantom{0}$ & $\phantom{0}94.2\phantom{0}$ & $\phantom{0}90.1\phantom{0}$ & $\phantom{0}95.5\phantom{0}$ & $\phantom{0}94.3\phantom{0}$ & $\phantom{0}95.4\phantom{0}$ \\
 & \nopagebreak $\;J=200$  & $\phantom{0}\phantom{-}1.3\phantom{0}$ & $\phantom{0}\phantom{-}1.7\phantom{0}$ & ${-}10.1\phantom{0}$ & $\phantom{0}\phantom{-}1.4\phantom{0}$ & $\phantom{0}\phantom{-}1.4\phantom{0}$ & $\phantom{0}\phantom{-}0.6\phantom{0}$ & $\phantom{0}0.15\phantom{0}$ & $\phantom{0}0.17\phantom{0}$ & $\phantom{0}0.18\phantom{0}$ & $\phantom{0}0.17\phantom{0}$ & $\phantom{0}0.17\phantom{0}$ & $\phantom{0}0.17\phantom{0}$ & $\phantom{0}94.8\phantom{0}$ & $\phantom{0}94.6\phantom{0}$ & $\phantom{0}87.6\phantom{0}$ & $\phantom{0}94.0\phantom{0}$ & $\phantom{0}94.6\phantom{0}$ & $\phantom{0}94.9\phantom{0}$ \\
 & \nopagebreak $\;J=500$  & $\phantom{0}\phantom{-}1.1\phantom{0}$ & $\phantom{0}\phantom{-}1.3\phantom{0}$ & ${-}10.1\phantom{0}$ & $\phantom{0}\phantom{-}1.4\phantom{0}$ & $\phantom{0}\phantom{-}1.3\phantom{0}$ & $\phantom{0}\phantom{-}1.0\phantom{0}$ & $\phantom{0}0.10\phantom{0}$ & $\phantom{0}0.11\phantom{0}$ & $\phantom{0}0.13\phantom{0}$ & $\phantom{0}0.11\phantom{0}$ & $\phantom{0}0.11\phantom{0}$ & $\phantom{0}0.11\phantom{0}$ & $\phantom{0}95.2\phantom{0}$ & $\phantom{0}93.9\phantom{0}$ & $\phantom{0}80.2\phantom{0}$ & $\phantom{0}94.7\phantom{0}$ & $\phantom{0}94.0\phantom{0}$ & $\phantom{0}94.6\phantom{0}$ \\
 & \nopagebreak $\;J=1000$  & $\phantom{0}{-}0.2\phantom{0}$ & $\phantom{0}{-}0.1\phantom{0}$ & ${-}11.3\phantom{0}$ & $\phantom{0}{-}0.1\phantom{0}$ & $\phantom{0}{-}0.1\phantom{0}$ & $\phantom{0}{-}0.3\phantom{0}$ & $\phantom{0}0.07\phantom{0}$ & $\phantom{0}0.07\phantom{0}$ & $\phantom{0}0.12\phantom{0}$ & $\phantom{0}0.07\phantom{0}$ & $\phantom{0}0.07\phantom{0}$ & $\phantom{0}0.07\phantom{0}$ & $\phantom{0}96.1\phantom{0}$ & $\phantom{0}95.4\phantom{0}$ & $\phantom{0}62.5\phantom{0}$ & $\phantom{0}94.7\phantom{0}$ & $\phantom{0}95.3\phantom{0}$ & $\phantom{0}95.9\phantom{0}$ \\
\multicolumn{4}{l}{$n=20$} \\  & \nopagebreak $\;J=30$  & $\phantom{0}\phantom{-}1.6\phantom{0}$ & $\phantom{0}\phantom{-}1.8\phantom{0}$ & ${-}13.1\phantom{0}$ & $\phantom{0}\phantom{-}1.3\phantom{0}$ & $\phantom{0}\phantom{-}1.8\phantom{0}$ & $\phantom{0}{-}2.9\phantom{0}$ & $\phantom{0}0.33\phantom{0}$ & $\phantom{0}0.38\phantom{0}$ & $\phantom{0}0.35\phantom{0}$ & $\phantom{0}0.38\phantom{0}$ & $\phantom{0}0.38\phantom{0}$ & $\phantom{0}0.37\phantom{0}$ & $\phantom{0}91.3\phantom{0}$ & $\phantom{0}90.9\phantom{0}$ & $\phantom{0}85.5\phantom{0}$ & $\phantom{0}93.1\phantom{0}$ & $\phantom{0}93.1\phantom{0}$ & $\phantom{0}93.5\phantom{0}$ \\
 & \nopagebreak $\;J=50$  & $\phantom{0}\phantom{-}1.0\phantom{0}$ & $\phantom{0}\phantom{-}1.6\phantom{0}$ & ${-}12.4\phantom{0}$ & $\phantom{0}\phantom{-}1.6\phantom{0}$ & $\phantom{0}\phantom{-}1.4\phantom{0}$ & $\phantom{0}{-}0.8\phantom{0}$ & $\phantom{0}0.25\phantom{0}$ & $\phantom{0}0.28\phantom{0}$ & $\phantom{0}0.27\phantom{0}$ & $\phantom{0}0.29\phantom{0}$ & $\phantom{0}0.28\phantom{0}$ & $\phantom{0}0.28\phantom{0}$ & $\phantom{0}93.1\phantom{0}$ & $\phantom{0}92.9\phantom{0}$ & $\phantom{0}86.3\phantom{0}$ & $\phantom{0}94.1\phantom{0}$ & $\phantom{0}93.8\phantom{0}$ & $\phantom{0}94.1\phantom{0}$ \\
 & \nopagebreak $\;J=100$  & $\phantom{0}\phantom{-}0.8\phantom{0}$ & $\phantom{0}\phantom{-}0.8\phantom{0}$ & ${-}13.0\phantom{0}$ & $\phantom{0}\phantom{-}0.8\phantom{0}$ & $\phantom{0}\phantom{-}0.6\phantom{0}$ & $\phantom{0}{-}0.4\phantom{0}$ & $\phantom{0}0.18\phantom{0}$ & $\phantom{0}0.19\phantom{0}$ & $\phantom{0}0.21\phantom{0}$ & $\phantom{0}0.19\phantom{0}$ & $\phantom{0}0.20\phantom{0}$ & $\phantom{0}0.19\phantom{0}$ & $\phantom{0}93.3\phantom{0}$ & $\phantom{0}94.1\phantom{0}$ & $\phantom{0}84.7\phantom{0}$ & $\phantom{0}94.1\phantom{0}$ & $\phantom{0}93.7\phantom{0}$ & $\phantom{0}94.5\phantom{0}$ \\
 & \nopagebreak $\;J=200$  & $\phantom{0}\phantom{-}0.4\phantom{0}$ & $\phantom{0}\phantom{-}0.4\phantom{0}$ & ${-}12.9\phantom{0}$ & $\phantom{0}\phantom{-}0.3\phantom{0}$ & $\phantom{0}\phantom{-}0.4\phantom{0}$ & $\phantom{0}{-}0.2\phantom{0}$ & $\phantom{0}0.12\phantom{0}$ & $\phantom{0}0.14\phantom{0}$ & $\phantom{0}0.17\phantom{0}$ & $\phantom{0}0.14\phantom{0}$ & $\phantom{0}0.14\phantom{0}$ & $\phantom{0}0.14\phantom{0}$ & $\phantom{0}93.2\phantom{0}$ & $\phantom{0}93.7\phantom{0}$ & $\phantom{0}77.6\phantom{0}$ & $\phantom{0}93.8\phantom{0}$ & $\phantom{0}94.1\phantom{0}$ & $\phantom{0}94.0\phantom{0}$ \\
 & \nopagebreak $\;J=500$  & $\phantom{0}{-}0.1\phantom{0}$ & $\phantom{0}{-}0.0\phantom{0}$ & ${-}13.4\phantom{0}$ & $\phantom{0}{-}0.1\phantom{0}$ & $\phantom{0}{-}0.0\phantom{0}$ & $\phantom{0}{-}0.2\phantom{0}$ & $\phantom{0}0.07\phantom{0}$ & $\phantom{0}0.08\phantom{0}$ & $\phantom{0}0.14\phantom{0}$ & $\phantom{0}0.09\phantom{0}$ & $\phantom{0}0.08\phantom{0}$ & $\phantom{0}0.08\phantom{0}$ & $\phantom{0}95.0\phantom{0}$ & $\phantom{0}95.4\phantom{0}$ & $\phantom{0}60.3\phantom{0}$ & $\phantom{0}95.2\phantom{0}$ & $\phantom{0}94.7\phantom{0}$ & $\phantom{0}95.8\phantom{0}$ \\
 & \nopagebreak $\;J=1000$  & $\phantom{0}\phantom{-}0.1\phantom{0}$ & $\phantom{0}\phantom{-}0.1\phantom{0}$ & ${-}13.2\phantom{0}$ & $\phantom{0}\phantom{-}0.1\phantom{0}$ & $\phantom{0}\phantom{-}0.1\phantom{0}$ & $\phantom{0}{-}0.1\phantom{0}$ & $\phantom{0}0.05\phantom{0}$ & $\phantom{0}0.06\phantom{0}$ & $\phantom{0}0.13\phantom{0}$ & $\phantom{0}0.06\phantom{0}$ & $\phantom{0}0.06\phantom{0}$ & $\phantom{0}0.06\phantom{0}$ & $\phantom{0}94.4\phantom{0}$ & $\phantom{0}93.3\phantom{0}$ & $\phantom{0}36.8\phantom{0}$ & $\phantom{0}93.0\phantom{0}$ & $\phantom{0}93.3\phantom{0}$ & $\phantom{0}94.0\phantom{0}$ \\
[0.5ex]\hline\\[-1.6ex] 
\end{tabular}
\begin{tablenotes}[para,flushleft]{\footnotesize \textit{Note.} $n$ = cluster size; $J$ = number of clusters; CD = complete data sets; LD = listwise deletion; FCS-SL = single-level FCS; FCS-MAN = two-level FCS with manifest cluster means; FCS-LAT = two-level FCS with latent cluster means; JM = joint modeling.}\end{tablenotes}
\end{threeparttable}
\end{sidewaystable}
\begin{sidewaystable}
\begin{threeparttable}
\setlength{\tabcolsep}{1.2pt}
\renewcommand{\arraystretch}{0.95}
\footnotesize
\caption{\small Study 1: Bias (in \%), RMSE, and Coverage of the 95\% Confidence Interval for the Regression Coefficient of $z$ on $y$ ($\hat\beta_{zy}$) With 20\% Missing Data (MAR, $\lambda=0.5$)}
\begin{tabular}{llcccccccccccccccccc}
\hline\\[-1.8ex]
& & \multicolumn{6}{c}{Bias (\%)} & \multicolumn{6}{c}{RMSE} & \multicolumn{6}{c}{Coverage (\%)} \\ \cmidrule(r){3-8}\cmidrule(r){9-14}\cmidrule(r){15-20}
 &  & CD & LD & \makecell{FCS-\\SL} & \makecell{FCS-\\MAN} & \makecell{FCS-\\LAT} & JM & CD & LD & \makecell{FCS-\\SL} & \makecell{FCS-\\MAN} & \makecell{FCS-\\LAT} & JM & CD & LD & \makecell{FCS-\\SL} & \makecell{FCS-\\MAN} & \makecell{FCS-\\LAT} & \multicolumn{1}{c}{JM} \\ 
[0.4ex]\hline\\[-1.8ex]
& & \multicolumn{18}{c}{Small intraclass correlation $(\rho_{Iy}=.10)$} \\[0.6ex]\hline\\[-1.8ex]
\multicolumn{4}{l}{$n=5$} \\  & \nopagebreak $\;J=30$  & $\phantom{-}43.7\phantom{0}$ & $\phantom{-}64.3\phantom{0}$ & $\phantom{-}14.5\phantom{0}$ & $\phantom{-}25.1\phantom{0}$ & $\phantom{-}30.1\phantom{0}$ & $\phantom{-}19.2\phantom{0}$ & $\phantom{0}2.90\phantom{0}$ & $\phantom{0}4.63\phantom{0}$ & $\phantom{0}2.39\phantom{0}$ & $\phantom{0}2.20\phantom{0}$ & $\phantom{0}2.37\phantom{0}$ & $\phantom{0}2.26\phantom{0}$ & $\phantom{0}87.8\phantom{0}$ & $\phantom{0}87.4\phantom{0}$ & $\phantom{0}86.3\phantom{0}$ & $\phantom{0}93.3\phantom{0}$ & $\phantom{0}93.4\phantom{0}$ & $\phantom{0}94.2\phantom{0}$ \\
 & \nopagebreak $\;J=50$  & $\phantom{-}32.1\phantom{0}$ & $\phantom{-}77.4\phantom{0}$ & $\phantom{0}\phantom{-}9.9\phantom{0}$ & $\phantom{-}27.9\phantom{0}$ & $\phantom{-}31.9\phantom{0}$ & $\phantom{-}20.5\phantom{0}$ & $\phantom{0}1.87\phantom{0}$ & $\phantom{0}3.18\phantom{0}$ & $\phantom{0}1.69\phantom{0}$ & $\phantom{0}1.65\phantom{0}$ & $\phantom{0}1.74\phantom{0}$ & $\phantom{0}1.68\phantom{0}$ & $\phantom{0}91.5\phantom{0}$ & $\phantom{0}91.2\phantom{0}$ & $\phantom{0}84.9\phantom{0}$ & $\phantom{0}93.5\phantom{0}$ & $\phantom{0}93.1\phantom{0}$ & $\phantom{0}92.9\phantom{0}$ \\
 & \nopagebreak $\;J=100$  & $\phantom{-}14.7\phantom{0}$ & $\phantom{-}61.5\phantom{0}$ & $\phantom{0}{-}7.8\phantom{0}$ & $\phantom{-}13.6\phantom{0}$ & $\phantom{-}16.9\phantom{0}$ & $\phantom{0}\phantom{-}7.0\phantom{0}$ & $\phantom{0}1.07\phantom{0}$ & $\phantom{0}2.32\phantom{0}$ & $\phantom{0}1.01\phantom{0}$ & $\phantom{0}1.09\phantom{0}$ & $\phantom{0}1.13\phantom{0}$ & $\phantom{0}1.04\phantom{0}$ & $\phantom{0}92.9\phantom{0}$ & $\phantom{0}94.4\phantom{0}$ & $\phantom{0}82.4\phantom{0}$ & $\phantom{0}94.2\phantom{0}$ & $\phantom{0}94.7\phantom{0}$ & $\phantom{0}94.0\phantom{0}$ \\
 & \nopagebreak $\;J=200$  & $\phantom{0}\phantom{-}7.6\phantom{0}$ & $\phantom{-}48.3\phantom{0}$ & ${-}15.0\phantom{0}$ & $\phantom{0}\phantom{-}7.1\phantom{0}$ & $\phantom{0}\phantom{-}9.4\phantom{0}$ & $\phantom{0}\phantom{-}2.5\phantom{0}$ & $\phantom{0}0.64\phantom{0}$ & $\phantom{0}1.43\phantom{0}$ & $\phantom{0}0.61\phantom{0}$ & $\phantom{0}0.65\phantom{0}$ & $\phantom{0}0.70\phantom{0}$ & $\phantom{0}0.65\phantom{0}$ & $\phantom{0}94.6\phantom{0}$ & $\phantom{0}97.4\phantom{0}$ & $\phantom{0}80.5\phantom{0}$ & $\phantom{0}95.1\phantom{0}$ & $\phantom{0}95.3\phantom{0}$ & $\phantom{0}93.8\phantom{0}$ \\
 & \nopagebreak $\;J=500$  & $\phantom{0}\phantom{-}2.8\phantom{0}$ & $\phantom{-}34.7\phantom{0}$ & ${-}18.7\phantom{0}$ & $\phantom{0}\phantom{-}2.8\phantom{0}$ & $\phantom{0}\phantom{-}3.4\phantom{0}$ & $\phantom{0}\phantom{-}0.6\phantom{0}$ & $\phantom{0}0.33\phantom{0}$ & $\phantom{0}0.87\phantom{0}$ & $\phantom{0}0.41\phantom{0}$ & $\phantom{0}0.36\phantom{0}$ & $\phantom{0}0.37\phantom{0}$ & $\phantom{0}0.34\phantom{0}$ & $\phantom{0}95.8\phantom{0}$ & $\phantom{0}99.7\phantom{0}$ & $\phantom{0}72.2\phantom{0}$ & $\phantom{0}96.8\phantom{0}$ & $\phantom{0}96.2\phantom{0}$ & $\phantom{0}96.1\phantom{0}$ \\
 & \nopagebreak $\;J=1000$  & $\phantom{0}\phantom{-}2.2\phantom{0}$ & $\phantom{-}32.8\phantom{0}$ & ${-}19.2\phantom{0}$ & $\phantom{0}\phantom{-}2.5\phantom{0}$ & $\phantom{0}\phantom{-}2.6\phantom{0}$ & $\phantom{0}\phantom{-}1.2\phantom{0}$ & $\phantom{0}0.23\phantom{0}$ & $\phantom{0}0.67\phantom{0}$ & $\phantom{0}0.36\phantom{0}$ & $\phantom{0}0.25\phantom{0}$ & $\phantom{0}0.25\phantom{0}$ & $\phantom{0}0.24\phantom{0}$ & $\phantom{0}95.9\phantom{0}$ & $\phantom{0}91.8\phantom{0}$ & $\phantom{0}59.0\phantom{0}$ & $\phantom{0}96.0\phantom{0}$ & $\phantom{0}95.5\phantom{0}$ & $\phantom{0}95.8\phantom{0}$ \\
\multicolumn{4}{l}{$n=20$} \\  & \nopagebreak $\;J=30$  & $\phantom{0}\phantom{-}9.5\phantom{0}$ & $\phantom{-}23.2\phantom{0}$ & ${-}16.8\phantom{0}$ & $\phantom{0}\phantom{-}9.8\phantom{0}$ & $\phantom{-}11.7\phantom{0}$ & $\phantom{0}{-}3.6\phantom{0}$ & $\phantom{0}0.90\phantom{0}$ & $\phantom{0}1.36\phantom{0}$ & $\phantom{0}0.89\phantom{0}$ & $\phantom{0}1.02\phantom{0}$ & $\phantom{0}1.05\phantom{0}$ & $\phantom{0}0.92\phantom{0}$ & $\phantom{0}91.5\phantom{0}$ & $\phantom{0}92.5\phantom{0}$ & $\phantom{0}82.7\phantom{0}$ & $\phantom{0}92.7\phantom{0}$ & $\phantom{0}92.7\phantom{0}$ & $\phantom{0}94.4\phantom{0}$ \\
 & \nopagebreak $\;J=50$  & $\phantom{0}\phantom{-}5.7\phantom{0}$ & $\phantom{-}14.0\phantom{0}$ & ${-}20.3\phantom{0}$ & $\phantom{0}\phantom{-}5.3\phantom{0}$ & $\phantom{0}\phantom{-}6.4\phantom{0}$ & $\phantom{0}{-}4.0\phantom{0}$ & $\phantom{0}0.58\phantom{0}$ & $\phantom{0}0.82\phantom{0}$ & $\phantom{0}0.63\phantom{0}$ & $\phantom{0}0.69\phantom{0}$ & $\phantom{0}0.70\phantom{0}$ & $\phantom{0}0.62\phantom{0}$ & $\phantom{0}93.3\phantom{0}$ & $\phantom{0}93.1\phantom{0}$ & $\phantom{0}81.5\phantom{0}$ & $\phantom{0}93.7\phantom{0}$ & $\phantom{0}93.9\phantom{0}$ & $\phantom{0}94.5\phantom{0}$ \\
 & \nopagebreak $\;J=100$  & $\phantom{0}\phantom{-}1.4\phantom{0}$ & $\phantom{0}\phantom{-}8.6\phantom{0}$ & ${-}23.4\phantom{0}$ & $\phantom{0}\phantom{-}1.4\phantom{0}$ & $\phantom{0}\phantom{-}1.8\phantom{0}$ & $\phantom{0}{-}3.7\phantom{0}$ & $\phantom{0}0.39\phantom{0}$ & $\phantom{0}0.52\phantom{0}$ & $\phantom{0}0.52\phantom{0}$ & $\phantom{0}0.46\phantom{0}$ & $\phantom{0}0.46\phantom{0}$ & $\phantom{0}0.43\phantom{0}$ & $\phantom{0}94.0\phantom{0}$ & $\phantom{0}94.2\phantom{0}$ & $\phantom{0}75.3\phantom{0}$ & $\phantom{0}94.2\phantom{0}$ & $\phantom{0}94.2\phantom{0}$ & $\phantom{0}95.2\phantom{0}$ \\
 & \nopagebreak $\;J=200$  & $\phantom{0}\phantom{-}1.3\phantom{0}$ & $\phantom{0}\phantom{-}7.1\phantom{0}$ & ${-}23.9\phantom{0}$ & $\phantom{0}\phantom{-}0.8\phantom{0}$ & $\phantom{0}\phantom{-}1.3\phantom{0}$ & $\phantom{0}{-}1.9\phantom{0}$ & $\phantom{0}0.26\phantom{0}$ & $\phantom{0}0.36\phantom{0}$ & $\phantom{0}0.45\phantom{0}$ & $\phantom{0}0.31\phantom{0}$ & $\phantom{0}0.32\phantom{0}$ & $\phantom{0}0.30\phantom{0}$ & $\phantom{0}95.1\phantom{0}$ & $\phantom{0}94.3\phantom{0}$ & $\phantom{0}60.4\phantom{0}$ & $\phantom{0}94.3\phantom{0}$ & $\phantom{0}93.4\phantom{0}$ & $\phantom{0}94.8\phantom{0}$ \\
 & \nopagebreak $\;J=500$  & $\phantom{0}\phantom{-}0.2\phantom{0}$ & $\phantom{0}\phantom{-}6.2\phantom{0}$ & ${-}24.4\phantom{0}$ & $\phantom{0}\phantom{-}0.4\phantom{0}$ & $\phantom{0}\phantom{-}0.5\phantom{0}$ & $\phantom{0}{-}0.9\phantom{0}$ & $\phantom{0}0.16\phantom{0}$ & $\phantom{0}0.22\phantom{0}$ & $\phantom{0}0.41\phantom{0}$ & $\phantom{0}0.19\phantom{0}$ & $\phantom{0}0.19\phantom{0}$ & $\phantom{0}0.18\phantom{0}$ & $\phantom{0}95.8\phantom{0}$ & $\phantom{0}93.7\phantom{0}$ & $\phantom{0}27.8\phantom{0}$ & $\phantom{0}95.1\phantom{0}$ & $\phantom{0}95.4\phantom{0}$ & $\phantom{0}95.0\phantom{0}$ \\
 & \nopagebreak $\;J=1000$  & $\phantom{0}\phantom{-}0.2\phantom{0}$ & $\phantom{0}\phantom{-}5.6\phantom{0}$ & ${-}24.6\phantom{0}$ & $\phantom{0}{-}0.1\phantom{0}$ & $\phantom{0}\phantom{-}0.0\phantom{0}$ & $\phantom{0}{-}0.7\phantom{0}$ & $\phantom{0}0.12\phantom{0}$ & $\phantom{0}0.17\phantom{0}$ & $\phantom{0}0.40\phantom{0}$ & $\phantom{0}0.14\phantom{0}$ & $\phantom{0}0.14\phantom{0}$ & $\phantom{0}0.13\phantom{0}$ & $\phantom{0}95.1\phantom{0}$ & $\phantom{0}91.1\phantom{0}$ & $\phantom{0}\phantom{0}6.0\phantom{0}$ & $\phantom{0}95.2\phantom{0}$ & $\phantom{0}94.4\phantom{0}$ & $\phantom{0}94.7\phantom{0}$ \\
[0.5ex]\hline\\[-1.6ex] 
& & \multicolumn{18}{c}{Moderate intraclass correlation $(\rho_{Iy}=.30)$} \\[0.6ex]\hline\\[-1.8ex]
\multicolumn{4}{l}{$n=5$} \\  & \nopagebreak $\;J=30$  & $\phantom{-}10.6\phantom{0}$ & $\phantom{-}26.6\phantom{0}$ & $\phantom{0}{-}8.7\phantom{0}$ & $\phantom{-}10.6\phantom{0}$ & $\phantom{-}13.5\phantom{0}$ & $\phantom{0}\phantom{-}2.6\phantom{0}$ & $\phantom{0}0.55\phantom{0}$ & $\phantom{0}0.87\phantom{0}$ & $\phantom{0}0.52\phantom{0}$ & $\phantom{0}0.62\phantom{0}$ & $\phantom{0}0.64\phantom{0}$ & $\phantom{0}0.57\phantom{0}$ & $\phantom{0}93.5\phantom{0}$ & $\phantom{0}93.9\phantom{0}$ & $\phantom{0}87.7\phantom{0}$ & $\phantom{0}94.5\phantom{0}$ & $\phantom{0}94.5\phantom{0}$ & $\phantom{0}95.3\phantom{0}$ \\
 & \nopagebreak $\;J=50$  & $\phantom{0}\phantom{-}5.0\phantom{0}$ & $\phantom{-}14.5\phantom{0}$ & ${-}12.6\phantom{0}$ & $\phantom{0}\phantom{-}5.4\phantom{0}$ & $\phantom{0}\phantom{-}6.5\phantom{0}$ & $\phantom{0}\phantom{-}0.1\phantom{0}$ & $\phantom{0}0.36\phantom{0}$ & $\phantom{0}0.54\phantom{0}$ & $\phantom{0}0.36\phantom{0}$ & $\phantom{0}0.42\phantom{0}$ & $\phantom{0}0.43\phantom{0}$ & $\phantom{0}0.39\phantom{0}$ & $\phantom{0}93.6\phantom{0}$ & $\phantom{0}94.6\phantom{0}$ & $\phantom{0}87.4\phantom{0}$ & $\phantom{0}94.4\phantom{0}$ & $\phantom{0}94.5\phantom{0}$ & $\phantom{0}94.7\phantom{0}$ \\
 & \nopagebreak $\;J=100$  & $\phantom{0}\phantom{-}1.1\phantom{0}$ & $\phantom{0}\phantom{-}7.5\phantom{0}$ & ${-}16.9\phantom{0}$ & $\phantom{0}{-}0.1\phantom{0}$ & $\phantom{0}\phantom{-}0.6\phantom{0}$ & $\phantom{0}{-}2.2\phantom{0}$ & $\phantom{0}0.22\phantom{0}$ & $\phantom{0}0.30\phantom{0}$ & $\phantom{0}0.27\phantom{0}$ & $\phantom{0}0.26\phantom{0}$ & $\phantom{0}0.27\phantom{0}$ & $\phantom{0}0.26\phantom{0}$ & $\phantom{0}93.7\phantom{0}$ & $\phantom{0}95.3\phantom{0}$ & $\phantom{0}83.1\phantom{0}$ & $\phantom{0}94.7\phantom{0}$ & $\phantom{0}94.7\phantom{0}$ & $\phantom{0}95.0\phantom{0}$ \\
 & \nopagebreak $\;J=200$  & $\phantom{0}\phantom{-}1.4\phantom{0}$ & $\phantom{0}\phantom{-}8.2\phantom{0}$ & ${-}15.8\phantom{0}$ & $\phantom{0}\phantom{-}1.5\phantom{0}$ & $\phantom{0}\phantom{-}1.6\phantom{0}$ & $\phantom{0}\phantom{-}0.3\phantom{0}$ & $\phantom{0}0.16\phantom{0}$ & $\phantom{0}0.22\phantom{0}$ & $\phantom{0}0.21\phantom{0}$ & $\phantom{0}0.19\phantom{0}$ & $\phantom{0}0.19\phantom{0}$ & $\phantom{0}0.18\phantom{0}$ & $\phantom{0}94.7\phantom{0}$ & $\phantom{0}93.7\phantom{0}$ & $\phantom{0}78.4\phantom{0}$ & $\phantom{0}93.9\phantom{0}$ & $\phantom{0}93.5\phantom{0}$ & $\phantom{0}94.3\phantom{0}$ \\
 & \nopagebreak $\;J=500$  & $\phantom{0}\phantom{-}0.2\phantom{0}$ & $\phantom{0}\phantom{-}5.8\phantom{0}$ & ${-}16.7\phantom{0}$ & $\phantom{0}{-}0.2\phantom{0}$ & $\phantom{0}{-}0.1\phantom{0}$ & $\phantom{0}{-}0.5\phantom{0}$ & $\phantom{0}0.10\phantom{0}$ & $\phantom{0}0.13\phantom{0}$ & $\phantom{0}0.18\phantom{0}$ & $\phantom{0}0.11\phantom{0}$ & $\phantom{0}0.11\phantom{0}$ & $\phantom{0}0.11\phantom{0}$ & $\phantom{0}94.0\phantom{0}$ & $\phantom{0}93.6\phantom{0}$ & $\phantom{0}60.7\phantom{0}$ & $\phantom{0}94.7\phantom{0}$ & $\phantom{0}95.0\phantom{0}$ & $\phantom{0}95.3\phantom{0}$ \\
 & \nopagebreak $\;J=1000$  & $\phantom{0}\phantom{-}0.3\phantom{0}$ & $\phantom{0}\phantom{-}6.2\phantom{0}$ & ${-}16.3\phantom{0}$ & $\phantom{0}\phantom{-}0.4\phantom{0}$ & $\phantom{0}\phantom{-}0.5\phantom{0}$ & $\phantom{0}\phantom{-}0.1\phantom{0}$ & $\phantom{0}0.07\phantom{0}$ & $\phantom{0}0.10\phantom{0}$ & $\phantom{0}0.16\phantom{0}$ & $\phantom{0}0.08\phantom{0}$ & $\phantom{0}0.08\phantom{0}$ & $\phantom{0}0.08\phantom{0}$ & $\phantom{0}95.1\phantom{0}$ & $\phantom{0}90.8\phantom{0}$ & $\phantom{0}36.1\phantom{0}$ & $\phantom{0}95.5\phantom{0}$ & $\phantom{0}94.5\phantom{0}$ & $\phantom{0}93.7\phantom{0}$ \\
\multicolumn{4}{l}{$n=20$} \\  & \nopagebreak $\;J=30$  & $\phantom{0}{-}0.5\phantom{0}$ & $\phantom{0}\phantom{-}1.3\phantom{0}$ & ${-}20.4\phantom{0}$ & $\phantom{0}{-}0.5\phantom{0}$ & $\phantom{0}{-}1.0\phantom{0}$ & $\phantom{0}{-}6.2\phantom{0}$ & $\phantom{0}0.33\phantom{0}$ & $\phantom{0}0.41\phantom{0}$ & $\phantom{0}0.39\phantom{0}$ & $\phantom{0}0.41\phantom{0}$ & $\phantom{0}0.42\phantom{0}$ & $\phantom{0}0.39\phantom{0}$ & $\phantom{0}91.0\phantom{0}$ & $\phantom{0}89.2\phantom{0}$ & $\phantom{0}80.7\phantom{0}$ & $\phantom{0}92.1\phantom{0}$ & $\phantom{0}92.1\phantom{0}$ & $\phantom{0}93.5\phantom{0}$ \\
 & \nopagebreak $\;J=50$  & $\phantom{0}\phantom{-}1.0\phantom{0}$ & $\phantom{0}\phantom{-}2.5\phantom{0}$ & ${-}19.1\phantom{0}$ & $\phantom{0}\phantom{-}0.7\phantom{0}$ & $\phantom{0}\phantom{-}0.7\phantom{0}$ & $\phantom{0}{-}3.0\phantom{0}$ & $\phantom{0}0.25\phantom{0}$ & $\phantom{0}0.30\phantom{0}$ & $\phantom{0}0.30\phantom{0}$ & $\phantom{0}0.29\phantom{0}$ & $\phantom{0}0.29\phantom{0}$ & $\phantom{0}0.29\phantom{0}$ & $\phantom{0}92.6\phantom{0}$ & $\phantom{0}92.9\phantom{0}$ & $\phantom{0}81.8\phantom{0}$ & $\phantom{0}94.5\phantom{0}$ & $\phantom{0}93.6\phantom{0}$ & $\phantom{0}94.4\phantom{0}$ \\
 & \nopagebreak $\;J=100$  & $\phantom{0}{-}0.2\phantom{0}$ & $\phantom{0}\phantom{-}1.0\phantom{0}$ & ${-}20.0\phantom{0}$ & $\phantom{0}{-}0.4\phantom{0}$ & $\phantom{0}{-}0.3\phantom{0}$ & $\phantom{0}{-}2.1\phantom{0}$ & $\phantom{0}0.17\phantom{0}$ & $\phantom{0}0.21\phantom{0}$ & $\phantom{0}0.25\phantom{0}$ & $\phantom{0}0.21\phantom{0}$ & $\phantom{0}0.21\phantom{0}$ & $\phantom{0}0.20\phantom{0}$ & $\phantom{0}94.1\phantom{0}$ & $\phantom{0}94.2\phantom{0}$ & $\phantom{0}74.6\phantom{0}$ & $\phantom{0}94.7\phantom{0}$ & $\phantom{0}94.9\phantom{0}$ & $\phantom{0}94.5\phantom{0}$ \\
 & \nopagebreak $\;J=200$  & $\phantom{0}\phantom{-}0.8\phantom{0}$ & $\phantom{0}\phantom{-}2.4\phantom{0}$ & ${-}19.0\phantom{0}$ & $\phantom{0}\phantom{-}0.9\phantom{0}$ & $\phantom{0}\phantom{-}1.0\phantom{0}$ & $\phantom{0}\phantom{-}0.1\phantom{0}$ & $\phantom{0}0.12\phantom{0}$ & $\phantom{0}0.14\phantom{0}$ & $\phantom{0}0.21\phantom{0}$ & $\phantom{0}0.14\phantom{0}$ & $\phantom{0}0.14\phantom{0}$ & $\phantom{0}0.14\phantom{0}$ & $\phantom{0}95.7\phantom{0}$ & $\phantom{0}95.9\phantom{0}$ & $\phantom{0}64.0\phantom{0}$ & $\phantom{0}95.7\phantom{0}$ & $\phantom{0}95.7\phantom{0}$ & $\phantom{0}96.0\phantom{0}$ \\
 & \nopagebreak $\;J=500$  & $\phantom{0}{-}0.1\phantom{0}$ & $\phantom{0}\phantom{-}1.7\phantom{0}$ & ${-}19.3\phantom{0}$ & $\phantom{0}\phantom{-}0.3\phantom{0}$ & $\phantom{0}\phantom{-}0.3\phantom{0}$ & $\phantom{0}{-}0.0\phantom{0}$ & $\phantom{0}0.08\phantom{0}$ & $\phantom{0}0.09\phantom{0}$ & $\phantom{0}0.19\phantom{0}$ & $\phantom{0}0.09\phantom{0}$ & $\phantom{0}0.09\phantom{0}$ & $\phantom{0}0.09\phantom{0}$ & $\phantom{0}94.5\phantom{0}$ & $\phantom{0}94.5\phantom{0}$ & $\phantom{0}31.0\phantom{0}$ & $\phantom{0}94.7\phantom{0}$ & $\phantom{0}94.9\phantom{0}$ & $\phantom{0}95.0\phantom{0}$ \\
 & \nopagebreak $\;J=1000$  & $\phantom{0}\phantom{-}0.0\phantom{0}$ & $\phantom{0}\phantom{-}1.6\phantom{0}$ & ${-}19.4\phantom{0}$ & $\phantom{0}\phantom{-}0.2\phantom{0}$ & $\phantom{0}\phantom{-}0.3\phantom{0}$ & $\phantom{0}\phantom{-}0.1\phantom{0}$ & $\phantom{0}0.05\phantom{0}$ & $\phantom{0}0.07\phantom{0}$ & $\phantom{0}0.19\phantom{0}$ & $\phantom{0}0.06\phantom{0}$ & $\phantom{0}0.07\phantom{0}$ & $\phantom{0}0.07\phantom{0}$ & $\phantom{0}94.4\phantom{0}$ & $\phantom{0}93.6\phantom{0}$ & $\phantom{0}\phantom{0}7.9\phantom{0}$ & $\phantom{0}95.8\phantom{0}$ & $\phantom{0}94.4\phantom{0}$ & $\phantom{0}94.3\phantom{0}$ \\
[0.5ex]\hline\\[-1.6ex] 
\end{tabular}
\begin{tablenotes}[para,flushleft]{\footnotesize \textit{Note.} $n$ = cluster size; $J$ = number of clusters; CD = complete data sets; LD = listwise deletion; FCS-SL = single-level FCS; FCS-MAN = two-level FCS with manifest cluster means; FCS-LAT = two-level FCS with latent cluster means; JM = joint modeling.}\end{tablenotes}
\end{threeparttable}
\end{sidewaystable}
\begin{sidewaystable}
\begin{threeparttable}
\setlength{\tabcolsep}{1.2pt}
\renewcommand{\arraystretch}{0.95}
\footnotesize
\caption{\small Study 1: Bias (in \%), RMSE, and Coverage of the 95\% Confidence Interval for the Regression Coefficient of $z$ on $y$ ($\hat\beta_{zy}$) With 20\% Missing Data (MAR, $\lambda=1$)}
\begin{tabular}{llcccccccccccccccccc}
\hline\\[-1.8ex]
& & \multicolumn{6}{c}{Bias (\%)} & \multicolumn{6}{c}{RMSE} & \multicolumn{6}{c}{Coverage (\%)} \\ \cmidrule(r){3-8}\cmidrule(r){9-14}\cmidrule(r){15-20}
 &  & CD & LD & \makecell{FCS-\\SL} & \makecell{FCS-\\MAN} & \makecell{FCS-\\LAT} & JM & CD & LD & \makecell{FCS-\\SL} & \makecell{FCS-\\MAN} & \makecell{FCS-\\LAT} & JM & CD & LD & \makecell{FCS-\\SL} & \makecell{FCS-\\MAN} & \makecell{FCS-\\LAT} & \multicolumn{1}{c}{JM} \\ 
[0.4ex]\hline\\[-1.8ex]
& & \multicolumn{18}{c}{Small intraclass correlation $(\rho_{Iy}=.10)$} \\[0.6ex]\hline\\[-1.8ex]
\multicolumn{4}{l}{$n=5$} \\  & \nopagebreak $\;J=30$  & $\phantom{0}\phantom{-}49.3\phantom{0}$ & $\phantom{-}182.8\phantom{0}$ & $\phantom{0}\phantom{0}{-}5.7\phantom{0}$ & $\phantom{0}\phantom{-}13.1\phantom{0}$ & $\phantom{0}\phantom{-}22.5\phantom{0}$ & $\phantom{0}\phantom{0}\phantom{-}1.2\phantom{0}$ & $\phantom{0}3.07\phantom{0}$ & $\phantom{0}7.90\phantom{0}$ & $\phantom{0}2.46\phantom{0}$ & $\phantom{0}2.22\phantom{0}$ & $\phantom{0}2.32\phantom{0}$ & $\phantom{0}2.12\phantom{0}$ & $\phantom{0}89.8\phantom{0}$ & $\phantom{0}68.0\phantom{0}$ & $\phantom{0}81.2\phantom{0}$ & $\phantom{0}94.3\phantom{0}$ & $\phantom{0}93.6\phantom{0}$ & $\phantom{0}95.5\phantom{0}$ \\
 & \nopagebreak $\;J=50$  & $\phantom{0}\phantom{-}35.0\phantom{0}$ & $\phantom{-}253.5\phantom{0}$ & $\phantom{0}{-}17.7\phantom{0}$ & $\phantom{0}\phantom{-}15.2\phantom{0}$ & $\phantom{0}\phantom{-}26.1\phantom{0}$ & $\phantom{0}\phantom{0}{-}1.1\phantom{0}$ & $\phantom{0}2.44\phantom{0}$ & $\phantom{0}6.97\phantom{0}$ & $\phantom{0}1.85\phantom{0}$ & $\phantom{0}1.72\phantom{0}$ & $\phantom{0}1.86\phantom{0}$ & $\phantom{0}1.65\phantom{0}$ & $\phantom{0}91.4\phantom{0}$ & $\phantom{0}67.0\phantom{0}$ & $\phantom{0}75.5\phantom{0}$ & $\phantom{0}94.8\phantom{0}$ & $\phantom{0}94.5\phantom{0}$ & $\phantom{0}95.2\phantom{0}$ \\
 & \nopagebreak $\;J=100$  & $\phantom{0}\phantom{-}22.6\phantom{0}$ & $\phantom{-}318.5\phantom{0}$ & $\phantom{0}{-}24.6\phantom{0}$ & $\phantom{0}\phantom{-}18.7\phantom{0}$ & $\phantom{0}\phantom{-}28.5\phantom{0}$ & $\phantom{0}\phantom{0}\phantom{-}1.5\phantom{0}$ & $\phantom{0}1.18\phantom{0}$ & $\phantom{0}6.68\phantom{0}$ & $\phantom{0}1.24\phantom{0}$ & $\phantom{0}1.27\phantom{0}$ & $\phantom{0}1.40\phantom{0}$ & $\phantom{0}1.11\phantom{0}$ & $\phantom{0}93.0\phantom{0}$ & $\phantom{0}54.0\phantom{0}$ & $\phantom{0}68.8\phantom{0}$ & $\phantom{0}93.8\phantom{0}$ & $\phantom{0}94.0\phantom{0}$ & $\phantom{0}93.2\phantom{0}$ \\
 & \nopagebreak $\;J=200$  & $\phantom{0}\phantom{0}\phantom{-}7.5\phantom{0}$ & $\phantom{-}359.0\phantom{0}$ & $\phantom{0}{-}38.9\phantom{0}$ & $\phantom{0}\phantom{0}\phantom{-}6.9\phantom{0}$ & $\phantom{0}\phantom{-}12.0\phantom{0}$ & $\phantom{0}\phantom{0}{-}5.3\phantom{0}$ & $\phantom{0}0.58\phantom{0}$ & $\phantom{0}6.51\phantom{0}$ & $\phantom{0}0.76\phantom{0}$ & $\phantom{0}0.71\phantom{0}$ & $\phantom{0}0.76\phantom{0}$ & $\phantom{0}0.62\phantom{0}$ & $\phantom{0}94.9\phantom{0}$ & $\phantom{0}34.6\phantom{0}$ & $\phantom{0}53.6\phantom{0}$ & $\phantom{0}95.7\phantom{0}$ & $\phantom{0}96.1\phantom{0}$ & $\phantom{0}94.8\phantom{0}$ \\
 & \nopagebreak $\;J=500$  & $\phantom{0}\phantom{0}\phantom{-}3.1\phantom{0}$ & $\phantom{-}395.7\phantom{0}$ & $\phantom{0}{-}41.6\phantom{0}$ & $\phantom{0}\phantom{0}\phantom{-}2.9\phantom{0}$ & $\phantom{0}\phantom{0}\phantom{-}4.9\phantom{0}$ & $\phantom{0}\phantom{0}{-}3.1\phantom{0}$ & $\phantom{0}0.33\phantom{0}$ & $\phantom{0}6.45\phantom{0}$ & $\phantom{0}0.70\phantom{0}$ & $\phantom{0}0.41\phantom{0}$ & $\phantom{0}0.43\phantom{0}$ & $\phantom{0}0.37\phantom{0}$ & $\phantom{0}96.6\phantom{0}$ & $\phantom{0}\phantom{0}8.5\phantom{0}$ & $\phantom{0}25.9\phantom{0}$ & $\phantom{0}96.9\phantom{0}$ & $\phantom{0}95.7\phantom{0}$ & $\phantom{0}95.6\phantom{0}$ \\
 & \nopagebreak $\;J=1000$  & $\phantom{0}\phantom{0}\phantom{-}1.0\phantom{0}$ & $\phantom{-}438.5\phantom{0}$ & $\phantom{0}{-}43.0\phantom{0}$ & $\phantom{0}\phantom{0}\phantom{-}0.9\phantom{0}$ & $\phantom{0}\phantom{0}\phantom{-}1.7\phantom{0}$ & $\phantom{0}\phantom{0}{-}2.5\phantom{0}$ & $\phantom{0}0.22\phantom{0}$ & $\phantom{0}7.06\phantom{0}$ & $\phantom{0}0.70\phantom{0}$ & $\phantom{0}0.29\phantom{0}$ & $\phantom{0}0.29\phantom{0}$ & $\phantom{0}0.27\phantom{0}$ & $\phantom{0}95.9\phantom{0}$ & $\phantom{0}\phantom{0}0.9\phantom{0}$ & $\phantom{0}\phantom{0}6.6\phantom{0}$ & $\phantom{0}94.6\phantom{0}$ & $\phantom{0}94.1\phantom{0}$ & $\phantom{0}94.8\phantom{0}$ \\
\multicolumn{4}{l}{$n=20$} \\  & \nopagebreak $\;J=30$  & $\phantom{0}\phantom{0}\phantom{-}9.4\phantom{0}$ & $\phantom{0}\phantom{-}90.1\phantom{0}$ & $\phantom{0}{-}41.5\phantom{0}$ & $\phantom{0}\phantom{-}10.1\phantom{0}$ & $\phantom{0}\phantom{-}13.6\phantom{0}$ & $\phantom{0}{-}20.1\phantom{0}$ & $\phantom{0}0.90\phantom{0}$ & $\phantom{0}4.63\phantom{0}$ & $\phantom{0}0.97\phantom{0}$ & $\phantom{0}1.26\phantom{0}$ & $\phantom{0}1.30\phantom{0}$ & $\phantom{0}0.96\phantom{0}$ & $\phantom{0}91.4\phantom{0}$ & $\phantom{0}91.2\phantom{0}$ & $\phantom{0}64.0\phantom{0}$ & $\phantom{0}93.0\phantom{0}$ & $\phantom{0}91.0\phantom{0}$ & $\phantom{0}94.8\phantom{0}$ \\
 & \nopagebreak $\;J=50$  & $\phantom{0}\phantom{0}\phantom{-}4.7\phantom{0}$ & $\phantom{0}\phantom{-}83.0\phantom{0}$ & $\phantom{0}{-}44.9\phantom{0}$ & $\phantom{0}\phantom{0}\phantom{-}6.2\phantom{0}$ & $\phantom{0}\phantom{0}\phantom{-}9.4\phantom{0}$ & $\phantom{0}{-}16.0\phantom{0}$ & $\phantom{0}0.59\phantom{0}$ & $\phantom{0}2.33\phantom{0}$ & $\phantom{0}0.86\phantom{0}$ & $\phantom{0}0.86\phantom{0}$ & $\phantom{0}0.90\phantom{0}$ & $\phantom{0}0.72\phantom{0}$ & $\phantom{0}93.2\phantom{0}$ & $\phantom{0}93.9\phantom{0}$ & $\phantom{0}55.4\phantom{0}$ & $\phantom{0}93.4\phantom{0}$ & $\phantom{0}91.4\phantom{0}$ & $\phantom{0}95.2\phantom{0}$ \\
 & \nopagebreak $\;J=100$  & $\phantom{0}\phantom{0}\phantom{-}1.7\phantom{0}$ & $\phantom{0}\phantom{-}63.6\phantom{0}$ & $\phantom{0}{-}47.1\phantom{0}$ & $\phantom{0}\phantom{0}\phantom{-}2.4\phantom{0}$ & $\phantom{0}\phantom{0}\phantom{-}3.1\phantom{0}$ & $\phantom{0}{-}11.1\phantom{0}$ & $\phantom{0}0.39\phantom{0}$ & $\phantom{0}1.59\phantom{0}$ & $\phantom{0}0.81\phantom{0}$ & $\phantom{0}0.56\phantom{0}$ & $\phantom{0}0.58\phantom{0}$ & $\phantom{0}0.50\phantom{0}$ & $\phantom{0}94.5\phantom{0}$ & $\phantom{0}92.0\phantom{0}$ & $\phantom{0}32.2\phantom{0}$ & $\phantom{0}93.2\phantom{0}$ & $\phantom{0}92.5\phantom{0}$ & $\phantom{0}94.9\phantom{0}$ \\
 & \nopagebreak $\;J=200$  & $\phantom{0}\phantom{0}\phantom{-}0.7\phantom{0}$ & $\phantom{0}\phantom{-}54.2\phantom{0}$ & $\phantom{0}{-}48.6\phantom{0}$ & $\phantom{0}\phantom{0}\phantom{-}0.2\phantom{0}$ & $\phantom{0}\phantom{0}\phantom{-}0.7\phantom{0}$ & $\phantom{0}\phantom{0}{-}7.9\phantom{0}$ & $\phantom{0}0.26\phantom{0}$ & $\phantom{0}1.09\phantom{0}$ & $\phantom{0}0.80\phantom{0}$ & $\phantom{0}0.37\phantom{0}$ & $\phantom{0}0.37\phantom{0}$ & $\phantom{0}0.37\phantom{0}$ & $\phantom{0}95.4\phantom{0}$ & $\phantom{0}84.2\phantom{0}$ & $\phantom{0}\phantom{0}8.3\phantom{0}$ & $\phantom{0}94.2\phantom{0}$ & $\phantom{0}94.7\phantom{0}$ & $\phantom{0}93.6\phantom{0}$ \\
 & \nopagebreak $\;J=500$  & $\phantom{0}\phantom{0}{-}0.5\phantom{0}$ & $\phantom{0}\phantom{-}48.7\phantom{0}$ & $\phantom{0}{-}48.8\phantom{0}$ & $\phantom{0}\phantom{0}{-}0.4\phantom{0}$ & $\phantom{0}\phantom{0}{-}0.4\phantom{0}$ & $\phantom{0}\phantom{0}{-}4.1\phantom{0}$ & $\phantom{0}0.16\phantom{0}$ & $\phantom{0}0.87\phantom{0}$ & $\phantom{0}0.78\phantom{0}$ & $\phantom{0}0.24\phantom{0}$ & $\phantom{0}0.24\phantom{0}$ & $\phantom{0}0.23\phantom{0}$ & $\phantom{0}94.4\phantom{0}$ & $\phantom{0}52.3\phantom{0}$ & $\phantom{0}\phantom{0}0.0\phantom{0}$ & $\phantom{0}93.7\phantom{0}$ & $\phantom{0}93.5\phantom{0}$ & $\phantom{0}94.2\phantom{0}$ \\
 & \nopagebreak $\;J=1000$  & $\phantom{0}\phantom{0}\phantom{-}0.2\phantom{0}$ & $\phantom{0}\phantom{-}48.7\phantom{0}$ & $\phantom{0}{-}48.4\phantom{0}$ & $\phantom{0}\phantom{0}\phantom{-}0.1\phantom{0}$ & $\phantom{0}\phantom{0}\phantom{-}0.2\phantom{0}$ & $\phantom{0}\phantom{0}{-}1.7\phantom{0}$ & $\phantom{0}0.12\phantom{0}$ & $\phantom{0}0.82\phantom{0}$ & $\phantom{0}0.77\phantom{0}$ & $\phantom{0}0.18\phantom{0}$ & $\phantom{0}0.18\phantom{0}$ & $\phantom{0}0.17\phantom{0}$ & $\phantom{0}94.9\phantom{0}$ & $\phantom{0}18.1\phantom{0}$ & $\phantom{0}\phantom{0}0.0\phantom{0}$ & $\phantom{0}92.9\phantom{0}$ & $\phantom{0}93.0\phantom{0}$ & $\phantom{0}94.3\phantom{0}$ \\
[0.5ex]\hline\\[-1.6ex] 
& & \multicolumn{18}{c}{Moderate intraclass correlation $(\rho_{Iy}=.30)$} \\[0.6ex]\hline\\[-1.8ex]
\multicolumn{4}{l}{$n=5$} \\  & \nopagebreak $\;J=30$  & $\phantom{0}\phantom{-}10.1\phantom{0}$ & $\phantom{-}117.1\phantom{0}$ & $\phantom{0}{-}31.1\phantom{0}$ & $\phantom{0}\phantom{0}\phantom{-}7.8\phantom{0}$ & $\phantom{0}\phantom{-}15.5\phantom{0}$ & $\phantom{0}{-}10.3\phantom{0}$ & $\phantom{0}0.52\phantom{0}$ & $\phantom{0}2.68\phantom{0}$ & $\phantom{0}0.53\phantom{0}$ & $\phantom{0}0.68\phantom{0}$ & $\phantom{0}0.75\phantom{0}$ & $\phantom{0}0.56\phantom{0}$ & $\phantom{0}92.9\phantom{0}$ & $\phantom{0}91.4\phantom{0}$ & $\phantom{0}78.3\phantom{0}$ & $\phantom{0}93.5\phantom{0}$ & $\phantom{0}91.5\phantom{0}$ & $\phantom{0}95.3\phantom{0}$ \\
 & \nopagebreak $\;J=50$  & $\phantom{0}\phantom{0}\phantom{-}4.5\phantom{0}$ & $\phantom{0}\phantom{-}92.7\phantom{0}$ & $\phantom{0}{-}34.1\phantom{0}$ & $\phantom{0}\phantom{0}\phantom{-}4.2\phantom{0}$ & $\phantom{0}\phantom{0}\phantom{-}8.1\phantom{0}$ & $\phantom{0}\phantom{0}{-}8.4\phantom{0}$ & $\phantom{0}0.35\phantom{0}$ & $\phantom{0}1.66\phantom{0}$ & $\phantom{0}0.44\phantom{0}$ & $\phantom{0}0.49\phantom{0}$ & $\phantom{0}0.52\phantom{0}$ & $\phantom{0}0.42\phantom{0}$ & $\phantom{0}93.7\phantom{0}$ & $\phantom{0}94.9\phantom{0}$ & $\phantom{0}72.7\phantom{0}$ & $\phantom{0}94.5\phantom{0}$ & $\phantom{0}93.1\phantom{0}$ & $\phantom{0}95.1\phantom{0}$ \\
 & \nopagebreak $\;J=100$  & $\phantom{0}\phantom{0}\phantom{-}2.1\phantom{0}$ & $\phantom{0}\phantom{-}70.5\phantom{0}$ & $\phantom{0}{-}35.6\phantom{0}$ & $\phantom{0}\phantom{0}\phantom{-}1.6\phantom{0}$ & $\phantom{0}\phantom{0}\phantom{-}3.0\phantom{0}$ & $\phantom{0}\phantom{0}{-}4.5\phantom{0}$ & $\phantom{0}0.22\phantom{0}$ & $\phantom{0}1.09\phantom{0}$ & $\phantom{0}0.39\phantom{0}$ & $\phantom{0}0.33\phantom{0}$ & $\phantom{0}0.34\phantom{0}$ & $\phantom{0}0.31\phantom{0}$ & $\phantom{0}94.7\phantom{0}$ & $\phantom{0}94.5\phantom{0}$ & $\phantom{0}56.9\phantom{0}$ & $\phantom{0}94.3\phantom{0}$ & $\phantom{0}93.0\phantom{0}$ & $\phantom{0}94.5\phantom{0}$ \\
 & \nopagebreak $\;J=200$  & $\phantom{0}\phantom{0}\phantom{-}0.7\phantom{0}$ & $\phantom{0}\phantom{-}57.9\phantom{0}$ & $\phantom{0}{-}36.4\phantom{0}$ & $\phantom{0}\phantom{0}\phantom{-}0.1\phantom{0}$ & $\phantom{0}\phantom{0}\phantom{-}0.5\phantom{0}$ & $\phantom{0}\phantom{0}{-}2.5\phantom{0}$ & $\phantom{0}0.15\phantom{0}$ & $\phantom{0}0.70\phantom{0}$ & $\phantom{0}0.36\phantom{0}$ & $\phantom{0}0.22\phantom{0}$ & $\phantom{0}0.22\phantom{0}$ & $\phantom{0}0.21\phantom{0}$ & $\phantom{0}95.5\phantom{0}$ & $\phantom{0}86.4\phantom{0}$ & $\phantom{0}33.0\phantom{0}$ & $\phantom{0}93.6\phantom{0}$ & $\phantom{0}92.5\phantom{0}$ & $\phantom{0}95.1\phantom{0}$ \\
 & \nopagebreak $\;J=500$  & $\phantom{0}\phantom{0}\phantom{-}0.1\phantom{0}$ & $\phantom{0}\phantom{-}51.9\phantom{0}$ & $\phantom{0}{-}36.8\phantom{0}$ & $\phantom{0}\phantom{0}{-}0.4\phantom{0}$ & $\phantom{0}\phantom{0}\phantom{-}0.0\phantom{0}$ & $\phantom{0}\phantom{0}{-}1.2\phantom{0}$ & $\phantom{0}0.09\phantom{0}$ & $\phantom{0}0.53\phantom{0}$ & $\phantom{0}0.35\phantom{0}$ & $\phantom{0}0.14\phantom{0}$ & $\phantom{0}0.14\phantom{0}$ & $\phantom{0}0.14\phantom{0}$ & $\phantom{0}95.6\phantom{0}$ & $\phantom{0}51.1\phantom{0}$ & $\phantom{0}\phantom{0}4.3\phantom{0}$ & $\phantom{0}94.9\phantom{0}$ & $\phantom{0}93.4\phantom{0}$ & $\phantom{0}95.3\phantom{0}$ \\
 & \nopagebreak $\;J=1000$  & $\phantom{0}\phantom{0}\phantom{-}0.4\phantom{0}$ & $\phantom{0}\phantom{-}51.6\phantom{0}$ & $\phantom{0}{-}36.5\phantom{0}$ & $\phantom{0}\phantom{0}{-}0.0\phantom{0}$ & $\phantom{0}\phantom{0}\phantom{-}0.3\phantom{0}$ & $\phantom{0}\phantom{0}{-}0.4\phantom{0}$ & $\phantom{0}0.07\phantom{0}$ & $\phantom{0}0.50\phantom{0}$ & $\phantom{0}0.34\phantom{0}$ & $\phantom{0}0.10\phantom{0}$ & $\phantom{0}0.10\phantom{0}$ & $\phantom{0}0.10\phantom{0}$ & $\phantom{0}94.7\phantom{0}$ & $\phantom{0}15.8\phantom{0}$ & $\phantom{0}\phantom{0}0.3\phantom{0}$ & $\phantom{0}93.8\phantom{0}$ & $\phantom{0}94.2\phantom{0}$ & $\phantom{0}94.5\phantom{0}$ \\
\multicolumn{4}{l}{$n=20$} \\  & \nopagebreak $\;J=30$  & $\phantom{0}\phantom{0}{-}1.3\phantom{0}$ & $\phantom{0}\phantom{-}12.3\phantom{0}$ & $\phantom{0}{-}42.1\phantom{0}$ & $\phantom{0}\phantom{0}{-}0.2\phantom{0}$ & $\phantom{0}\phantom{0}\phantom{-}0.4\phantom{0}$ & $\phantom{0}{-}14.2\phantom{0}$ & $\phantom{0}0.33\phantom{0}$ & $\phantom{0}0.60\phantom{0}$ & $\phantom{0}0.48\phantom{0}$ & $\phantom{0}0.50\phantom{0}$ & $\phantom{0}0.51\phantom{0}$ & $\phantom{0}0.45\phantom{0}$ & $\phantom{0}90.7\phantom{0}$ & $\phantom{0}89.6\phantom{0}$ & $\phantom{0}60.9\phantom{0}$ & $\phantom{0}93.8\phantom{0}$ & $\phantom{0}93.1\phantom{0}$ & $\phantom{0}94.2\phantom{0}$ \\
 & \nopagebreak $\;J=50$  & $\phantom{0}\phantom{0}\phantom{-}1.0\phantom{0}$ & $\phantom{0}\phantom{-}12.2\phantom{0}$ & $\phantom{0}{-}40.2\phantom{0}$ & $\phantom{0}\phantom{0}\phantom{-}1.4\phantom{0}$ & $\phantom{0}\phantom{0}\phantom{-}1.5\phantom{0}$ & $\phantom{0}\phantom{0}{-}7.8\phantom{0}$ & $\phantom{0}0.26\phantom{0}$ & $\phantom{0}0.44\phantom{0}$ & $\phantom{0}0.44\phantom{0}$ & $\phantom{0}0.39\phantom{0}$ & $\phantom{0}0.39\phantom{0}$ & $\phantom{0}0.36\phantom{0}$ & $\phantom{0}93.1\phantom{0}$ & $\phantom{0}91.3\phantom{0}$ & $\phantom{0}53.4\phantom{0}$ & $\phantom{0}91.9\phantom{0}$ & $\phantom{0}92.3\phantom{0}$ & $\phantom{0}94.8\phantom{0}$ \\
 & \nopagebreak $\;J=100$  & $\phantom{0}\phantom{0}\phantom{-}0.8\phantom{0}$ & $\phantom{0}\phantom{-}11.1\phantom{0}$ & $\phantom{0}{-}40.6\phantom{0}$ & $\phantom{0}\phantom{0}\phantom{-}1.0\phantom{0}$ & $\phantom{0}\phantom{0}\phantom{-}1.2\phantom{0}$ & $\phantom{0}\phantom{0}{-}3.6\phantom{0}$ & $\phantom{0}0.18\phantom{0}$ & $\phantom{0}0.31\phantom{0}$ & $\phantom{0}0.40\phantom{0}$ & $\phantom{0}0.27\phantom{0}$ & $\phantom{0}0.27\phantom{0}$ & $\phantom{0}0.25\phantom{0}$ & $\phantom{0}93.9\phantom{0}$ & $\phantom{0}92.4\phantom{0}$ & $\phantom{0}30.6\phantom{0}$ & $\phantom{0}92.5\phantom{0}$ & $\phantom{0}93.4\phantom{0}$ & $\phantom{0}93.8\phantom{0}$ \\
 & \nopagebreak $\;J=200$  & $\phantom{0}\phantom{0}\phantom{-}0.6\phantom{0}$ & $\phantom{0}\phantom{-}10.3\phantom{0}$ & $\phantom{0}{-}41.1\phantom{0}$ & $\phantom{0}\phantom{0}\phantom{-}1.2\phantom{0}$ & $\phantom{0}\phantom{0}\phantom{-}0.8\phantom{0}$ & $\phantom{0}\phantom{0}{-}1.6\phantom{0}$ & $\phantom{0}0.12\phantom{0}$ & $\phantom{0}0.21\phantom{0}$ & $\phantom{0}0.39\phantom{0}$ & $\phantom{0}0.18\phantom{0}$ & $\phantom{0}0.18\phantom{0}$ & $\phantom{0}0.17\phantom{0}$ & $\phantom{0}95.2\phantom{0}$ & $\phantom{0}92.7\phantom{0}$ & $\phantom{0}\phantom{0}7.9\phantom{0}$ & $\phantom{0}94.7\phantom{0}$ & $\phantom{0}95.0\phantom{0}$ & $\phantom{0}94.8\phantom{0}$ \\
 & \nopagebreak $\;J=500$  & $\phantom{0}\phantom{0}\phantom{-}0.1\phantom{0}$ & $\phantom{0}\phantom{0}\phantom{-}9.0\phantom{0}$ & $\phantom{0}{-}41.4\phantom{0}$ & $\phantom{0}\phantom{0}{-}0.3\phantom{0}$ & $\phantom{0}\phantom{0}{-}0.2\phantom{0}$ & $\phantom{0}\phantom{0}{-}1.1\phantom{0}$ & $\phantom{0}0.08\phantom{0}$ & $\phantom{0}0.15\phantom{0}$ & $\phantom{0}0.38\phantom{0}$ & $\phantom{0}0.12\phantom{0}$ & $\phantom{0}0.12\phantom{0}$ & $\phantom{0}0.12\phantom{0}$ & $\phantom{0}94.2\phantom{0}$ & $\phantom{0}89.8\phantom{0}$ & $\phantom{0}\phantom{0}0.0\phantom{0}$ & $\phantom{0}93.1\phantom{0}$ & $\phantom{0}94.2\phantom{0}$ & $\phantom{0}94.3\phantom{0}$ \\
 & \nopagebreak $\;J=1000$  & $\phantom{0}\phantom{0}{-}0.1\phantom{0}$ & $\phantom{0}\phantom{0}\phantom{-}9.1\phantom{0}$ & $\phantom{0}{-}41.1\phantom{0}$ & $\phantom{0}\phantom{0}{-}0.0\phantom{0}$ & $\phantom{0}\phantom{0}\phantom{-}0.0\phantom{0}$ & $\phantom{0}\phantom{0}{-}0.5\phantom{0}$ & $\phantom{0}0.05\phantom{0}$ & $\phantom{0}0.12\phantom{0}$ & $\phantom{0}0.38\phantom{0}$ & $\phantom{0}0.08\phantom{0}$ & $\phantom{0}0.08\phantom{0}$ & $\phantom{0}0.08\phantom{0}$ & $\phantom{0}95.6\phantom{0}$ & $\phantom{0}83.5\phantom{0}$ & $\phantom{0}\phantom{0}0.0\phantom{0}$ & $\phantom{0}93.8\phantom{0}$ & $\phantom{0}93.7\phantom{0}$ & $\phantom{0}94.5\phantom{0}$ \\
[0.5ex]\hline\\[-1.6ex] 
\end{tabular}
\begin{tablenotes}[para,flushleft]{\footnotesize \textit{Note.} $n$ = cluster size; $J$ = number of clusters; CD = complete data sets; LD = listwise deletion; FCS-SL = single-level FCS; FCS-MAN = two-level FCS with manifest cluster means; FCS-LAT = two-level FCS with latent cluster means; JM = joint modeling.}\end{tablenotes}
\end{threeparttable}
\end{sidewaystable}
\begin{sidewaystable}
\begin{threeparttable}
\setlength{\tabcolsep}{1.2pt}
\renewcommand{\arraystretch}{0.95}
\footnotesize
\caption{\small Study 1: Bias (in \%), RMSE, and Coverage of the 95\% Confidence Interval for the Regression Coefficient of $z$ on $y$ ($\hat\beta_{zy}$) With 40\% Missing Data (MCAR, $\lambda=0$)}
\begin{tabular}{llcccccccccccccccccc}
\hline\\[-1.8ex]
& & \multicolumn{6}{c}{Bias (\%)} & \multicolumn{6}{c}{RMSE} & \multicolumn{6}{c}{Coverage (\%)} \\ \cmidrule(r){3-8}\cmidrule(r){9-14}\cmidrule(r){15-20}
 &  & CD & LD & \makecell{FCS-\\SL} & \makecell{FCS-\\MAN} & \makecell{FCS-\\LAT} & JM & CD & LD & \makecell{FCS-\\SL} & \makecell{FCS-\\MAN} & \makecell{FCS-\\LAT} & JM & CD & LD & \makecell{FCS-\\SL} & \makecell{FCS-\\MAN} & \makecell{FCS-\\LAT} & \multicolumn{1}{c}{JM} \\ 
[0.4ex]\hline\\[-1.8ex]
& & \multicolumn{18}{c}{Small intraclass correlation $(\rho_{Iy}=.10)$} \\[0.6ex]\hline\\[-1.8ex]
\multicolumn{4}{l}{$n=5$} \\  & \nopagebreak $\;J=30$  & $\phantom{-}42.3\phantom{0}$ & $\phantom{-}37.4\phantom{0}$ & $\phantom{0}{-}5.4\phantom{0}$ & $\phantom{-}16.0\phantom{0}$ & $\phantom{-}23.0\phantom{0}$ & $\phantom{0}\phantom{-}7.6\phantom{0}$ & $\phantom{0}3.20\phantom{0}$ & $\phantom{0}3.63\phantom{0}$ & $\phantom{0}1.95\phantom{0}$ & $\phantom{0}2.13\phantom{0}$ & $\phantom{0}2.29\phantom{0}$ & $\phantom{0}2.16\phantom{0}$ & $\phantom{0}88.3\phantom{0}$ & $\phantom{0}84.2\phantom{0}$ & $\phantom{0}82.7\phantom{0}$ & $\phantom{0}93.8\phantom{0}$ & $\phantom{0}94.9\phantom{0}$ & $\phantom{0}94.0\phantom{0}$ \\
 & \nopagebreak $\;J=50$  & $\phantom{-}32.3\phantom{0}$ & $\phantom{-}42.6\phantom{0}$ & $\phantom{0}{-}7.2\phantom{0}$ & $\phantom{-}20.6\phantom{0}$ & $\phantom{-}26.1\phantom{0}$ & $\phantom{0}\phantom{-}10.0\phantom{0}$ & $\phantom{0}2.12\phantom{0}$ & $\phantom{0}3.15\phantom{0}$ & $\phantom{0}1.50\phantom{0}$ & $\phantom{0}1.71\phantom{0}$ & $\phantom{0}1.72\phantom{0}$ & $\phantom{0}1.67\phantom{0}$ & $\phantom{0}91.6\phantom{0}$ & $\phantom{0}89.0\phantom{0}$ & $\phantom{0}80.6\phantom{0}$ & $\phantom{0}94.6\phantom{0}$ & $\phantom{0}94.2\phantom{0}$ & $\phantom{0}93.5\phantom{0}$ \\
 & \nopagebreak $\;J=100$  & $\phantom{-}21.8\phantom{0}$ & $\phantom{-}32.5\phantom{0}$ & ${-}13.6\phantom{0}$ & $\phantom{-}16.8\phantom{0}$ & $\phantom{-}22.0\phantom{0}$ & $\phantom{0}\phantom{-}7.4\phantom{0}$ & $\phantom{0}1.26\phantom{0}$ & $\phantom{0}1.65\phantom{0}$ & $\phantom{0}1.13\phantom{0}$ & $\phantom{0}1.19\phantom{0}$ & $\phantom{0}1.25\phantom{0}$ & $\phantom{0}1.21\phantom{0}$ & $\phantom{0}93.4\phantom{0}$ & $\phantom{0}92.2\phantom{0}$ & $\phantom{0}79.6\phantom{0}$ & $\phantom{0}94.6\phantom{0}$ & $\phantom{0}95.3\phantom{0}$ & $\phantom{0}94.2\phantom{0}$ \\
 & \nopagebreak $\;J=200$  & $\phantom{-}11.5\phantom{0}$ & $\phantom{-}21.5\phantom{0}$ & ${-}21.4\phantom{0}$ & $\phantom{-}11.6\phantom{0}$ & $\phantom{-}15.1\phantom{0}$ & $\phantom{0}\phantom{-}3.1\phantom{0}$ & $\phantom{0}0.67\phantom{0}$ & $\phantom{0}1.04\phantom{0}$ & $\phantom{0}0.64\phantom{0}$ & $\phantom{0}0.76\phantom{0}$ & $\phantom{0}0.80\phantom{0}$ & $\phantom{0}0.67\phantom{0}$ & $\phantom{0}96.5\phantom{0}$ & $\phantom{0}96.2\phantom{0}$ & $\phantom{0}73.4\phantom{0}$ & $\phantom{0}96.2\phantom{0}$ & $\phantom{0}96.5\phantom{0}$ & $\phantom{0}96.1\phantom{0}$ \\
 & \nopagebreak $\;J=500$  & $\phantom{0}\phantom{-}1.5\phantom{0}$ & $\phantom{0}\phantom{-}3.7\phantom{0}$ & ${-}28.2\phantom{0}$ & $\phantom{0}\phantom{-}1.1\phantom{0}$ & $\phantom{0}\phantom{-}1.9\phantom{0}$ & $\phantom{0}{-}2.8\phantom{0}$ & $\phantom{0}0.32\phantom{0}$ & $\phantom{0}0.44\phantom{0}$ & $\phantom{0}0.52\phantom{0}$ & $\phantom{0}0.38\phantom{0}$ & $\phantom{0}0.38\phantom{0}$ & $\phantom{0}0.35\phantom{0}$ & $\phantom{0}95.2\phantom{0}$ & $\phantom{0}95.0\phantom{0}$ & $\phantom{0}53.2\phantom{0}$ & $\phantom{0}95.3\phantom{0}$ & $\phantom{0}94.9\phantom{0}$ & $\phantom{0}94.8\phantom{0}$ \\
 & \nopagebreak $\;J=1000$  & $\phantom{0}\phantom{-}1.5\phantom{0}$ & $\phantom{0}\phantom{-}2.5\phantom{0}$ & ${-}27.9\phantom{0}$ & $\phantom{0}\phantom{-}1.5\phantom{0}$ & $\phantom{0}\phantom{-}1.9\phantom{0}$ & $\phantom{0}{-}0.7\phantom{0}$ & $\phantom{0}0.23\phantom{0}$ & $\phantom{0}0.30\phantom{0}$ & $\phantom{0}0.48\phantom{0}$ & $\phantom{0}0.27\phantom{0}$ & $\phantom{0}0.27\phantom{0}$ & $\phantom{0}0.25\phantom{0}$ & $\phantom{0}95.3\phantom{0}$ & $\phantom{0}95.9\phantom{0}$ & $\phantom{0}35.0\phantom{0}$ & $\phantom{0}95.6\phantom{0}$ & $\phantom{0}95.7\phantom{0}$ & $\phantom{0}95.0\phantom{0}$ \\
\multicolumn{4}{l}{$n=20$} \\  & \nopagebreak $\;J=30$  & $\phantom{0}\phantom{-}6.9\phantom{0}$ & $\phantom{-}15.0\phantom{0}$ & ${-}32.6\phantom{0}$ & $\phantom{0}\phantom{-}5.2\phantom{0}$ & $\phantom{0}\phantom{-}9.8\phantom{0}$ & ${-}16.1\phantom{0}$ & $\phantom{0}0.84\phantom{0}$ & $\phantom{0}1.49\phantom{0}$ & $\phantom{0}0.87\phantom{0}$ & $\phantom{0}1.08\phantom{0}$ & $\phantom{0}1.14\phantom{0}$ & $\phantom{0}0.88\phantom{0}$ & $\phantom{0}92.9\phantom{0}$ & $\phantom{0}89.9\phantom{0}$ & $\phantom{0}72.3\phantom{0}$ & $\phantom{0}93.9\phantom{0}$ & $\phantom{0}93.2\phantom{0}$ & $\phantom{0}93.9\phantom{0}$ \\
 & \nopagebreak $\;J=50$  & $\phantom{0}\phantom{-}4.4\phantom{0}$ & $\phantom{0}\phantom{-}9.4\phantom{0}$ & ${-}32.7\phantom{0}$ & $\phantom{0}\phantom{-}4.1\phantom{0}$ & $\phantom{0}\phantom{-}6.1\phantom{0}$ & ${-}11.7\phantom{0}$ & $\phantom{0}0.57\phantom{0}$ & $\phantom{0}0.91\phantom{0}$ & $\phantom{0}0.71\phantom{0}$ & $\phantom{0}0.76\phantom{0}$ & $\phantom{0}0.80\phantom{0}$ & $\phantom{0}0.65\phantom{0}$ & $\phantom{0}92.9\phantom{0}$ & $\phantom{0}92.3\phantom{0}$ & $\phantom{0}68.9\phantom{0}$ & $\phantom{0}94.1\phantom{0}$ & $\phantom{0}93.0\phantom{0}$ & $\phantom{0}95.0\phantom{0}$ \\
 & \nopagebreak $\;J=100$  & $\phantom{0}\phantom{-}3.2\phantom{0}$ & $\phantom{0}\phantom{-}5.1\phantom{0}$ & ${-}33.1\phantom{0}$ & $\phantom{0}\phantom{-}2.9\phantom{0}$ & $\phantom{0}\phantom{-}3.5\phantom{0}$ & $\phantom{0}{-}6.3\phantom{0}$ & $\phantom{0}0.38\phantom{0}$ & $\phantom{0}0.54\phantom{0}$ & $\phantom{0}0.62\phantom{0}$ & $\phantom{0}0.50\phantom{0}$ & $\phantom{0}0.51\phantom{0}$ & $\phantom{0}0.45\phantom{0}$ & $\phantom{0}94.0\phantom{0}$ & $\phantom{0}93.3\phantom{0}$ & $\phantom{0}54.9\phantom{0}$ & $\phantom{0}94.2\phantom{0}$ & $\phantom{0}93.5\phantom{0}$ & $\phantom{0}95.3\phantom{0}$ \\
 & \nopagebreak $\;J=200$  & $\phantom{0}\phantom{-}0.1\phantom{0}$ & $\phantom{0}\phantom{-}1.6\phantom{0}$ & ${-}34.1\phantom{0}$ & $\phantom{0}\phantom{-}0.8\phantom{0}$ & $\phantom{0}\phantom{-}1.3\phantom{0}$ & $\phantom{0}{-}4.3\phantom{0}$ & $\phantom{0}0.26\phantom{0}$ & $\phantom{0}0.35\phantom{0}$ & $\phantom{0}0.59\phantom{0}$ & $\phantom{0}0.34\phantom{0}$ & $\phantom{0}0.34\phantom{0}$ & $\phantom{0}0.32\phantom{0}$ & $\phantom{0}94.6\phantom{0}$ & $\phantom{0}94.3\phantom{0}$ & $\phantom{0}32.4\phantom{0}$ & $\phantom{0}94.3\phantom{0}$ & $\phantom{0}94.0\phantom{0}$ & $\phantom{0}96.1\phantom{0}$ \\
 & \nopagebreak $\;J=500$  & $\phantom{0}\phantom{-}0.1\phantom{0}$ & $\phantom{0}\phantom{-}0.6\phantom{0}$ & ${-}34.0\phantom{0}$ & $\phantom{0}\phantom{-}0.4\phantom{0}$ & $\phantom{0}\phantom{-}0.7\phantom{0}$ & $\phantom{0}{-}1.6\phantom{0}$ & $\phantom{0}0.16\phantom{0}$ & $\phantom{0}0.21\phantom{0}$ & $\phantom{0}0.56\phantom{0}$ & $\phantom{0}0.21\phantom{0}$ & $\phantom{0}0.21\phantom{0}$ & $\phantom{0}0.20\phantom{0}$ & $\phantom{0}95.2\phantom{0}$ & $\phantom{0}96.0\phantom{0}$ & $\phantom{0}\phantom{0}5.7\phantom{0}$ & $\phantom{0}94.5\phantom{0}$ & $\phantom{0}95.1\phantom{0}$ & $\phantom{0}94.5\phantom{0}$ \\
 & \nopagebreak $\;J=1000$  & $\phantom{0}\phantom{-}0.1\phantom{0}$ & $\phantom{0}\phantom{-}0.1\phantom{0}$ & ${-}34.2\phantom{0}$ & $\phantom{0}\phantom{-}0.0\phantom{0}$ & $\phantom{0}\phantom{-}0.1\phantom{0}$ & $\phantom{0}{-}1.1\phantom{0}$ & $\phantom{0}0.11\phantom{0}$ & $\phantom{0}0.14\phantom{0}$ & $\phantom{0}0.55\phantom{0}$ & $\phantom{0}0.14\phantom{0}$ & $\phantom{0}0.14\phantom{0}$ & $\phantom{0}0.14\phantom{0}$ & $\phantom{0}95.3\phantom{0}$ & $\phantom{0}95.8\phantom{0}$ & $\phantom{0}\phantom{0}0.1\phantom{0}$ & $\phantom{0}95.4\phantom{0}$ & $\phantom{0}94.6\phantom{0}$ & $\phantom{0}95.0\phantom{0}$ \\
[0.5ex]\hline\\[-1.6ex] 
& & \multicolumn{18}{c}{Moderate intraclass correlation $(\rho_{Iy}=.30)$} \\[0.6ex]\hline\\[-1.8ex]
\multicolumn{4}{l}{$n=5$} \\  & \nopagebreak $\;J=30$  & $\phantom{0}\phantom{-}7.3\phantom{0}$ & $\phantom{-}14.4\phantom{0}$ & ${-}19.0\phantom{0}$ & $\phantom{0}\phantom{-}6.2\phantom{0}$ & $\phantom{-}10.5\phantom{0}$ & $\phantom{0}{-}6.0\phantom{0}$ & $\phantom{0}0.53\phantom{0}$ & $\phantom{0}0.82\phantom{0}$ & $\phantom{0}0.50\phantom{0}$ & $\phantom{0}0.63\phantom{0}$ & $\phantom{0}0.67\phantom{0}$ & $\phantom{0}0.55\phantom{0}$ & $\phantom{0}93.6\phantom{0}$ & $\phantom{0}90.9\phantom{0}$ & $\phantom{0}82.4\phantom{0}$ & $\phantom{0}92.5\phantom{0}$ & $\phantom{0}93.1\phantom{0}$ & $\phantom{0}95.3\phantom{0}$ \\
 & \nopagebreak $\;J=50$  & $\phantom{0}\phantom{-}4.1\phantom{0}$ & $\phantom{-}11.7\phantom{0}$ & ${-}20.9\phantom{0}$ & $\phantom{0}\phantom{-}3.7\phantom{0}$ & $\phantom{0}\phantom{-}6.3\phantom{0}$ & $\phantom{0}{-}4.8\phantom{0}$ & $\phantom{0}0.34\phantom{0}$ & $\phantom{0}1.11\phantom{0}$ & $\phantom{0}0.38\phantom{0}$ & $\phantom{0}0.45\phantom{0}$ & $\phantom{0}0.47\phantom{0}$ & $\phantom{0}0.39\phantom{0}$ & $\phantom{0}93.5\phantom{0}$ & $\phantom{0}92.7\phantom{0}$ & $\phantom{0}81.7\phantom{0}$ & $\phantom{0}93.9\phantom{0}$ & $\phantom{0}93.3\phantom{0}$ & $\phantom{0}94.9\phantom{0}$ \\
 & \nopagebreak $\;J=100$  & $\phantom{0}\phantom{-}2.6\phantom{0}$ & $\phantom{0}\phantom{-}4.5\phantom{0}$ & ${-}21.4\phantom{0}$ & $\phantom{0}\phantom{-}2.6\phantom{0}$ & $\phantom{0}\phantom{-}3.4\phantom{0}$ & $\phantom{0}{-}1.3\phantom{0}$ & $\phantom{0}0.23\phantom{0}$ & $\phantom{0}0.32\phantom{0}$ & $\phantom{0}0.30\phantom{0}$ & $\phantom{0}0.30\phantom{0}$ & $\phantom{0}0.31\phantom{0}$ & $\phantom{0}0.28\phantom{0}$ & $\phantom{0}94.2\phantom{0}$ & $\phantom{0}93.5\phantom{0}$ & $\phantom{0}75.8\phantom{0}$ & $\phantom{0}94.0\phantom{0}$ & $\phantom{0}94.0\phantom{0}$ & $\phantom{0}94.6\phantom{0}$ \\
 & \nopagebreak $\;J=200$  & $\phantom{0}\phantom{-}0.1\phantom{0}$ & $\phantom{0}\phantom{-}1.8\phantom{0}$ & ${-}21.9\phantom{0}$ & $\phantom{0}\phantom{-}1.0\phantom{0}$ & $\phantom{0}\phantom{-}1.3\phantom{0}$ & $\phantom{0}{-}0.5\phantom{0}$ & $\phantom{0}0.15\phantom{0}$ & $\phantom{0}0.20\phantom{0}$ & $\phantom{0}0.25\phantom{0}$ & $\phantom{0}0.19\phantom{0}$ & $\phantom{0}0.19\phantom{0}$ & $\phantom{0}0.19\phantom{0}$ & $\phantom{0}94.7\phantom{0}$ & $\phantom{0}95.2\phantom{0}$ & $\phantom{0}65.4\phantom{0}$ & $\phantom{0}94.6\phantom{0}$ & $\phantom{0}95.3\phantom{0}$ & $\phantom{0}95.2\phantom{0}$ \\
 & \nopagebreak $\;J=500$  & $\phantom{0}\phantom{-}0.3\phantom{0}$ & $\phantom{0}\phantom{-}0.5\phantom{0}$ & ${-}22.1\phantom{0}$ & $\phantom{0}\phantom{-}0.3\phantom{0}$ & $\phantom{0}\phantom{-}0.4\phantom{0}$ & $\phantom{0}{-}0.5\phantom{0}$ & $\phantom{0}0.10\phantom{0}$ & $\phantom{0}0.12\phantom{0}$ & $\phantom{0}0.22\phantom{0}$ & $\phantom{0}0.12\phantom{0}$ & $\phantom{0}0.12\phantom{0}$ & $\phantom{0}0.12\phantom{0}$ & $\phantom{0}95.6\phantom{0}$ & $\phantom{0}94.6\phantom{0}$ & $\phantom{0}38.8\phantom{0}$ & $\phantom{0}95.0\phantom{0}$ & $\phantom{0}94.4\phantom{0}$ & $\phantom{0}94.6\phantom{0}$ \\
 & \nopagebreak $\;J=1000$  & $\phantom{0}\phantom{-}0.1\phantom{0}$ & $\phantom{0}\phantom{-}0.1\phantom{0}$ & ${-}22.5\phantom{0}$ & $\phantom{0}{-}0.1\phantom{0}$ & $\phantom{0}\phantom{-}0.0\phantom{0}$ & $\phantom{0}{-}0.3\phantom{0}$ & $\phantom{0}0.06\phantom{0}$ & $\phantom{0}0.09\phantom{0}$ & $\phantom{0}0.22\phantom{0}$ & $\phantom{0}0.08\phantom{0}$ & $\phantom{0}0.08\phantom{0}$ & $\phantom{0}0.08\phantom{0}$ & $\phantom{0}95.4\phantom{0}$ & $\phantom{0}95.7\phantom{0}$ & $\phantom{0}10.0\phantom{0}$ & $\phantom{0}95.5\phantom{0}$ & $\phantom{0}95.1\phantom{0}$ & $\phantom{0}95.5\phantom{0}$ \\
\multicolumn{4}{l}{$n=20$} \\  & \nopagebreak $\;J=30$  & $\phantom{0}\phantom{-}0.1\phantom{0}$ & $\phantom{0}\phantom{-}2.0\phantom{0}$ & ${-}28.8\phantom{0}$ & $\phantom{0}\phantom{-}0.7\phantom{0}$ & $\phantom{0}\phantom{-}1.5\phantom{0}$ & ${-}10.2\phantom{0}$ & $\phantom{0}0.34\phantom{0}$ & $\phantom{0}0.48\phantom{0}$ & $\phantom{0}0.42\phantom{0}$ & $\phantom{0}0.48\phantom{0}$ & $\phantom{0}0.48\phantom{0}$ & $\phantom{0}0.42\phantom{0}$ & $\phantom{0}91.0\phantom{0}$ & $\phantom{0}87.3\phantom{0}$ & $\phantom{0}69.8\phantom{0}$ & $\phantom{0}92.0\phantom{0}$ & $\phantom{0}92.1\phantom{0}$ & $\phantom{0}93.8\phantom{0}$ \\
 & \nopagebreak $\;J=50$  & $\phantom{0}\phantom{-}1.4\phantom{0}$ & $\phantom{0}\phantom{-}3.2\phantom{0}$ & ${-}26.6\phantom{0}$ & $\phantom{0}\phantom{-}2.2\phantom{0}$ & $\phantom{0}\phantom{-}3.1\phantom{0}$ & $\phantom{0}{-}4.2\phantom{0}$ & $\phantom{0}0.24\phantom{0}$ & $\phantom{0}0.33\phantom{0}$ & $\phantom{0}0.34\phantom{0}$ & $\phantom{0}0.33\phantom{0}$ & $\phantom{0}0.33\phantom{0}$ & $\phantom{0}0.30\phantom{0}$ & $\phantom{0}92.3\phantom{0}$ & $\phantom{0}91.8\phantom{0}$ & $\phantom{0}69.5\phantom{0}$ & $\phantom{0}94.7\phantom{0}$ & $\phantom{0}94.1\phantom{0}$ & $\phantom{0}95.0\phantom{0}$ \\
 & \nopagebreak $\;J=100$  & $\phantom{0}\phantom{-}0.4\phantom{0}$ & $\phantom{0}\phantom{-}0.6\phantom{0}$ & ${-}26.8\phantom{0}$ & $\phantom{0}\phantom{-}0.4\phantom{0}$ & $\phantom{0}\phantom{-}0.5\phantom{0}$ & $\phantom{0}{-}2.5\phantom{0}$ & $\phantom{0}0.17\phantom{0}$ & $\phantom{0}0.22\phantom{0}$ & $\phantom{0}0.30\phantom{0}$ & $\phantom{0}0.23\phantom{0}$ & $\phantom{0}0.22\phantom{0}$ & $\phantom{0}0.22\phantom{0}$ & $\phantom{0}95.3\phantom{0}$ & $\phantom{0}93.9\phantom{0}$ & $\phantom{0}56.4\phantom{0}$ & $\phantom{0}94.3\phantom{0}$ & $\phantom{0}94.7\phantom{0}$ & $\phantom{0}94.5\phantom{0}$ \\
 & \nopagebreak $\;J=200$  & $\phantom{0}{-}0.4\phantom{0}$ & $\phantom{0}\phantom{-}0.3\phantom{0}$ & ${-}27.0\phantom{0}$ & $\phantom{0}\phantom{-}0.2\phantom{0}$ & $\phantom{0}\phantom{-}0.3\phantom{0}$ & $\phantom{0}{-}1.4\phantom{0}$ & $\phantom{0}0.12\phantom{0}$ & $\phantom{0}0.16\phantom{0}$ & $\phantom{0}0.28\phantom{0}$ & $\phantom{0}0.16\phantom{0}$ & $\phantom{0}0.16\phantom{0}$ & $\phantom{0}0.16\phantom{0}$ & $\phantom{0}93.8\phantom{0}$ & $\phantom{0}93.8\phantom{0}$ & $\phantom{0}38.4\phantom{0}$ & $\phantom{0}93.8\phantom{0}$ & $\phantom{0}94.3\phantom{0}$ & $\phantom{0}94.2\phantom{0}$ \\
 & \nopagebreak $\;J=500$  & $\phantom{0}\phantom{-}0.2\phantom{0}$ & $\phantom{0}\phantom{-}0.3\phantom{0}$ & ${-}26.5\phantom{0}$ & $\phantom{0}\phantom{-}0.2\phantom{0}$ & $\phantom{0}\phantom{-}0.2\phantom{0}$ & $\phantom{0}{-}0.4\phantom{0}$ & $\phantom{0}0.07\phantom{0}$ & $\phantom{0}0.10\phantom{0}$ & $\phantom{0}0.25\phantom{0}$ & $\phantom{0}0.10\phantom{0}$ & $\phantom{0}0.10\phantom{0}$ & $\phantom{0}0.10\phantom{0}$ & $\phantom{0}95.3\phantom{0}$ & $\phantom{0}95.6\phantom{0}$ & $\phantom{0}\phantom{0}8.3\phantom{0}$ & $\phantom{0}95.7\phantom{0}$ & $\phantom{0}95.6\phantom{0}$ & $\phantom{0}95.3\phantom{0}$ \\
 & \nopagebreak $\;J=1000$  & $\phantom{0}{-}0.0\phantom{0}$ & $\phantom{0}{-}0.2\phantom{0}$ & ${-}26.8\phantom{0}$ & $\phantom{0}{-}0.1\phantom{0}$ & $\phantom{0}{-}0.1\phantom{0}$ & $\phantom{0}{-}0.5\phantom{0}$ & $\phantom{0}0.05\phantom{0}$ & $\phantom{0}0.07\phantom{0}$ & $\phantom{0}0.25\phantom{0}$ & $\phantom{0}0.07\phantom{0}$ & $\phantom{0}0.07\phantom{0}$ & $\phantom{0}0.07\phantom{0}$ & $\phantom{0}95.4\phantom{0}$ & $\phantom{0}95.9\phantom{0}$ & $\phantom{0}\phantom{0}0.3\phantom{0}$ & $\phantom{0}95.3\phantom{0}$ & $\phantom{0}95.1\phantom{0}$ & $\phantom{0}95.4\phantom{0}$ \\
[0.5ex]\hline\\[-1.6ex] 
\end{tabular}
\begin{tablenotes}[para,flushleft]{\footnotesize \textit{Note.} $n$ = cluster size; $J$ = number of clusters; CD = complete data sets; LD = listwise deletion; FCS-SL = single-level FCS; FCS-MAN = two-level FCS with manifest cluster means; FCS-LAT = two-level FCS with latent cluster means; JM = joint modeling.}\end{tablenotes}
\end{threeparttable}
\end{sidewaystable}
\begin{sidewaystable}
\begin{threeparttable}
\setlength{\tabcolsep}{1.2pt}
\renewcommand{\arraystretch}{0.95}
\footnotesize
\caption{\small Study 1: Bias (in \%), RMSE, and Coverage of the 95\% Confidence Interval for the Regression Coefficient of $z$ on $y$ ($\hat\beta_{zy}$) With 40\% Missing Data (MAR, $\lambda=0.5$)}
\begin{tabular}{llcccccccccccccccccc}
\hline\\[-1.8ex]
& & \multicolumn{6}{c}{Bias (\%)} & \multicolumn{6}{c}{RMSE} & \multicolumn{6}{c}{Coverage (\%)} \\ \cmidrule(r){3-8}\cmidrule(r){9-14}\cmidrule(r){15-20}
 &  & CD & LD & \makecell{FCS-\\SL} & \makecell{FCS-\\MAN} & \makecell{FCS-\\LAT} & JM & CD & LD & \makecell{FCS-\\SL} & \makecell{FCS-\\MAN} & \makecell{FCS-\\LAT} & JM & CD & LD & \makecell{FCS-\\SL} & \makecell{FCS-\\MAN} & \makecell{FCS-\\LAT} & \multicolumn{1}{c}{JM} \\ 
[0.4ex]\hline\\[-1.8ex]
& & \multicolumn{18}{c}{Small intraclass correlation $(\rho_{Iy}=.10)$} \\[0.6ex]\hline\\[-1.8ex]
\multicolumn{4}{l}{$n=5$} \\  & \nopagebreak $\;J=30$  & $\phantom{-}35.9\phantom{0}$ & $\phantom{-}70.1\phantom{0}$ & $\phantom{0}{-}9.7\phantom{0}$ & $\phantom{-}12.7\phantom{0}$ & $\phantom{-}20.8\phantom{0}$ & $\phantom{0}\phantom{-}0.6\phantom{0}$ & $\phantom{0}2.53\phantom{0}$ & $\phantom{0}4.50\phantom{0}$ & $\phantom{0}1.92\phantom{0}$ & $\phantom{0}2.10\phantom{0}$ & $\phantom{0}2.09\phantom{0}$ & $\phantom{0}2.01\phantom{0}$ & $\phantom{0}89.4\phantom{0}$ & $\phantom{0}83.7\phantom{0}$ & $\phantom{0}81.0\phantom{0}$ & $\phantom{0}94.2\phantom{0}$ & $\phantom{0}95.1\phantom{0}$ & $\phantom{0}94.8\phantom{0}$ \\
 & \nopagebreak $\;J=50$  & $\phantom{-}38.3\phantom{0}$ & $\phantom{-}96.8\phantom{0}$ & $\phantom{0}{-}9.0\phantom{0}$ & $\phantom{-}23.6\phantom{0}$ & $\phantom{-}31.4\phantom{0}$ & $\phantom{0}\phantom{-}5.7\phantom{0}$ & $\phantom{0}2.11\phantom{0}$ & $\phantom{0}4.14\phantom{0}$ & $\phantom{0}1.49\phantom{0}$ & $\phantom{0}1.71\phantom{0}$ & $\phantom{0}1.82\phantom{0}$ & $\phantom{0}1.63\phantom{0}$ & $\phantom{0}91.4\phantom{0}$ & $\phantom{0}88.0\phantom{0}$ & $\phantom{0}78.3\phantom{0}$ & $\phantom{0}94.1\phantom{0}$ & $\phantom{0}94.5\phantom{0}$ & $\phantom{0}94.1\phantom{0}$ \\
 & \nopagebreak $\;J=100$  & $\phantom{-}20.9\phantom{0}$ & $\phantom{-}93.1\phantom{0}$ & ${-}21.3\phantom{0}$ & $\phantom{-}19.3\phantom{0}$ & $\phantom{-}24.8\phantom{0}$ & $\phantom{0}\phantom{-}3.1\phantom{0}$ & $\phantom{0}1.16\phantom{0}$ & $\phantom{0}2.83\phantom{0}$ & $\phantom{0}0.97\phantom{0}$ & $\phantom{0}1.20\phantom{0}$ & $\phantom{0}1.24\phantom{0}$ & $\phantom{0}1.09\phantom{0}$ & $\phantom{0}93.7\phantom{0}$ & $\phantom{0}91.3\phantom{0}$ & $\phantom{0}72.9\phantom{0}$ & $\phantom{0}94.5\phantom{0}$ & $\phantom{0}94.9\phantom{0}$ & $\phantom{0}94.0\phantom{0}$ \\
 & \nopagebreak $\;J=200$  & $\phantom{0}\phantom{-}7.6\phantom{0}$ & $\phantom{-}81.4\phantom{0}$ & ${-}31.4\phantom{0}$ & $\phantom{0}\phantom{-}7.7\phantom{0}$ & $\phantom{-}12.1\phantom{0}$ & $\phantom{0}{-}2.8\phantom{0}$ & $\phantom{0}0.65\phantom{0}$ & $\phantom{0}2.44\phantom{0}$ & $\phantom{0}0.72\phantom{0}$ & $\phantom{0}0.76\phantom{0}$ & $\phantom{0}0.82\phantom{0}$ & $\phantom{0}0.68\phantom{0}$ & $\phantom{0}95.2\phantom{0}$ & $\phantom{0}93.9\phantom{0}$ & $\phantom{0}61.3\phantom{0}$ & $\phantom{0}95.4\phantom{0}$ & $\phantom{0}94.6\phantom{0}$ & $\phantom{0}93.7\phantom{0}$ \\
 & \nopagebreak $\;J=500$  & $\phantom{0}\phantom{-}3.8\phantom{0}$ & $\phantom{-}64.2\phantom{0}$ & ${-}34.8\phantom{0}$ & $\phantom{0}\phantom{-}3.8\phantom{0}$ & $\phantom{0}\phantom{-}5.1\phantom{0}$ & $\phantom{0}{-}2.1\phantom{0}$ & $\phantom{0}0.36\phantom{0}$ & $\phantom{0}1.49\phantom{0}$ & $\phantom{0}0.61\phantom{0}$ & $\phantom{0}0.43\phantom{0}$ & $\phantom{0}0.45\phantom{0}$ & $\phantom{0}0.38\phantom{0}$ & $\phantom{0}96.6\phantom{0}$ & $\phantom{0}97.5\phantom{0}$ & $\phantom{0}39.6\phantom{0}$ & $\phantom{0}95.7\phantom{0}$ & $\phantom{0}95.0\phantom{0}$ & $\phantom{0}95.6\phantom{0}$ \\
 & \nopagebreak $\;J=1000$  & $\phantom{0}\phantom{-}1.2\phantom{0}$ & $\phantom{-}53.4\phantom{0}$ & ${-}36.0\phantom{0}$ & $\phantom{0}\phantom{-}1.0\phantom{0}$ & $\phantom{0}\phantom{-}1.7\phantom{0}$ & $\phantom{0}{-}1.8\phantom{0}$ & $\phantom{0}0.22\phantom{0}$ & $\phantom{0}1.10\phantom{0}$ & $\phantom{0}0.59\phantom{0}$ & $\phantom{0}0.27\phantom{0}$ & $\phantom{0}0.27\phantom{0}$ & $\phantom{0}0.25\phantom{0}$ & $\phantom{0}94.6\phantom{0}$ & $\phantom{0}97.2\phantom{0}$ & $\phantom{0}14.6\phantom{0}$ & $\phantom{0}95.8\phantom{0}$ & $\phantom{0}95.7\phantom{0}$ & $\phantom{0}96.0\phantom{0}$ \\
\multicolumn{4}{l}{$n=20$} \\  & \nopagebreak $\;J=30$  & $\phantom{0}\phantom{-}9.4\phantom{0}$ & $\phantom{-}31.6\phantom{0}$ & ${-}38.0\phantom{0}$ & $\phantom{0}\phantom{-}8.0\phantom{0}$ & $\phantom{-}13.6\phantom{0}$ & ${-}18.4\phantom{0}$ & $\phantom{0}0.94\phantom{0}$ & $\phantom{0}2.10\phantom{0}$ & $\phantom{0}0.96\phantom{0}$ & $\phantom{0}1.20\phantom{0}$ & $\phantom{0}1.26\phantom{0}$ & $\phantom{0}0.94\phantom{0}$ & $\phantom{0}91.2\phantom{0}$ & $\phantom{0}90.9\phantom{0}$ & $\phantom{0}67.6\phantom{0}$ & $\phantom{0}93.4\phantom{0}$ & $\phantom{0}92.2\phantom{0}$ & $\phantom{0}94.8\phantom{0}$ \\
 & \nopagebreak $\;J=50$  & $\phantom{0}\phantom{-}4.1\phantom{0}$ & $\phantom{-}21.1\phantom{0}$ & ${-}40.8\phantom{0}$ & $\phantom{0}\phantom{-}3.6\phantom{0}$ & $\phantom{0}\phantom{-}6.1\phantom{0}$ & ${-}16.5\phantom{0}$ & $\phantom{0}0.57\phantom{0}$ & $\phantom{0}1.14\phantom{0}$ & $\phantom{0}0.80\phantom{0}$ & $\phantom{0}0.79\phantom{0}$ & $\phantom{0}0.84\phantom{0}$ & $\phantom{0}0.68\phantom{0}$ & $\phantom{0}93.7\phantom{0}$ & $\phantom{0}93.2\phantom{0}$ & $\phantom{0}59.5\phantom{0}$ & $\phantom{0}94.5\phantom{0}$ & $\phantom{0}92.9\phantom{0}$ & $\phantom{0}94.8\phantom{0}$ \\
 & \nopagebreak $\;J=100$  & $\phantom{0}\phantom{-}1.2\phantom{0}$ & $\phantom{-}13.4\phantom{0}$ & ${-}41.5\phantom{0}$ & $\phantom{0}\phantom{-}2.1\phantom{0}$ & $\phantom{0}\phantom{-}2.9\phantom{0}$ & $\phantom{0}{-}9.9\phantom{0}$ & $\phantom{0}0.39\phantom{0}$ & $\phantom{0}0.66\phantom{0}$ & $\phantom{0}0.73\phantom{0}$ & $\phantom{0}0.53\phantom{0}$ & $\phantom{0}0.54\phantom{0}$ & $\phantom{0}0.49\phantom{0}$ & $\phantom{0}93.3\phantom{0}$ & $\phantom{0}94.3\phantom{0}$ & $\phantom{0}40.5\phantom{0}$ & $\phantom{0}94.0\phantom{0}$ & $\phantom{0}93.5\phantom{0}$ & $\phantom{0}95.3\phantom{0}$ \\
 & \nopagebreak $\;J=200$  & $\phantom{0}\phantom{-}1.2\phantom{0}$ & $\phantom{-}11.0\phantom{0}$ & ${-}41.5\phantom{0}$ & $\phantom{0}\phantom{-}1.6\phantom{0}$ & $\phantom{0}\phantom{-}1.9\phantom{0}$ & $\phantom{0}{-}5.5\phantom{0}$ & $\phantom{0}0.27\phantom{0}$ & $\phantom{0}0.44\phantom{0}$ & $\phantom{0}0.69\phantom{0}$ & $\phantom{0}0.36\phantom{0}$ & $\phantom{0}0.36\phantom{0}$ & $\phantom{0}0.34\phantom{0}$ & $\phantom{0}94.4\phantom{0}$ & $\phantom{0}94.8\phantom{0}$ & $\phantom{0}18.4\phantom{0}$ & $\phantom{0}95.0\phantom{0}$ & $\phantom{0}93.6\phantom{0}$ & $\phantom{0}95.1\phantom{0}$ \\
 & \nopagebreak $\;J=500$  & $\phantom{0}\phantom{-}0.6\phantom{0}$ & $\phantom{0}\phantom{-}9.2\phantom{0}$ & ${-}42.4\phantom{0}$ & $\phantom{0}\phantom{-}0.4\phantom{0}$ & $\phantom{0}\phantom{-}0.5\phantom{0}$ & $\phantom{0}{-}2.4\phantom{0}$ & $\phantom{0}0.17\phantom{0}$ & $\phantom{0}0.30\phantom{0}$ & $\phantom{0}0.68\phantom{0}$ & $\phantom{0}0.23\phantom{0}$ & $\phantom{0}0.23\phantom{0}$ & $\phantom{0}0.23\phantom{0}$ & $\phantom{0}93.8\phantom{0}$ & $\phantom{0}91.8\phantom{0}$ & $\phantom{0}\phantom{0}0.3\phantom{0}$ & $\phantom{0}94.8\phantom{0}$ & $\phantom{0}93.7\phantom{0}$ & $\phantom{0}95.5\phantom{0}$ \\
 & \nopagebreak $\;J=1000$  & $\phantom{0}{-}0.1\phantom{0}$ & $\phantom{0}\phantom{-}8.6\phantom{0}$ & ${-}42.5\phantom{0}$ & $\phantom{0}\phantom{-}0.0\phantom{0}$ & $\phantom{0}\phantom{-}0.1\phantom{0}$ & $\phantom{0}{-}1.3\phantom{0}$ & $\phantom{0}0.11\phantom{0}$ & $\phantom{0}0.22\phantom{0}$ & $\phantom{0}0.68\phantom{0}$ & $\phantom{0}0.16\phantom{0}$ & $\phantom{0}0.16\phantom{0}$ & $\phantom{0}0.16\phantom{0}$ & $\phantom{0}95.0\phantom{0}$ & $\phantom{0}89.2\phantom{0}$ & $\phantom{0}\phantom{0}0.0\phantom{0}$ & $\phantom{0}95.0\phantom{0}$ & $\phantom{0}94.5\phantom{0}$ & $\phantom{0}94.9\phantom{0}$ \\
[0.5ex]\hline\\[-1.6ex] 
& & \multicolumn{18}{c}{Moderate intraclass correlation $(\rho_{Iy}=.30)$} \\[0.6ex]\hline\\[-1.8ex]
\multicolumn{4}{l}{$n=5$} \\  & \nopagebreak $\;J=30$  & $\phantom{-}12.1\phantom{0}$ & $\phantom{-}41.5\phantom{0}$ & ${-}24.1\phantom{0}$ & $\phantom{0}\phantom{-}9.8\phantom{0}$ & $\phantom{-}15.6\phantom{0}$ & $\phantom{0}{-}7.1\phantom{0}$ & $\phantom{0}0.56\phantom{0}$ & $\phantom{0}1.68\phantom{0}$ & $\phantom{0}0.54\phantom{0}$ & $\phantom{0}0.71\phantom{0}$ & $\phantom{0}0.77\phantom{0}$ & $\phantom{0}0.59\phantom{0}$ & $\phantom{0}92.7\phantom{0}$ & $\phantom{0}90.7\phantom{0}$ & $\phantom{0}78.9\phantom{0}$ & $\phantom{0}93.4\phantom{0}$ & $\phantom{0}92.6\phantom{0}$ & $\phantom{0}95.1\phantom{0}$ \\
 & \nopagebreak $\;J=50$  & $\phantom{0}\phantom{-}2.8\phantom{0}$ & $\phantom{-}23.5\phantom{0}$ & ${-}29.0\phantom{0}$ & $\phantom{0}\phantom{-}2.9\phantom{0}$ & $\phantom{0}\phantom{-}5.9\phantom{0}$ & $\phantom{0}{-}8.3\phantom{0}$ & $\phantom{0}0.35\phantom{0}$ & $\phantom{0}0.78\phantom{0}$ & $\phantom{0}0.42\phantom{0}$ & $\phantom{0}0.49\phantom{0}$ & $\phantom{0}0.52\phantom{0}$ & $\phantom{0}0.43\phantom{0}$ & $\phantom{0}93.6\phantom{0}$ & $\phantom{0}94.5\phantom{0}$ & $\phantom{0}75.5\phantom{0}$ & $\phantom{0}93.5\phantom{0}$ & $\phantom{0}93.7\phantom{0}$ & $\phantom{0}95.1\phantom{0}$ \\
 & \nopagebreak $\;J=100$  & $\phantom{0}\phantom{-}1.6\phantom{0}$ & $\phantom{-}14.4\phantom{0}$ & ${-}29.2\phantom{0}$ & $\phantom{0}\phantom{-}1.2\phantom{0}$ & $\phantom{0}\phantom{-}2.6\phantom{0}$ & $\phantom{0}{-}4.4\phantom{0}$ & $\phantom{0}0.24\phantom{0}$ & $\phantom{0}0.42\phantom{0}$ & $\phantom{0}0.35\phantom{0}$ & $\phantom{0}0.33\phantom{0}$ & $\phantom{0}0.33\phantom{0}$ & $\phantom{0}0.30\phantom{0}$ & $\phantom{0}93.8\phantom{0}$ & $\phantom{0}94.9\phantom{0}$ & $\phantom{0}66.7\phantom{0}$ & $\phantom{0}93.2\phantom{0}$ & $\phantom{0}94.3\phantom{0}$ & $\phantom{0}94.4\phantom{0}$ \\
 & \nopagebreak $\;J=200$  & $\phantom{0}\phantom{-}1.4\phantom{0}$ & $\phantom{-}11.9\phantom{0}$ & ${-}27.8\phantom{0}$ & $\phantom{0}\phantom{-}2.2\phantom{0}$ & $\phantom{0}\phantom{-}2.1\phantom{0}$ & $\phantom{0}{-}0.5\phantom{0}$ & $\phantom{0}0.16\phantom{0}$ & $\phantom{0}0.27\phantom{0}$ & $\phantom{0}0.30\phantom{0}$ & $\phantom{0}0.21\phantom{0}$ & $\phantom{0}0.22\phantom{0}$ & $\phantom{0}0.21\phantom{0}$ & $\phantom{0}93.6\phantom{0}$ & $\phantom{0}94.8\phantom{0}$ & $\phantom{0}52.6\phantom{0}$ & $\phantom{0}94.8\phantom{0}$ & $\phantom{0}93.8\phantom{0}$ & $\phantom{0}95.3\phantom{0}$ \\
 & \nopagebreak $\;J=500$  & $\phantom{0}\phantom{-}0.2\phantom{0}$ & $\phantom{0}\phantom{-}9.8\phantom{0}$ & ${-}28.9\phantom{0}$ & $\phantom{0}\phantom{-}0.5\phantom{0}$ & $\phantom{0}\phantom{-}0.7\phantom{0}$ & $\phantom{0}{-}0.6\phantom{0}$ & $\phantom{0}0.10\phantom{0}$ & $\phantom{0}0.18\phantom{0}$ & $\phantom{0}0.28\phantom{0}$ & $\phantom{0}0.13\phantom{0}$ & $\phantom{0}0.14\phantom{0}$ & $\phantom{0}0.13\phantom{0}$ & $\phantom{0}94.3\phantom{0}$ & $\phantom{0}91.6\phantom{0}$ & $\phantom{0}19.0\phantom{0}$ & $\phantom{0}94.0\phantom{0}$ & $\phantom{0}93.3\phantom{0}$ & $\phantom{0}94.5\phantom{0}$ \\
 & \nopagebreak $\;J=1000$  & $\phantom{0}\phantom{-}0.3\phantom{0}$ & $\phantom{0}\phantom{-}9.5\phantom{0}$ & ${-}28.6\phantom{0}$ & $\phantom{0}\phantom{-}0.7\phantom{0}$ & $\phantom{0}\phantom{-}0.6\phantom{0}$ & $\phantom{0}\phantom{-}0.1\phantom{0}$ & $\phantom{0}0.07\phantom{0}$ & $\phantom{0}0.14\phantom{0}$ & $\phantom{0}0.27\phantom{0}$ & $\phantom{0}0.10\phantom{0}$ & $\phantom{0}0.10\phantom{0}$ & $\phantom{0}0.09\phantom{0}$ & $\phantom{0}93.3\phantom{0}$ & $\phantom{0}88.8\phantom{0}$ & $\phantom{0}\phantom{0}3.3\phantom{0}$ & $\phantom{0}94.1\phantom{0}$ & $\phantom{0}93.8\phantom{0}$ & $\phantom{0}94.6\phantom{0}$ \\
\multicolumn{4}{l}{$n=20$} \\  & \nopagebreak $\;J=30$  & $\phantom{0}\phantom{-}0.1\phantom{0}$ & $\phantom{0}\phantom{-}3.9\phantom{0}$ & ${-}36.2\phantom{0}$ & $\phantom{0}{-}0.3\phantom{0}$ & $\phantom{0}\phantom{-}0.5\phantom{0}$ & ${-}13.1\phantom{0}$ & $\phantom{0}0.34\phantom{0}$ & $\phantom{0}0.54\phantom{0}$ & $\phantom{0}0.46\phantom{0}$ & $\phantom{0}0.50\phantom{0}$ & $\phantom{0}0.50\phantom{0}$ & $\phantom{0}0.45\phantom{0}$ & $\phantom{0}91.2\phantom{0}$ & $\phantom{0}87.3\phantom{0}$ & $\phantom{0}63.7\phantom{0}$ & $\phantom{0}91.6\phantom{0}$ & $\phantom{0}91.5\phantom{0}$ & $\phantom{0}93.4\phantom{0}$ \\
 & \nopagebreak $\;J=50$  & $\phantom{0}\phantom{-}1.4\phantom{0}$ & $\phantom{0}\phantom{-}5.4\phantom{0}$ & ${-}33.5\phantom{0}$ & $\phantom{0}\phantom{-}2.2\phantom{0}$ & $\phantom{0}\phantom{-}2.3\phantom{0}$ & $\phantom{0}{-}6.0\phantom{0}$ & $\phantom{0}0.25\phantom{0}$ & $\phantom{0}0.38\phantom{0}$ & $\phantom{0}0.39\phantom{0}$ & $\phantom{0}0.37\phantom{0}$ & $\phantom{0}0.37\phantom{0}$ & $\phantom{0}0.34\phantom{0}$ & $\phantom{0}92.4\phantom{0}$ & $\phantom{0}90.9\phantom{0}$ & $\phantom{0}60.5\phantom{0}$ & $\phantom{0}93.5\phantom{0}$ & $\phantom{0}93.1\phantom{0}$ & $\phantom{0}95.1\phantom{0}$ \\
 & \nopagebreak $\;J=100$  & $\phantom{0}\phantom{-}0.6\phantom{0}$ & $\phantom{0}\phantom{-}3.2\phantom{0}$ & ${-}34.1\phantom{0}$ & $\phantom{0}\phantom{-}0.7\phantom{0}$ & $\phantom{0}\phantom{-}0.8\phantom{0}$ & $\phantom{0}{-}3.5\phantom{0}$ & $\phantom{0}0.17\phantom{0}$ & $\phantom{0}0.26\phantom{0}$ & $\phantom{0}0.36\phantom{0}$ & $\phantom{0}0.26\phantom{0}$ & $\phantom{0}0.26\phantom{0}$ & $\phantom{0}0.25\phantom{0}$ & $\phantom{0}94.1\phantom{0}$ & $\phantom{0}92.6\phantom{0}$ & $\phantom{0}42.6\phantom{0}$ & $\phantom{0}92.9\phantom{0}$ & $\phantom{0}93.7\phantom{0}$ & $\phantom{0}93.5\phantom{0}$ \\
 & \nopagebreak $\;J=200$  & $\phantom{0}\phantom{-}0.6\phantom{0}$ & $\phantom{0}\phantom{-}2.6\phantom{0}$ & ${-}34.0\phantom{0}$ & $\phantom{0}\phantom{-}0.4\phantom{0}$ & $\phantom{0}\phantom{-}0.4\phantom{0}$ & $\phantom{0}{-}1.7\phantom{0}$ & $\phantom{0}0.12\phantom{0}$ & $\phantom{0}0.18\phantom{0}$ & $\phantom{0}0.33\phantom{0}$ & $\phantom{0}0.17\phantom{0}$ & $\phantom{0}0.18\phantom{0}$ & $\phantom{0}0.17\phantom{0}$ & $\phantom{0}94.2\phantom{0}$ & $\phantom{0}94.2\phantom{0}$ & $\phantom{0}20.4\phantom{0}$ & $\phantom{0}94.7\phantom{0}$ & $\phantom{0}93.7\phantom{0}$ & $\phantom{0}95.0\phantom{0}$ \\
 & \nopagebreak $\;J=500$  & $\phantom{0}{-}0.0\phantom{0}$ & $\phantom{0}\phantom{-}2.0\phantom{0}$ & ${-}34.2\phantom{0}$ & $\phantom{0}{-}0.0\phantom{0}$ & $\phantom{0}{-}0.0\phantom{0}$ & $\phantom{0}{-}0.9\phantom{0}$ & $\phantom{0}0.08\phantom{0}$ & $\phantom{0}0.11\phantom{0}$ & $\phantom{0}0.32\phantom{0}$ & $\phantom{0}0.11\phantom{0}$ & $\phantom{0}0.11\phantom{0}$ & $\phantom{0}0.11\phantom{0}$ & $\phantom{0}94.4\phantom{0}$ & $\phantom{0}94.3\phantom{0}$ & $\phantom{0}\phantom{0}0.8\phantom{0}$ & $\phantom{0}94.0\phantom{0}$ & $\phantom{0}93.9\phantom{0}$ & $\phantom{0}93.9\phantom{0}$ \\
 & \nopagebreak $\;J=1000$  & $\phantom{0}\phantom{-}0.0\phantom{0}$ & $\phantom{0}\phantom{-}2.3\phantom{0}$ & ${-}33.8\phantom{0}$ & $\phantom{0}\phantom{-}0.3\phantom{0}$ & $\phantom{0}\phantom{-}0.2\phantom{0}$ & $\phantom{0}{-}0.1\phantom{0}$ & $\phantom{0}0.05\phantom{0}$ & $\phantom{0}0.08\phantom{0}$ & $\phantom{0}0.31\phantom{0}$ & $\phantom{0}0.07\phantom{0}$ & $\phantom{0}0.07\phantom{0}$ & $\phantom{0}0.07\phantom{0}$ & $\phantom{0}95.7\phantom{0}$ & $\phantom{0}95.8\phantom{0}$ & $\phantom{0}\phantom{0}0.0\phantom{0}$ & $\phantom{0}94.8\phantom{0}$ & $\phantom{0}95.6\phantom{0}$ & $\phantom{0}95.8\phantom{0}$ \\
[0.5ex]\hline\\[-1.6ex] 
\end{tabular}
\begin{tablenotes}[para,flushleft]{\footnotesize \textit{Note.} $n$ = cluster size; $J$ = number of clusters; CD = complete data sets; LD = listwise deletion; FCS-SL = single-level FCS; FCS-MAN = two-level FCS with manifest cluster means; FCS-LAT = two-level FCS with latent cluster means; JM = joint modeling.}\end{tablenotes}
\end{threeparttable}
\end{sidewaystable}
\begin{sidewaystable}
\begin{threeparttable}
\setlength{\tabcolsep}{1.2pt}
\renewcommand{\arraystretch}{0.95}
\footnotesize
\caption{\small Study 1: Bias (in \%), RMSE, and Coverage of the 95\% Confidence Interval for the Regression Coefficient of $z$ on $y$ ($\hat\beta_{zy}$) With 40\% Missing Data (MAR, $\lambda=1$)}
\begin{tabular}{llcccccccccccccccccc}
\hline\\[-1.8ex]
& & \multicolumn{6}{c}{Bias (\%)} & \multicolumn{6}{c}{RMSE} & \multicolumn{6}{c}{Coverage (\%)} \\ \cmidrule(r){3-8}\cmidrule(r){9-14}\cmidrule(r){15-20}
 &  & CD & LD & \makecell{FCS-\\SL} & \makecell{FCS-\\MAN} & \makecell{FCS-\\LAT} & JM & CD & LD & \makecell{FCS-\\SL} & \makecell{FCS-\\MAN} & \makecell{FCS-\\LAT} & JM & CD & LD & \makecell{FCS-\\SL} & \makecell{FCS-\\MAN} & \makecell{FCS-\\LAT} & \multicolumn{1}{c}{JM} \\ 
[0.4ex]\hline\\[-1.8ex]
& & \multicolumn{18}{c}{Small intraclass correlation $(\rho_{Iy}=.10)$} \\[0.6ex]\hline\\[-1.8ex]
\multicolumn{4}{l}{$n=5$} \\  & \nopagebreak $\;J=30$  & $\phantom{0}\phantom{-}43.2\phantom{0}$ & $\phantom{-}180.3\phantom{0}$ & $\phantom{0}{-}43.9\phantom{0}$ & $\phantom{0}{-}12.5\phantom{0}$ & $\phantom{0}\phantom{0}{-}4.4\phantom{0}$ & $\phantom{0}{-}38.9\phantom{0}$ & $\phantom{0}2.79\phantom{0}$ & $\phantom{0}9.10\phantom{0}$ & $\phantom{0}2.02\phantom{0}$ & $\phantom{0}2.07\phantom{0}$ & $\phantom{0}2.19\phantom{0}$ & $\phantom{0}1.94\phantom{0}$ & $\phantom{0}88.7\phantom{0}$ & $\phantom{0}60.0\phantom{0}$ & $\phantom{0}74.8\phantom{0}$ & $\phantom{0}94.2\phantom{0}$ & $\phantom{0}94.0\phantom{0}$ & $\phantom{0}95.1\phantom{0}$ \\
 & \nopagebreak $\;J=50$  & $\phantom{0}\phantom{-}35.9\phantom{0}$ & $\phantom{-}301.8\phantom{0}$ & $\phantom{0}{-}46.6\phantom{0}$ & $\phantom{0}\phantom{0}{-}0.8\phantom{0}$ & $\phantom{0}\phantom{-}12.6\phantom{0}$ & $\phantom{0}{-}29.8\phantom{0}$ & $\phantom{0}1.83\phantom{0}$ & $\phantom{0}9.47\phantom{0}$ & $\phantom{0}1.57\phantom{0}$ & $\phantom{0}1.64\phantom{0}$ & $\phantom{0}1.78\phantom{0}$ & $\phantom{0}1.45\phantom{0}$ & $\phantom{0}91.8\phantom{0}$ & $\phantom{0}54.9\phantom{0}$ & $\phantom{0}68.2\phantom{0}$ & $\phantom{0}93.9\phantom{0}$ & $\phantom{0}94.3\phantom{0}$ & $\phantom{0}94.6\phantom{0}$ \\
 & \nopagebreak $\;J=100$  & $\phantom{0}\phantom{-}20.6\phantom{0}$ & $\phantom{-}443.2\phantom{0}$ & $\phantom{0}{-}53.6\phantom{0}$ & $\phantom{0}\phantom{0}\phantom{-}8.5\phantom{0}$ & $\phantom{0}\phantom{-}23.1\phantom{0}$ & $\phantom{0}{-}24.0\phantom{0}$ & $\phantom{0}1.23\phantom{0}$ & $\phantom{0}9.55\phantom{0}$ & $\phantom{0}1.23\phantom{0}$ & $\phantom{0}1.24\phantom{0}$ & $\phantom{0}1.34\phantom{0}$ & $\phantom{0}1.14\phantom{0}$ & $\phantom{0}94.5\phantom{0}$ & $\phantom{0}41.1\phantom{0}$ & $\phantom{0}49.2\phantom{0}$ & $\phantom{0}94.8\phantom{0}$ & $\phantom{0}95.1\phantom{0}$ & $\phantom{0}93.7\phantom{0}$ \\
 & \nopagebreak $\;J=200$  & $\phantom{0}\phantom{0}\phantom{-}7.9\phantom{0}$ & $\phantom{-}523.4\phantom{0}$ & $\phantom{0}{-}61.8\phantom{0}$ & $\phantom{0}\phantom{0}\phantom{-}5.7\phantom{0}$ & $\phantom{0}\phantom{-}16.6\phantom{0}$ & $\phantom{0}{-}20.1\phantom{0}$ & $\phantom{0}0.58\phantom{0}$ & $\phantom{0}9.66\phantom{0}$ & $\phantom{0}1.03\phantom{0}$ & $\phantom{0}0.86\phantom{0}$ & $\phantom{0}0.95\phantom{0}$ & $\phantom{0}0.72\phantom{0}$ & $\phantom{0}96.0\phantom{0}$ & $\phantom{0}21.9\phantom{0}$ & $\phantom{0}26.7\phantom{0}$ & $\phantom{0}95.5\phantom{0}$ & $\phantom{0}95.0\phantom{0}$ & $\phantom{0}94.5\phantom{0}$ \\
 & \nopagebreak $\;J=500$  & $\phantom{0}\phantom{0}\phantom{-}2.8\phantom{0}$ & $\phantom{-}615.3\phantom{0}$ & $\phantom{0}{-}62.9\phantom{0}$ & $\phantom{0}\phantom{0}\phantom{-}3.4\phantom{0}$ & $\phantom{0}\phantom{0}\phantom{-}7.8\phantom{0}$ & $\phantom{0}{-}11.2\phantom{0}$ & $\phantom{0}0.35\phantom{0}$ & $\phantom{0}10.26\phantom{0}$ & $\phantom{0}1.01\phantom{0}$ & $\phantom{0}0.54\phantom{0}$ & $\phantom{0}0.58\phantom{0}$ & $\phantom{0}0.47\phantom{0}$ & $\phantom{0}95.6\phantom{0}$ & $\phantom{0}\phantom{0}2.2\phantom{0}$ & $\phantom{0}\phantom{0}3.5\phantom{0}$ & $\phantom{0}94.5\phantom{0}$ & $\phantom{0}93.1\phantom{0}$ & $\phantom{0}92.8\phantom{0}$ \\
 & \nopagebreak $\;J=1000$  & $\phantom{0}\phantom{0}\phantom{-}1.3\phantom{0}$ & $\phantom{-}667.9\phantom{0}$ & $\phantom{0}{-}64.2\phantom{0}$ & $\phantom{0}\phantom{0}\phantom{-}0.6\phantom{0}$ & $\phantom{0}\phantom{0}\phantom{-}2.1\phantom{0}$ & $\phantom{0}\phantom{0}{-}8.0\phantom{0}$ & $\phantom{0}0.22\phantom{0}$ & $\phantom{0}10.74\phantom{0}$ & $\phantom{0}1.02\phantom{0}$ & $\phantom{0}0.36\phantom{0}$ & $\phantom{0}0.37\phantom{0}$ & $\phantom{0}0.33\phantom{0}$ & $\phantom{0}95.4\phantom{0}$ & $\phantom{0}\phantom{0}0.0\phantom{0}$ & $\phantom{0}\phantom{0}0.0\phantom{0}$ & $\phantom{0}94.7\phantom{0}$ & $\phantom{0}94.0\phantom{0}$ & $\phantom{0}94.0\phantom{0}$ \\
\multicolumn{4}{l}{$n=20$} \\  & \nopagebreak $\;J=30$  & $\phantom{0}\phantom{0}\phantom{-}8.8\phantom{0}$ & $\phantom{-}192.4\phantom{0}$ & $\phantom{0}{-}66.7\phantom{0}$ & $\phantom{0}\phantom{0}\phantom{-}3.9\phantom{0}$ & $\phantom{0}\phantom{-}14.7\phantom{0}$ & $\phantom{0}{-}45.6\phantom{0}$ & $\phantom{0}0.89\phantom{0}$ & $\phantom{0}8.88\phantom{0}$ & $\phantom{0}1.20\phantom{0}$ & $\phantom{0}1.58\phantom{0}$ & $\phantom{0}1.76\phantom{0}$ & $\phantom{0}1.11\phantom{0}$ & $\phantom{0}93.2\phantom{0}$ & $\phantom{0}80.1\phantom{0}$ & $\phantom{0}41.2\phantom{0}$ & $\phantom{0}91.8\phantom{0}$ & $\phantom{0}88.2\phantom{0}$ & $\phantom{0}93.6\phantom{0}$ \\
 & \nopagebreak $\;J=50$  & $\phantom{0}\phantom{0}\phantom{-}6.2\phantom{0}$ & $\phantom{-}222.1\phantom{0}$ & $\phantom{0}{-}67.3\phantom{0}$ & $\phantom{0}\phantom{0}\phantom{-}4.8\phantom{0}$ & $\phantom{0}\phantom{-}14.5\phantom{0}$ & $\phantom{0}{-}36.5\phantom{0}$ & $\phantom{0}0.61\phantom{0}$ & $\phantom{0}7.47\phantom{0}$ & $\phantom{0}1.13\phantom{0}$ & $\phantom{0}1.18\phantom{0}$ & $\phantom{0}1.32\phantom{0}$ & $\phantom{0}0.91\phantom{0}$ & $\phantom{0}93.5\phantom{0}$ & $\phantom{0}84.6\phantom{0}$ & $\phantom{0}25.4\phantom{0}$ & $\phantom{0}91.7\phantom{0}$ & $\phantom{0}89.4\phantom{0}$ & $\phantom{0}92.7\phantom{0}$ \\
 & \nopagebreak $\;J=100$  & $\phantom{0}\phantom{0}\phantom{-}1.9\phantom{0}$ & $\phantom{-}256.8\phantom{0}$ & $\phantom{0}{-}69.4\phantom{0}$ & $\phantom{0}\phantom{0}\phantom{-}1.7\phantom{0}$ & $\phantom{0}\phantom{0}\phantom{-}5.3\phantom{0}$ & $\phantom{0}{-}28.1\phantom{0}$ & $\phantom{0}0.39\phantom{0}$ & $\phantom{0}6.57\phantom{0}$ & $\phantom{0}1.12\phantom{0}$ & $\phantom{0}0.76\phantom{0}$ & $\phantom{0}0.82\phantom{0}$ & $\phantom{0}0.69\phantom{0}$ & $\phantom{0}93.4\phantom{0}$ & $\phantom{0}88.7\phantom{0}$ & $\phantom{0}\phantom{0}4.5\phantom{0}$ & $\phantom{0}93.3\phantom{0}$ & $\phantom{0}91.7\phantom{0}$ & $\phantom{0}91.8\phantom{0}$ \\
 & \nopagebreak $\;J=200$  & $\phantom{0}\phantom{0}\phantom{-}0.3\phantom{0}$ & $\phantom{-}211.8\phantom{0}$ & $\phantom{0}{-}69.7\phantom{0}$ & $\phantom{0}\phantom{0}\phantom{-}0.0\phantom{0}$ & $\phantom{0}\phantom{0}\phantom{-}1.0\phantom{0}$ & $\phantom{0}{-}18.2\phantom{0}$ & $\phantom{0}0.27\phantom{0}$ & $\phantom{0}4.59\phantom{0}$ & $\phantom{0}1.11\phantom{0}$ & $\phantom{0}0.52\phantom{0}$ & $\phantom{0}0.53\phantom{0}$ & $\phantom{0}0.50\phantom{0}$ & $\phantom{0}94.3\phantom{0}$ & $\phantom{0}93.3\phantom{0}$ & $\phantom{0}\phantom{0}0.2\phantom{0}$ & $\phantom{0}93.5\phantom{0}$ & $\phantom{0}94.0\phantom{0}$ & $\phantom{0}92.5\phantom{0}$ \\
 & \nopagebreak $\;J=500$  & $\phantom{0}\phantom{0}\phantom{-}0.7\phantom{0}$ & $\phantom{-}191.2\phantom{0}$ & $\phantom{0}{-}69.4\phantom{0}$ & $\phantom{0}\phantom{0}\phantom{-}0.7\phantom{0}$ & $\phantom{0}\phantom{0}\phantom{-}0.9\phantom{0}$ & $\phantom{0}\phantom{0}{-}8.2\phantom{0}$ & $\phantom{0}0.17\phantom{0}$ & $\phantom{0}3.54\phantom{0}$ & $\phantom{0}1.10\phantom{0}$ & $\phantom{0}0.33\phantom{0}$ & $\phantom{0}0.34\phantom{0}$ & $\phantom{0}0.33\phantom{0}$ & $\phantom{0}94.3\phantom{0}$ & $\phantom{0}64.7\phantom{0}$ & $\phantom{0}\phantom{0}0.0\phantom{0}$ & $\phantom{0}93.6\phantom{0}$ & $\phantom{0}91.7\phantom{0}$ & $\phantom{0}93.0\phantom{0}$ \\
 & \nopagebreak $\;J=1000$  & $\phantom{0}\phantom{0}{-}0.1\phantom{0}$ & $\phantom{-}174.2\phantom{0}$ & $\phantom{0}{-}69.9\phantom{0}$ & $\phantom{0}\phantom{0}{-}0.9\phantom{0}$ & $\phantom{0}\phantom{0}{-}0.7\phantom{0}$ & $\phantom{0}\phantom{0}{-}5.3\phantom{0}$ & $\phantom{0}0.12\phantom{0}$ & $\phantom{0}2.99\phantom{0}$ & $\phantom{0}1.11\phantom{0}$ & $\phantom{0}0.22\phantom{0}$ & $\phantom{0}0.22\phantom{0}$ & $\phantom{0}0.23\phantom{0}$ & $\phantom{0}93.9\phantom{0}$ & $\phantom{0}\phantom{0}9.3\phantom{0}$ & $\phantom{0}\phantom{0}0.0\phantom{0}$ & $\phantom{0}94.4\phantom{0}$ & $\phantom{0}93.3\phantom{0}$ & $\phantom{0}93.6\phantom{0}$ \\
[0.5ex]\hline\\[-1.6ex] 
& & \multicolumn{18}{c}{Moderate intraclass correlation $(\rho_{Iy}=.30)$} \\[0.6ex]\hline\\[-1.8ex]
\multicolumn{4}{l}{$n=5$} \\  & \nopagebreak $\;J=30$  & $\phantom{0}\phantom{0}\phantom{-}7.6\phantom{0}$ & $\phantom{-}233.3\phantom{0}$ & $\phantom{0}{-}55.4\phantom{0}$ & $\phantom{0}\phantom{0}\phantom{-}1.6\phantom{0}$ & $\phantom{0}\phantom{-}14.7\phantom{0}$ & $\phantom{0}{-}34.4\phantom{0}$ & $\phantom{0}0.52\phantom{0}$ & $\phantom{0}5.82\phantom{0}$ & $\phantom{0}0.65\phantom{0}$ & $\phantom{0}0.93\phantom{0}$ & $\phantom{0}1.02\phantom{0}$ & $\phantom{0}0.67\phantom{0}$ & $\phantom{0}92.0\phantom{0}$ & $\phantom{0}80.1\phantom{0}$ & $\phantom{0}63.5\phantom{0}$ & $\phantom{0}92.9\phantom{0}$ & $\phantom{0}89.8\phantom{0}$ & $\phantom{0}94.1\phantom{0}$ \\
 & \nopagebreak $\;J=50$  & $\phantom{0}\phantom{0}\phantom{-}5.1\phantom{0}$ & $\phantom{-}269.1\phantom{0}$ & $\phantom{0}{-}56.0\phantom{0}$ & $\phantom{0}\phantom{0}\phantom{-}3.4\phantom{0}$ & $\phantom{0}\phantom{-}14.9\phantom{0}$ & $\phantom{0}{-}24.4\phantom{0}$ & $\phantom{0}0.35\phantom{0}$ & $\phantom{0}4.84\phantom{0}$ & $\phantom{0}0.59\phantom{0}$ & $\phantom{0}0.69\phantom{0}$ & $\phantom{0}0.78\phantom{0}$ & $\phantom{0}0.54\phantom{0}$ & $\phantom{0}93.5\phantom{0}$ & $\phantom{0}83.8\phantom{0}$ & $\phantom{0}52.5\phantom{0}$ & $\phantom{0}92.5\phantom{0}$ & $\phantom{0}88.4\phantom{0}$ & $\phantom{0}94.7\phantom{0}$ \\
 & \nopagebreak $\;J=100$  & $\phantom{0}\phantom{0}\phantom{-}2.0\phantom{0}$ & $\phantom{-}286.2\phantom{0}$ & $\phantom{0}{-}55.5\phantom{0}$ & $\phantom{0}\phantom{0}\phantom{-}3.0\phantom{0}$ & $\phantom{0}\phantom{0}\phantom{-}7.0\phantom{0}$ & $\phantom{0}{-}12.3\phantom{0}$ & $\phantom{0}0.23\phantom{0}$ & $\phantom{0}4.14\phantom{0}$ & $\phantom{0}0.54\phantom{0}$ & $\phantom{0}0.45\phantom{0}$ & $\phantom{0}0.49\phantom{0}$ & $\phantom{0}0.39\phantom{0}$ & $\phantom{0}93.7\phantom{0}$ & $\phantom{0}87.5\phantom{0}$ & $\phantom{0}29.8\phantom{0}$ & $\phantom{0}93.9\phantom{0}$ & $\phantom{0}91.0\phantom{0}$ & $\phantom{0}94.3\phantom{0}$ \\
 & \nopagebreak $\;J=200$  & $\phantom{0}\phantom{0}\phantom{-}1.5\phantom{0}$ & $\phantom{-}254.0\phantom{0}$ & $\phantom{0}{-}56.0\phantom{0}$ & $\phantom{0}\phantom{0}\phantom{-}1.4\phantom{0}$ & $\phantom{0}\phantom{0}\phantom{-}2.5\phantom{0}$ & $\phantom{0}\phantom{0}{-}6.3\phantom{0}$ & $\phantom{0}0.16\phantom{0}$ & $\phantom{0}3.03\phantom{0}$ & $\phantom{0}0.53\phantom{0}$ & $\phantom{0}0.31\phantom{0}$ & $\phantom{0}0.32\phantom{0}$ & $\phantom{0}0.29\phantom{0}$ & $\phantom{0}94.3\phantom{0}$ & $\phantom{0}91.0\phantom{0}$ & $\phantom{0}\phantom{0}5.5\phantom{0}$ & $\phantom{0}93.9\phantom{0}$ & $\phantom{0}92.4\phantom{0}$ & $\phantom{0}94.2\phantom{0}$ \\
 & \nopagebreak $\;J=500$  & $\phantom{0}\phantom{0}\phantom{-}0.3\phantom{0}$ & $\phantom{-}219.2\phantom{0}$ & $\phantom{0}{-}56.0\phantom{0}$ & $\phantom{0}\phantom{0}\phantom{-}0.2\phantom{0}$ & $\phantom{0}\phantom{0}\phantom{-}0.9\phantom{0}$ & $\phantom{0}\phantom{0}{-}2.5\phantom{0}$ & $\phantom{0}0.10\phantom{0}$ & $\phantom{0}2.44\phantom{0}$ & $\phantom{0}0.52\phantom{0}$ & $\phantom{0}0.19\phantom{0}$ & $\phantom{0}0.19\phantom{0}$ & $\phantom{0}0.18\phantom{0}$ & $\phantom{0}94.4\phantom{0}$ & $\phantom{0}75.7\phantom{0}$ & $\phantom{0}\phantom{0}0.0\phantom{0}$ & $\phantom{0}93.3\phantom{0}$ & $\phantom{0}92.4\phantom{0}$ & $\phantom{0}94.0\phantom{0}$ \\
 & \nopagebreak $\;J=1000$  & $\phantom{0}\phantom{0}\phantom{-}0.1\phantom{0}$ & $\phantom{-}197.5\phantom{0}$ & $\phantom{0}{-}55.8\phantom{0}$ & $\phantom{0}\phantom{0}\phantom{-}0.8\phantom{0}$ & $\phantom{0}\phantom{0}\phantom{-}1.0\phantom{0}$ & $\phantom{0}\phantom{0}{-}0.6\phantom{0}$ & $\phantom{0}0.07\phantom{0}$ & $\phantom{0}1.97\phantom{0}$ & $\phantom{0}0.51\phantom{0}$ & $\phantom{0}0.13\phantom{0}$ & $\phantom{0}0.14\phantom{0}$ & $\phantom{0}0.13\phantom{0}$ & $\phantom{0}95.1\phantom{0}$ & $\phantom{0}15.6\phantom{0}$ & $\phantom{0}\phantom{0}0.0\phantom{0}$ & $\phantom{0}93.5\phantom{0}$ & $\phantom{0}91.9\phantom{0}$ & $\phantom{0}93.4\phantom{0}$ \\
\multicolumn{4}{l}{$n=20$} \\  & \nopagebreak $\;J=30$  & $\phantom{0}\phantom{0}\phantom{-}2.2\phantom{0}$ & $\phantom{0}\phantom{-}35.6\phantom{0}$ & $\phantom{0}{-}62.1\phantom{0}$ & $\phantom{0}\phantom{0}\phantom{-}1.4\phantom{0}$ & $\phantom{0}\phantom{0}\phantom{-}3.8\phantom{0}$ & $\phantom{0}{-}28.7\phantom{0}$ & $\phantom{0}0.34\phantom{0}$ & $\phantom{0}1.24\phantom{0}$ & $\phantom{0}0.62\phantom{0}$ & $\phantom{0}0.72\phantom{0}$ & $\phantom{0}0.75\phantom{0}$ & $\phantom{0}0.56\phantom{0}$ & $\phantom{0}91.7\phantom{0}$ & $\phantom{0}89.2\phantom{0}$ & $\phantom{0}35.5\phantom{0}$ & $\phantom{0}93.3\phantom{0}$ & $\phantom{0}92.9\phantom{0}$ & $\phantom{0}95.5\phantom{0}$ \\
 & \nopagebreak $\;J=50$  & $\phantom{0}\phantom{0}\phantom{-}0.7\phantom{0}$ & $\phantom{0}\phantom{-}29.4\phantom{0}$ & $\phantom{0}{-}62.1\phantom{0}$ & $\phantom{0}\phantom{0}\phantom{-}1.0\phantom{0}$ & $\phantom{0}\phantom{0}\phantom{-}2.4\phantom{0}$ & $\phantom{0}{-}19.5\phantom{0}$ & $\phantom{0}0.26\phantom{0}$ & $\phantom{0}0.83\phantom{0}$ & $\phantom{0}0.60\phantom{0}$ & $\phantom{0}0.54\phantom{0}$ & $\phantom{0}0.55\phantom{0}$ & $\phantom{0}0.46\phantom{0}$ & $\phantom{0}92.6\phantom{0}$ & $\phantom{0}90.5\phantom{0}$ & $\phantom{0}21.4\phantom{0}$ & $\phantom{0}92.9\phantom{0}$ & $\phantom{0}92.7\phantom{0}$ & $\phantom{0}94.4\phantom{0}$ \\
 & \nopagebreak $\;J=100$  & $\phantom{0}\phantom{0}\phantom{-}0.6\phantom{0}$ & $\phantom{0}\phantom{-}22.6\phantom{0}$ & $\phantom{0}{-}62.4\phantom{0}$ & $\phantom{0}\phantom{0}\phantom{-}0.3\phantom{0}$ & $\phantom{0}\phantom{0}\phantom{-}0.3\phantom{0}$ & $\phantom{0}{-}10.6\phantom{0}$ & $\phantom{0}0.17\phantom{0}$ & $\phantom{0}0.49\phantom{0}$ & $\phantom{0}0.59\phantom{0}$ & $\phantom{0}0.36\phantom{0}$ & $\phantom{0}0.36\phantom{0}$ & $\phantom{0}0.33\phantom{0}$ & $\phantom{0}93.9\phantom{0}$ & $\phantom{0}91.7\phantom{0}$ & $\phantom{0}\phantom{0}2.9\phantom{0}$ & $\phantom{0}93.5\phantom{0}$ & $\phantom{0}93.6\phantom{0}$ & $\phantom{0}95.1\phantom{0}$ \\
 & \nopagebreak $\;J=200$  & $\phantom{0}\phantom{0}\phantom{-}0.1\phantom{0}$ & $\phantom{0}\phantom{-}19.7\phantom{0}$ & $\phantom{0}{-}61.9\phantom{0}$ & $\phantom{0}\phantom{0}{-}0.5\phantom{0}$ & $\phantom{0}\phantom{0}{-}0.4\phantom{0}$ & $\phantom{0}\phantom{0}{-}6.1\phantom{0}$ & $\phantom{0}0.13\phantom{0}$ & $\phantom{0}0.34\phantom{0}$ & $\phantom{0}0.57\phantom{0}$ & $\phantom{0}0.25\phantom{0}$ & $\phantom{0}0.25\phantom{0}$ & $\phantom{0}0.24\phantom{0}$ & $\phantom{0}93.5\phantom{0}$ & $\phantom{0}90.5\phantom{0}$ & $\phantom{0}\phantom{0}0.0\phantom{0}$ & $\phantom{0}93.6\phantom{0}$ & $\phantom{0}94.0\phantom{0}$ & $\phantom{0}95.1\phantom{0}$ \\
 & \nopagebreak $\;J=500$  & $\phantom{0}\phantom{0}\phantom{-}0.2\phantom{0}$ & $\phantom{0}\phantom{-}20.7\phantom{0}$ & $\phantom{0}{-}61.2\phantom{0}$ & $\phantom{0}\phantom{0}\phantom{-}1.1\phantom{0}$ & $\phantom{0}\phantom{0}\phantom{-}1.1\phantom{0}$ & $\phantom{0}\phantom{0}{-}1.2\phantom{0}$ & $\phantom{0}0.07\phantom{0}$ & $\phantom{0}0.27\phantom{0}$ & $\phantom{0}0.56\phantom{0}$ & $\phantom{0}0.16\phantom{0}$ & $\phantom{0}0.16\phantom{0}$ & $\phantom{0}0.16\phantom{0}$ & $\phantom{0}95.0\phantom{0}$ & $\phantom{0}81.8\phantom{0}$ & $\phantom{0}\phantom{0}0.0\phantom{0}$ & $\phantom{0}93.5\phantom{0}$ & $\phantom{0}93.3\phantom{0}$ & $\phantom{0}95.9\phantom{0}$ \\
 & \nopagebreak $\;J=1000$  & $\phantom{0}\phantom{0}\phantom{-}0.0\phantom{0}$ & $\phantom{0}\phantom{-}19.6\phantom{0}$ & $\phantom{0}{-}61.3\phantom{0}$ & $\phantom{0}\phantom{0}\phantom{-}0.6\phantom{0}$ & $\phantom{0}\phantom{0}\phantom{-}0.4\phantom{0}$ & $\phantom{0}\phantom{0}{-}0.7\phantom{0}$ & $\phantom{0}0.05\phantom{0}$ & $\phantom{0}0.22\phantom{0}$ & $\phantom{0}0.56\phantom{0}$ & $\phantom{0}0.12\phantom{0}$ & $\phantom{0}0.12\phantom{0}$ & $\phantom{0}0.11\phantom{0}$ & $\phantom{0}93.8\phantom{0}$ & $\phantom{0}70.7\phantom{0}$ & $\phantom{0}\phantom{0}0.0\phantom{0}$ & $\phantom{0}92.0\phantom{0}$ & $\phantom{0}91.3\phantom{0}$ & $\phantom{0}93.9\phantom{0}$ \\
[0.5ex]\hline\\[-1.6ex] 
\end{tabular}
\begin{tablenotes}[para,flushleft]{\footnotesize \textit{Note.} $n$ = cluster size; $J$ = number of clusters; CD = complete data sets; LD = listwise deletion; FCS-SL = single-level FCS; FCS-MAN = two-level FCS with manifest cluster means; FCS-LAT = two-level FCS with latent cluster means; JM = joint modeling.}\end{tablenotes}
\end{threeparttable}
\end{sidewaystable}


\clearpage
\newgeometry{headsep=1.1in,textheight=5.8in}

\setcounter{table}{0}
\renewcommand{\thetable}{2-\arabic{table}}
\begin{sidewaystable}
\begin{threeparttable}
\setlength{\tabcolsep}{1.0pt}
\renewcommand{\arraystretch}{0.95}
\footnotesize
\caption{\small Study 2: Bias, Relative RMSE, and Coverage of the 95\% Confidence Interval for the Mean of $z$ ($\hat\mu_z$) With Moderately Unbalanced Data (Uniform, $\pm 40\%$) and 20\% Missing Data (MAR, $\lambda=0.5$)}
\begin{tabular}{llccccccccccccccc}
\hline\\[-1.8ex]
& & \multicolumn{5}{c}{Bias (\%)} & \multicolumn{5}{c}{Rel. RMSE} & \multicolumn{5}{c}{Coverage (\%)} \\ \cmidrule(r){3-7}\cmidrule(r){8-12}\cmidrule(r){13-17}
 &  & CD & \makecell{FCS-\\MAN} & \makecell{FCS-\\NJ} & \makecell{FCS-\\LAT} & JM & CD & \makecell{FCS-\\MAN} & \makecell{FCS-\\NJ} & \makecell{FCS-\\LAT} & JM & CD & \makecell{FCS-\\MAN} & \makecell{FCS-\\NJ} & \makecell{FCS-\\LAT} & \multicolumn{1}{c}{JM} \\ 
[0.4ex]\hline\\[-1.8ex]
& & \multicolumn{15}{c}{Small intraclass correlation $(\rho_{Iy}=.10)$} \\[0.6ex]\hline\\[-1.8ex]
\multicolumn{4}{l}{$\bar{n}=5$} \\  & \nopagebreak $\;J=50$  & ${-}0.01\phantom{0}$ & ${-}0.00\phantom{0}$ & ${-}0.00\phantom{0}$ & $\phantom{-}0.00\phantom{0}$ & ${-}0.02\phantom{0}$ & $\phantom{0}0.15\phantom{0}$ & $\phantom{0}0.17\phantom{0}$ & $\phantom{0}0.17\phantom{0}$ & $\phantom{0}0.17\phantom{0}$ & $\phantom{0}0.17\phantom{0}$ & $\phantom{0}93.9\phantom{0}$ & $\phantom{0}93.0\phantom{0}$ & $\phantom{0}93.8\phantom{0}$ & $\phantom{0}93.2\phantom{0}$ & $\phantom{0}93.4\phantom{0}$ \\
 & \nopagebreak $\;J=200$  & $\phantom{-}0.00\phantom{0}$ & $\phantom{-}0.00\phantom{0}$ & ${-}0.00\phantom{0}$ & $\phantom{-}0.00\phantom{0}$ & ${-}0.01\phantom{0}$ & $\phantom{0}0.07\phantom{0}$ & $\phantom{0}0.08\phantom{0}$ & $\phantom{0}0.08\phantom{0}$ & $\phantom{0}0.08\phantom{0}$ & $\phantom{0}0.08\phantom{0}$ & $\phantom{0}93.7\phantom{0}$ & $\phantom{0}94.4\phantom{0}$ & $\phantom{0}93.9\phantom{0}$ & $\phantom{0}93.9\phantom{0}$ & $\phantom{0}93.9\phantom{0}$ \\
 & \nopagebreak $\;J=1000$  & $\phantom{-}0.00\phantom{0}$ & $\phantom{-}0.00\phantom{0}$ & ${-}0.00\phantom{0}$ & $\phantom{-}0.00\phantom{0}$ & ${-}0.00\phantom{0}$ & $\phantom{0}0.03\phantom{0}$ & $\phantom{0}0.04\phantom{0}$ & $\phantom{0}0.04\phantom{0}$ & $\phantom{0}0.04\phantom{0}$ & $\phantom{0}0.04\phantom{0}$ & $\phantom{0}95.3\phantom{0}$ & $\phantom{0}94.9\phantom{0}$ & $\phantom{0}95.4\phantom{0}$ & $\phantom{0}94.7\phantom{0}$ & $\phantom{0}94.5\phantom{0}$ \\
\multicolumn{4}{l}{$\bar{n}=20$} \\  & \nopagebreak $\;J=50$  & $\phantom{-}0.01\phantom{0}$ & $\phantom{-}0.01\phantom{0}$ & $\phantom{-}0.01\phantom{0}$ & $\phantom{-}0.01\phantom{0}$ & ${-}0.01\phantom{0}$ & $\phantom{0}0.14\phantom{0}$ & $\phantom{0}0.16\phantom{0}$ & $\phantom{0}0.16\phantom{0}$ & $\phantom{0}0.16\phantom{0}$ & $\phantom{0}0.16\phantom{0}$ & $\phantom{0}94.4\phantom{0}$ & $\phantom{0}95.6\phantom{0}$ & $\phantom{0}95.2\phantom{0}$ & $\phantom{0}94.5\phantom{0}$ & $\phantom{0}94.4\phantom{0}$ \\
 & \nopagebreak $\;J=200$  & ${-}0.00\phantom{0}$ & ${-}0.00\phantom{0}$ & ${-}0.00\phantom{0}$ & ${-}0.00\phantom{0}$ & ${-}0.01\phantom{0}$ & $\phantom{0}0.07\phantom{0}$ & $\phantom{0}0.08\phantom{0}$ & $\phantom{0}0.08\phantom{0}$ & $\phantom{0}0.08\phantom{0}$ & $\phantom{0}0.08\phantom{0}$ & $\phantom{0}94.4\phantom{0}$ & $\phantom{0}95.7\phantom{0}$ & $\phantom{0}96.0\phantom{0}$ & $\phantom{0}95.3\phantom{0}$ & $\phantom{0}95.6\phantom{0}$ \\
 & \nopagebreak $\;J=1000$  & ${-}0.00\phantom{0}$ & ${-}0.00\phantom{0}$ & ${-}0.00\phantom{0}$ & ${-}0.00\phantom{0}$ & ${-}0.00\phantom{0}$ & $\phantom{0}0.03\phantom{0}$ & $\phantom{0}0.04\phantom{0}$ & $\phantom{0}0.04\phantom{0}$ & $\phantom{0}0.04\phantom{0}$ & $\phantom{0}0.04\phantom{0}$ & $\phantom{0}95.0\phantom{0}$ & $\phantom{0}95.0\phantom{0}$ & $\phantom{0}94.2\phantom{0}$ & $\phantom{0}94.0\phantom{0}$ & $\phantom{0}93.8\phantom{0}$ \\
[0.5ex]\hline\\[-1.6ex] 
& & \multicolumn{15}{c}{Moderate intraclass correlation $(\rho_{Iy}=.30)$} \\[0.6ex]\hline\\[-1.8ex]
\multicolumn{4}{l}{$\bar{n}=5$} \\  & \nopagebreak $\;J=50$  & $\phantom{-}0.00\phantom{0}$ & $\phantom{-}0.01\phantom{0}$ & $\phantom{-}0.01\phantom{0}$ & $\phantom{-}0.01\phantom{0}$ & ${-}0.01\phantom{0}$ & $\phantom{0}0.14\phantom{0}$ & $\phantom{0}0.16\phantom{0}$ & $\phantom{0}0.17\phantom{0}$ & $\phantom{0}0.16\phantom{0}$ & $\phantom{0}0.16\phantom{0}$ & $\phantom{0}94.8\phantom{0}$ & $\phantom{0}94.2\phantom{0}$ & $\phantom{0}94.3\phantom{0}$ & $\phantom{0}94.9\phantom{0}$ & $\phantom{0}94.2\phantom{0}$ \\
 & \nopagebreak $\;J=200$  & $\phantom{-}0.00\phantom{0}$ & $\phantom{-}0.00\phantom{0}$ & $\phantom{-}0.00\phantom{0}$ & $\phantom{-}0.00\phantom{0}$ & $\phantom{-}0.00\phantom{0}$ & $\phantom{0}0.07\phantom{0}$ & $\phantom{0}0.08\phantom{0}$ & $\phantom{0}0.08\phantom{0}$ & $\phantom{0}0.08\phantom{0}$ & $\phantom{0}0.08\phantom{0}$ & $\phantom{0}95.2\phantom{0}$ & $\phantom{0}95.0\phantom{0}$ & $\phantom{0}95.1\phantom{0}$ & $\phantom{0}94.8\phantom{0}$ & $\phantom{0}95.5\phantom{0}$ \\
 & \nopagebreak $\;J=1000$  & $\phantom{-}0.00\phantom{0}$ & $\phantom{-}0.00\phantom{0}$ & $\phantom{-}0.00\phantom{0}$ & $\phantom{-}0.00\phantom{0}$ & ${-}0.00\phantom{0}$ & $\phantom{0}0.03\phantom{0}$ & $\phantom{0}0.03\phantom{0}$ & $\phantom{0}0.03\phantom{0}$ & $\phantom{0}0.03\phantom{0}$ & $\phantom{0}0.03\phantom{0}$ & $\phantom{0}95.7\phantom{0}$ & $\phantom{0}95.8\phantom{0}$ & $\phantom{0}96.0\phantom{0}$ & $\phantom{0}96.0\phantom{0}$ & $\phantom{0}95.9\phantom{0}$ \\
\multicolumn{4}{l}{$\bar{n}=20$} \\  & \nopagebreak $\;J=50$  & $\phantom{-}0.01\phantom{0}$ & $\phantom{-}0.01\phantom{0}$ & $\phantom{-}0.01\phantom{0}$ & $\phantom{-}0.01\phantom{0}$ & ${-}0.00\phantom{0}$ & $\phantom{0}0.14\phantom{0}$ & $\phantom{0}0.16\phantom{0}$ & $\phantom{0}0.16\phantom{0}$ & $\phantom{0}0.16\phantom{0}$ & $\phantom{0}0.15\phantom{0}$ & $\phantom{0}95.3\phantom{0}$ & $\phantom{0}94.1\phantom{0}$ & $\phantom{0}94.9\phantom{0}$ & $\phantom{0}94.2\phantom{0}$ & $\phantom{0}94.2\phantom{0}$ \\
 & \nopagebreak $\;J=200$  & $\phantom{-}0.00\phantom{0}$ & $\phantom{-}0.00\phantom{0}$ & ${-}0.00\phantom{0}$ & ${-}0.00\phantom{0}$ & ${-}0.00\phantom{0}$ & $\phantom{0}0.07\phantom{0}$ & $\phantom{0}0.08\phantom{0}$ & $\phantom{0}0.08\phantom{0}$ & $\phantom{0}0.08\phantom{0}$ & $\phantom{0}0.08\phantom{0}$ & $\phantom{0}95.7\phantom{0}$ & $\phantom{0}95.5\phantom{0}$ & $\phantom{0}95.4\phantom{0}$ & $\phantom{0}95.5\phantom{0}$ & $\phantom{0}95.8\phantom{0}$ \\
 & \nopagebreak $\;J=1000$  & ${-}0.00\phantom{0}$ & $\phantom{-}0.00\phantom{0}$ & $\phantom{-}0.00\phantom{0}$ & $\phantom{-}0.00\phantom{0}$ & ${-}0.00\phantom{0}$ & $\phantom{0}0.03\phantom{0}$ & $\phantom{0}0.04\phantom{0}$ & $\phantom{0}0.04\phantom{0}$ & $\phantom{0}0.03\phantom{0}$ & $\phantom{0}0.03\phantom{0}$ & $\phantom{0}94.7\phantom{0}$ & $\phantom{0}95.0\phantom{0}$ & $\phantom{0}94.6\phantom{0}$ & $\phantom{0}94.4\phantom{0}$ & $\phantom{0}94.8\phantom{0}$ \\
[0.5ex]\hline\\[-1.6ex] 
\end{tabular}
\begin{tablenotes}[para,flushleft]{\footnotesize \textit{Note.} $\bar{n}$ = average cluster size; $J$ = number of clusters; CD = complete data sets; LD = listwise deletion; FCS-SL = single-level FCS; FCS-MAN = two-level FCS with manifest cluster means; FCS-LAT = two-level FCS with latent cluster means; JM = joint modeling.}\end{tablenotes}
\end{threeparttable}
\end{sidewaystable}
\begin{sidewaystable}
\begin{threeparttable}
\setlength{\tabcolsep}{1.0pt}
\renewcommand{\arraystretch}{0.95}
\footnotesize
\caption{\small Study 2: Bias, Relative RMSE, and Coverage of the 95\% Confidence Interval for the Mean of $z$ ($\hat\mu_z$) With Strongly Unbalanced Data (Uniform, $\pm 80\%$) and 20\% Missing Data (MAR, $\lambda=0.5$)}
\begin{tabular}{llccccccccccccccc}
\hline\\[-1.8ex]
& & \multicolumn{5}{c}{Bias (\%)} & \multicolumn{5}{c}{Rel. RMSE} & \multicolumn{5}{c}{Coverage (\%)} \\ \cmidrule(r){3-7}\cmidrule(r){8-12}\cmidrule(r){13-17}
 &  & CD & \makecell{FCS-\\MAN} & \makecell{FCS-\\NJ} & \makecell{FCS-\\LAT} & JM & CD & \makecell{FCS-\\MAN} & \makecell{FCS-\\NJ} & \makecell{FCS-\\LAT} & JM & CD & \makecell{FCS-\\MAN} & \makecell{FCS-\\NJ} & \makecell{FCS-\\LAT} & \multicolumn{1}{c}{JM} \\ 
[0.4ex]\hline\\[-1.8ex]
& & \multicolumn{15}{c}{Small intraclass correlation $(\rho_{Iy}=.10)$} \\[0.6ex]\hline\\[-1.8ex]
\multicolumn{4}{l}{$\bar{n}=5$} \\  & \nopagebreak $\;J=50$  & $\phantom{-}0.00\phantom{0}$ & $\phantom{-}0.00\phantom{0}$ & $\phantom{-}0.00\phantom{0}$ & $\phantom{-}0.01\phantom{0}$ & ${-}0.01\phantom{0}$ & $\phantom{0}0.14\phantom{0}$ & $\phantom{0}0.16\phantom{0}$ & $\phantom{0}0.16\phantom{0}$ & $\phantom{0}0.16\phantom{0}$ & $\phantom{0}0.16\phantom{0}$ & $\phantom{0}95.1\phantom{0}$ & $\phantom{0}95.5\phantom{0}$ & $\phantom{0}95.2\phantom{0}$ & $\phantom{0}94.5\phantom{0}$ & $\phantom{0}95.2\phantom{0}$ \\
 & \nopagebreak $\;J=200$  & $\phantom{-}0.00\phantom{0}$ & $\phantom{-}0.01\phantom{0}$ & $\phantom{-}0.00\phantom{0}$ & $\phantom{-}0.01\phantom{0}$ & ${-}0.00\phantom{0}$ & $\phantom{0}0.07\phantom{0}$ & $\phantom{0}0.08\phantom{0}$ & $\phantom{0}0.08\phantom{0}$ & $\phantom{0}0.08\phantom{0}$ & $\phantom{0}0.08\phantom{0}$ & $\phantom{0}94.1\phantom{0}$ & $\phantom{0}94.4\phantom{0}$ & $\phantom{0}93.6\phantom{0}$ & $\phantom{0}94.3\phantom{0}$ & $\phantom{0}94.4\phantom{0}$ \\
 & \nopagebreak $\;J=1000$  & $\phantom{-}0.00\phantom{0}$ & $\phantom{-}0.00\phantom{0}$ & $\phantom{-}0.00\phantom{0}$ & $\phantom{-}0.00\phantom{0}$ & $\phantom{-}0.00\phantom{0}$ & $\phantom{0}0.03\phantom{0}$ & $\phantom{0}0.03\phantom{0}$ & $\phantom{0}0.03\phantom{0}$ & $\phantom{0}0.03\phantom{0}$ & $\phantom{0}0.03\phantom{0}$ & $\phantom{0}96.3\phantom{0}$ & $\phantom{0}96.1\phantom{0}$ & $\phantom{0}95.4\phantom{0}$ & $\phantom{0}95.7\phantom{0}$ & $\phantom{0}96.0\phantom{0}$ \\
\multicolumn{4}{l}{$\bar{n}=20$} \\  & \nopagebreak $\;J=50$  & $\phantom{-}0.00\phantom{0}$ & $\phantom{-}0.01\phantom{0}$ & $\phantom{-}0.01\phantom{0}$ & $\phantom{-}0.01\phantom{0}$ & ${-}0.02\phantom{0}$ & $\phantom{0}0.14\phantom{0}$ & $\phantom{0}0.16\phantom{0}$ & $\phantom{0}0.16\phantom{0}$ & $\phantom{0}0.16\phantom{0}$ & $\phantom{0}0.16\phantom{0}$ & $\phantom{0}94.7\phantom{0}$ & $\phantom{0}94.5\phantom{0}$ & $\phantom{0}95.2\phantom{0}$ & $\phantom{0}95.0\phantom{0}$ & $\phantom{0}94.2\phantom{0}$ \\
 & \nopagebreak $\;J=200$  & ${-}0.00\phantom{0}$ & ${-}0.00\phantom{0}$ & ${-}0.00\phantom{0}$ & ${-}0.00\phantom{0}$ & ${-}0.01\phantom{0}$ & $\phantom{0}0.07\phantom{0}$ & $\phantom{0}0.08\phantom{0}$ & $\phantom{0}0.08\phantom{0}$ & $\phantom{0}0.08\phantom{0}$ & $\phantom{0}0.08\phantom{0}$ & $\phantom{0}94.6\phantom{0}$ & $\phantom{0}94.9\phantom{0}$ & $\phantom{0}95.1\phantom{0}$ & $\phantom{0}95.5\phantom{0}$ & $\phantom{0}94.5\phantom{0}$ \\
 & \nopagebreak $\;J=1000$  & ${-}0.00\phantom{0}$ & ${-}0.00\phantom{0}$ & ${-}0.00\phantom{0}$ & ${-}0.00\phantom{0}$ & ${-}0.00\phantom{0}$ & $\phantom{0}0.03\phantom{0}$ & $\phantom{0}0.04\phantom{0}$ & $\phantom{0}0.04\phantom{0}$ & $\phantom{0}0.04\phantom{0}$ & $\phantom{0}0.04\phantom{0}$ & $\phantom{0}95.1\phantom{0}$ & $\phantom{0}95.0\phantom{0}$ & $\phantom{0}95.0\phantom{0}$ & $\phantom{0}95.0\phantom{0}$ & $\phantom{0}94.4\phantom{0}$ \\
[0.5ex]\hline\\[-1.6ex] 
& & \multicolumn{15}{c}{Moderate intraclass correlation $(\rho_{Iy}=.30)$} \\[0.6ex]\hline\\[-1.8ex]
\multicolumn{4}{l}{$\bar{n}=5$} \\  & \nopagebreak $\;J=50$  & $\phantom{-}0.00\phantom{0}$ & $\phantom{-}0.01\phantom{0}$ & $\phantom{-}0.01\phantom{0}$ & $\phantom{-}0.01\phantom{0}$ & ${-}0.01\phantom{0}$ & $\phantom{0}0.14\phantom{0}$ & $\phantom{0}0.16\phantom{0}$ & $\phantom{0}0.16\phantom{0}$ & $\phantom{0}0.16\phantom{0}$ & $\phantom{0}0.16\phantom{0}$ & $\phantom{0}94.6\phantom{0}$ & $\phantom{0}95.2\phantom{0}$ & $\phantom{0}94.6\phantom{0}$ & $\phantom{0}95.0\phantom{0}$ & $\phantom{0}94.7\phantom{0}$ \\
 & \nopagebreak $\;J=200$  & $\phantom{-}0.00\phantom{0}$ & $\phantom{-}0.00\phantom{0}$ & ${-}0.00\phantom{0}$ & ${-}0.00\phantom{0}$ & ${-}0.00\phantom{0}$ & $\phantom{0}0.07\phantom{0}$ & $\phantom{0}0.08\phantom{0}$ & $\phantom{0}0.08\phantom{0}$ & $\phantom{0}0.07\phantom{0}$ & $\phantom{0}0.08\phantom{0}$ & $\phantom{0}95.6\phantom{0}$ & $\phantom{0}95.2\phantom{0}$ & $\phantom{0}95.8\phantom{0}$ & $\phantom{0}96.1\phantom{0}$ & $\phantom{0}95.6\phantom{0}$ \\
 & \nopagebreak $\;J=1000$  & ${-}0.00\phantom{0}$ & ${-}0.00\phantom{0}$ & ${-}0.00\phantom{0}$ & ${-}0.00\phantom{0}$ & ${-}0.00\phantom{0}$ & $\phantom{0}0.03\phantom{0}$ & $\phantom{0}0.03\phantom{0}$ & $\phantom{0}0.03\phantom{0}$ & $\phantom{0}0.03\phantom{0}$ & $\phantom{0}0.03\phantom{0}$ & $\phantom{0}95.7\phantom{0}$ & $\phantom{0}95.0\phantom{0}$ & $\phantom{0}95.1\phantom{0}$ & $\phantom{0}95.0\phantom{0}$ & $\phantom{0}95.5\phantom{0}$ \\
\multicolumn{4}{l}{$\bar{n}=20$} \\  & \nopagebreak $\;J=50$  & $\phantom{-}0.00\phantom{0}$ & $\phantom{-}0.01\phantom{0}$ & $\phantom{-}0.01\phantom{0}$ & $\phantom{-}0.01\phantom{0}$ & ${-}0.00\phantom{0}$ & $\phantom{0}0.14\phantom{0}$ & $\phantom{0}0.16\phantom{0}$ & $\phantom{0}0.16\phantom{0}$ & $\phantom{0}0.16\phantom{0}$ & $\phantom{0}0.16\phantom{0}$ & $\phantom{0}94.5\phantom{0}$ & $\phantom{0}94.9\phantom{0}$ & $\phantom{0}96.0\phantom{0}$ & $\phantom{0}94.7\phantom{0}$ & $\phantom{0}95.2\phantom{0}$ \\
 & \nopagebreak $\;J=200$  & ${-}0.00\phantom{0}$ & ${-}0.00\phantom{0}$ & $\phantom{-}0.00\phantom{0}$ & ${-}0.00\phantom{0}$ & ${-}0.00\phantom{0}$ & $\phantom{0}0.07\phantom{0}$ & $\phantom{0}0.08\phantom{0}$ & $\phantom{0}0.08\phantom{0}$ & $\phantom{0}0.08\phantom{0}$ & $\phantom{0}0.08\phantom{0}$ & $\phantom{0}94.9\phantom{0}$ & $\phantom{0}94.5\phantom{0}$ & $\phantom{0}94.6\phantom{0}$ & $\phantom{0}94.8\phantom{0}$ & $\phantom{0}94.5\phantom{0}$ \\
 & \nopagebreak $\;J=1000$  & ${-}0.00\phantom{0}$ & ${-}0.00\phantom{0}$ & ${-}0.00\phantom{0}$ & ${-}0.00\phantom{0}$ & ${-}0.00\phantom{0}$ & $\phantom{0}0.03\phantom{0}$ & $\phantom{0}0.04\phantom{0}$ & $\phantom{0}0.04\phantom{0}$ & $\phantom{0}0.04\phantom{0}$ & $\phantom{0}0.04\phantom{0}$ & $\phantom{0}94.1\phantom{0}$ & $\phantom{0}93.9\phantom{0}$ & $\phantom{0}93.7\phantom{0}$ & $\phantom{0}93.7\phantom{0}$ & $\phantom{0}93.8\phantom{0}$ \\
[0.5ex]\hline\\[-1.6ex] 
\end{tabular}
\begin{tablenotes}[para,flushleft]{\footnotesize \textit{Note.} $\bar{n}$ = average cluster size; $J$ = number of clusters; CD = complete data sets; LD = listwise deletion; FCS-SL = single-level FCS; FCS-MAN = two-level FCS with manifest cluster means; FCS-LAT = two-level FCS with latent cluster means; JM = joint modeling.}\end{tablenotes}
\end{threeparttable}
\end{sidewaystable}
\begin{sidewaystable}
\begin{threeparttable}
\setlength{\tabcolsep}{1.0pt}
\renewcommand{\arraystretch}{0.95}
\footnotesize
\caption{\small Study 2: Bias, Relative RMSE, and Coverage of the 95\% Confidence Interval for the Mean of $z$ ($\hat\mu_z$) With Moderately Unbalanced Data (Bimodal, $\pm 40\%$) and 20\% Missing Data (MAR, $\lambda=0.5$)}
\begin{tabular}{llccccccccccccccc}
\hline\\[-1.8ex]
& & \multicolumn{5}{c}{Bias (\%)} & \multicolumn{5}{c}{Rel. RMSE} & \multicolumn{5}{c}{Coverage (\%)} \\ \cmidrule(r){3-7}\cmidrule(r){8-12}\cmidrule(r){13-17}
 &  & CD & \makecell{FCS-\\MAN} & \makecell{FCS-\\NJ} & \makecell{FCS-\\LAT} & JM & CD & \makecell{FCS-\\MAN} & \makecell{FCS-\\NJ} & \makecell{FCS-\\LAT} & JM & CD & \makecell{FCS-\\MAN} & \makecell{FCS-\\NJ} & \makecell{FCS-\\LAT} & \multicolumn{1}{c}{JM} \\ 
[0.4ex]\hline\\[-1.8ex]
& & \multicolumn{15}{c}{Small intraclass correlation $(\rho_{Iy}=.10)$} \\[0.6ex]\hline\\[-1.8ex]
\multicolumn{4}{l}{$\bar{n}=5$} \\  & \nopagebreak $\;J=50$  & $\phantom{-}0.00\phantom{0}$ & $\phantom{-}0.01\phantom{0}$ & $\phantom{-}0.01\phantom{0}$ & $\phantom{-}0.01\phantom{0}$ & ${-}0.01\phantom{0}$ & $\phantom{0}0.15\phantom{0}$ & $\phantom{0}0.17\phantom{0}$ & $\phantom{0}0.17\phantom{0}$ & $\phantom{0}0.17\phantom{0}$ & $\phantom{0}0.16\phantom{0}$ & $\phantom{0}94.0\phantom{0}$ & $\phantom{0}94.0\phantom{0}$ & $\phantom{0}95.2\phantom{0}$ & $\phantom{0}94.4\phantom{0}$ & $\phantom{0}94.4\phantom{0}$ \\
 & \nopagebreak $\;J=200$  & $\phantom{-}0.00\phantom{0}$ & $\phantom{-}0.00\phantom{0}$ & $\phantom{-}0.00\phantom{0}$ & $\phantom{-}0.00\phantom{0}$ & ${-}0.01\phantom{0}$ & $\phantom{0}0.07\phantom{0}$ & $\phantom{0}0.08\phantom{0}$ & $\phantom{0}0.08\phantom{0}$ & $\phantom{0}0.08\phantom{0}$ & $\phantom{0}0.08\phantom{0}$ & $\phantom{0}94.6\phantom{0}$ & $\phantom{0}94.9\phantom{0}$ & $\phantom{0}95.2\phantom{0}$ & $\phantom{0}95.3\phantom{0}$ & $\phantom{0}94.2\phantom{0}$ \\
 & \nopagebreak $\;J=1000$  & ${-}0.00\phantom{0}$ & ${-}0.00\phantom{0}$ & ${-}0.00\phantom{0}$ & ${-}0.00\phantom{0}$ & ${-}0.00\phantom{0}$ & $\phantom{0}0.03\phantom{0}$ & $\phantom{0}0.04\phantom{0}$ & $\phantom{0}0.04\phantom{0}$ & $\phantom{0}0.04\phantom{0}$ & $\phantom{0}0.04\phantom{0}$ & $\phantom{0}94.7\phantom{0}$ & $\phantom{0}95.0\phantom{0}$ & $\phantom{0}94.5\phantom{0}$ & $\phantom{0}94.7\phantom{0}$ & $\phantom{0}94.6\phantom{0}$ \\
\multicolumn{4}{l}{$\bar{n}=20$} \\  & \nopagebreak $\;J=50$  & ${-}0.00\phantom{0}$ & ${-}0.00\phantom{0}$ & ${-}0.00\phantom{0}$ & ${-}0.00\phantom{0}$ & ${-}0.03\phantom{0}$ & $\phantom{0}0.14\phantom{0}$ & $\phantom{0}0.16\phantom{0}$ & $\phantom{0}0.16\phantom{0}$ & $\phantom{0}0.16\phantom{0}$ & $\phantom{0}0.16\phantom{0}$ & $\phantom{0}93.6\phantom{0}$ & $\phantom{0}94.4\phantom{0}$ & $\phantom{0}94.5\phantom{0}$ & $\phantom{0}93.7\phantom{0}$ & $\phantom{0}93.6\phantom{0}$ \\
 & \nopagebreak $\;J=200$  & $\phantom{-}0.00\phantom{0}$ & $\phantom{-}0.00\phantom{0}$ & $\phantom{-}0.00\phantom{0}$ & $\phantom{-}0.00\phantom{0}$ & ${-}0.00\phantom{0}$ & $\phantom{0}0.07\phantom{0}$ & $\phantom{0}0.08\phantom{0}$ & $\phantom{0}0.08\phantom{0}$ & $\phantom{0}0.08\phantom{0}$ & $\phantom{0}0.08\phantom{0}$ & $\phantom{0}93.9\phantom{0}$ & $\phantom{0}94.3\phantom{0}$ & $\phantom{0}94.2\phantom{0}$ & $\phantom{0}93.8\phantom{0}$ & $\phantom{0}94.5\phantom{0}$ \\
 & \nopagebreak $\;J=1000$  & ${-}0.00\phantom{0}$ & $\phantom{-}0.00\phantom{0}$ & $\phantom{-}0.00\phantom{0}$ & $\phantom{-}0.00\phantom{0}$ & ${-}0.00\phantom{0}$ & $\phantom{0}0.03\phantom{0}$ & $\phantom{0}0.04\phantom{0}$ & $\phantom{0}0.04\phantom{0}$ & $\phantom{0}0.04\phantom{0}$ & $\phantom{0}0.04\phantom{0}$ & $\phantom{0}94.6\phantom{0}$ & $\phantom{0}95.4\phantom{0}$ & $\phantom{0}95.1\phantom{0}$ & $\phantom{0}95.1\phantom{0}$ & $\phantom{0}94.7\phantom{0}$ \\
[0.5ex]\hline\\[-1.6ex] 
& & \multicolumn{15}{c}{Moderate intraclass correlation $(\rho_{Iy}=.30)$} \\[0.6ex]\hline\\[-1.8ex]
\multicolumn{4}{l}{$\bar{n}=5$} \\  & \nopagebreak $\;J=50$  & $\phantom{-}0.01\phantom{0}$ & $\phantom{-}0.00\phantom{0}$ & $\phantom{-}0.01\phantom{0}$ & $\phantom{-}0.01\phantom{0}$ & ${-}0.01\phantom{0}$ & $\phantom{0}0.14\phantom{0}$ & $\phantom{0}0.16\phantom{0}$ & $\phantom{0}0.16\phantom{0}$ & $\phantom{0}0.16\phantom{0}$ & $\phantom{0}0.16\phantom{0}$ & $\phantom{0}94.6\phantom{0}$ & $\phantom{0}95.4\phantom{0}$ & $\phantom{0}95.7\phantom{0}$ & $\phantom{0}95.3\phantom{0}$ & $\phantom{0}95.4\phantom{0}$ \\
 & \nopagebreak $\;J=200$  & $\phantom{-}0.00\phantom{0}$ & $\phantom{-}0.00\phantom{0}$ & $\phantom{-}0.00\phantom{0}$ & $\phantom{-}0.00\phantom{0}$ & ${-}0.00\phantom{0}$ & $\phantom{0}0.07\phantom{0}$ & $\phantom{0}0.08\phantom{0}$ & $\phantom{0}0.08\phantom{0}$ & $\phantom{0}0.08\phantom{0}$ & $\phantom{0}0.08\phantom{0}$ & $\phantom{0}92.7\phantom{0}$ & $\phantom{0}93.4\phantom{0}$ & $\phantom{0}93.4\phantom{0}$ & $\phantom{0}93.4\phantom{0}$ & $\phantom{0}94.0\phantom{0}$ \\
 & \nopagebreak $\;J=1000$  & ${-}0.00\phantom{0}$ & ${-}0.00\phantom{0}$ & ${-}0.00\phantom{0}$ & ${-}0.00\phantom{0}$ & ${-}0.00\phantom{0}$ & $\phantom{0}0.03\phantom{0}$ & $\phantom{0}0.04\phantom{0}$ & $\phantom{0}0.04\phantom{0}$ & $\phantom{0}0.04\phantom{0}$ & $\phantom{0}0.04\phantom{0}$ & $\phantom{0}96.0\phantom{0}$ & $\phantom{0}95.2\phantom{0}$ & $\phantom{0}95.0\phantom{0}$ & $\phantom{0}95.0\phantom{0}$ & $\phantom{0}95.6\phantom{0}$ \\
\multicolumn{4}{l}{$\bar{n}=20$} \\  & \nopagebreak $\;J=50$  & ${-}0.00\phantom{0}$ & ${-}0.00\phantom{0}$ & ${-}0.00\phantom{0}$ & ${-}0.00\phantom{0}$ & ${-}0.01\phantom{0}$ & $\phantom{0}0.14\phantom{0}$ & $\phantom{0}0.16\phantom{0}$ & $\phantom{0}0.16\phantom{0}$ & $\phantom{0}0.16\phantom{0}$ & $\phantom{0}0.16\phantom{0}$ & $\phantom{0}95.5\phantom{0}$ & $\phantom{0}94.2\phantom{0}$ & $\phantom{0}94.8\phantom{0}$ & $\phantom{0}94.9\phantom{0}$ & $\phantom{0}94.0\phantom{0}$ \\
 & \nopagebreak $\;J=200$  & $\phantom{-}0.00\phantom{0}$ & $\phantom{-}0.00\phantom{0}$ & $\phantom{-}0.00\phantom{0}$ & $\phantom{-}0.00\phantom{0}$ & $\phantom{-}0.00\phantom{0}$ & $\phantom{0}0.07\phantom{0}$ & $\phantom{0}0.08\phantom{0}$ & $\phantom{0}0.08\phantom{0}$ & $\phantom{0}0.08\phantom{0}$ & $\phantom{0}0.08\phantom{0}$ & $\phantom{0}95.3\phantom{0}$ & $\phantom{0}95.4\phantom{0}$ & $\phantom{0}95.4\phantom{0}$ & $\phantom{0}94.5\phantom{0}$ & $\phantom{0}95.3\phantom{0}$ \\
 & \nopagebreak $\;J=1000$  & $\phantom{-}0.00\phantom{0}$ & $\phantom{-}0.00\phantom{0}$ & $\phantom{-}0.00\phantom{0}$ & $\phantom{-}0.00\phantom{0}$ & $\phantom{-}0.00\phantom{0}$ & $\phantom{0}0.03\phantom{0}$ & $\phantom{0}0.04\phantom{0}$ & $\phantom{0}0.04\phantom{0}$ & $\phantom{0}0.04\phantom{0}$ & $\phantom{0}0.04\phantom{0}$ & $\phantom{0}94.4\phantom{0}$ & $\phantom{0}93.1\phantom{0}$ & $\phantom{0}93.3\phantom{0}$ & $\phantom{0}93.3\phantom{0}$ & $\phantom{0}93.0\phantom{0}$ \\
[0.5ex]\hline\\[-1.6ex] 
\end{tabular}
\begin{tablenotes}[para,flushleft]{\footnotesize \textit{Note.} $\bar{n}$ = average cluster size; $J$ = number of clusters; CD = complete data sets; LD = listwise deletion; FCS-SL = single-level FCS; FCS-MAN = two-level FCS with manifest cluster means; FCS-LAT = two-level FCS with latent cluster means; JM = joint modeling.}\end{tablenotes}
\end{threeparttable}
\end{sidewaystable}
\begin{sidewaystable}
\begin{threeparttable}
\setlength{\tabcolsep}{1.0pt}
\renewcommand{\arraystretch}{0.95}
\footnotesize
\caption{\small Study 2: Bias, Relative RMSE, and Coverage of the 95\% Confidence Interval for the Mean of $z$ ($\hat\mu_z$) With Strongly Unbalanced Data (Bimodal, $\pm 80\%$) and 20\% Missing Data (MAR, $\lambda=0.5$)}
\begin{tabular}{llccccccccccccccc}
\hline\\[-1.8ex]
& & \multicolumn{5}{c}{Bias (\%)} & \multicolumn{5}{c}{Rel. RMSE} & \multicolumn{5}{c}{Coverage (\%)} \\ \cmidrule(r){3-7}\cmidrule(r){8-12}\cmidrule(r){13-17}
 &  & CD & \makecell{FCS-\\MAN} & \makecell{FCS-\\NJ} & \makecell{FCS-\\LAT} & JM & CD & \makecell{FCS-\\MAN} & \makecell{FCS-\\NJ} & \makecell{FCS-\\LAT} & JM & CD & \makecell{FCS-\\MAN} & \makecell{FCS-\\NJ} & \makecell{FCS-\\LAT} & \multicolumn{1}{c}{JM} \\ 
[0.4ex]\hline\\[-1.8ex]
& & \multicolumn{15}{c}{Small intraclass correlation $(\rho_{Iy}=.10)$} \\[0.6ex]\hline\\[-1.8ex]
\multicolumn{4}{l}{$\bar{n}=5$} \\  & \nopagebreak $\;J=50$  & ${-}0.00\phantom{0}$ & $\phantom{-}0.00\phantom{0}$ & ${-}0.00\phantom{0}$ & $\phantom{-}0.00\phantom{0}$ & ${-}0.01\phantom{0}$ & $\phantom{0}0.14\phantom{0}$ & $\phantom{0}0.17\phantom{0}$ & $\phantom{0}0.17\phantom{0}$ & $\phantom{0}0.17\phantom{0}$ & $\phantom{0}0.17\phantom{0}$ & $\phantom{0}94.1\phantom{0}$ & $\phantom{0}93.5\phantom{0}$ & $\phantom{0}93.9\phantom{0}$ & $\phantom{0}93.5\phantom{0}$ & $\phantom{0}93.3\phantom{0}$ \\
 & \nopagebreak $\;J=200$  & $\phantom{-}0.00\phantom{0}$ & $\phantom{-}0.00\phantom{0}$ & $\phantom{-}0.00\phantom{0}$ & $\phantom{-}0.00\phantom{0}$ & ${-}0.00\phantom{0}$ & $\phantom{0}0.07\phantom{0}$ & $\phantom{0}0.08\phantom{0}$ & $\phantom{0}0.08\phantom{0}$ & $\phantom{0}0.08\phantom{0}$ & $\phantom{0}0.08\phantom{0}$ & $\phantom{0}95.3\phantom{0}$ & $\phantom{0}95.9\phantom{0}$ & $\phantom{0}95.6\phantom{0}$ & $\phantom{0}96.1\phantom{0}$ & $\phantom{0}95.8\phantom{0}$ \\
 & \nopagebreak $\;J=1000$  & $\phantom{-}0.00\phantom{0}$ & $\phantom{-}0.00\phantom{0}$ & $\phantom{-}0.00\phantom{0}$ & $\phantom{-}0.00\phantom{0}$ & $\phantom{-}0.00\phantom{0}$ & $\phantom{0}0.03\phantom{0}$ & $\phantom{0}0.04\phantom{0}$ & $\phantom{0}0.04\phantom{0}$ & $\phantom{0}0.04\phantom{0}$ & $\phantom{0}0.04\phantom{0}$ & $\phantom{0}95.2\phantom{0}$ & $\phantom{0}95.1\phantom{0}$ & $\phantom{0}95.8\phantom{0}$ & $\phantom{0}95.5\phantom{0}$ & $\phantom{0}95.6\phantom{0}$ \\
\multicolumn{4}{l}{$\bar{n}=20$} \\  & \nopagebreak $\;J=50$  & $\phantom{-}0.00\phantom{0}$ & $\phantom{-}0.01\phantom{0}$ & $\phantom{-}0.01\phantom{0}$ & $\phantom{-}0.01\phantom{0}$ & ${-}0.01\phantom{0}$ & $\phantom{0}0.14\phantom{0}$ & $\phantom{0}0.16\phantom{0}$ & $\phantom{0}0.16\phantom{0}$ & $\phantom{0}0.16\phantom{0}$ & $\phantom{0}0.16\phantom{0}$ & $\phantom{0}93.7\phantom{0}$ & $\phantom{0}94.6\phantom{0}$ & $\phantom{0}93.9\phantom{0}$ & $\phantom{0}94.2\phantom{0}$ & $\phantom{0}93.6\phantom{0}$ \\
 & \nopagebreak $\;J=200$  & $\phantom{-}0.00\phantom{0}$ & $\phantom{-}0.01\phantom{0}$ & $\phantom{-}0.00\phantom{0}$ & $\phantom{-}0.00\phantom{0}$ & ${-}0.00\phantom{0}$ & $\phantom{0}0.07\phantom{0}$ & $\phantom{0}0.08\phantom{0}$ & $\phantom{0}0.08\phantom{0}$ & $\phantom{0}0.08\phantom{0}$ & $\phantom{0}0.08\phantom{0}$ & $\phantom{0}94.8\phantom{0}$ & $\phantom{0}95.9\phantom{0}$ & $\phantom{0}95.9\phantom{0}$ & $\phantom{0}95.3\phantom{0}$ & $\phantom{0}95.7\phantom{0}$ \\
 & \nopagebreak $\;J=1000$  & $\phantom{-}0.00\phantom{0}$ & $\phantom{-}0.00\phantom{0}$ & $\phantom{-}0.00\phantom{0}$ & $\phantom{-}0.00\phantom{0}$ & ${-}0.00\phantom{0}$ & $\phantom{0}0.03\phantom{0}$ & $\phantom{0}0.04\phantom{0}$ & $\phantom{0}0.04\phantom{0}$ & $\phantom{0}0.04\phantom{0}$ & $\phantom{0}0.04\phantom{0}$ & $\phantom{0}93.2\phantom{0}$ & $\phantom{0}93.0\phantom{0}$ & $\phantom{0}92.5\phantom{0}$ & $\phantom{0}92.3\phantom{0}$ & $\phantom{0}92.4\phantom{0}$ \\
[0.5ex]\hline\\[-1.6ex] 
& & \multicolumn{15}{c}{Moderate intraclass correlation $(\rho_{Iy}=.30)$} \\[0.6ex]\hline\\[-1.8ex]
\multicolumn{4}{l}{$\bar{n}=5$} \\  & \nopagebreak $\;J=50$  & ${-}0.00\phantom{0}$ & $\phantom{-}0.00\phantom{0}$ & ${-}0.00\phantom{0}$ & $\phantom{-}0.00\phantom{0}$ & ${-}0.01\phantom{0}$ & $\phantom{0}0.13\phantom{0}$ & $\phantom{0}0.15\phantom{0}$ & $\phantom{0}0.16\phantom{0}$ & $\phantom{0}0.15\phantom{0}$ & $\phantom{0}0.15\phantom{0}$ & $\phantom{0}96.0\phantom{0}$ & $\phantom{0}95.7\phantom{0}$ & $\phantom{0}94.8\phantom{0}$ & $\phantom{0}95.0\phantom{0}$ & $\phantom{0}95.1\phantom{0}$ \\
 & \nopagebreak $\;J=200$  & ${-}0.00\phantom{0}$ & $\phantom{-}0.00\phantom{0}$ & ${-}0.00\phantom{0}$ & ${-}0.00\phantom{0}$ & ${-}0.00\phantom{0}$ & $\phantom{0}0.07\phantom{0}$ & $\phantom{0}0.08\phantom{0}$ & $\phantom{0}0.08\phantom{0}$ & $\phantom{0}0.08\phantom{0}$ & $\phantom{0}0.08\phantom{0}$ & $\phantom{0}94.7\phantom{0}$ & $\phantom{0}95.3\phantom{0}$ & $\phantom{0}95.7\phantom{0}$ & $\phantom{0}95.0\phantom{0}$ & $\phantom{0}94.8\phantom{0}$ \\
 & \nopagebreak $\;J=1000$  & $\phantom{-}0.00\phantom{0}$ & $\phantom{-}0.00\phantom{0}$ & ${-}0.00\phantom{0}$ & ${-}0.00\phantom{0}$ & ${-}0.00\phantom{0}$ & $\phantom{0}0.03\phantom{0}$ & $\phantom{0}0.04\phantom{0}$ & $\phantom{0}0.04\phantom{0}$ & $\phantom{0}0.04\phantom{0}$ & $\phantom{0}0.04\phantom{0}$ & $\phantom{0}95.3\phantom{0}$ & $\phantom{0}95.4\phantom{0}$ & $\phantom{0}95.8\phantom{0}$ & $\phantom{0}95.5\phantom{0}$ & $\phantom{0}95.4\phantom{0}$ \\
\multicolumn{4}{l}{$\bar{n}=20$} \\  & \nopagebreak $\;J=50$  & ${-}0.00\phantom{0}$ & $\phantom{-}0.00\phantom{0}$ & ${-}0.00\phantom{0}$ & $\phantom{-}0.00\phantom{0}$ & ${-}0.01\phantom{0}$ & $\phantom{0}0.14\phantom{0}$ & $\phantom{0}0.16\phantom{0}$ & $\phantom{0}0.16\phantom{0}$ & $\phantom{0}0.16\phantom{0}$ & $\phantom{0}0.16\phantom{0}$ & $\phantom{0}94.5\phantom{0}$ & $\phantom{0}95.1\phantom{0}$ & $\phantom{0}95.4\phantom{0}$ & $\phantom{0}94.9\phantom{0}$ & $\phantom{0}95.1\phantom{0}$ \\
 & \nopagebreak $\;J=200$  & $\phantom{-}0.00\phantom{0}$ & $\phantom{-}0.00\phantom{0}$ & $\phantom{-}0.00\phantom{0}$ & $\phantom{-}0.00\phantom{0}$ & $\phantom{-}0.00\phantom{0}$ & $\phantom{0}0.07\phantom{0}$ & $\phantom{0}0.08\phantom{0}$ & $\phantom{0}0.08\phantom{0}$ & $\phantom{0}0.08\phantom{0}$ & $\phantom{0}0.08\phantom{0}$ & $\phantom{0}93.5\phantom{0}$ & $\phantom{0}95.2\phantom{0}$ & $\phantom{0}94.6\phantom{0}$ & $\phantom{0}93.8\phantom{0}$ & $\phantom{0}94.7\phantom{0}$ \\
 & \nopagebreak $\;J=1000$  & ${-}0.00\phantom{0}$ & ${-}0.00\phantom{0}$ & ${-}0.00\phantom{0}$ & ${-}0.00\phantom{0}$ & ${-}0.00\phantom{0}$ & $\phantom{0}0.03\phantom{0}$ & $\phantom{0}0.04\phantom{0}$ & $\phantom{0}0.04\phantom{0}$ & $\phantom{0}0.04\phantom{0}$ & $\phantom{0}0.04\phantom{0}$ & $\phantom{0}94.5\phantom{0}$ & $\phantom{0}94.0\phantom{0}$ & $\phantom{0}93.6\phantom{0}$ & $\phantom{0}94.0\phantom{0}$ & $\phantom{0}93.8\phantom{0}$ \\
[0.5ex]\hline\\[-1.6ex] 
\end{tabular}
\begin{tablenotes}[para,flushleft]{\footnotesize \textit{Note.} $\bar{n}$ = average cluster size; $J$ = number of clusters; CD = complete data sets; LD = listwise deletion; FCS-SL = single-level FCS; FCS-MAN = two-level FCS with manifest cluster means; FCS-LAT = two-level FCS with latent cluster means; JM = joint modeling.}\end{tablenotes}
\end{threeparttable}
\end{sidewaystable}
\begin{sidewaystable}
\begin{threeparttable}
\setlength{\tabcolsep}{1.0pt}
\renewcommand{\arraystretch}{0.95}
\footnotesize
\caption{\small Study 2: Bias, Relative RMSE, and Coverage of the 95\% Confidence Interval for the Mean of $z$ ($\hat\mu_z$) With Moderately Unbalanced Data (Uniform, $\pm 40\%$) and 40\% Missing Data (MAR, $\lambda=0.5$)}
\begin{tabular}{llccccccccccccccc}
\hline\\[-1.8ex]
& & \multicolumn{5}{c}{Bias (\%)} & \multicolumn{5}{c}{Rel. RMSE} & \multicolumn{5}{c}{Coverage (\%)} \\ \cmidrule(r){3-7}\cmidrule(r){8-12}\cmidrule(r){13-17}
 &  & CD & \makecell{FCS-\\MAN} & \makecell{FCS-\\NJ} & \makecell{FCS-\\LAT} & JM & CD & \makecell{FCS-\\MAN} & \makecell{FCS-\\NJ} & \makecell{FCS-\\LAT} & JM & CD & \makecell{FCS-\\MAN} & \makecell{FCS-\\NJ} & \makecell{FCS-\\LAT} & \multicolumn{1}{c}{JM} \\ 
[0.4ex]\hline\\[-1.8ex]
& & \multicolumn{15}{c}{Small intraclass correlation $(\rho_{Iy}=.10)$} \\[0.6ex]\hline\\[-1.8ex]
\multicolumn{4}{l}{$\bar{n}=5$} \\  & \nopagebreak $\;J=50$  & ${-}0.00\phantom{0}$ & $\phantom{-}0.01\phantom{0}$ & $\phantom{-}0.01\phantom{0}$ & $\phantom{-}0.02\phantom{0}$ & ${-}0.04\phantom{0}$ & $\phantom{0}0.15\phantom{0}$ & $\phantom{0}0.20\phantom{0}$ & $\phantom{0}0.21\phantom{0}$ & $\phantom{0}0.21\phantom{0}$ & $\phantom{0}0.20\phantom{0}$ & $\phantom{0}92.6\phantom{0}$ & $\phantom{0}95.1\phantom{0}$ & $\phantom{0}94.2\phantom{0}$ & $\phantom{0}93.3\phantom{0}$ & $\phantom{0}92.8\phantom{0}$ \\
 & \nopagebreak $\;J=200$  & ${-}0.00\phantom{0}$ & ${-}0.01\phantom{0}$ & ${-}0.00\phantom{0}$ & $\phantom{-}0.00\phantom{0}$ & ${-}0.02\phantom{0}$ & $\phantom{0}0.07\phantom{0}$ & $\phantom{0}0.09\phantom{0}$ & $\phantom{0}0.09\phantom{0}$ & $\phantom{0}0.09\phantom{0}$ & $\phantom{0}0.09\phantom{0}$ & $\phantom{0}95.2\phantom{0}$ & $\phantom{0}95.4\phantom{0}$ & $\phantom{0}94.8\phantom{0}$ & $\phantom{0}94.9\phantom{0}$ & $\phantom{0}94.7\phantom{0}$ \\
 & \nopagebreak $\;J=1000$  & ${-}0.00\phantom{0}$ & ${-}0.00\phantom{0}$ & ${-}0.00\phantom{0}$ & ${-}0.00\phantom{0}$ & ${-}0.01\phantom{0}$ & $\phantom{0}0.03\phantom{0}$ & $\phantom{0}0.04\phantom{0}$ & $\phantom{0}0.04\phantom{0}$ & $\phantom{0}0.04\phantom{0}$ & $\phantom{0}0.04\phantom{0}$ & $\phantom{0}94.7\phantom{0}$ & $\phantom{0}93.7\phantom{0}$ & $\phantom{0}93.4\phantom{0}$ & $\phantom{0}94.4\phantom{0}$ & $\phantom{0}94.8\phantom{0}$ \\
\multicolumn{4}{l}{$\bar{n}=20$} \\  & \nopagebreak $\;J=50$  & ${-}0.00\phantom{0}$ & ${-}0.00\phantom{0}$ & ${-}0.00\phantom{0}$ & $\phantom{-}0.00\phantom{0}$ & ${-}0.05\phantom{0}$ & $\phantom{0}0.14\phantom{0}$ & $\phantom{0}0.18\phantom{0}$ & $\phantom{0}0.19\phantom{0}$ & $\phantom{0}0.19\phantom{0}$ & $\phantom{0}0.19\phantom{0}$ & $\phantom{0}95.2\phantom{0}$ & $\phantom{0}95.5\phantom{0}$ & $\phantom{0}95.9\phantom{0}$ & $\phantom{0}95.0\phantom{0}$ & $\phantom{0}93.8\phantom{0}$ \\
 & \nopagebreak $\;J=200$  & ${-}0.00\phantom{0}$ & $\phantom{-}0.00\phantom{0}$ & $\phantom{-}0.00\phantom{0}$ & $\phantom{-}0.00\phantom{0}$ & ${-}0.02\phantom{0}$ & $\phantom{0}0.07\phantom{0}$ & $\phantom{0}0.09\phantom{0}$ & $\phantom{0}0.09\phantom{0}$ & $\phantom{0}0.09\phantom{0}$ & $\phantom{0}0.09\phantom{0}$ & $\phantom{0}94.6\phantom{0}$ & $\phantom{0}93.7\phantom{0}$ & $\phantom{0}95.4\phantom{0}$ & $\phantom{0}95.0\phantom{0}$ & $\phantom{0}94.9\phantom{0}$ \\
 & \nopagebreak $\;J=1000$  & ${-}0.00\phantom{0}$ & ${-}0.00\phantom{0}$ & ${-}0.00\phantom{0}$ & ${-}0.00\phantom{0}$ & ${-}0.01\phantom{0}$ & $\phantom{0}0.03\phantom{0}$ & $\phantom{0}0.04\phantom{0}$ & $\phantom{0}0.04\phantom{0}$ & $\phantom{0}0.04\phantom{0}$ & $\phantom{0}0.04\phantom{0}$ & $\phantom{0}95.6\phantom{0}$ & $\phantom{0}95.6\phantom{0}$ & $\phantom{0}94.1\phantom{0}$ & $\phantom{0}95.2\phantom{0}$ & $\phantom{0}94.8\phantom{0}$ \\
[0.5ex]\hline\\[-1.6ex] 
& & \multicolumn{15}{c}{Moderate intraclass correlation $(\rho_{Iy}=.30)$} \\[0.6ex]\hline\\[-1.8ex]
\multicolumn{4}{l}{$\bar{n}=5$} \\  & \nopagebreak $\;J=50$  & ${-}0.00\phantom{0}$ & ${-}0.00\phantom{0}$ & ${-}0.00\phantom{0}$ & $\phantom{-}0.00\phantom{0}$ & ${-}0.03\phantom{0}$ & $\phantom{0}0.14\phantom{0}$ & $\phantom{0}0.19\phantom{0}$ & $\phantom{0}0.20\phantom{0}$ & $\phantom{0}0.19\phantom{0}$ & $\phantom{0}0.19\phantom{0}$ & $\phantom{0}94.0\phantom{0}$ & $\phantom{0}94.0\phantom{0}$ & $\phantom{0}95.4\phantom{0}$ & $\phantom{0}94.0\phantom{0}$ & $\phantom{0}93.5\phantom{0}$ \\
 & \nopagebreak $\;J=200$  & ${-}0.00\phantom{0}$ & ${-}0.00\phantom{0}$ & ${-}0.00\phantom{0}$ & ${-}0.00\phantom{0}$ & ${-}0.01\phantom{0}$ & $\phantom{0}0.07\phantom{0}$ & $\phantom{0}0.10\phantom{0}$ & $\phantom{0}0.10\phantom{0}$ & $\phantom{0}0.10\phantom{0}$ & $\phantom{0}0.09\phantom{0}$ & $\phantom{0}95.3\phantom{0}$ & $\phantom{0}94.6\phantom{0}$ & $\phantom{0}94.7\phantom{0}$ & $\phantom{0}93.8\phantom{0}$ & $\phantom{0}95.1\phantom{0}$ \\
 & \nopagebreak $\;J=1000$  & $\phantom{-}0.00\phantom{0}$ & ${-}0.00\phantom{0}$ & ${-}0.00\phantom{0}$ & $\phantom{-}0.00\phantom{0}$ & ${-}0.00\phantom{0}$ & $\phantom{0}0.03\phantom{0}$ & $\phantom{0}0.04\phantom{0}$ & $\phantom{0}0.04\phantom{0}$ & $\phantom{0}0.04\phantom{0}$ & $\phantom{0}0.04\phantom{0}$ & $\phantom{0}93.8\phantom{0}$ & $\phantom{0}94.1\phantom{0}$ & $\phantom{0}93.8\phantom{0}$ & $\phantom{0}93.7\phantom{0}$ & $\phantom{0}93.6\phantom{0}$ \\
\multicolumn{4}{l}{$\bar{n}=20$} \\  & \nopagebreak $\;J=50$  & $\phantom{-}0.00\phantom{0}$ & ${-}0.01\phantom{0}$ & ${-}0.01\phantom{0}$ & ${-}0.01\phantom{0}$ & ${-}0.03\phantom{0}$ & $\phantom{0}0.14\phantom{0}$ & $\phantom{0}0.19\phantom{0}$ & $\phantom{0}0.19\phantom{0}$ & $\phantom{0}0.19\phantom{0}$ & $\phantom{0}0.19\phantom{0}$ & $\phantom{0}94.3\phantom{0}$ & $\phantom{0}94.0\phantom{0}$ & $\phantom{0}95.5\phantom{0}$ & $\phantom{0}94.4\phantom{0}$ & $\phantom{0}95.2\phantom{0}$ \\
 & \nopagebreak $\;J=200$  & $\phantom{-}0.00\phantom{0}$ & $\phantom{-}0.00\phantom{0}$ & ${-}0.00\phantom{0}$ & $\phantom{-}0.00\phantom{0}$ & ${-}0.00\phantom{0}$ & $\phantom{0}0.07\phantom{0}$ & $\phantom{0}0.10\phantom{0}$ & $\phantom{0}0.10\phantom{0}$ & $\phantom{0}0.10\phantom{0}$ & $\phantom{0}0.10\phantom{0}$ & $\phantom{0}95.5\phantom{0}$ & $\phantom{0}94.3\phantom{0}$ & $\phantom{0}94.5\phantom{0}$ & $\phantom{0}94.2\phantom{0}$ & $\phantom{0}94.9\phantom{0}$ \\
 & \nopagebreak $\;J=1000$  & ${-}0.00\phantom{0}$ & $\phantom{-}0.00\phantom{0}$ & ${-}0.00\phantom{0}$ & $\phantom{-}0.00\phantom{0}$ & ${-}0.00\phantom{0}$ & $\phantom{0}0.03\phantom{0}$ & $\phantom{0}0.04\phantom{0}$ & $\phantom{0}0.04\phantom{0}$ & $\phantom{0}0.04\phantom{0}$ & $\phantom{0}0.04\phantom{0}$ & $\phantom{0}94.2\phantom{0}$ & $\phantom{0}94.3\phantom{0}$ & $\phantom{0}94.4\phantom{0}$ & $\phantom{0}94.1\phantom{0}$ & $\phantom{0}93.9\phantom{0}$ \\
[0.5ex]\hline\\[-1.6ex] 
\end{tabular}
\begin{tablenotes}[para,flushleft]{\footnotesize \textit{Note.} $\bar{n}$ = average cluster size; $J$ = number of clusters; CD = complete data sets; LD = listwise deletion; FCS-SL = single-level FCS; FCS-MAN = two-level FCS with manifest cluster means; FCS-LAT = two-level FCS with latent cluster means; JM = joint modeling.}\end{tablenotes}
\end{threeparttable}
\end{sidewaystable}
\begin{sidewaystable}
\begin{threeparttable}
\setlength{\tabcolsep}{1.0pt}
\renewcommand{\arraystretch}{0.95}
\footnotesize
\caption{\small Study 2: Bias, Relative RMSE, and Coverage of the 95\% Confidence Interval for the Mean of $z$ ($\hat\mu_z$) With Strongly Unbalanced Data (Uniform, $\pm 80\%$) and 40\% Missing Data (MAR, $\lambda=0.5$)}
\begin{tabular}{llccccccccccccccc}
\hline\\[-1.8ex]
& & \multicolumn{5}{c}{Bias (\%)} & \multicolumn{5}{c}{Rel. RMSE} & \multicolumn{5}{c}{Coverage (\%)} \\ \cmidrule(r){3-7}\cmidrule(r){8-12}\cmidrule(r){13-17}
 &  & CD & \makecell{FCS-\\MAN} & \makecell{FCS-\\NJ} & \makecell{FCS-\\LAT} & JM & CD & \makecell{FCS-\\MAN} & \makecell{FCS-\\NJ} & \makecell{FCS-\\LAT} & JM & CD & \makecell{FCS-\\MAN} & \makecell{FCS-\\NJ} & \makecell{FCS-\\LAT} & \multicolumn{1}{c}{JM} \\ 
[0.4ex]\hline\\[-1.8ex]
& & \multicolumn{15}{c}{Small intraclass correlation $(\rho_{Iy}=.10)$} \\[0.6ex]\hline\\[-1.8ex]
\multicolumn{4}{l}{$\bar{n}=5$} \\  & \nopagebreak $\;J=50$  & $\phantom{-}0.00\phantom{0}$ & ${-}0.00\phantom{0}$ & ${-}0.00\phantom{0}$ & $\phantom{-}0.01\phantom{0}$ & ${-}0.04\phantom{0}$ & $\phantom{0}0.15\phantom{0}$ & $\phantom{0}0.22\phantom{0}$ & $\phantom{0}0.22\phantom{0}$ & $\phantom{0}0.21\phantom{0}$ & $\phantom{0}0.20\phantom{0}$ & $\phantom{0}92.2\phantom{0}$ & $\phantom{0}91.4\phantom{0}$ & $\phantom{0}93.1\phantom{0}$ & $\phantom{0}92.3\phantom{0}$ & $\phantom{0}91.2\phantom{0}$ \\
 & \nopagebreak $\;J=200$  & $\phantom{-}0.00\phantom{0}$ & ${-}0.00\phantom{0}$ & $\phantom{-}0.00\phantom{0}$ & $\phantom{-}0.01\phantom{0}$ & ${-}0.02\phantom{0}$ & $\phantom{0}0.07\phantom{0}$ & $\phantom{0}0.10\phantom{0}$ & $\phantom{0}0.10\phantom{0}$ & $\phantom{0}0.10\phantom{0}$ & $\phantom{0}0.10\phantom{0}$ & $\phantom{0}95.7\phantom{0}$ & $\phantom{0}94.1\phantom{0}$ & $\phantom{0}94.7\phantom{0}$ & $\phantom{0}93.9\phantom{0}$ & $\phantom{0}93.5\phantom{0}$ \\
 & \nopagebreak $\;J=1000$  & ${-}0.00\phantom{0}$ & ${-}0.00\phantom{0}$ & ${-}0.00\phantom{0}$ & ${-}0.00\phantom{0}$ & ${-}0.01\phantom{0}$ & $\phantom{0}0.03\phantom{0}$ & $\phantom{0}0.04\phantom{0}$ & $\phantom{0}0.04\phantom{0}$ & $\phantom{0}0.04\phantom{0}$ & $\phantom{0}0.04\phantom{0}$ & $\phantom{0}95.5\phantom{0}$ & $\phantom{0}94.7\phantom{0}$ & $\phantom{0}95.0\phantom{0}$ & $\phantom{0}94.3\phantom{0}$ & $\phantom{0}93.7\phantom{0}$ \\
\multicolumn{4}{l}{$\bar{n}=20$} \\  & \nopagebreak $\;J=50$  & ${-}0.00\phantom{0}$ & ${-}0.00\phantom{0}$ & ${-}0.00\phantom{0}$ & $\phantom{-}0.00\phantom{0}$ & ${-}0.05\phantom{0}$ & $\phantom{0}0.14\phantom{0}$ & $\phantom{0}0.19\phantom{0}$ & $\phantom{0}0.19\phantom{0}$ & $\phantom{0}0.19\phantom{0}$ & $\phantom{0}0.19\phantom{0}$ & $\phantom{0}94.6\phantom{0}$ & $\phantom{0}95.0\phantom{0}$ & $\phantom{0}95.3\phantom{0}$ & $\phantom{0}95.8\phantom{0}$ & $\phantom{0}94.4\phantom{0}$ \\
 & \nopagebreak $\;J=200$  & ${-}0.00\phantom{0}$ & ${-}0.01\phantom{0}$ & ${-}0.01\phantom{0}$ & ${-}0.01\phantom{0}$ & ${-}0.03\phantom{0}$ & $\phantom{0}0.07\phantom{0}$ & $\phantom{0}0.09\phantom{0}$ & $\phantom{0}0.09\phantom{0}$ & $\phantom{0}0.09\phantom{0}$ & $\phantom{0}0.10\phantom{0}$ & $\phantom{0}94.9\phantom{0}$ & $\phantom{0}94.5\phantom{0}$ & $\phantom{0}94.8\phantom{0}$ & $\phantom{0}94.3\phantom{0}$ & $\phantom{0}94.1\phantom{0}$ \\
 & \nopagebreak $\;J=1000$  & $\phantom{-}0.00\phantom{0}$ & $\phantom{-}0.00\phantom{0}$ & $\phantom{-}0.00\phantom{0}$ & $\phantom{-}0.00\phantom{0}$ & ${-}0.00\phantom{0}$ & $\phantom{0}0.03\phantom{0}$ & $\phantom{0}0.04\phantom{0}$ & $\phantom{0}0.04\phantom{0}$ & $\phantom{0}0.04\phantom{0}$ & $\phantom{0}0.04\phantom{0}$ & $\phantom{0}94.7\phantom{0}$ & $\phantom{0}94.2\phantom{0}$ & $\phantom{0}94.8\phantom{0}$ & $\phantom{0}95.0\phantom{0}$ & $\phantom{0}95.1\phantom{0}$ \\
[0.5ex]\hline\\[-1.6ex] 
& & \multicolumn{15}{c}{Moderate intraclass correlation $(\rho_{Iy}=.30)$} \\[0.6ex]\hline\\[-1.8ex]
\multicolumn{4}{l}{$\bar{n}=5$} \\  & \nopagebreak $\;J=50$  & ${-}0.00\phantom{0}$ & ${-}0.01\phantom{0}$ & ${-}0.01\phantom{0}$ & ${-}0.00\phantom{0}$ & ${-}0.04\phantom{0}$ & $\phantom{0}0.14\phantom{0}$ & $\phantom{0}0.19\phantom{0}$ & $\phantom{0}0.20\phantom{0}$ & $\phantom{0}0.20\phantom{0}$ & $\phantom{0}0.19\phantom{0}$ & $\phantom{0}94.9\phantom{0}$ & $\phantom{0}94.1\phantom{0}$ & $\phantom{0}95.0\phantom{0}$ & $\phantom{0}92.9\phantom{0}$ & $\phantom{0}91.9\phantom{0}$ \\
 & \nopagebreak $\;J=200$  & ${-}0.00\phantom{0}$ & ${-}0.00\phantom{0}$ & ${-}0.00\phantom{0}$ & $\phantom{-}0.00\phantom{0}$ & ${-}0.01\phantom{0}$ & $\phantom{0}0.07\phantom{0}$ & $\phantom{0}0.09\phantom{0}$ & $\phantom{0}0.10\phantom{0}$ & $\phantom{0}0.09\phantom{0}$ & $\phantom{0}0.09\phantom{0}$ & $\phantom{0}95.5\phantom{0}$ & $\phantom{0}94.6\phantom{0}$ & $\phantom{0}95.3\phantom{0}$ & $\phantom{0}95.2\phantom{0}$ & $\phantom{0}95.2\phantom{0}$ \\
 & \nopagebreak $\;J=1000$  & ${-}0.00\phantom{0}$ & $\phantom{-}0.00\phantom{0}$ & $\phantom{-}0.00\phantom{0}$ & $\phantom{-}0.00\phantom{0}$ & $\phantom{-}0.00\phantom{0}$ & $\phantom{0}0.03\phantom{0}$ & $\phantom{0}0.04\phantom{0}$ & $\phantom{0}0.04\phantom{0}$ & $\phantom{0}0.04\phantom{0}$ & $\phantom{0}0.04\phantom{0}$ & $\phantom{0}94.4\phantom{0}$ & $\phantom{0}95.3\phantom{0}$ & $\phantom{0}94.9\phantom{0}$ & $\phantom{0}94.4\phantom{0}$ & $\phantom{0}94.6\phantom{0}$ \\
\multicolumn{4}{l}{$\bar{n}=20$} \\  & \nopagebreak $\;J=50$  & $\phantom{-}0.01\phantom{0}$ & $\phantom{-}0.01\phantom{0}$ & $\phantom{-}0.01\phantom{0}$ & $\phantom{-}0.01\phantom{0}$ & ${-}0.01\phantom{0}$ & $\phantom{0}0.14\phantom{0}$ & $\phantom{0}0.19\phantom{0}$ & $\phantom{0}0.20\phantom{0}$ & $\phantom{0}0.19\phantom{0}$ & $\phantom{0}0.19\phantom{0}$ & $\phantom{0}94.2\phantom{0}$ & $\phantom{0}94.1\phantom{0}$ & $\phantom{0}96.0\phantom{0}$ & $\phantom{0}94.5\phantom{0}$ & $\phantom{0}94.4\phantom{0}$ \\
 & \nopagebreak $\;J=200$  & $\phantom{-}0.00\phantom{0}$ & ${-}0.00\phantom{0}$ & ${-}0.00\phantom{0}$ & ${-}0.00\phantom{0}$ & ${-}0.01\phantom{0}$ & $\phantom{0}0.07\phantom{0}$ & $\phantom{0}0.09\phantom{0}$ & $\phantom{0}0.09\phantom{0}$ & $\phantom{0}0.09\phantom{0}$ & $\phantom{0}0.09\phantom{0}$ & $\phantom{0}94.5\phantom{0}$ & $\phantom{0}94.6\phantom{0}$ & $\phantom{0}94.5\phantom{0}$ & $\phantom{0}94.6\phantom{0}$ & $\phantom{0}95.3\phantom{0}$ \\
 & \nopagebreak $\;J=1000$  & ${-}0.00\phantom{0}$ & $\phantom{-}0.00\phantom{0}$ & ${-}0.00\phantom{0}$ & $\phantom{-}0.00\phantom{0}$ & ${-}0.00\phantom{0}$ & $\phantom{0}0.03\phantom{0}$ & $\phantom{0}0.04\phantom{0}$ & $\phantom{0}0.04\phantom{0}$ & $\phantom{0}0.04\phantom{0}$ & $\phantom{0}0.04\phantom{0}$ & $\phantom{0}95.7\phantom{0}$ & $\phantom{0}94.9\phantom{0}$ & $\phantom{0}95.9\phantom{0}$ & $\phantom{0}95.1\phantom{0}$ & $\phantom{0}95.6\phantom{0}$ \\
[0.5ex]\hline\\[-1.6ex] 
\end{tabular}
\begin{tablenotes}[para,flushleft]{\footnotesize \textit{Note.} $\bar{n}$ = average cluster size; $J$ = number of clusters; CD = complete data sets; LD = listwise deletion; FCS-SL = single-level FCS; FCS-MAN = two-level FCS with manifest cluster means; FCS-LAT = two-level FCS with latent cluster means; JM = joint modeling.}\end{tablenotes}
\end{threeparttable}
\end{sidewaystable}
\begin{sidewaystable}
\begin{threeparttable}
\setlength{\tabcolsep}{1.0pt}
\renewcommand{\arraystretch}{0.95}
\footnotesize
\caption{\small Study 2: Bias, Relative RMSE, and Coverage of the 95\% Confidence Interval for the Mean of $z$ ($\hat\mu_z$) With Moderately Unbalanced Data (Bimodal, $\pm 40\%$) and 40\% Missing Data (MAR, $\lambda=0.5$)}
\begin{tabular}{llccccccccccccccc}
\hline\\[-1.8ex]
& & \multicolumn{5}{c}{Bias (\%)} & \multicolumn{5}{c}{Rel. RMSE} & \multicolumn{5}{c}{Coverage (\%)} \\ \cmidrule(r){3-7}\cmidrule(r){8-12}\cmidrule(r){13-17}
 &  & CD & \makecell{FCS-\\MAN} & \makecell{FCS-\\NJ} & \makecell{FCS-\\LAT} & JM & CD & \makecell{FCS-\\MAN} & \makecell{FCS-\\NJ} & \makecell{FCS-\\LAT} & JM & CD & \makecell{FCS-\\MAN} & \makecell{FCS-\\NJ} & \makecell{FCS-\\LAT} & \multicolumn{1}{c}{JM} \\ 
[0.4ex]\hline\\[-1.8ex]
& & \multicolumn{15}{c}{Small intraclass correlation $(\rho_{Iy}=.10)$} \\[0.6ex]\hline\\[-1.8ex]
\multicolumn{4}{l}{$\bar{n}=5$} \\  & \nopagebreak $\;J=50$  & ${-}0.00\phantom{0}$ & ${-}0.00\phantom{0}$ & ${-}0.00\phantom{0}$ & $\phantom{-}0.01\phantom{0}$ & ${-}0.04\phantom{0}$ & $\phantom{0}0.14\phantom{0}$ & $\phantom{0}0.19\phantom{0}$ & $\phantom{0}0.20\phantom{0}$ & $\phantom{0}0.19\phantom{0}$ & $\phantom{0}0.19\phantom{0}$ & $\phantom{0}93.6\phantom{0}$ & $\phantom{0}94.7\phantom{0}$ & $\phantom{0}95.3\phantom{0}$ & $\phantom{0}93.3\phantom{0}$ & $\phantom{0}93.5\phantom{0}$ \\
 & \nopagebreak $\;J=200$  & $\phantom{-}0.00\phantom{0}$ & $\phantom{-}0.00\phantom{0}$ & $\phantom{-}0.00\phantom{0}$ & $\phantom{-}0.01\phantom{0}$ & ${-}0.01\phantom{0}$ & $\phantom{0}0.07\phantom{0}$ & $\phantom{0}0.09\phantom{0}$ & $\phantom{0}0.09\phantom{0}$ & $\phantom{0}0.09\phantom{0}$ & $\phantom{0}0.09\phantom{0}$ & $\phantom{0}95.0\phantom{0}$ & $\phantom{0}94.1\phantom{0}$ & $\phantom{0}95.1\phantom{0}$ & $\phantom{0}94.1\phantom{0}$ & $\phantom{0}95.2\phantom{0}$ \\
 & \nopagebreak $\;J=1000$  & ${-}0.00\phantom{0}$ & ${-}0.00\phantom{0}$ & $\phantom{-}0.00\phantom{0}$ & $\phantom{-}0.00\phantom{0}$ & ${-}0.00\phantom{0}$ & $\phantom{0}0.03\phantom{0}$ & $\phantom{0}0.04\phantom{0}$ & $\phantom{0}0.04\phantom{0}$ & $\phantom{0}0.04\phantom{0}$ & $\phantom{0}0.04\phantom{0}$ & $\phantom{0}95.0\phantom{0}$ & $\phantom{0}94.1\phantom{0}$ & $\phantom{0}94.6\phantom{0}$ & $\phantom{0}93.9\phantom{0}$ & $\phantom{0}94.6\phantom{0}$ \\
\multicolumn{4}{l}{$\bar{n}=20$} \\  & \nopagebreak $\;J=50$  & ${-}0.00\phantom{0}$ & $\phantom{-}0.00\phantom{0}$ & ${-}0.00\phantom{0}$ & $\phantom{-}0.00\phantom{0}$ & ${-}0.05\phantom{0}$ & $\phantom{0}0.14\phantom{0}$ & $\phantom{0}0.20\phantom{0}$ & $\phantom{0}0.20\phantom{0}$ & $\phantom{0}0.20\phantom{0}$ & $\phantom{0}0.20\phantom{0}$ & $\phantom{0}94.8\phantom{0}$ & $\phantom{0}94.8\phantom{0}$ & $\phantom{0}94.8\phantom{0}$ & $\phantom{0}93.9\phantom{0}$ & $\phantom{0}93.7\phantom{0}$ \\
 & \nopagebreak $\;J=200$  & $\phantom{-}0.00\phantom{0}$ & ${-}0.00\phantom{0}$ & $\phantom{-}0.00\phantom{0}$ & $\phantom{-}0.00\phantom{0}$ & ${-}0.02\phantom{0}$ & $\phantom{0}0.07\phantom{0}$ & $\phantom{0}0.10\phantom{0}$ & $\phantom{0}0.10\phantom{0}$ & $\phantom{0}0.10\phantom{0}$ & $\phantom{0}0.10\phantom{0}$ & $\phantom{0}94.0\phantom{0}$ & $\phantom{0}93.1\phantom{0}$ & $\phantom{0}93.2\phantom{0}$ & $\phantom{0}93.0\phantom{0}$ & $\phantom{0}93.5\phantom{0}$ \\
 & \nopagebreak $\;J=1000$  & $\phantom{-}0.00\phantom{0}$ & ${-}0.00\phantom{0}$ & ${-}0.00\phantom{0}$ & ${-}0.00\phantom{0}$ & ${-}0.01\phantom{0}$ & $\phantom{0}0.03\phantom{0}$ & $\phantom{0}0.04\phantom{0}$ & $\phantom{0}0.04\phantom{0}$ & $\phantom{0}0.04\phantom{0}$ & $\phantom{0}0.04\phantom{0}$ & $\phantom{0}95.3\phantom{0}$ & $\phantom{0}93.6\phantom{0}$ & $\phantom{0}94.5\phantom{0}$ & $\phantom{0}94.8\phantom{0}$ & $\phantom{0}94.4\phantom{0}$ \\
[0.5ex]\hline\\[-1.6ex] 
& & \multicolumn{15}{c}{Moderate intraclass correlation $(\rho_{Iy}=.30)$} \\[0.6ex]\hline\\[-1.8ex]
\multicolumn{4}{l}{$\bar{n}=5$} \\  & \nopagebreak $\;J=50$  & $\phantom{-}0.00\phantom{0}$ & ${-}0.00\phantom{0}$ & ${-}0.00\phantom{0}$ & $\phantom{-}0.00\phantom{0}$ & ${-}0.03\phantom{0}$ & $\phantom{0}0.15\phantom{0}$ & $\phantom{0}0.19\phantom{0}$ & $\phantom{0}0.20\phantom{0}$ & $\phantom{0}0.19\phantom{0}$ & $\phantom{0}0.19\phantom{0}$ & $\phantom{0}93.0\phantom{0}$ & $\phantom{0}94.8\phantom{0}$ & $\phantom{0}94.8\phantom{0}$ & $\phantom{0}93.8\phantom{0}$ & $\phantom{0}94.8\phantom{0}$ \\
 & \nopagebreak $\;J=200$  & ${-}0.00\phantom{0}$ & ${-}0.00\phantom{0}$ & ${-}0.00\phantom{0}$ & ${-}0.00\phantom{0}$ & ${-}0.01\phantom{0}$ & $\phantom{0}0.07\phantom{0}$ & $\phantom{0}0.10\phantom{0}$ & $\phantom{0}0.10\phantom{0}$ & $\phantom{0}0.10\phantom{0}$ & $\phantom{0}0.10\phantom{0}$ & $\phantom{0}94.9\phantom{0}$ & $\phantom{0}94.2\phantom{0}$ & $\phantom{0}94.1\phantom{0}$ & $\phantom{0}93.2\phantom{0}$ & $\phantom{0}93.5\phantom{0}$ \\
 & \nopagebreak $\;J=1000$  & $\phantom{-}0.00\phantom{0}$ & $\phantom{-}0.00\phantom{0}$ & $\phantom{-}0.00\phantom{0}$ & $\phantom{-}0.00\phantom{0}$ & $\phantom{-}0.00\phantom{0}$ & $\phantom{0}0.03\phantom{0}$ & $\phantom{0}0.04\phantom{0}$ & $\phantom{0}0.04\phantom{0}$ & $\phantom{0}0.04\phantom{0}$ & $\phantom{0}0.04\phantom{0}$ & $\phantom{0}95.6\phantom{0}$ & $\phantom{0}95.8\phantom{0}$ & $\phantom{0}95.3\phantom{0}$ & $\phantom{0}95.3\phantom{0}$ & $\phantom{0}95.7\phantom{0}$ \\
\multicolumn{4}{l}{$\bar{n}=20$} \\  & \nopagebreak $\;J=50$  & $\phantom{-}0.00\phantom{0}$ & ${-}0.00\phantom{0}$ & ${-}0.00\phantom{0}$ & ${-}0.00\phantom{0}$ & ${-}0.02\phantom{0}$ & $\phantom{0}0.14\phantom{0}$ & $\phantom{0}0.20\phantom{0}$ & $\phantom{0}0.21\phantom{0}$ & $\phantom{0}0.20\phantom{0}$ & $\phantom{0}0.20\phantom{0}$ & $\phantom{0}93.6\phantom{0}$ & $\phantom{0}93.4\phantom{0}$ & $\phantom{0}94.2\phantom{0}$ & $\phantom{0}92.4\phantom{0}$ & $\phantom{0}93.9\phantom{0}$ \\
 & \nopagebreak $\;J=200$  & ${-}0.00\phantom{0}$ & ${-}0.00\phantom{0}$ & ${-}0.00\phantom{0}$ & ${-}0.00\phantom{0}$ & ${-}0.01\phantom{0}$ & $\phantom{0}0.07\phantom{0}$ & $\phantom{0}0.09\phantom{0}$ & $\phantom{0}0.10\phantom{0}$ & $\phantom{0}0.09\phantom{0}$ & $\phantom{0}0.10\phantom{0}$ & $\phantom{0}94.9\phantom{0}$ & $\phantom{0}94.3\phantom{0}$ & $\phantom{0}94.4\phantom{0}$ & $\phantom{0}94.7\phantom{0}$ & $\phantom{0}95.2\phantom{0}$ \\
 & \nopagebreak $\;J=1000$  & $\phantom{-}0.00\phantom{0}$ & $\phantom{-}0.00\phantom{0}$ & $\phantom{-}0.00\phantom{0}$ & $\phantom{-}0.00\phantom{0}$ & $\phantom{-}0.00\phantom{0}$ & $\phantom{0}0.03\phantom{0}$ & $\phantom{0}0.04\phantom{0}$ & $\phantom{0}0.04\phantom{0}$ & $\phantom{0}0.04\phantom{0}$ & $\phantom{0}0.04\phantom{0}$ & $\phantom{0}93.5\phantom{0}$ & $\phantom{0}95.0\phantom{0}$ & $\phantom{0}95.4\phantom{0}$ & $\phantom{0}94.4\phantom{0}$ & $\phantom{0}95.4\phantom{0}$ \\
[0.5ex]\hline\\[-1.6ex] 
\end{tabular}
\begin{tablenotes}[para,flushleft]{\footnotesize \textit{Note.} $\bar{n}$ = average cluster size; $J$ = number of clusters; CD = complete data sets; LD = listwise deletion; FCS-SL = single-level FCS; FCS-MAN = two-level FCS with manifest cluster means; FCS-LAT = two-level FCS with latent cluster means; JM = joint modeling.}\end{tablenotes}
\end{threeparttable}
\end{sidewaystable}
\begin{sidewaystable}
\begin{threeparttable}
\setlength{\tabcolsep}{1.0pt}
\renewcommand{\arraystretch}{0.95}
\footnotesize
\caption{\small Study 2: Bias, Relative RMSE, and Coverage of the 95\% Confidence Interval for the Mean of $z$ ($\hat\mu_z$) With Strongly Unbalanced Data (Bimodal, $\pm 80\%$) and 40\% Missing Data (MAR, $\lambda=0.5$)}
\begin{tabular}{llccccccccccccccc}
\hline\\[-1.8ex]
& & \multicolumn{5}{c}{Bias (\%)} & \multicolumn{5}{c}{Rel. RMSE} & \multicolumn{5}{c}{Coverage (\%)} \\ \cmidrule(r){3-7}\cmidrule(r){8-12}\cmidrule(r){13-17}
 &  & CD & \makecell{FCS-\\MAN} & \makecell{FCS-\\NJ} & \makecell{FCS-\\LAT} & JM & CD & \makecell{FCS-\\MAN} & \makecell{FCS-\\NJ} & \makecell{FCS-\\LAT} & JM & CD & \makecell{FCS-\\MAN} & \makecell{FCS-\\NJ} & \makecell{FCS-\\LAT} & \multicolumn{1}{c}{JM} \\ 
[0.4ex]\hline\\[-1.8ex]
& & \multicolumn{15}{c}{Small intraclass correlation $(\rho_{Iy}=.10)$} \\[0.6ex]\hline\\[-1.8ex]
\multicolumn{4}{l}{$\bar{n}=5$} \\  & \nopagebreak $\;J=50$  & ${-}0.00\phantom{0}$ & ${-}0.00\phantom{0}$ & ${-}0.00\phantom{0}$ & $\phantom{-}0.00\phantom{0}$ & ${-}0.03\phantom{0}$ & $\phantom{0}0.14\phantom{0}$ & $\phantom{0}0.20\phantom{0}$ & $\phantom{0}0.21\phantom{0}$ & $\phantom{0}0.19\phantom{0}$ & $\phantom{0}0.19\phantom{0}$ & $\phantom{0}94.0\phantom{0}$ & $\phantom{0}94.0\phantom{0}$ & $\phantom{0}93.6\phantom{0}$ & $\phantom{0}93.6\phantom{0}$ & $\phantom{0}93.3\phantom{0}$ \\
 & \nopagebreak $\;J=200$  & $\phantom{-}0.00\phantom{0}$ & ${-}0.00\phantom{0}$ & ${-}0.00\phantom{0}$ & $\phantom{-}0.00\phantom{0}$ & ${-}0.01\phantom{0}$ & $\phantom{0}0.07\phantom{0}$ & $\phantom{0}0.10\phantom{0}$ & $\phantom{0}0.10\phantom{0}$ & $\phantom{0}0.10\phantom{0}$ & $\phantom{0}0.10\phantom{0}$ & $\phantom{0}94.7\phantom{0}$ & $\phantom{0}94.0\phantom{0}$ & $\phantom{0}94.1\phantom{0}$ & $\phantom{0}93.5\phantom{0}$ & $\phantom{0}94.3\phantom{0}$ \\
 & \nopagebreak $\;J=1000$  & $\phantom{-}0.00\phantom{0}$ & ${-}0.00\phantom{0}$ & ${-}0.00\phantom{0}$ & $\phantom{-}0.00\phantom{0}$ & ${-}0.00\phantom{0}$ & $\phantom{0}0.03\phantom{0}$ & $\phantom{0}0.04\phantom{0}$ & $\phantom{0}0.04\phantom{0}$ & $\phantom{0}0.04\phantom{0}$ & $\phantom{0}0.04\phantom{0}$ & $\phantom{0}96.5\phantom{0}$ & $\phantom{0}95.8\phantom{0}$ & $\phantom{0}95.7\phantom{0}$ & $\phantom{0}95.9\phantom{0}$ & $\phantom{0}96.2\phantom{0}$ \\
\multicolumn{4}{l}{$\bar{n}=20$} \\  & \nopagebreak $\;J=50$  & $\phantom{-}0.00\phantom{0}$ & $\phantom{-}0.00\phantom{0}$ & $\phantom{-}0.00\phantom{0}$ & $\phantom{-}0.00\phantom{0}$ & ${-}0.04\phantom{0}$ & $\phantom{0}0.14\phantom{0}$ & $\phantom{0}0.19\phantom{0}$ & $\phantom{0}0.20\phantom{0}$ & $\phantom{0}0.19\phantom{0}$ & $\phantom{0}0.19\phantom{0}$ & $\phantom{0}95.0\phantom{0}$ & $\phantom{0}94.4\phantom{0}$ & $\phantom{0}95.0\phantom{0}$ & $\phantom{0}94.9\phantom{0}$ & $\phantom{0}93.6\phantom{0}$ \\
 & \nopagebreak $\;J=200$  & $\phantom{-}0.00\phantom{0}$ & $\phantom{-}0.00\phantom{0}$ & ${-}0.00\phantom{0}$ & $\phantom{-}0.00\phantom{0}$ & ${-}0.02\phantom{0}$ & $\phantom{0}0.07\phantom{0}$ & $\phantom{0}0.10\phantom{0}$ & $\phantom{0}0.10\phantom{0}$ & $\phantom{0}0.09\phantom{0}$ & $\phantom{0}0.10\phantom{0}$ & $\phantom{0}95.4\phantom{0}$ & $\phantom{0}94.4\phantom{0}$ & $\phantom{0}95.0\phantom{0}$ & $\phantom{0}94.3\phantom{0}$ & $\phantom{0}93.7\phantom{0}$ \\
 & \nopagebreak $\;J=1000$  & $\phantom{-}0.00\phantom{0}$ & ${-}0.00\phantom{0}$ & $\phantom{-}0.00\phantom{0}$ & $\phantom{-}0.00\phantom{0}$ & ${-}0.00\phantom{0}$ & $\phantom{0}0.03\phantom{0}$ & $\phantom{0}0.04\phantom{0}$ & $\phantom{0}0.04\phantom{0}$ & $\phantom{0}0.04\phantom{0}$ & $\phantom{0}0.04\phantom{0}$ & $\phantom{0}94.8\phantom{0}$ & $\phantom{0}94.8\phantom{0}$ & $\phantom{0}95.2\phantom{0}$ & $\phantom{0}94.4\phantom{0}$ & $\phantom{0}94.9\phantom{0}$ \\
[0.5ex]\hline\\[-1.6ex] 
& & \multicolumn{15}{c}{Moderate intraclass correlation $(\rho_{Iy}=.30)$} \\[0.6ex]\hline\\[-1.8ex]
\multicolumn{4}{l}{$\bar{n}=5$} \\  & \nopagebreak $\;J=50$  & ${-}0.00\phantom{0}$ & ${-}0.01\phantom{0}$ & ${-}0.01\phantom{0}$ & ${-}0.01\phantom{0}$ & ${-}0.03\phantom{0}$ & $\phantom{0}0.15\phantom{0}$ & $\phantom{0}0.21\phantom{0}$ & $\phantom{0}0.21\phantom{0}$ & $\phantom{0}0.20\phantom{0}$ & $\phantom{0}0.20\phantom{0}$ & $\phantom{0}92.9\phantom{0}$ & $\phantom{0}93.8\phantom{0}$ & $\phantom{0}93.1\phantom{0}$ & $\phantom{0}93.4\phantom{0}$ & $\phantom{0}93.9\phantom{0}$ \\
 & \nopagebreak $\;J=200$  & $\phantom{-}0.00\phantom{0}$ & ${-}0.00\phantom{0}$ & $\phantom{-}0.00\phantom{0}$ & $\phantom{-}0.00\phantom{0}$ & ${-}0.01\phantom{0}$ & $\phantom{0}0.07\phantom{0}$ & $\phantom{0}0.10\phantom{0}$ & $\phantom{0}0.10\phantom{0}$ & $\phantom{0}0.10\phantom{0}$ & $\phantom{0}0.10\phantom{0}$ & $\phantom{0}94.4\phantom{0}$ & $\phantom{0}93.7\phantom{0}$ & $\phantom{0}92.9\phantom{0}$ & $\phantom{0}93.5\phantom{0}$ & $\phantom{0}94.0\phantom{0}$ \\
 & \nopagebreak $\;J=1000$  & ${-}0.00\phantom{0}$ & ${-}0.00\phantom{0}$ & ${-}0.00\phantom{0}$ & ${-}0.00\phantom{0}$ & ${-}0.00\phantom{0}$ & $\phantom{0}0.03\phantom{0}$ & $\phantom{0}0.04\phantom{0}$ & $\phantom{0}0.04\phantom{0}$ & $\phantom{0}0.04\phantom{0}$ & $\phantom{0}0.04\phantom{0}$ & $\phantom{0}94.5\phantom{0}$ & $\phantom{0}93.9\phantom{0}$ & $\phantom{0}93.6\phantom{0}$ & $\phantom{0}93.6\phantom{0}$ & $\phantom{0}94.6\phantom{0}$ \\
\multicolumn{4}{l}{$\bar{n}=20$} \\  & \nopagebreak $\;J=50$  & ${-}0.00\phantom{0}$ & ${-}0.00\phantom{0}$ & ${-}0.01\phantom{0}$ & ${-}0.00\phantom{0}$ & ${-}0.03\phantom{0}$ & $\phantom{0}0.14\phantom{0}$ & $\phantom{0}0.19\phantom{0}$ & $\phantom{0}0.20\phantom{0}$ & $\phantom{0}0.19\phantom{0}$ & $\phantom{0}0.19\phantom{0}$ & $\phantom{0}93.3\phantom{0}$ & $\phantom{0}95.4\phantom{0}$ & $\phantom{0}95.4\phantom{0}$ & $\phantom{0}93.8\phantom{0}$ & $\phantom{0}94.2\phantom{0}$ \\
 & \nopagebreak $\;J=200$  & $\phantom{-}0.00\phantom{0}$ & ${-}0.00\phantom{0}$ & ${-}0.00\phantom{0}$ & $\phantom{-}0.00\phantom{0}$ & ${-}0.01\phantom{0}$ & $\phantom{0}0.07\phantom{0}$ & $\phantom{0}0.09\phantom{0}$ & $\phantom{0}0.10\phantom{0}$ & $\phantom{0}0.09\phantom{0}$ & $\phantom{0}0.09\phantom{0}$ & $\phantom{0}95.9\phantom{0}$ & $\phantom{0}94.8\phantom{0}$ & $\phantom{0}94.4\phantom{0}$ & $\phantom{0}95.0\phantom{0}$ & $\phantom{0}94.9\phantom{0}$ \\
 & \nopagebreak $\;J=1000$  & ${-}0.00\phantom{0}$ & ${-}0.00\phantom{0}$ & $\phantom{-}0.00\phantom{0}$ & $\phantom{-}0.00\phantom{0}$ & $\phantom{-}0.00\phantom{0}$ & $\phantom{0}0.03\phantom{0}$ & $\phantom{0}0.04\phantom{0}$ & $\phantom{0}0.04\phantom{0}$ & $\phantom{0}0.04\phantom{0}$ & $\phantom{0}0.04\phantom{0}$ & $\phantom{0}95.1\phantom{0}$ & $\phantom{0}93.7\phantom{0}$ & $\phantom{0}94.3\phantom{0}$ & $\phantom{0}94.0\phantom{0}$ & $\phantom{0}94.8\phantom{0}$ \\
[0.5ex]\hline\\[-1.6ex] 
\end{tabular}
\begin{tablenotes}[para,flushleft]{\footnotesize \textit{Note.} $\bar{n}$ = average cluster size; $J$ = number of clusters; CD = complete data sets; LD = listwise deletion; FCS-SL = single-level FCS; FCS-MAN = two-level FCS with manifest cluster means; FCS-LAT = two-level FCS with latent cluster means; JM = joint modeling.}\end{tablenotes}
\end{threeparttable}
\end{sidewaystable}
\begin{sidewaystable}
\begin{threeparttable}
\setlength{\tabcolsep}{1.0pt}
\renewcommand{\arraystretch}{0.95}
\footnotesize
\caption{\small Study 2: Bias (in \%), Relative RMSE, and Coverage of the 95\% Confidence Interval for the Variance of $z$ ($\hat\sigma_z^2$) With Moderately Unbalanced Data (Uniform, $\pm 40\%$) and 20\% Missing Data (MAR, $\lambda=0.5$)}
\begin{tabular}{llccccccccccccccc}
\hline\\[-1.8ex]
& & \multicolumn{5}{c}{Bias (\%)} & \multicolumn{5}{c}{Rel. RMSE} & \multicolumn{5}{c}{Coverage (\%)} \\ \cmidrule(r){3-7}\cmidrule(r){8-12}\cmidrule(r){13-17}
 &  & CD & \makecell{FCS-\\MAN} & \makecell{FCS-\\NJ} & \makecell{FCS-\\LAT} & JM & CD & \makecell{FCS-\\MAN} & \makecell{FCS-\\NJ} & \makecell{FCS-\\LAT} & JM & CD & \makecell{FCS-\\MAN} & \makecell{FCS-\\NJ} & \makecell{FCS-\\LAT} & \multicolumn{1}{c}{JM} \\ 
[0.4ex]\hline\\[-1.8ex]
& & \multicolumn{15}{c}{Small intraclass correlation $(\rho_{Iy}=.10)$} \\[0.6ex]\hline\\[-1.8ex]
\multicolumn{4}{l}{$\bar{n}=5$} \\  & \nopagebreak $\;J=50$  & ${-}1.3\phantom{0}$ & $\phantom{-}0.9\phantom{0}$ & $\phantom{-}4.5\phantom{0}$ & $\phantom{-}1.5\phantom{0}$ & ${-}1.4\phantom{0}$ & $\phantom{0}0.20\phantom{0}$ & $\phantom{0}0.24\phantom{0}$ & $\phantom{0}0.25\phantom{0}$ & $\phantom{0}0.24\phantom{0}$ & $\phantom{0}0.22\phantom{0}$ & $\phantom{0}90.2\phantom{0}$ & $\phantom{0}92.3\phantom{0}$ & $\phantom{0}94.4\phantom{0}$ & $\phantom{0}92.7\phantom{0}$ & $\phantom{0}92.5\phantom{0}$ \\
 & \nopagebreak $\;J=200$  & $\phantom{-}0.4\phantom{0}$ & $\phantom{-}1.1\phantom{0}$ & $\phantom{-}1.7\phantom{0}$ & $\phantom{-}1.2\phantom{0}$ & $\phantom{-}0.2\phantom{0}$ & $\phantom{0}0.10\phantom{0}$ & $\phantom{0}0.11\phantom{0}$ & $\phantom{0}0.12\phantom{0}$ & $\phantom{0}0.11\phantom{0}$ & $\phantom{0}0.11\phantom{0}$ & $\phantom{0}95.3\phantom{0}$ & $\phantom{0}95.3\phantom{0}$ & $\phantom{0}96.0\phantom{0}$ & $\phantom{0}95.6\phantom{0}$ & $\phantom{0}94.9\phantom{0}$ \\
 & \nopagebreak $\;J=1000$  & ${-}0.1\phantom{0}$ & ${-}0.0\phantom{0}$ & $\phantom{-}0.0\phantom{0}$ & ${-}0.1\phantom{0}$ & ${-}0.2\phantom{0}$ & $\phantom{0}0.05\phantom{0}$ & $\phantom{0}0.05\phantom{0}$ & $\phantom{0}0.05\phantom{0}$ & $\phantom{0}0.05\phantom{0}$ & $\phantom{0}0.05\phantom{0}$ & $\phantom{0}94.2\phantom{0}$ & $\phantom{0}93.7\phantom{0}$ & $\phantom{0}94.3\phantom{0}$ & $\phantom{0}94.7\phantom{0}$ & $\phantom{0}94.1\phantom{0}$ \\
\multicolumn{4}{l}{$\bar{n}=20$} \\  & \nopagebreak $\;J=50$  & ${-}1.8\phantom{0}$ & $\phantom{-}0.7\phantom{0}$ & $\phantom{-}4.6\phantom{0}$ & $\phantom{-}0.8\phantom{0}$ & ${-}2.0\phantom{0}$ & $\phantom{0}0.20\phantom{0}$ & $\phantom{0}0.24\phantom{0}$ & $\phantom{0}0.27\phantom{0}$ & $\phantom{0}0.24\phantom{0}$ & $\phantom{0}0.23\phantom{0}$ & $\phantom{0}89.7\phantom{0}$ & $\phantom{0}90.6\phantom{0}$ & $\phantom{0}93.1\phantom{0}$ & $\phantom{0}90.8\phantom{0}$ & $\phantom{0}89.4\phantom{0}$ \\
 & \nopagebreak $\;J=200$  & ${-}0.5\phantom{0}$ & $\phantom{-}0.0\phantom{0}$ & $\phantom{-}0.6\phantom{0}$ & ${-}0.0\phantom{0}$ & ${-}0.8\phantom{0}$ & $\phantom{0}0.10\phantom{0}$ & $\phantom{0}0.12\phantom{0}$ & $\phantom{0}0.12\phantom{0}$ & $\phantom{0}0.12\phantom{0}$ & $\phantom{0}0.11\phantom{0}$ & $\phantom{0}93.4\phantom{0}$ & $\phantom{0}94.0\phantom{0}$ & $\phantom{0}94.4\phantom{0}$ & $\phantom{0}93.6\phantom{0}$ & $\phantom{0}93.6\phantom{0}$ \\
 & \nopagebreak $\;J=1000$  & ${-}0.2\phantom{0}$ & $\phantom{-}0.0\phantom{0}$ & $\phantom{-}0.1\phantom{0}$ & ${-}0.0\phantom{0}$ & ${-}0.1\phantom{0}$ & $\phantom{0}0.04\phantom{0}$ & $\phantom{0}0.05\phantom{0}$ & $\phantom{0}0.05\phantom{0}$ & $\phantom{0}0.05\phantom{0}$ & $\phantom{0}0.05\phantom{0}$ & $\phantom{0}94.6\phantom{0}$ & $\phantom{0}95.1\phantom{0}$ & $\phantom{0}95.4\phantom{0}$ & $\phantom{0}95.7\phantom{0}$ & $\phantom{0}95.0\phantom{0}$ \\
[0.5ex]\hline\\[-1.6ex] 
& & \multicolumn{15}{c}{Moderate intraclass correlation $(\rho_{Iy}=.30)$} \\[0.6ex]\hline\\[-1.8ex]
\multicolumn{4}{l}{$\bar{n}=5$} \\  & \nopagebreak $\;J=50$  & ${-}3.4\phantom{0}$ & ${-}1.0\phantom{0}$ & $\phantom{-}2.0\phantom{0}$ & ${-}0.8\phantom{0}$ & ${-}2.9\phantom{0}$ & $\phantom{0}0.21\phantom{0}$ & $\phantom{0}0.23\phantom{0}$ & $\phantom{0}0.24\phantom{0}$ & $\phantom{0}0.23\phantom{0}$ & $\phantom{0}0.23\phantom{0}$ & $\phantom{0}88.2\phantom{0}$ & $\phantom{0}91.9\phantom{0}$ & $\phantom{0}95.0\phantom{0}$ & $\phantom{0}92.0\phantom{0}$ & $\phantom{0}90.0\phantom{0}$ \\
 & \nopagebreak $\;J=200$  & ${-}0.6\phantom{0}$ & ${-}0.2\phantom{0}$ & $\phantom{-}0.5\phantom{0}$ & ${-}0.1\phantom{0}$ & ${-}0.6\phantom{0}$ & $\phantom{0}0.10\phantom{0}$ & $\phantom{0}0.12\phantom{0}$ & $\phantom{0}0.12\phantom{0}$ & $\phantom{0}0.12\phantom{0}$ & $\phantom{0}0.11\phantom{0}$ & $\phantom{0}93.5\phantom{0}$ & $\phantom{0}93.7\phantom{0}$ & $\phantom{0}94.0\phantom{0}$ & $\phantom{0}93.0\phantom{0}$ & $\phantom{0}93.2\phantom{0}$ \\
 & \nopagebreak $\;J=1000$  & $\phantom{-}0.0\phantom{0}$ & $\phantom{-}0.1\phantom{0}$ & $\phantom{-}0.2\phantom{0}$ & $\phantom{-}0.1\phantom{0}$ & $\phantom{-}0.0\phantom{0}$ & $\phantom{0}0.04\phantom{0}$ & $\phantom{0}0.05\phantom{0}$ & $\phantom{0}0.05\phantom{0}$ & $\phantom{0}0.05\phantom{0}$ & $\phantom{0}0.05\phantom{0}$ & $\phantom{0}95.5\phantom{0}$ & $\phantom{0}95.4\phantom{0}$ & $\phantom{0}95.7\phantom{0}$ & $\phantom{0}94.0\phantom{0}$ & $\phantom{0}95.0\phantom{0}$ \\
\multicolumn{4}{l}{$\bar{n}=20$} \\  & \nopagebreak $\;J=50$  & ${-}1.5\phantom{0}$ & $\phantom{-}0.8\phantom{0}$ & $\phantom{-}3.6\phantom{0}$ & $\phantom{-}0.6\phantom{0}$ & ${-}0.9\phantom{0}$ & $\phantom{0}0.20\phantom{0}$ & $\phantom{0}0.24\phantom{0}$ & $\phantom{0}0.25\phantom{0}$ & $\phantom{0}0.24\phantom{0}$ & $\phantom{0}0.24\phantom{0}$ & $\phantom{0}90.1\phantom{0}$ & $\phantom{0}91.2\phantom{0}$ & $\phantom{0}93.4\phantom{0}$ & $\phantom{0}91.5\phantom{0}$ & $\phantom{0}90.2\phantom{0}$ \\
 & \nopagebreak $\;J=200$  & ${-}0.0\phantom{0}$ & $\phantom{-}0.7\phantom{0}$ & $\phantom{-}1.3\phantom{0}$ & $\phantom{-}0.7\phantom{0}$ & $\phantom{-}0.4\phantom{0}$ & $\phantom{0}0.10\phantom{0}$ & $\phantom{0}0.12\phantom{0}$ & $\phantom{0}0.12\phantom{0}$ & $\phantom{0}0.12\phantom{0}$ & $\phantom{0}0.12\phantom{0}$ & $\phantom{0}93.6\phantom{0}$ & $\phantom{0}93.7\phantom{0}$ & $\phantom{0}93.4\phantom{0}$ & $\phantom{0}93.1\phantom{0}$ & $\phantom{0}93.5\phantom{0}$ \\
 & \nopagebreak $\;J=1000$  & ${-}0.1\phantom{0}$ & $\phantom{-}0.1\phantom{0}$ & $\phantom{-}0.2\phantom{0}$ & $\phantom{-}0.1\phantom{0}$ & $\phantom{-}0.0\phantom{0}$ & $\phantom{0}0.05\phantom{0}$ & $\phantom{0}0.05\phantom{0}$ & $\phantom{0}0.05\phantom{0}$ & $\phantom{0}0.05\phantom{0}$ & $\phantom{0}0.05\phantom{0}$ & $\phantom{0}94.6\phantom{0}$ & $\phantom{0}94.5\phantom{0}$ & $\phantom{0}94.0\phantom{0}$ & $\phantom{0}94.4\phantom{0}$ & $\phantom{0}94.1\phantom{0}$ \\
[0.5ex]\hline\\[-1.6ex] 
\end{tabular}
\begin{tablenotes}[para,flushleft]{\footnotesize \textit{Note.} $\bar{n}$ = average cluster size; $J$ = number of clusters; CD = complete data sets; LD = listwise deletion; FCS-SL = single-level FCS; FCS-MAN = two-level FCS with manifest cluster means; FCS-LAT = two-level FCS with latent cluster means; JM = joint modeling.}\end{tablenotes}
\end{threeparttable}
\end{sidewaystable}
\begin{sidewaystable}
\begin{threeparttable}
\setlength{\tabcolsep}{1.0pt}
\renewcommand{\arraystretch}{0.95}
\footnotesize
\caption{\small Study 2: Bias (in \%), Relative RMSE, and Coverage of the 95\% Confidence Interval for the Variance of $z$ ($\hat\sigma_z^2$) With Strongly Unbalanced Data (Uniform, $\pm 80\%$) and 20\% Missing Data (MAR, $\lambda=0.5$)}
\begin{tabular}{llccccccccccccccc}
\hline\\[-1.8ex]
& & \multicolumn{5}{c}{Bias (\%)} & \multicolumn{5}{c}{Rel. RMSE} & \multicolumn{5}{c}{Coverage (\%)} \\ \cmidrule(r){3-7}\cmidrule(r){8-12}\cmidrule(r){13-17}
 &  & CD & \makecell{FCS-\\MAN} & \makecell{FCS-\\NJ} & \makecell{FCS-\\LAT} & JM & CD & \makecell{FCS-\\MAN} & \makecell{FCS-\\NJ} & \makecell{FCS-\\LAT} & JM & CD & \makecell{FCS-\\MAN} & \makecell{FCS-\\NJ} & \makecell{FCS-\\LAT} & \multicolumn{1}{c}{JM} \\ 
[0.4ex]\hline\\[-1.8ex]
& & \multicolumn{15}{c}{Small intraclass correlation $(\rho_{Iy}=.10)$} \\[0.6ex]\hline\\[-1.8ex]
\multicolumn{4}{l}{$\bar{n}=5$} \\  & \nopagebreak $\;J=50$  & ${-}3.2\phantom{0}$ & ${-}0.2\phantom{0}$ & $\phantom{-}3.4\phantom{0}$ & ${-}0.6\phantom{0}$ & ${-}2.8\phantom{0}$ & $\phantom{0}0.19\phantom{0}$ & $\phantom{0}0.22\phantom{0}$ & $\phantom{0}0.24\phantom{0}$ & $\phantom{0}0.22\phantom{0}$ & $\phantom{0}0.21\phantom{0}$ & $\phantom{0}90.2\phantom{0}$ & $\phantom{0}92.3\phantom{0}$ & $\phantom{0}93.5\phantom{0}$ & $\phantom{0}91.6\phantom{0}$ & $\phantom{0}91.3\phantom{0}$ \\
 & \nopagebreak $\;J=200$  & ${-}0.4\phantom{0}$ & $\phantom{-}0.3\phantom{0}$ & $\phantom{-}0.8\phantom{0}$ & $\phantom{-}0.1\phantom{0}$ & ${-}0.5\phantom{0}$ & $\phantom{0}0.10\phantom{0}$ & $\phantom{0}0.11\phantom{0}$ & $\phantom{0}0.11\phantom{0}$ & $\phantom{0}0.11\phantom{0}$ & $\phantom{0}0.11\phantom{0}$ & $\phantom{0}93.4\phantom{0}$ & $\phantom{0}94.5\phantom{0}$ & $\phantom{0}94.8\phantom{0}$ & $\phantom{0}93.9\phantom{0}$ & $\phantom{0}93.7\phantom{0}$ \\
 & \nopagebreak $\;J=1000$  & $\phantom{-}0.0\phantom{0}$ & $\phantom{-}0.3\phantom{0}$ & $\phantom{-}0.2\phantom{0}$ & $\phantom{-}0.1\phantom{0}$ & ${-}0.1\phantom{0}$ & $\phantom{0}0.05\phantom{0}$ & $\phantom{0}0.05\phantom{0}$ & $\phantom{0}0.05\phantom{0}$ & $\phantom{0}0.05\phantom{0}$ & $\phantom{0}0.05\phantom{0}$ & $\phantom{0}94.9\phantom{0}$ & $\phantom{0}94.4\phantom{0}$ & $\phantom{0}95.0\phantom{0}$ & $\phantom{0}94.1\phantom{0}$ & $\phantom{0}94.1\phantom{0}$ \\
\multicolumn{4}{l}{$\bar{n}=20$} \\  & \nopagebreak $\;J=50$  & ${-}1.4\phantom{0}$ & $\phantom{-}1.6\phantom{0}$ & $\phantom{-}4.9\phantom{0}$ & $\phantom{-}1.5\phantom{0}$ & ${-}1.6\phantom{0}$ & $\phantom{0}0.20\phantom{0}$ & $\phantom{0}0.24\phantom{0}$ & $\phantom{0}0.26\phantom{0}$ & $\phantom{0}0.24\phantom{0}$ & $\phantom{0}0.23\phantom{0}$ & $\phantom{0}91.4\phantom{0}$ & $\phantom{0}92.1\phantom{0}$ & $\phantom{0}94.3\phantom{0}$ & $\phantom{0}93.3\phantom{0}$ & $\phantom{0}91.7\phantom{0}$ \\
 & \nopagebreak $\;J=200$  & ${-}0.6\phantom{0}$ & $\phantom{-}0.4\phantom{0}$ & $\phantom{-}0.8\phantom{0}$ & $\phantom{-}0.2\phantom{0}$ & ${-}0.6\phantom{0}$ & $\phantom{0}0.10\phantom{0}$ & $\phantom{0}0.11\phantom{0}$ & $\phantom{0}0.11\phantom{0}$ & $\phantom{0}0.11\phantom{0}$ & $\phantom{0}0.11\phantom{0}$ & $\phantom{0}94.7\phantom{0}$ & $\phantom{0}95.7\phantom{0}$ & $\phantom{0}96.0\phantom{0}$ & $\phantom{0}95.4\phantom{0}$ & $\phantom{0}94.4\phantom{0}$ \\
 & \nopagebreak $\;J=1000$  & ${-}0.2\phantom{0}$ & $\phantom{-}0.1\phantom{0}$ & $\phantom{-}0.1\phantom{0}$ & ${-}0.0\phantom{0}$ & ${-}0.2\phantom{0}$ & $\phantom{0}0.05\phantom{0}$ & $\phantom{0}0.05\phantom{0}$ & $\phantom{0}0.05\phantom{0}$ & $\phantom{0}0.05\phantom{0}$ & $\phantom{0}0.05\phantom{0}$ & $\phantom{0}94.8\phantom{0}$ & $\phantom{0}94.4\phantom{0}$ & $\phantom{0}94.5\phantom{0}$ & $\phantom{0}94.7\phantom{0}$ & $\phantom{0}94.9\phantom{0}$ \\
[0.5ex]\hline\\[-1.6ex] 
& & \multicolumn{15}{c}{Moderate intraclass correlation $(\rho_{Iy}=.30)$} \\[0.6ex]\hline\\[-1.8ex]
\multicolumn{4}{l}{$\bar{n}=5$} \\  & \nopagebreak $\;J=50$  & ${-}2.5\phantom{0}$ & $\phantom{-}0.8\phantom{0}$ & $\phantom{-}3.9\phantom{0}$ & $\phantom{-}0.5\phantom{0}$ & ${-}1.7\phantom{0}$ & $\phantom{0}0.19\phantom{0}$ & $\phantom{0}0.24\phantom{0}$ & $\phantom{0}0.25\phantom{0}$ & $\phantom{0}0.23\phantom{0}$ & $\phantom{0}0.22\phantom{0}$ & $\phantom{0}91.1\phantom{0}$ & $\phantom{0}92.1\phantom{0}$ & $\phantom{0}93.5\phantom{0}$ & $\phantom{0}93.0\phantom{0}$ & $\phantom{0}91.6\phantom{0}$ \\
 & \nopagebreak $\;J=200$  & ${-}0.2\phantom{0}$ & $\phantom{-}0.5\phantom{0}$ & $\phantom{-}0.9\phantom{0}$ & $\phantom{-}0.2\phantom{0}$ & ${-}0.1\phantom{0}$ & $\phantom{0}0.10\phantom{0}$ & $\phantom{0}0.11\phantom{0}$ & $\phantom{0}0.11\phantom{0}$ & $\phantom{0}0.11\phantom{0}$ & $\phantom{0}0.11\phantom{0}$ & $\phantom{0}94.5\phantom{0}$ & $\phantom{0}93.9\phantom{0}$ & $\phantom{0}94.5\phantom{0}$ & $\phantom{0}93.8\phantom{0}$ & $\phantom{0}93.8\phantom{0}$ \\
 & \nopagebreak $\;J=1000$  & ${-}0.1\phantom{0}$ & $\phantom{-}0.1\phantom{0}$ & $\phantom{-}0.0\phantom{0}$ & ${-}0.1\phantom{0}$ & ${-}0.1\phantom{0}$ & $\phantom{0}0.04\phantom{0}$ & $\phantom{0}0.05\phantom{0}$ & $\phantom{0}0.05\phantom{0}$ & $\phantom{0}0.05\phantom{0}$ & $\phantom{0}0.05\phantom{0}$ & $\phantom{0}94.9\phantom{0}$ & $\phantom{0}94.9\phantom{0}$ & $\phantom{0}95.1\phantom{0}$ & $\phantom{0}94.6\phantom{0}$ & $\phantom{0}94.5\phantom{0}$ \\
\multicolumn{4}{l}{$\bar{n}=20$} \\  & \nopagebreak $\;J=50$  & ${-}1.3\phantom{0}$ & $\phantom{-}1.2\phantom{0}$ & $\phantom{-}4.4\phantom{0}$ & $\phantom{-}1.4\phantom{0}$ & ${-}0.5\phantom{0}$ & $\phantom{0}0.21\phantom{0}$ & $\phantom{0}0.25\phantom{0}$ & $\phantom{0}0.26\phantom{0}$ & $\phantom{0}0.25\phantom{0}$ & $\phantom{0}0.24\phantom{0}$ & $\phantom{0}88.7\phantom{0}$ & $\phantom{0}91.6\phantom{0}$ & $\phantom{0}93.5\phantom{0}$ & $\phantom{0}91.0\phantom{0}$ & $\phantom{0}90.6\phantom{0}$ \\
 & \nopagebreak $\;J=200$  & ${-}0.6\phantom{0}$ & $\phantom{-}0.1\phantom{0}$ & $\phantom{-}0.7\phantom{0}$ & $\phantom{-}0.0\phantom{0}$ & ${-}0.3\phantom{0}$ & $\phantom{0}0.10\phantom{0}$ & $\phantom{0}0.12\phantom{0}$ & $\phantom{0}0.12\phantom{0}$ & $\phantom{0}0.12\phantom{0}$ & $\phantom{0}0.12\phantom{0}$ & $\phantom{0}92.3\phantom{0}$ & $\phantom{0}93.2\phantom{0}$ & $\phantom{0}93.7\phantom{0}$ & $\phantom{0}92.4\phantom{0}$ & $\phantom{0}93.0\phantom{0}$ \\
 & \nopagebreak $\;J=1000$  & ${-}0.2\phantom{0}$ & ${-}0.1\phantom{0}$ & ${-}0.0\phantom{0}$ & ${-}0.1\phantom{0}$ & ${-}0.2\phantom{0}$ & $\phantom{0}0.04\phantom{0}$ & $\phantom{0}0.05\phantom{0}$ & $\phantom{0}0.05\phantom{0}$ & $\phantom{0}0.05\phantom{0}$ & $\phantom{0}0.05\phantom{0}$ & $\phantom{0}94.4\phantom{0}$ & $\phantom{0}94.4\phantom{0}$ & $\phantom{0}94.7\phantom{0}$ & $\phantom{0}94.7\phantom{0}$ & $\phantom{0}94.8\phantom{0}$ \\
[0.5ex]\hline\\[-1.6ex] 
\end{tabular}
\begin{tablenotes}[para,flushleft]{\footnotesize \textit{Note.} $\bar{n}$ = average cluster size; $J$ = number of clusters; CD = complete data sets; LD = listwise deletion; FCS-SL = single-level FCS; FCS-MAN = two-level FCS with manifest cluster means; FCS-LAT = two-level FCS with latent cluster means; JM = joint modeling.}\end{tablenotes}
\end{threeparttable}
\end{sidewaystable}
\begin{sidewaystable}
\begin{threeparttable}
\setlength{\tabcolsep}{1.0pt}
\renewcommand{\arraystretch}{0.95}
\footnotesize
\caption{\small Study 2: Bias (in \%), Relative RMSE, and Coverage of the 95\% Confidence Interval for the Variance of $z$ ($\hat\sigma_z^2$) With Moderately Unbalanced Data (Bimodal, $\pm 40\%$) and 20\% Missing Data (MAR, $\lambda=0.5$)}
\begin{tabular}{llccccccccccccccc}
\hline\\[-1.8ex]
& & \multicolumn{5}{c}{Bias (\%)} & \multicolumn{5}{c}{Rel. RMSE} & \multicolumn{5}{c}{Coverage (\%)} \\ \cmidrule(r){3-7}\cmidrule(r){8-12}\cmidrule(r){13-17}
 &  & CD & \makecell{FCS-\\MAN} & \makecell{FCS-\\NJ} & \makecell{FCS-\\LAT} & JM & CD & \makecell{FCS-\\MAN} & \makecell{FCS-\\NJ} & \makecell{FCS-\\LAT} & JM & CD & \makecell{FCS-\\MAN} & \makecell{FCS-\\NJ} & \makecell{FCS-\\LAT} & \multicolumn{1}{c}{JM} \\ 
[0.4ex]\hline\\[-1.8ex]
& & \multicolumn{15}{c}{Small intraclass correlation $(\rho_{Iy}=.10)$} \\[0.6ex]\hline\\[-1.8ex]
\multicolumn{4}{l}{$\bar{n}=5$} \\  & \nopagebreak $\;J=50$  & ${-}1.6\phantom{0}$ & $\phantom{-}0.6\phantom{0}$ & $\phantom{-}3.9\phantom{0}$ & $\phantom{-}0.7\phantom{0}$ & ${-}2.0\phantom{0}$ & $\phantom{0}0.21\phantom{0}$ & $\phantom{0}0.24\phantom{0}$ & $\phantom{0}0.26\phantom{0}$ & $\phantom{0}0.24\phantom{0}$ & $\phantom{0}0.23\phantom{0}$ & $\phantom{0}90.2\phantom{0}$ & $\phantom{0}91.6\phantom{0}$ & $\phantom{0}93.1\phantom{0}$ & $\phantom{0}91.1\phantom{0}$ & $\phantom{0}91.0\phantom{0}$ \\
 & \nopagebreak $\;J=200$  & ${-}0.9\phantom{0}$ & ${-}0.3\phantom{0}$ & $\phantom{-}0.3\phantom{0}$ & ${-}0.2\phantom{0}$ & ${-}1.0\phantom{0}$ & $\phantom{0}0.10\phantom{0}$ & $\phantom{0}0.11\phantom{0}$ & $\phantom{0}0.11\phantom{0}$ & $\phantom{0}0.11\phantom{0}$ & $\phantom{0}0.11\phantom{0}$ & $\phantom{0}93.7\phantom{0}$ & $\phantom{0}93.9\phantom{0}$ & $\phantom{0}94.2\phantom{0}$ & $\phantom{0}94.0\phantom{0}$ & $\phantom{0}94.3\phantom{0}$ \\
 & \nopagebreak $\;J=1000$  & $\phantom{-}0.0\phantom{0}$ & $\phantom{-}0.3\phantom{0}$ & $\phantom{-}0.3\phantom{0}$ & $\phantom{-}0.2\phantom{0}$ & $\phantom{-}0.1\phantom{0}$ & $\phantom{0}0.05\phantom{0}$ & $\phantom{0}0.05\phantom{0}$ & $\phantom{0}0.05\phantom{0}$ & $\phantom{0}0.05\phantom{0}$ & $\phantom{0}0.05\phantom{0}$ & $\phantom{0}93.3\phantom{0}$ & $\phantom{0}94.0\phantom{0}$ & $\phantom{0}93.8\phantom{0}$ & $\phantom{0}93.4\phantom{0}$ & $\phantom{0}94.6\phantom{0}$ \\
\multicolumn{4}{l}{$\bar{n}=20$} \\  & \nopagebreak $\;J=50$  & ${-}2.5\phantom{0}$ & $\phantom{-}0.7\phantom{0}$ & $\phantom{-}3.5\phantom{0}$ & $\phantom{-}0.7\phantom{0}$ & ${-}2.5\phantom{0}$ & $\phantom{0}0.19\phantom{0}$ & $\phantom{0}0.23\phantom{0}$ & $\phantom{0}0.25\phantom{0}$ & $\phantom{0}0.23\phantom{0}$ & $\phantom{0}0.22\phantom{0}$ & $\phantom{0}90.7\phantom{0}$ & $\phantom{0}92.3\phantom{0}$ & $\phantom{0}94.0\phantom{0}$ & $\phantom{0}92.3\phantom{0}$ & $\phantom{0}90.7\phantom{0}$ \\
 & \nopagebreak $\;J=200$  & ${-}0.5\phantom{0}$ & ${-}0.2\phantom{0}$ & $\phantom{-}0.2\phantom{0}$ & ${-}0.3\phantom{0}$ & ${-}1.1\phantom{0}$ & $\phantom{0}0.10\phantom{0}$ & $\phantom{0}0.11\phantom{0}$ & $\phantom{0}0.11\phantom{0}$ & $\phantom{0}0.11\phantom{0}$ & $\phantom{0}0.11\phantom{0}$ & $\phantom{0}93.9\phantom{0}$ & $\phantom{0}94.7\phantom{0}$ & $\phantom{0}94.7\phantom{0}$ & $\phantom{0}94.5\phantom{0}$ & $\phantom{0}94.0\phantom{0}$ \\
 & \nopagebreak $\;J=1000$  & $\phantom{-}0.2\phantom{0}$ & $\phantom{-}0.4\phantom{0}$ & $\phantom{-}0.5\phantom{0}$ & $\phantom{-}0.4\phantom{0}$ & $\phantom{-}0.2\phantom{0}$ & $\phantom{0}0.04\phantom{0}$ & $\phantom{0}0.05\phantom{0}$ & $\phantom{0}0.05\phantom{0}$ & $\phantom{0}0.05\phantom{0}$ & $\phantom{0}0.05\phantom{0}$ & $\phantom{0}96.2\phantom{0}$ & $\phantom{0}96.3\phantom{0}$ & $\phantom{0}96.4\phantom{0}$ & $\phantom{0}96.6\phantom{0}$ & $\phantom{0}95.9\phantom{0}$ \\
[0.5ex]\hline\\[-1.6ex] 
& & \multicolumn{15}{c}{Moderate intraclass correlation $(\rho_{Iy}=.30)$} \\[0.6ex]\hline\\[-1.8ex]
\multicolumn{4}{l}{$\bar{n}=5$} \\  & \nopagebreak $\;J=50$  & ${-}3.4\phantom{0}$ & ${-}0.3\phantom{0}$ & $\phantom{-}2.4\phantom{0}$ & ${-}0.5\phantom{0}$ & ${-}2.2\phantom{0}$ & $\phantom{0}0.19\phantom{0}$ & $\phantom{0}0.22\phantom{0}$ & $\phantom{0}0.24\phantom{0}$ & $\phantom{0}0.22\phantom{0}$ & $\phantom{0}0.22\phantom{0}$ & $\phantom{0}89.8\phantom{0}$ & $\phantom{0}93.4\phantom{0}$ & $\phantom{0}94.6\phantom{0}$ & $\phantom{0}92.5\phantom{0}$ & $\phantom{0}91.3\phantom{0}$ \\
 & \nopagebreak $\;J=200$  & ${-}0.5\phantom{0}$ & $\phantom{-}0.1\phantom{0}$ & $\phantom{-}0.6\phantom{0}$ & $\phantom{-}0.1\phantom{0}$ & ${-}0.2\phantom{0}$ & $\phantom{0}0.10\phantom{0}$ & $\phantom{0}0.11\phantom{0}$ & $\phantom{0}0.12\phantom{0}$ & $\phantom{0}0.11\phantom{0}$ & $\phantom{0}0.11\phantom{0}$ & $\phantom{0}94.3\phantom{0}$ & $\phantom{0}92.8\phantom{0}$ & $\phantom{0}93.3\phantom{0}$ & $\phantom{0}93.6\phantom{0}$ & $\phantom{0}93.7\phantom{0}$ \\
 & \nopagebreak $\;J=1000$  & ${-}0.0\phantom{0}$ & $\phantom{-}0.1\phantom{0}$ & $\phantom{-}0.2\phantom{0}$ & $\phantom{-}0.2\phantom{0}$ & $\phantom{-}0.1\phantom{0}$ & $\phantom{0}0.04\phantom{0}$ & $\phantom{0}0.05\phantom{0}$ & $\phantom{0}0.05\phantom{0}$ & $\phantom{0}0.05\phantom{0}$ & $\phantom{0}0.05\phantom{0}$ & $\phantom{0}94.2\phantom{0}$ & $\phantom{0}93.9\phantom{0}$ & $\phantom{0}94.2\phantom{0}$ & $\phantom{0}94.5\phantom{0}$ & $\phantom{0}93.9\phantom{0}$ \\
\multicolumn{4}{l}{$\bar{n}=20$} \\  & \nopagebreak $\;J=50$  & ${-}2.1\phantom{0}$ & $\phantom{-}0.4\phantom{0}$ & $\phantom{-}3.5\phantom{0}$ & $\phantom{-}0.4\phantom{0}$ & ${-}1.2\phantom{0}$ & $\phantom{0}0.20\phantom{0}$ & $\phantom{0}0.24\phantom{0}$ & $\phantom{0}0.25\phantom{0}$ & $\phantom{0}0.24\phantom{0}$ & $\phantom{0}0.24\phantom{0}$ & $\phantom{0}89.6\phantom{0}$ & $\phantom{0}90.9\phantom{0}$ & $\phantom{0}93.0\phantom{0}$ & $\phantom{0}91.1\phantom{0}$ & $\phantom{0}89.3\phantom{0}$ \\
 & \nopagebreak $\;J=200$  & ${-}0.5\phantom{0}$ & $\phantom{-}0.1\phantom{0}$ & $\phantom{-}0.7\phantom{0}$ & $\phantom{-}0.1\phantom{0}$ & ${-}0.2\phantom{0}$ & $\phantom{0}0.10\phantom{0}$ & $\phantom{0}0.11\phantom{0}$ & $\phantom{0}0.11\phantom{0}$ & $\phantom{0}0.11\phantom{0}$ & $\phantom{0}0.11\phantom{0}$ & $\phantom{0}93.8\phantom{0}$ & $\phantom{0}94.7\phantom{0}$ & $\phantom{0}95.1\phantom{0}$ & $\phantom{0}94.3\phantom{0}$ & $\phantom{0}94.6\phantom{0}$ \\
 & \nopagebreak $\;J=1000$  & ${-}0.0\phantom{0}$ & $\phantom{-}0.2\phantom{0}$ & $\phantom{-}0.3\phantom{0}$ & $\phantom{-}0.2\phantom{0}$ & $\phantom{-}0.1\phantom{0}$ & $\phantom{0}0.04\phantom{0}$ & $\phantom{0}0.05\phantom{0}$ & $\phantom{0}0.05\phantom{0}$ & $\phantom{0}0.05\phantom{0}$ & $\phantom{0}0.05\phantom{0}$ & $\phantom{0}95.4\phantom{0}$ & $\phantom{0}94.7\phantom{0}$ & $\phantom{0}95.0\phantom{0}$ & $\phantom{0}94.9\phantom{0}$ & $\phantom{0}94.6\phantom{0}$ \\
[0.5ex]\hline\\[-1.6ex] 
\end{tabular}
\begin{tablenotes}[para,flushleft]{\footnotesize \textit{Note.} $\bar{n}$ = average cluster size; $J$ = number of clusters; CD = complete data sets; LD = listwise deletion; FCS-SL = single-level FCS; FCS-MAN = two-level FCS with manifest cluster means; FCS-LAT = two-level FCS with latent cluster means; JM = joint modeling.}\end{tablenotes}
\end{threeparttable}
\end{sidewaystable}
\begin{sidewaystable}
\begin{threeparttable}
\setlength{\tabcolsep}{1.0pt}
\renewcommand{\arraystretch}{0.95}
\footnotesize
\caption{\small Study 2: Bias (in \%), Relative RMSE, and Coverage of the 95\% Confidence Interval for the Variance of $z$ ($\hat\sigma_z^2$) With Strongly Unbalanced Data (Bimodal, $\pm 80\%$) and 20\% Missing Data (MAR, $\lambda=0.5$)}
\begin{tabular}{llccccccccccccccc}
\hline\\[-1.8ex]
& & \multicolumn{5}{c}{Bias (\%)} & \multicolumn{5}{c}{Rel. RMSE} & \multicolumn{5}{c}{Coverage (\%)} \\ \cmidrule(r){3-7}\cmidrule(r){8-12}\cmidrule(r){13-17}
 &  & CD & \makecell{FCS-\\MAN} & \makecell{FCS-\\NJ} & \makecell{FCS-\\LAT} & JM & CD & \makecell{FCS-\\MAN} & \makecell{FCS-\\NJ} & \makecell{FCS-\\LAT} & JM & CD & \makecell{FCS-\\MAN} & \makecell{FCS-\\NJ} & \makecell{FCS-\\LAT} & \multicolumn{1}{c}{JM} \\ 
[0.4ex]\hline\\[-1.8ex]
& & \multicolumn{15}{c}{Small intraclass correlation $(\rho_{Iy}=.10)$} \\[0.6ex]\hline\\[-1.8ex]
\multicolumn{4}{l}{$\bar{n}=5$} \\  & \nopagebreak $\;J=50$  & ${-}2.2\phantom{0}$ & $\phantom{-}1.4\phantom{0}$ & $\phantom{-}4.1\phantom{0}$ & $\phantom{-}0.7\phantom{0}$ & ${-}1.4\phantom{0}$ & $\phantom{0}0.21\phantom{0}$ & $\phantom{0}0.24\phantom{0}$ & $\phantom{0}0.25\phantom{0}$ & $\phantom{0}0.24\phantom{0}$ & $\phantom{0}0.23\phantom{0}$ & $\phantom{0}88.9\phantom{0}$ & $\phantom{0}91.7\phantom{0}$ & $\phantom{0}93.8\phantom{0}$ & $\phantom{0}91.4\phantom{0}$ & $\phantom{0}90.6\phantom{0}$ \\
 & \nopagebreak $\;J=200$  & ${-}1.0\phantom{0}$ & ${-}0.0\phantom{0}$ & $\phantom{-}0.3\phantom{0}$ & ${-}0.3\phantom{0}$ & ${-}0.9\phantom{0}$ & $\phantom{0}0.10\phantom{0}$ & $\phantom{0}0.12\phantom{0}$ & $\phantom{0}0.11\phantom{0}$ & $\phantom{0}0.11\phantom{0}$ & $\phantom{0}0.11\phantom{0}$ & $\phantom{0}92.3\phantom{0}$ & $\phantom{0}92.8\phantom{0}$ & $\phantom{0}94.1\phantom{0}$ & $\phantom{0}93.0\phantom{0}$ & $\phantom{0}92.5\phantom{0}$ \\
 & \nopagebreak $\;J=1000$  & ${-}0.1\phantom{0}$ & $\phantom{-}0.3\phantom{0}$ & $\phantom{-}0.1\phantom{0}$ & $\phantom{-}0.1\phantom{0}$ & ${-}0.1\phantom{0}$ & $\phantom{0}0.05\phantom{0}$ & $\phantom{0}0.05\phantom{0}$ & $\phantom{0}0.05\phantom{0}$ & $\phantom{0}0.05\phantom{0}$ & $\phantom{0}0.05\phantom{0}$ & $\phantom{0}94.6\phantom{0}$ & $\phantom{0}94.0\phantom{0}$ & $\phantom{0}94.5\phantom{0}$ & $\phantom{0}94.6\phantom{0}$ & $\phantom{0}94.7\phantom{0}$ \\
\multicolumn{4}{l}{$\bar{n}=20$} \\  & \nopagebreak $\;J=50$  & ${-}2.9\phantom{0}$ & $\phantom{-}0.6\phantom{0}$ & $\phantom{-}3.1\phantom{0}$ & $\phantom{-}0.5\phantom{0}$ & ${-}2.4\phantom{0}$ & $\phantom{0}0.19\phantom{0}$ & $\phantom{0}0.23\phantom{0}$ & $\phantom{0}0.23\phantom{0}$ & $\phantom{0}0.23\phantom{0}$ & $\phantom{0}0.22\phantom{0}$ & $\phantom{0}90.7\phantom{0}$ & $\phantom{0}93.1\phantom{0}$ & $\phantom{0}95.0\phantom{0}$ & $\phantom{0}92.6\phantom{0}$ & $\phantom{0}90.9\phantom{0}$ \\
 & \nopagebreak $\;J=200$  & ${-}0.7\phantom{0}$ & $\phantom{-}0.3\phantom{0}$ & $\phantom{-}0.5\phantom{0}$ & ${-}0.0\phantom{0}$ & ${-}0.7\phantom{0}$ & $\phantom{0}0.10\phantom{0}$ & $\phantom{0}0.11\phantom{0}$ & $\phantom{0}0.11\phantom{0}$ & $\phantom{0}0.11\phantom{0}$ & $\phantom{0}0.11\phantom{0}$ & $\phantom{0}92.4\phantom{0}$ & $\phantom{0}94.7\phantom{0}$ & $\phantom{0}94.5\phantom{0}$ & $\phantom{0}93.9\phantom{0}$ & $\phantom{0}92.8\phantom{0}$ \\
 & \nopagebreak $\;J=1000$  & ${-}0.1\phantom{0}$ & $\phantom{-}0.2\phantom{0}$ & $\phantom{-}0.0\phantom{0}$ & ${-}0.0\phantom{0}$ & ${-}0.2\phantom{0}$ & $\phantom{0}0.05\phantom{0}$ & $\phantom{0}0.05\phantom{0}$ & $\phantom{0}0.05\phantom{0}$ & $\phantom{0}0.05\phantom{0}$ & $\phantom{0}0.05\phantom{0}$ & $\phantom{0}93.9\phantom{0}$ & $\phantom{0}92.7\phantom{0}$ & $\phantom{0}93.0\phantom{0}$ & $\phantom{0}93.3\phantom{0}$ & $\phantom{0}92.7\phantom{0}$ \\
[0.5ex]\hline\\[-1.6ex] 
& & \multicolumn{15}{c}{Moderate intraclass correlation $(\rho_{Iy}=.30)$} \\[0.6ex]\hline\\[-1.8ex]
\multicolumn{4}{l}{$\bar{n}=5$} \\  & \nopagebreak $\;J=50$  & ${-}1.7\phantom{0}$ & $\phantom{-}1.8\phantom{0}$ & $\phantom{-}4.2\phantom{0}$ & $\phantom{-}1.1\phantom{0}$ & ${-}0.6\phantom{0}$ & $\phantom{0}0.19\phantom{0}$ & $\phantom{0}0.23\phantom{0}$ & $\phantom{0}0.24\phantom{0}$ & $\phantom{0}0.23\phantom{0}$ & $\phantom{0}0.22\phantom{0}$ & $\phantom{0}90.5\phantom{0}$ & $\phantom{0}92.5\phantom{0}$ & $\phantom{0}94.2\phantom{0}$ & $\phantom{0}92.4\phantom{0}$ & $\phantom{0}91.9\phantom{0}$ \\
 & \nopagebreak $\;J=200$  & ${-}0.8\phantom{0}$ & ${-}0.0\phantom{0}$ & $\phantom{-}0.3\phantom{0}$ & ${-}0.2\phantom{0}$ & ${-}0.7\phantom{0}$ & $\phantom{0}0.10\phantom{0}$ & $\phantom{0}0.11\phantom{0}$ & $\phantom{0}0.11\phantom{0}$ & $\phantom{0}0.11\phantom{0}$ & $\phantom{0}0.11\phantom{0}$ & $\phantom{0}94.5\phantom{0}$ & $\phantom{0}94.7\phantom{0}$ & $\phantom{0}95.2\phantom{0}$ & $\phantom{0}94.8\phantom{0}$ & $\phantom{0}94.7\phantom{0}$ \\
 & \nopagebreak $\;J=1000$  & ${-}0.2\phantom{0}$ & $\phantom{-}0.1\phantom{0}$ & ${-}0.1\phantom{0}$ & ${-}0.1\phantom{0}$ & ${-}0.1\phantom{0}$ & $\phantom{0}0.05\phantom{0}$ & $\phantom{0}0.05\phantom{0}$ & $\phantom{0}0.05\phantom{0}$ & $\phantom{0}0.05\phantom{0}$ & $\phantom{0}0.05\phantom{0}$ & $\phantom{0}95.0\phantom{0}$ & $\phantom{0}94.7\phantom{0}$ & $\phantom{0}93.7\phantom{0}$ & $\phantom{0}94.0\phantom{0}$ & $\phantom{0}94.0\phantom{0}$ \\
\multicolumn{4}{l}{$\bar{n}=20$} \\  & \nopagebreak $\;J=50$  & ${-}3.2\phantom{0}$ & $\phantom{-}0.2\phantom{0}$ & $\phantom{-}2.8\phantom{0}$ & ${-}0.0\phantom{0}$ & ${-}1.7\phantom{0}$ & $\phantom{0}0.20\phantom{0}$ & $\phantom{0}0.23\phantom{0}$ & $\phantom{0}0.25\phantom{0}$ & $\phantom{0}0.23\phantom{0}$ & $\phantom{0}0.23\phantom{0}$ & $\phantom{0}90.6\phantom{0}$ & $\phantom{0}92.0\phantom{0}$ & $\phantom{0}94.5\phantom{0}$ & $\phantom{0}92.4\phantom{0}$ & $\phantom{0}91.4\phantom{0}$ \\
 & \nopagebreak $\;J=200$  & ${-}0.6\phantom{0}$ & ${-}0.0\phantom{0}$ & $\phantom{-}0.4\phantom{0}$ & ${-}0.1\phantom{0}$ & ${-}0.5\phantom{0}$ & $\phantom{0}0.10\phantom{0}$ & $\phantom{0}0.11\phantom{0}$ & $\phantom{0}0.11\phantom{0}$ & $\phantom{0}0.11\phantom{0}$ & $\phantom{0}0.11\phantom{0}$ & $\phantom{0}94.1\phantom{0}$ & $\phantom{0}93.9\phantom{0}$ & $\phantom{0}94.7\phantom{0}$ & $\phantom{0}94.2\phantom{0}$ & $\phantom{0}94.1\phantom{0}$ \\
 & \nopagebreak $\;J=1000$  & ${-}0.2\phantom{0}$ & $\phantom{-}0.0\phantom{0}$ & $\phantom{-}0.0\phantom{0}$ & ${-}0.1\phantom{0}$ & ${-}0.1\phantom{0}$ & $\phantom{0}0.04\phantom{0}$ & $\phantom{0}0.05\phantom{0}$ & $\phantom{0}0.05\phantom{0}$ & $\phantom{0}0.05\phantom{0}$ & $\phantom{0}0.05\phantom{0}$ & $\phantom{0}94.7\phantom{0}$ & $\phantom{0}95.0\phantom{0}$ & $\phantom{0}95.4\phantom{0}$ & $\phantom{0}94.7\phantom{0}$ & $\phantom{0}95.7\phantom{0}$ \\
[0.5ex]\hline\\[-1.6ex] 
\end{tabular}
\begin{tablenotes}[para,flushleft]{\footnotesize \textit{Note.} $\bar{n}$ = average cluster size; $J$ = number of clusters; CD = complete data sets; LD = listwise deletion; FCS-SL = single-level FCS; FCS-MAN = two-level FCS with manifest cluster means; FCS-LAT = two-level FCS with latent cluster means; JM = joint modeling.}\end{tablenotes}
\end{threeparttable}
\end{sidewaystable}
\begin{sidewaystable}
\begin{threeparttable}
\setlength{\tabcolsep}{1.0pt}
\renewcommand{\arraystretch}{0.95}
\footnotesize
\caption{\small Study 2: Bias (in \%), Relative RMSE, and Coverage of the 95\% Confidence Interval for the Variance of $z$ ($\hat\sigma_z^2$) With Moderately Unbalanced Data (Uniform, $\pm 40\%$) and 40\% Missing Data (MAR, $\lambda=0.5$)}
\begin{tabular}{llccccccccccccccc}
\hline\\[-1.8ex]
& & \multicolumn{5}{c}{Bias (\%)} & \multicolumn{5}{c}{Rel. RMSE} & \multicolumn{5}{c}{Coverage (\%)} \\ \cmidrule(r){3-7}\cmidrule(r){8-12}\cmidrule(r){13-17}
 &  & CD & \makecell{FCS-\\MAN} & \makecell{FCS-\\NJ} & \makecell{FCS-\\LAT} & JM & CD & \makecell{FCS-\\MAN} & \makecell{FCS-\\NJ} & \makecell{FCS-\\LAT} & JM & CD & \makecell{FCS-\\MAN} & \makecell{FCS-\\NJ} & \makecell{FCS-\\LAT} & \multicolumn{1}{c}{JM} \\ 
[0.4ex]\hline\\[-1.8ex]
& & \multicolumn{15}{c}{Small intraclass correlation $(\rho_{Iy}=.10)$} \\[0.6ex]\hline\\[-1.8ex]
\multicolumn{4}{l}{$\bar{n}=5$} \\  & \nopagebreak $\;J=50$  & $\phantom{0}{-}2.4\phantom{0}$ & $\phantom{0}\phantom{-}4.0\phantom{0}$ & $\phantom{-}16.0\phantom{0}$ & $\phantom{0}\phantom{-}4.6\phantom{0}$ & $\phantom{0}{-}2.0\phantom{0}$ & $\phantom{0}0.20\phantom{0}$ & $\phantom{0}0.29\phantom{0}$ & $\phantom{0}0.38\phantom{0}$ & $\phantom{0}0.29\phantom{0}$ & $\phantom{0}0.26\phantom{0}$ & $\phantom{0}90.0\phantom{0}$ & $\phantom{0}94.1\phantom{0}$ & $\phantom{0}96.3\phantom{0}$ & $\phantom{0}93.1\phantom{0}$ & $\phantom{0}90.7\phantom{0}$ \\
 & \nopagebreak $\;J=200$  & $\phantom{0}{-}0.2\phantom{0}$ & $\phantom{0}\phantom{-}1.0\phantom{0}$ & $\phantom{0}\phantom{-}2.9\phantom{0}$ & $\phantom{0}\phantom{-}1.3\phantom{0}$ & $\phantom{0}{-}0.5\phantom{0}$ & $\phantom{0}0.10\phantom{0}$ & $\phantom{0}0.14\phantom{0}$ & $\phantom{0}0.14\phantom{0}$ & $\phantom{0}0.14\phantom{0}$ & $\phantom{0}0.13\phantom{0}$ & $\phantom{0}93.5\phantom{0}$ & $\phantom{0}94.6\phantom{0}$ & $\phantom{0}95.9\phantom{0}$ & $\phantom{0}95.7\phantom{0}$ & $\phantom{0}93.8\phantom{0}$ \\
 & \nopagebreak $\;J=1000$  & $\phantom{0}{-}0.3\phantom{0}$ & $\phantom{0}{-}0.2\phantom{0}$ & $\phantom{0}\phantom{-}0.2\phantom{0}$ & $\phantom{0}{-}0.1\phantom{0}$ & $\phantom{0}{-}0.5\phantom{0}$ & $\phantom{0}0.04\phantom{0}$ & $\phantom{0}0.06\phantom{0}$ & $\phantom{0}0.06\phantom{0}$ & $\phantom{0}0.06\phantom{0}$ & $\phantom{0}0.06\phantom{0}$ & $\phantom{0}93.6\phantom{0}$ & $\phantom{0}92.9\phantom{0}$ & $\phantom{0}93.6\phantom{0}$ & $\phantom{0}93.3\phantom{0}$ & $\phantom{0}93.4\phantom{0}$ \\
\multicolumn{4}{l}{$\bar{n}=20$} \\  & \nopagebreak $\;J=50$  & $\phantom{0}{-}2.0\phantom{0}$ & $\phantom{0}\phantom{-}4.6\phantom{0}$ & $\phantom{-}14.5\phantom{0}$ & $\phantom{0}\phantom{-}5.3\phantom{0}$ & $\phantom{0}{-}1.5\phantom{0}$ & $\phantom{0}0.19\phantom{0}$ & $\phantom{0}0.28\phantom{0}$ & $\phantom{0}0.36\phantom{0}$ & $\phantom{0}0.29\phantom{0}$ & $\phantom{0}0.25\phantom{0}$ & $\phantom{0}90.6\phantom{0}$ & $\phantom{0}94.0\phantom{0}$ & $\phantom{0}96.0\phantom{0}$ & $\phantom{0}92.7\phantom{0}$ & $\phantom{0}91.5\phantom{0}$ \\
 & \nopagebreak $\;J=200$  & $\phantom{0}{-}0.1\phantom{0}$ & $\phantom{0}\phantom{-}1.5\phantom{0}$ & $\phantom{0}\phantom{-}2.9\phantom{0}$ & $\phantom{0}\phantom{-}1.3\phantom{0}$ & $\phantom{0}{-}0.5\phantom{0}$ & $\phantom{0}0.10\phantom{0}$ & $\phantom{0}0.14\phantom{0}$ & $\phantom{0}0.14\phantom{0}$ & $\phantom{0}0.14\phantom{0}$ & $\phantom{0}0.13\phantom{0}$ & $\phantom{0}94.8\phantom{0}$ & $\phantom{0}94.6\phantom{0}$ & $\phantom{0}94.4\phantom{0}$ & $\phantom{0}95.1\phantom{0}$ & $\phantom{0}93.8\phantom{0}$ \\
 & \nopagebreak $\;J=1000$  & $\phantom{0}{-}0.1\phantom{0}$ & $\phantom{0}\phantom{-}0.3\phantom{0}$ & $\phantom{0}\phantom{-}0.6\phantom{0}$ & $\phantom{0}\phantom{-}0.3\phantom{0}$ & $\phantom{0}{-}0.0\phantom{0}$ & $\phantom{0}0.05\phantom{0}$ & $\phantom{0}0.06\phantom{0}$ & $\phantom{0}0.06\phantom{0}$ & $\phantom{0}0.06\phantom{0}$ & $\phantom{0}0.06\phantom{0}$ & $\phantom{0}93.9\phantom{0}$ & $\phantom{0}93.4\phantom{0}$ & $\phantom{0}94.9\phantom{0}$ & $\phantom{0}94.4\phantom{0}$ & $\phantom{0}95.0\phantom{0}$ \\
[0.5ex]\hline\\[-1.6ex] 
& & \multicolumn{15}{c}{Moderate intraclass correlation $(\rho_{Iy}=.30)$} \\[0.6ex]\hline\\[-1.8ex]
\multicolumn{4}{l}{$\bar{n}=5$} \\  & \nopagebreak $\;J=50$  & $\phantom{0}{-}2.0\phantom{0}$ & $\phantom{0}\phantom{-}5.0\phantom{0}$ & $\phantom{-}15.9\phantom{0}$ & $\phantom{0}\phantom{-}6.0\phantom{0}$ & $\phantom{0}\phantom{-}0.4\phantom{0}$ & $\phantom{0}0.20\phantom{0}$ & $\phantom{0}0.28\phantom{0}$ & $\phantom{0}0.39\phantom{0}$ & $\phantom{0}0.29\phantom{0}$ & $\phantom{0}0.26\phantom{0}$ & $\phantom{0}90.1\phantom{0}$ & $\phantom{0}94.5\phantom{0}$ & $\phantom{0}97.2\phantom{0}$ & $\phantom{0}94.0\phantom{0}$ & $\phantom{0}91.6\phantom{0}$ \\
 & \nopagebreak $\;J=200$  & $\phantom{0}{-}0.3\phantom{0}$ & $\phantom{0}\phantom{-}1.2\phantom{0}$ & $\phantom{0}\phantom{-}2.9\phantom{0}$ & $\phantom{0}\phantom{-}1.2\phantom{0}$ & $\phantom{0}\phantom{-}0.3\phantom{0}$ & $\phantom{0}0.10\phantom{0}$ & $\phantom{0}0.13\phantom{0}$ & $\phantom{0}0.14\phantom{0}$ & $\phantom{0}0.14\phantom{0}$ & $\phantom{0}0.13\phantom{0}$ & $\phantom{0}93.4\phantom{0}$ & $\phantom{0}94.7\phantom{0}$ & $\phantom{0}95.7\phantom{0}$ & $\phantom{0}94.8\phantom{0}$ & $\phantom{0}94.8\phantom{0}$ \\
 & \nopagebreak $\;J=1000$  & $\phantom{0}{-}0.1\phantom{0}$ & $\phantom{0}\phantom{-}0.4\phantom{0}$ & $\phantom{0}\phantom{-}0.6\phantom{0}$ & $\phantom{0}\phantom{-}0.3\phantom{0}$ & $\phantom{0}\phantom{-}0.2\phantom{0}$ & $\phantom{0}0.04\phantom{0}$ & $\phantom{0}0.06\phantom{0}$ & $\phantom{0}0.06\phantom{0}$ & $\phantom{0}0.06\phantom{0}$ & $\phantom{0}0.06\phantom{0}$ & $\phantom{0}96.1\phantom{0}$ & $\phantom{0}93.3\phantom{0}$ & $\phantom{0}94.5\phantom{0}$ & $\phantom{0}95.0\phantom{0}$ & $\phantom{0}94.3\phantom{0}$ \\
\multicolumn{4}{l}{$\bar{n}=20$} \\  & \nopagebreak $\;J=50$  & $\phantom{0}{-}2.0\phantom{0}$ & $\phantom{0}\phantom{-}3.4\phantom{0}$ & $\phantom{-}13.1\phantom{0}$ & $\phantom{0}\phantom{-}4.0\phantom{0}$ & $\phantom{0}{-}0.0\phantom{0}$ & $\phantom{0}0.21\phantom{0}$ & $\phantom{0}0.29\phantom{0}$ & $\phantom{0}0.36\phantom{0}$ & $\phantom{0}0.29\phantom{0}$ & $\phantom{0}0.28\phantom{0}$ & $\phantom{0}88.4\phantom{0}$ & $\phantom{0}92.2\phantom{0}$ & $\phantom{0}95.3\phantom{0}$ & $\phantom{0}92.7\phantom{0}$ & $\phantom{0}90.7\phantom{0}$ \\
 & \nopagebreak $\;J=200$  & $\phantom{0}{-}0.6\phantom{0}$ & $\phantom{0}\phantom{-}0.8\phantom{0}$ & $\phantom{0}\phantom{-}2.2\phantom{0}$ & $\phantom{0}\phantom{-}0.8\phantom{0}$ & $\phantom{0}\phantom{-}0.0\phantom{0}$ & $\phantom{0}0.10\phantom{0}$ & $\phantom{0}0.14\phantom{0}$ & $\phantom{0}0.14\phantom{0}$ & $\phantom{0}0.14\phantom{0}$ & $\phantom{0}0.14\phantom{0}$ & $\phantom{0}93.0\phantom{0}$ & $\phantom{0}93.1\phantom{0}$ & $\phantom{0}93.7\phantom{0}$ & $\phantom{0}92.6\phantom{0}$ & $\phantom{0}93.2\phantom{0}$ \\
 & \nopagebreak $\;J=1000$  & $\phantom{0}{-}0.0\phantom{0}$ & $\phantom{0}\phantom{-}0.1\phantom{0}$ & $\phantom{0}\phantom{-}0.4\phantom{0}$ & $\phantom{0}\phantom{-}0.1\phantom{0}$ & $\phantom{0}{-}0.1\phantom{0}$ & $\phantom{0}0.05\phantom{0}$ & $\phantom{0}0.06\phantom{0}$ & $\phantom{0}0.06\phantom{0}$ & $\phantom{0}0.06\phantom{0}$ & $\phantom{0}0.06\phantom{0}$ & $\phantom{0}93.9\phantom{0}$ & $\phantom{0}93.8\phantom{0}$ & $\phantom{0}94.2\phantom{0}$ & $\phantom{0}94.2\phantom{0}$ & $\phantom{0}93.8\phantom{0}$ \\
[0.5ex]\hline\\[-1.6ex] 
\end{tabular}
\begin{tablenotes}[para,flushleft]{\footnotesize \textit{Note.} $\bar{n}$ = average cluster size; $J$ = number of clusters; CD = complete data sets; LD = listwise deletion; FCS-SL = single-level FCS; FCS-MAN = two-level FCS with manifest cluster means; FCS-LAT = two-level FCS with latent cluster means; JM = joint modeling.}\end{tablenotes}
\end{threeparttable}
\end{sidewaystable}
\begin{sidewaystable}
\begin{threeparttable}
\setlength{\tabcolsep}{1.0pt}
\renewcommand{\arraystretch}{0.95}
\footnotesize
\caption{\small Study 2: Bias (in \%), Relative RMSE, and Coverage of the 95\% Confidence Interval for the Variance of $z$ ($\hat\sigma_z^2$) With Strongly Unbalanced Data (Uniform, $\pm 80\%$) and 40\% Missing Data (MAR, $\lambda=0.5$)}
\begin{tabular}{llccccccccccccccc}
\hline\\[-1.8ex]
& & \multicolumn{5}{c}{Bias (\%)} & \multicolumn{5}{c}{Rel. RMSE} & \multicolumn{5}{c}{Coverage (\%)} \\ \cmidrule(r){3-7}\cmidrule(r){8-12}\cmidrule(r){13-17}
 &  & CD & \makecell{FCS-\\MAN} & \makecell{FCS-\\NJ} & \makecell{FCS-\\LAT} & JM & CD & \makecell{FCS-\\MAN} & \makecell{FCS-\\NJ} & \makecell{FCS-\\LAT} & JM & CD & \makecell{FCS-\\MAN} & \makecell{FCS-\\NJ} & \makecell{FCS-\\LAT} & \multicolumn{1}{c}{JM} \\ 
[0.4ex]\hline\\[-1.8ex]
& & \multicolumn{15}{c}{Small intraclass correlation $(\rho_{Iy}=.10)$} \\[0.6ex]\hline\\[-1.8ex]
\multicolumn{4}{l}{$\bar{n}=5$} \\  & \nopagebreak $\;J=50$  & $\phantom{0}{-}1.9\phantom{0}$ & $\phantom{0}\phantom{-}5.3\phantom{0}$ & $\phantom{-}17.2\phantom{0}$ & $\phantom{0}\phantom{-}5.6\phantom{0}$ & $\phantom{0}{-}0.4\phantom{0}$ & $\phantom{0}0.21\phantom{0}$ & $\phantom{0}0.31\phantom{0}$ & $\phantom{0}0.41\phantom{0}$ & $\phantom{0}0.31\phantom{0}$ & $\phantom{0}0.28\phantom{0}$ & $\phantom{0}89.4\phantom{0}$ & $\phantom{0}91.7\phantom{0}$ & $\phantom{0}95.4\phantom{0}$ & $\phantom{0}91.8\phantom{0}$ & $\phantom{0}88.4\phantom{0}$ \\
 & \nopagebreak $\;J=200$  & $\phantom{0}{-}0.7\phantom{0}$ & $\phantom{0}\phantom{-}1.2\phantom{0}$ & $\phantom{0}\phantom{-}3.0\phantom{0}$ & $\phantom{0}\phantom{-}1.2\phantom{0}$ & $\phantom{0}{-}0.8\phantom{0}$ & $\phantom{0}0.10\phantom{0}$ & $\phantom{0}0.14\phantom{0}$ & $\phantom{0}0.15\phantom{0}$ & $\phantom{0}0.14\phantom{0}$ & $\phantom{0}0.13\phantom{0}$ & $\phantom{0}93.2\phantom{0}$ & $\phantom{0}94.3\phantom{0}$ & $\phantom{0}94.2\phantom{0}$ & $\phantom{0}94.0\phantom{0}$ & $\phantom{0}93.8\phantom{0}$ \\
 & \nopagebreak $\;J=1000$  & $\phantom{0}{-}0.2\phantom{0}$ & $\phantom{0}\phantom{-}0.2\phantom{0}$ & $\phantom{0}\phantom{-}0.4\phantom{0}$ & $\phantom{0}\phantom{-}0.1\phantom{0}$ & $\phantom{0}{-}0.4\phantom{0}$ & $\phantom{0}0.05\phantom{0}$ & $\phantom{0}0.06\phantom{0}$ & $\phantom{0}0.06\phantom{0}$ & $\phantom{0}0.06\phantom{0}$ & $\phantom{0}0.06\phantom{0}$ & $\phantom{0}94.0\phantom{0}$ & $\phantom{0}94.0\phantom{0}$ & $\phantom{0}94.6\phantom{0}$ & $\phantom{0}93.9\phantom{0}$ & $\phantom{0}93.4\phantom{0}$ \\
\multicolumn{4}{l}{$\bar{n}=20$} \\  & \nopagebreak $\;J=50$  & $\phantom{0}{-}2.1\phantom{0}$ & $\phantom{0}\phantom{-}5.3\phantom{0}$ & $\phantom{-}15.4\phantom{0}$ & $\phantom{0}\phantom{-}5.0\phantom{0}$ & $\phantom{0}{-}1.8\phantom{0}$ & $\phantom{0}0.20\phantom{0}$ & $\phantom{0}0.30\phantom{0}$ & $\phantom{0}0.39\phantom{0}$ & $\phantom{0}0.31\phantom{0}$ & $\phantom{0}0.26\phantom{0}$ & $\phantom{0}88.7\phantom{0}$ & $\phantom{0}91.6\phantom{0}$ & $\phantom{0}95.7\phantom{0}$ & $\phantom{0}92.3\phantom{0}$ & $\phantom{0}90.9\phantom{0}$ \\
 & \nopagebreak $\;J=200$  & $\phantom{0}{-}0.8\phantom{0}$ & $\phantom{0}{-}0.0\phantom{0}$ & $\phantom{0}\phantom{-}1.8\phantom{0}$ & $\phantom{0}\phantom{-}0.0\phantom{0}$ & $\phantom{0}{-}1.8\phantom{0}$ & $\phantom{0}0.10\phantom{0}$ & $\phantom{0}0.14\phantom{0}$ & $\phantom{0}0.14\phantom{0}$ & $\phantom{0}0.14\phantom{0}$ & $\phantom{0}0.14\phantom{0}$ & $\phantom{0}93.4\phantom{0}$ & $\phantom{0}92.6\phantom{0}$ & $\phantom{0}93.5\phantom{0}$ & $\phantom{0}92.3\phantom{0}$ & $\phantom{0}91.9\phantom{0}$ \\
 & \nopagebreak $\;J=1000$  & $\phantom{0}{-}0.1\phantom{0}$ & $\phantom{0}\phantom{-}0.4\phantom{0}$ & $\phantom{0}\phantom{-}0.7\phantom{0}$ & $\phantom{0}\phantom{-}0.4\phantom{0}$ & $\phantom{0}{-}0.0\phantom{0}$ & $\phantom{0}0.04\phantom{0}$ & $\phantom{0}0.06\phantom{0}$ & $\phantom{0}0.06\phantom{0}$ & $\phantom{0}0.06\phantom{0}$ & $\phantom{0}0.06\phantom{0}$ & $\phantom{0}95.4\phantom{0}$ & $\phantom{0}95.0\phantom{0}$ & $\phantom{0}95.1\phantom{0}$ & $\phantom{0}95.9\phantom{0}$ & $\phantom{0}94.8\phantom{0}$ \\
[0.5ex]\hline\\[-1.6ex] 
& & \multicolumn{15}{c}{Moderate intraclass correlation $(\rho_{Iy}=.30)$} \\[0.6ex]\hline\\[-1.8ex]
\multicolumn{4}{l}{$\bar{n}=5$} \\  & \nopagebreak $\;J=50$  & $\phantom{0}{-}1.5\phantom{0}$ & $\phantom{0}\phantom{-}6.9\phantom{0}$ & $\phantom{-}17.1\phantom{0}$ & $\phantom{0}\phantom{-}7.3\phantom{0}$ & $\phantom{0}\phantom{-}1.3\phantom{0}$ & $\phantom{0}0.19\phantom{0}$ & $\phantom{0}0.30\phantom{0}$ & $\phantom{0}0.38\phantom{0}$ & $\phantom{0}0.30\phantom{0}$ & $\phantom{0}0.26\phantom{0}$ & $\phantom{0}90.5\phantom{0}$ & $\phantom{0}95.2\phantom{0}$ & $\phantom{0}97.4\phantom{0}$ & $\phantom{0}94.1\phantom{0}$ & $\phantom{0}93.3\phantom{0}$ \\
 & \nopagebreak $\;J=200$  & $\phantom{0}{-}1.0\phantom{0}$ & $\phantom{0}\phantom{-}0.6\phantom{0}$ & $\phantom{0}\phantom{-}2.2\phantom{0}$ & $\phantom{0}\phantom{-}0.6\phantom{0}$ & $\phantom{0}{-}0.5\phantom{0}$ & $\phantom{0}0.10\phantom{0}$ & $\phantom{0}0.14\phantom{0}$ & $\phantom{0}0.14\phantom{0}$ & $\phantom{0}0.14\phantom{0}$ & $\phantom{0}0.14\phantom{0}$ & $\phantom{0}93.3\phantom{0}$ & $\phantom{0}93.0\phantom{0}$ & $\phantom{0}94.8\phantom{0}$ & $\phantom{0}93.7\phantom{0}$ & $\phantom{0}93.0\phantom{0}$ \\
 & \nopagebreak $\;J=1000$  & $\phantom{0}{-}0.4\phantom{0}$ & $\phantom{0}\phantom{-}0.0\phantom{0}$ & $\phantom{0}\phantom{-}0.2\phantom{0}$ & $\phantom{0}{-}0.1\phantom{0}$ & $\phantom{0}{-}0.3\phantom{0}$ & $\phantom{0}0.04\phantom{0}$ & $\phantom{0}0.06\phantom{0}$ & $\phantom{0}0.06\phantom{0}$ & $\phantom{0}0.06\phantom{0}$ & $\phantom{0}0.06\phantom{0}$ & $\phantom{0}95.4\phantom{0}$ & $\phantom{0}95.6\phantom{0}$ & $\phantom{0}96.1\phantom{0}$ & $\phantom{0}95.4\phantom{0}$ & $\phantom{0}94.8\phantom{0}$ \\
\multicolumn{4}{l}{$\bar{n}=20$} \\  & \nopagebreak $\;J=50$  & $\phantom{0}{-}1.0\phantom{0}$ & $\phantom{0}\phantom{-}6.1\phantom{0}$ & $\phantom{-}16.0\phantom{0}$ & $\phantom{0}\phantom{-}5.8\phantom{0}$ & $\phantom{0}\phantom{-}2.2\phantom{0}$ & $\phantom{0}0.20\phantom{0}$ & $\phantom{0}0.30\phantom{0}$ & $\phantom{0}0.37\phantom{0}$ & $\phantom{0}0.29\phantom{0}$ & $\phantom{0}0.28\phantom{0}$ & $\phantom{0}89.1\phantom{0}$ & $\phantom{0}94.1\phantom{0}$ & $\phantom{0}96.6\phantom{0}$ & $\phantom{0}92.5\phantom{0}$ & $\phantom{0}92.7\phantom{0}$ \\
 & \nopagebreak $\;J=200$  & $\phantom{0}{-}0.5\phantom{0}$ & $\phantom{0}\phantom{-}0.4\phantom{0}$ & $\phantom{0}\phantom{-}2.1\phantom{0}$ & $\phantom{0}\phantom{-}0.5\phantom{0}$ & $\phantom{0}{-}0.3\phantom{0}$ & $\phantom{0}0.10\phantom{0}$ & $\phantom{0}0.14\phantom{0}$ & $\phantom{0}0.15\phantom{0}$ & $\phantom{0}0.14\phantom{0}$ & $\phantom{0}0.14\phantom{0}$ & $\phantom{0}93.4\phantom{0}$ & $\phantom{0}93.4\phantom{0}$ & $\phantom{0}93.4\phantom{0}$ & $\phantom{0}92.5\phantom{0}$ & $\phantom{0}93.2\phantom{0}$ \\
 & \nopagebreak $\;J=1000$  & $\phantom{0}{-}0.1\phantom{0}$ & $\phantom{0}\phantom{-}0.1\phantom{0}$ & $\phantom{0}\phantom{-}0.4\phantom{0}$ & $\phantom{0}\phantom{-}0.1\phantom{0}$ & $\phantom{0}{-}0.1\phantom{0}$ & $\phantom{0}0.05\phantom{0}$ & $\phantom{0}0.06\phantom{0}$ & $\phantom{0}0.06\phantom{0}$ & $\phantom{0}0.06\phantom{0}$ & $\phantom{0}0.06\phantom{0}$ & $\phantom{0}94.8\phantom{0}$ & $\phantom{0}94.1\phantom{0}$ & $\phantom{0}93.8\phantom{0}$ & $\phantom{0}93.9\phantom{0}$ & $\phantom{0}93.2\phantom{0}$ \\
[0.5ex]\hline\\[-1.6ex] 
\end{tabular}
\begin{tablenotes}[para,flushleft]{\footnotesize \textit{Note.} $\bar{n}$ = average cluster size; $J$ = number of clusters; CD = complete data sets; LD = listwise deletion; FCS-SL = single-level FCS; FCS-MAN = two-level FCS with manifest cluster means; FCS-LAT = two-level FCS with latent cluster means; JM = joint modeling.}\end{tablenotes}
\end{threeparttable}
\end{sidewaystable}
\begin{sidewaystable}
\begin{threeparttable}
\setlength{\tabcolsep}{1.0pt}
\renewcommand{\arraystretch}{0.95}
\footnotesize
\caption{\small Study 2: Bias (in \%), Relative RMSE, and Coverage of the 95\% Confidence Interval for the Variance of $z$ ($\hat\sigma_z^2$) With Moderately Unbalanced Data (Bimodal, $\pm 40\%$) and 40\% Missing Data (MAR, $\lambda=0.5$)}
\begin{tabular}{llccccccccccccccc}
\hline\\[-1.8ex]
& & \multicolumn{5}{c}{Bias (\%)} & \multicolumn{5}{c}{Rel. RMSE} & \multicolumn{5}{c}{Coverage (\%)} \\ \cmidrule(r){3-7}\cmidrule(r){8-12}\cmidrule(r){13-17}
 &  & CD & \makecell{FCS-\\MAN} & \makecell{FCS-\\NJ} & \makecell{FCS-\\LAT} & JM & CD & \makecell{FCS-\\MAN} & \makecell{FCS-\\NJ} & \makecell{FCS-\\LAT} & JM & CD & \makecell{FCS-\\MAN} & \makecell{FCS-\\NJ} & \makecell{FCS-\\LAT} & \multicolumn{1}{c}{JM} \\ 
[0.4ex]\hline\\[-1.8ex]
& & \multicolumn{15}{c}{Small intraclass correlation $(\rho_{Iy}=.10)$} \\[0.6ex]\hline\\[-1.8ex]
\multicolumn{4}{l}{$\bar{n}=5$} \\  & \nopagebreak $\;J=50$  & $\phantom{0}{-}2.2\phantom{0}$ & $\phantom{0}\phantom{-}5.1\phantom{0}$ & $\phantom{-}14.8\phantom{0}$ & $\phantom{0}\phantom{-}4.9\phantom{0}$ & $\phantom{0}{-}1.3\phantom{0}$ & $\phantom{0}0.20\phantom{0}$ & $\phantom{0}0.30\phantom{0}$ & $\phantom{0}0.37\phantom{0}$ & $\phantom{0}0.30\phantom{0}$ & $\phantom{0}0.27\phantom{0}$ & $\phantom{0}89.6\phantom{0}$ & $\phantom{0}93.3\phantom{0}$ & $\phantom{0}95.6\phantom{0}$ & $\phantom{0}92.1\phantom{0}$ & $\phantom{0}89.5\phantom{0}$ \\
 & \nopagebreak $\;J=200$  & $\phantom{0}{-}0.4\phantom{0}$ & $\phantom{0}\phantom{-}1.0\phantom{0}$ & $\phantom{0}\phantom{-}2.7\phantom{0}$ & $\phantom{0}\phantom{-}1.4\phantom{0}$ & $\phantom{0}{-}0.7\phantom{0}$ & $\phantom{0}0.10\phantom{0}$ & $\phantom{0}0.14\phantom{0}$ & $\phantom{0}0.14\phantom{0}$ & $\phantom{0}0.14\phantom{0}$ & $\phantom{0}0.13\phantom{0}$ & $\phantom{0}93.1\phantom{0}$ & $\phantom{0}93.1\phantom{0}$ & $\phantom{0}94.9\phantom{0}$ & $\phantom{0}93.9\phantom{0}$ & $\phantom{0}93.4\phantom{0}$ \\
 & \nopagebreak $\;J=1000$  & $\phantom{0}{-}0.3\phantom{0}$ & $\phantom{0}{-}0.1\phantom{0}$ & $\phantom{0}\phantom{-}0.2\phantom{0}$ & $\phantom{0}{-}0.0\phantom{0}$ & $\phantom{0}{-}0.5\phantom{0}$ & $\phantom{0}0.04\phantom{0}$ & $\phantom{0}0.06\phantom{0}$ & $\phantom{0}0.06\phantom{0}$ & $\phantom{0}0.06\phantom{0}$ & $\phantom{0}0.06\phantom{0}$ & $\phantom{0}94.9\phantom{0}$ & $\phantom{0}94.5\phantom{0}$ & $\phantom{0}94.9\phantom{0}$ & $\phantom{0}94.4\phantom{0}$ & $\phantom{0}94.4\phantom{0}$ \\
\multicolumn{4}{l}{$\bar{n}=20$} \\  & \nopagebreak $\;J=50$  & $\phantom{0}{-}2.2\phantom{0}$ & $\phantom{0}\phantom{-}5.4\phantom{0}$ & $\phantom{-}14.2\phantom{0}$ & $\phantom{0}\phantom{-}6.0\phantom{0}$ & $\phantom{0}{-}0.8\phantom{0}$ & $\phantom{0}0.19\phantom{0}$ & $\phantom{0}0.30\phantom{0}$ & $\phantom{0}0.36\phantom{0}$ & $\phantom{0}0.30\phantom{0}$ & $\phantom{0}0.26\phantom{0}$ & $\phantom{0}89.4\phantom{0}$ & $\phantom{0}94.0\phantom{0}$ & $\phantom{0}97.3\phantom{0}$ & $\phantom{0}94.1\phantom{0}$ & $\phantom{0}91.3\phantom{0}$ \\
 & \nopagebreak $\;J=200$  & $\phantom{0}{-}0.4\phantom{0}$ & $\phantom{0}\phantom{-}0.6\phantom{0}$ & $\phantom{0}\phantom{-}2.4\phantom{0}$ & $\phantom{0}\phantom{-}0.7\phantom{0}$ & $\phantom{0}{-}1.3\phantom{0}$ & $\phantom{0}0.10\phantom{0}$ & $\phantom{0}0.14\phantom{0}$ & $\phantom{0}0.14\phantom{0}$ & $\phantom{0}0.14\phantom{0}$ & $\phantom{0}0.13\phantom{0}$ & $\phantom{0}94.6\phantom{0}$ & $\phantom{0}93.3\phantom{0}$ & $\phantom{0}93.9\phantom{0}$ & $\phantom{0}93.9\phantom{0}$ & $\phantom{0}92.3\phantom{0}$ \\
 & \nopagebreak $\;J=1000$  & $\phantom{0}{-}0.5\phantom{0}$ & $\phantom{0}{-}0.2\phantom{0}$ & $\phantom{0}\phantom{-}0.1\phantom{0}$ & $\phantom{0}{-}0.3\phantom{0}$ & $\phantom{0}{-}0.7\phantom{0}$ & $\phantom{0}0.05\phantom{0}$ & $\phantom{0}0.06\phantom{0}$ & $\phantom{0}0.06\phantom{0}$ & $\phantom{0}0.06\phantom{0}$ & $\phantom{0}0.06\phantom{0}$ & $\phantom{0}93.9\phantom{0}$ & $\phantom{0}94.4\phantom{0}$ & $\phantom{0}94.0\phantom{0}$ & $\phantom{0}93.0\phantom{0}$ & $\phantom{0}92.7\phantom{0}$ \\
[0.5ex]\hline\\[-1.6ex] 
& & \multicolumn{15}{c}{Moderate intraclass correlation $(\rho_{Iy}=.30)$} \\[0.6ex]\hline\\[-1.8ex]
\multicolumn{4}{l}{$\bar{n}=5$} \\  & \nopagebreak $\;J=50$  & $\phantom{0}{-}1.1\phantom{0}$ & $\phantom{0}\phantom{-}4.8\phantom{0}$ & $\phantom{-}15.1\phantom{0}$ & $\phantom{0}\phantom{-}5.5\phantom{0}$ & $\phantom{0}\phantom{-}0.6\phantom{0}$ & $\phantom{0}0.20\phantom{0}$ & $\phantom{0}0.30\phantom{0}$ & $\phantom{0}0.50\phantom{0}$ & $\phantom{0}0.31\phantom{0}$ & $\phantom{0}0.27\phantom{0}$ & $\phantom{0}91.4\phantom{0}$ & $\phantom{0}92.9\phantom{0}$ & $\phantom{0}95.7\phantom{0}$ & $\phantom{0}93.9\phantom{0}$ & $\phantom{0}91.2\phantom{0}$ \\
 & \nopagebreak $\;J=200$  & $\phantom{0}{-}0.6\phantom{0}$ & $\phantom{0}\phantom{-}1.0\phantom{0}$ & $\phantom{0}\phantom{-}2.7\phantom{0}$ & $\phantom{0}\phantom{-}1.0\phantom{0}$ & $\phantom{0}{-}0.2\phantom{0}$ & $\phantom{0}0.10\phantom{0}$ & $\phantom{0}0.14\phantom{0}$ & $\phantom{0}0.14\phantom{0}$ & $\phantom{0}0.14\phantom{0}$ & $\phantom{0}0.13\phantom{0}$ & $\phantom{0}93.6\phantom{0}$ & $\phantom{0}93.6\phantom{0}$ & $\phantom{0}95.5\phantom{0}$ & $\phantom{0}94.4\phantom{0}$ & $\phantom{0}93.5\phantom{0}$ \\
 & \nopagebreak $\;J=1000$  & $\phantom{0}{-}0.2\phantom{0}$ & $\phantom{0}\phantom{-}0.3\phantom{0}$ & $\phantom{0}\phantom{-}0.5\phantom{0}$ & $\phantom{0}\phantom{-}0.3\phantom{0}$ & $\phantom{0}\phantom{-}0.2\phantom{0}$ & $\phantom{0}0.04\phantom{0}$ & $\phantom{0}0.06\phantom{0}$ & $\phantom{0}0.06\phantom{0}$ & $\phantom{0}0.06\phantom{0}$ & $\phantom{0}0.06\phantom{0}$ & $\phantom{0}94.4\phantom{0}$ & $\phantom{0}94.9\phantom{0}$ & $\phantom{0}94.9\phantom{0}$ & $\phantom{0}95.7\phantom{0}$ & $\phantom{0}94.7\phantom{0}$ \\
\multicolumn{4}{l}{$\bar{n}=20$} \\  & \nopagebreak $\;J=50$  & $\phantom{0}{-}1.6\phantom{0}$ & $\phantom{0}\phantom{-}3.8\phantom{0}$ & $\phantom{-}13.0\phantom{0}$ & $\phantom{0}\phantom{-}4.2\phantom{0}$ & $\phantom{0}\phantom{-}0.3\phantom{0}$ & $\phantom{0}0.20\phantom{0}$ & $\phantom{0}0.29\phantom{0}$ & $\phantom{0}0.35\phantom{0}$ & $\phantom{0}0.29\phantom{0}$ & $\phantom{0}0.27\phantom{0}$ & $\phantom{0}89.4\phantom{0}$ & $\phantom{0}93.1\phantom{0}$ & $\phantom{0}95.8\phantom{0}$ & $\phantom{0}93.2\phantom{0}$ & $\phantom{0}91.7\phantom{0}$ \\
 & \nopagebreak $\;J=200$  & $\phantom{0}{-}0.6\phantom{0}$ & $\phantom{0}\phantom{-}0.9\phantom{0}$ & $\phantom{0}\phantom{-}2.5\phantom{0}$ & $\phantom{0}\phantom{-}0.8\phantom{0}$ & $\phantom{0}\phantom{-}0.1\phantom{0}$ & $\phantom{0}0.10\phantom{0}$ & $\phantom{0}0.14\phantom{0}$ & $\phantom{0}0.14\phantom{0}$ & $\phantom{0}0.14\phantom{0}$ & $\phantom{0}0.14\phantom{0}$ & $\phantom{0}94.1\phantom{0}$ & $\phantom{0}93.8\phantom{0}$ & $\phantom{0}93.8\phantom{0}$ & $\phantom{0}93.4\phantom{0}$ & $\phantom{0}93.8\phantom{0}$ \\
 & \nopagebreak $\;J=1000$  & $\phantom{0}{-}0.1\phantom{0}$ & $\phantom{0}\phantom{-}0.2\phantom{0}$ & $\phantom{0}\phantom{-}0.5\phantom{0}$ & $\phantom{0}\phantom{-}0.2\phantom{0}$ & $\phantom{0}\phantom{-}0.0\phantom{0}$ & $\phantom{0}0.04\phantom{0}$ & $\phantom{0}0.06\phantom{0}$ & $\phantom{0}0.06\phantom{0}$ & $\phantom{0}0.06\phantom{0}$ & $\phantom{0}0.06\phantom{0}$ & $\phantom{0}95.2\phantom{0}$ & $\phantom{0}94.1\phantom{0}$ & $\phantom{0}94.0\phantom{0}$ & $\phantom{0}93.3\phantom{0}$ & $\phantom{0}93.6\phantom{0}$ \\
[0.5ex]\hline\\[-1.6ex] 
\end{tabular}
\begin{tablenotes}[para,flushleft]{\footnotesize \textit{Note.} $\bar{n}$ = average cluster size; $J$ = number of clusters; CD = complete data sets; LD = listwise deletion; FCS-SL = single-level FCS; FCS-MAN = two-level FCS with manifest cluster means; FCS-LAT = two-level FCS with latent cluster means; JM = joint modeling.}\end{tablenotes}
\end{threeparttable}
\end{sidewaystable}
\begin{sidewaystable}
\begin{threeparttable}
\setlength{\tabcolsep}{1.0pt}
\renewcommand{\arraystretch}{0.95}
\footnotesize
\caption{\small Study 2: Bias (in \%), Relative RMSE, and Coverage of the 95\% Confidence Interval for the Variance of $z$ ($\hat\sigma_z^2$) With Strongly Unbalanced Data (Bimodal, $\pm 80\%$) and 40\% Missing Data (MAR, $\lambda=0.5$)}
\begin{tabular}{llccccccccccccccc}
\hline\\[-1.8ex]
& & \multicolumn{5}{c}{Bias (\%)} & \multicolumn{5}{c}{Rel. RMSE} & \multicolumn{5}{c}{Coverage (\%)} \\ \cmidrule(r){3-7}\cmidrule(r){8-12}\cmidrule(r){13-17}
 &  & CD & \makecell{FCS-\\MAN} & \makecell{FCS-\\NJ} & \makecell{FCS-\\LAT} & JM & CD & \makecell{FCS-\\MAN} & \makecell{FCS-\\NJ} & \makecell{FCS-\\LAT} & JM & CD & \makecell{FCS-\\MAN} & \makecell{FCS-\\NJ} & \makecell{FCS-\\LAT} & \multicolumn{1}{c}{JM} \\ 
[0.4ex]\hline\\[-1.8ex]
& & \multicolumn{15}{c}{Small intraclass correlation $(\rho_{Iy}=.10)$} \\[0.6ex]\hline\\[-1.8ex]
\multicolumn{4}{l}{$\bar{n}=5$} \\  & \nopagebreak $\;J=50$  & $\phantom{0}{-}2.2\phantom{0}$ & $\phantom{0}\phantom{-}5.3\phantom{0}$ & $\phantom{-}14.5\phantom{0}$ & $\phantom{0}\phantom{-}4.1\phantom{0}$ & $\phantom{0}{-}1.2\phantom{0}$ & $\phantom{0}0.20\phantom{0}$ & $\phantom{0}0.29\phantom{0}$ & $\phantom{0}0.36\phantom{0}$ & $\phantom{0}0.28\phantom{0}$ & $\phantom{0}0.26\phantom{0}$ & $\phantom{0}89.5\phantom{0}$ & $\phantom{0}93.7\phantom{0}$ & $\phantom{0}96.3\phantom{0}$ & $\phantom{0}93.2\phantom{0}$ & $\phantom{0}90.2\phantom{0}$ \\
 & \nopagebreak $\;J=200$  & $\phantom{0}{-}1.0\phantom{0}$ & $\phantom{0}\phantom{-}0.8\phantom{0}$ & $\phantom{0}\phantom{-}2.5\phantom{0}$ & $\phantom{0}\phantom{-}0.8\phantom{0}$ & $\phantom{0}{-}0.6\phantom{0}$ & $\phantom{0}0.10\phantom{0}$ & $\phantom{0}0.13\phantom{0}$ & $\phantom{0}0.14\phantom{0}$ & $\phantom{0}0.14\phantom{0}$ & $\phantom{0}0.13\phantom{0}$ & $\phantom{0}93.4\phantom{0}$ & $\phantom{0}93.5\phantom{0}$ & $\phantom{0}95.5\phantom{0}$ & $\phantom{0}93.0\phantom{0}$ & $\phantom{0}92.9\phantom{0}$ \\
 & \nopagebreak $\;J=1000$  & $\phantom{0}{-}0.2\phantom{0}$ & $\phantom{0}\phantom{-}0.1\phantom{0}$ & $\phantom{0}\phantom{-}0.3\phantom{0}$ & $\phantom{0}\phantom{-}0.1\phantom{0}$ & $\phantom{0}{-}0.1\phantom{0}$ & $\phantom{0}0.04\phantom{0}$ & $\phantom{0}0.06\phantom{0}$ & $\phantom{0}0.06\phantom{0}$ & $\phantom{0}0.06\phantom{0}$ & $\phantom{0}0.06\phantom{0}$ & $\phantom{0}95.3\phantom{0}$ & $\phantom{0}94.9\phantom{0}$ & $\phantom{0}94.2\phantom{0}$ & $\phantom{0}94.7\phantom{0}$ & $\phantom{0}94.0\phantom{0}$ \\
\multicolumn{4}{l}{$\bar{n}=20$} \\  & \nopagebreak $\;J=50$  & $\phantom{0}{-}3.0\phantom{0}$ & $\phantom{0}\phantom{-}2.9\phantom{0}$ & $\phantom{-}11.6\phantom{0}$ & $\phantom{0}\phantom{-}2.5\phantom{0}$ & $\phantom{0}{-}3.3\phantom{0}$ & $\phantom{0}0.20\phantom{0}$ & $\phantom{0}0.28\phantom{0}$ & $\phantom{0}0.33\phantom{0}$ & $\phantom{0}0.27\phantom{0}$ & $\phantom{0}0.25\phantom{0}$ & $\phantom{0}88.4\phantom{0}$ & $\phantom{0}93.5\phantom{0}$ & $\phantom{0}96.2\phantom{0}$ & $\phantom{0}92.7\phantom{0}$ & $\phantom{0}90.8\phantom{0}$ \\
 & \nopagebreak $\;J=200$  & $\phantom{0}{-}0.1\phantom{0}$ & $\phantom{0}\phantom{-}2.0\phantom{0}$ & $\phantom{0}\phantom{-}3.6\phantom{0}$ & $\phantom{0}\phantom{-}1.9\phantom{0}$ & $\phantom{0}\phantom{-}0.0\phantom{0}$ & $\phantom{0}0.10\phantom{0}$ & $\phantom{0}0.13\phantom{0}$ & $\phantom{0}0.14\phantom{0}$ & $\phantom{0}0.13\phantom{0}$ & $\phantom{0}0.13\phantom{0}$ & $\phantom{0}93.5\phantom{0}$ & $\phantom{0}95.0\phantom{0}$ & $\phantom{0}95.8\phantom{0}$ & $\phantom{0}95.2\phantom{0}$ & $\phantom{0}93.9\phantom{0}$ \\
 & \nopagebreak $\;J=1000$  & $\phantom{0}\phantom{-}0.0\phantom{0}$ & $\phantom{0}\phantom{-}0.4\phantom{0}$ & $\phantom{0}\phantom{-}0.6\phantom{0}$ & $\phantom{0}\phantom{-}0.2\phantom{0}$ & $\phantom{0}{-}0.1\phantom{0}$ & $\phantom{0}0.04\phantom{0}$ & $\phantom{0}0.06\phantom{0}$ & $\phantom{0}0.06\phantom{0}$ & $\phantom{0}0.06\phantom{0}$ & $\phantom{0}0.06\phantom{0}$ & $\phantom{0}96.7\phantom{0}$ & $\phantom{0}94.8\phantom{0}$ & $\phantom{0}94.5\phantom{0}$ & $\phantom{0}95.4\phantom{0}$ & $\phantom{0}95.2\phantom{0}$ \\
[0.5ex]\hline\\[-1.6ex] 
& & \multicolumn{15}{c}{Moderate intraclass correlation $(\rho_{Iy}=.30)$} \\[0.6ex]\hline\\[-1.8ex]
\multicolumn{4}{l}{$\bar{n}=5$} \\  & \nopagebreak $\;J=50$  & $\phantom{0}{-}0.4\phantom{0}$ & $\phantom{0}\phantom{-}6.5\phantom{0}$ & $\phantom{-}16.4\phantom{0}$ & $\phantom{0}\phantom{-}6.8\phantom{0}$ & $\phantom{0}\phantom{-}0.8\phantom{0}$ & $\phantom{0}0.20\phantom{0}$ & $\phantom{0}0.31\phantom{0}$ & $\phantom{0}0.38\phantom{0}$ & $\phantom{0}0.31\phantom{0}$ & $\phantom{0}0.27\phantom{0}$ & $\phantom{0}91.9\phantom{0}$ & $\phantom{0}94.3\phantom{0}$ & $\phantom{0}96.2\phantom{0}$ & $\phantom{0}93.7\phantom{0}$ & $\phantom{0}91.5\phantom{0}$ \\
 & \nopagebreak $\;J=200$  & $\phantom{0}{-}0.6\phantom{0}$ & $\phantom{0}\phantom{-}0.3\phantom{0}$ & $\phantom{0}\phantom{-}2.0\phantom{0}$ & $\phantom{0}\phantom{-}0.2\phantom{0}$ & $\phantom{0}{-}0.8\phantom{0}$ & $\phantom{0}0.10\phantom{0}$ & $\phantom{0}0.14\phantom{0}$ & $\phantom{0}0.14\phantom{0}$ & $\phantom{0}0.13\phantom{0}$ & $\phantom{0}0.13\phantom{0}$ & $\phantom{0}93.4\phantom{0}$ & $\phantom{0}94.0\phantom{0}$ & $\phantom{0}95.1\phantom{0}$ & $\phantom{0}94.9\phantom{0}$ & $\phantom{0}93.2\phantom{0}$ \\
 & \nopagebreak $\;J=1000$  & $\phantom{0}{-}0.0\phantom{0}$ & $\phantom{0}\phantom{-}0.1\phantom{0}$ & $\phantom{0}\phantom{-}0.4\phantom{0}$ & $\phantom{0}\phantom{-}0.1\phantom{0}$ & $\phantom{0}{-}0.1\phantom{0}$ & $\phantom{0}0.04\phantom{0}$ & $\phantom{0}0.06\phantom{0}$ & $\phantom{0}0.06\phantom{0}$ & $\phantom{0}0.06\phantom{0}$ & $\phantom{0}0.06\phantom{0}$ & $\phantom{0}95.5\phantom{0}$ & $\phantom{0}93.8\phantom{0}$ & $\phantom{0}94.5\phantom{0}$ & $\phantom{0}94.2\phantom{0}$ & $\phantom{0}93.7\phantom{0}$ \\
\multicolumn{4}{l}{$\bar{n}=20$} \\  & \nopagebreak $\;J=50$  & $\phantom{0}{-}2.5\phantom{0}$ & $\phantom{0}\phantom{-}2.8\phantom{0}$ & $\phantom{-}12.1\phantom{0}$ & $\phantom{0}\phantom{-}3.1\phantom{0}$ & $\phantom{0}{-}1.3\phantom{0}$ & $\phantom{0}0.20\phantom{0}$ & $\phantom{0}0.28\phantom{0}$ & $\phantom{0}0.35\phantom{0}$ & $\phantom{0}0.28\phantom{0}$ & $\phantom{0}0.26\phantom{0}$ & $\phantom{0}89.2\phantom{0}$ & $\phantom{0}92.5\phantom{0}$ & $\phantom{0}96.1\phantom{0}$ & $\phantom{0}92.6\phantom{0}$ & $\phantom{0}91.2\phantom{0}$ \\
 & \nopagebreak $\;J=200$  & $\phantom{0}{-}0.5\phantom{0}$ & $\phantom{0}\phantom{-}1.0\phantom{0}$ & $\phantom{0}\phantom{-}2.7\phantom{0}$ & $\phantom{0}\phantom{-}1.1\phantom{0}$ & $\phantom{0}\phantom{-}0.2\phantom{0}$ & $\phantom{0}0.10\phantom{0}$ & $\phantom{0}0.14\phantom{0}$ & $\phantom{0}0.14\phantom{0}$ & $\phantom{0}0.14\phantom{0}$ & $\phantom{0}0.14\phantom{0}$ & $\phantom{0}94.4\phantom{0}$ & $\phantom{0}93.2\phantom{0}$ & $\phantom{0}95.7\phantom{0}$ & $\phantom{0}93.7\phantom{0}$ & $\phantom{0}93.3\phantom{0}$ \\
 & \nopagebreak $\;J=1000$  & $\phantom{0}\phantom{-}0.0\phantom{0}$ & $\phantom{0}\phantom{-}0.4\phantom{0}$ & $\phantom{0}\phantom{-}0.6\phantom{0}$ & $\phantom{0}\phantom{-}0.3\phantom{0}$ & $\phantom{0}\phantom{-}0.2\phantom{0}$ & $\phantom{0}0.05\phantom{0}$ & $\phantom{0}0.06\phantom{0}$ & $\phantom{0}0.06\phantom{0}$ & $\phantom{0}0.06\phantom{0}$ & $\phantom{0}0.06\phantom{0}$ & $\phantom{0}94.9\phantom{0}$ & $\phantom{0}93.7\phantom{0}$ & $\phantom{0}93.2\phantom{0}$ & $\phantom{0}93.4\phantom{0}$ & $\phantom{0}94.1\phantom{0}$ \\
[0.5ex]\hline\\[-1.6ex] 
\end{tabular}
\begin{tablenotes}[para,flushleft]{\footnotesize \textit{Note.} $\bar{n}$ = average cluster size; $J$ = number of clusters; CD = complete data sets; LD = listwise deletion; FCS-SL = single-level FCS; FCS-MAN = two-level FCS with manifest cluster means; FCS-LAT = two-level FCS with latent cluster means; JM = joint modeling.}\end{tablenotes}
\end{threeparttable}
\end{sidewaystable}
\begin{sidewaystable}
\begin{threeparttable}
\setlength{\tabcolsep}{1.0pt}
\renewcommand{\arraystretch}{0.95}
\footnotesize
\caption{\small Study 2: Bias (in \%), Relative RMSE, and Coverage of the 95\% Confidence Interval for the Covariance of $y$ With $z$ ($\hat\sigma_{yz}$) With Moderately Unbalanced Data (Uniform, $\pm 40\%$) and 20\% Missing Data (MAR, $\lambda=0.5$)}
\begin{tabular}{llccccccccccccccc}
\hline\\[-1.8ex]
& & \multicolumn{5}{c}{Bias (\%)} & \multicolumn{5}{c}{Rel. RMSE} & \multicolumn{5}{c}{Coverage (\%)} \\ \cmidrule(r){3-7}\cmidrule(r){8-12}\cmidrule(r){13-17}
 &  & CD & \makecell{FCS-\\MAN} & \makecell{FCS-\\NJ} & \makecell{FCS-\\LAT} & JM & CD & \makecell{FCS-\\MAN} & \makecell{FCS-\\NJ} & \makecell{FCS-\\LAT} & JM & CD & \makecell{FCS-\\MAN} & \makecell{FCS-\\NJ} & \makecell{FCS-\\LAT} & \multicolumn{1}{c}{JM} \\ 
[0.4ex]\hline\\[-1.8ex]
& & \multicolumn{15}{c}{Small intraclass correlation $(\rho_{Iy}=.10)$} \\[0.6ex]\hline\\[-1.8ex]
\multicolumn{4}{l}{$\bar{n}=5$} \\  & \nopagebreak $\;J=50$  & $\phantom{-}0.2\phantom{0}$ & $\phantom{-}0.7\phantom{0}$ & $\phantom{-}1.2\phantom{0}$ & $\phantom{-}5.4\phantom{0}$ & ${-}8.2\phantom{0}$ & $\phantom{0}0.08\phantom{0}$ & $\phantom{0}0.09\phantom{0}$ & $\phantom{0}0.10\phantom{0}$ & $\phantom{0}0.10\phantom{0}$ & $\phantom{0}0.09\phantom{0}$ & $\phantom{0}93.2\phantom{0}$ & $\phantom{0}94.5\phantom{0}$ & $\phantom{0}95.3\phantom{0}$ & $\phantom{0}94.0\phantom{0}$ & $\phantom{0}94.5\phantom{0}$ \\
 & \nopagebreak $\;J=200$  & $\phantom{-}0.2\phantom{0}$ & $\phantom{-}0.2\phantom{0}$ & $\phantom{-}0.4\phantom{0}$ & $\phantom{-}2.3\phantom{0}$ & ${-}3.6\phantom{0}$ & $\phantom{0}0.04\phantom{0}$ & $\phantom{0}0.05\phantom{0}$ & $\phantom{0}0.04\phantom{0}$ & $\phantom{0}0.05\phantom{0}$ & $\phantom{0}0.04\phantom{0}$ & $\phantom{0}95.9\phantom{0}$ & $\phantom{0}96.1\phantom{0}$ & $\phantom{0}96.5\phantom{0}$ & $\phantom{0}95.5\phantom{0}$ & $\phantom{0}96.7\phantom{0}$ \\
 & \nopagebreak $\;J=1000$  & $\phantom{-}0.1\phantom{0}$ & ${-}0.9\phantom{0}$ & ${-}0.4\phantom{0}$ & ${-}0.1\phantom{0}$ & ${-}1.5\phantom{0}$ & $\phantom{0}0.02\phantom{0}$ & $\phantom{0}0.02\phantom{0}$ & $\phantom{0}0.02\phantom{0}$ & $\phantom{0}0.02\phantom{0}$ & $\phantom{0}0.02\phantom{0}$ & $\phantom{0}94.9\phantom{0}$ & $\phantom{0}93.8\phantom{0}$ & $\phantom{0}94.3\phantom{0}$ & $\phantom{0}94.5\phantom{0}$ & $\phantom{0}94.1\phantom{0}$ \\
\multicolumn{4}{l}{$\bar{n}=20$} \\  & \nopagebreak $\;J=50$  & ${-}1.8\phantom{0}$ & ${-}1.8\phantom{0}$ & ${-}0.9\phantom{0}$ & ${-}0.8\phantom{0}$ & ${-}9.7\phantom{0}$ & $\phantom{0}0.06\phantom{0}$ & $\phantom{0}0.07\phantom{0}$ & $\phantom{0}0.07\phantom{0}$ & $\phantom{0}0.07\phantom{0}$ & $\phantom{0}0.07\phantom{0}$ & $\phantom{0}91.8\phantom{0}$ & $\phantom{0}92.6\phantom{0}$ & $\phantom{0}93.9\phantom{0}$ & $\phantom{0}92.4\phantom{0}$ & $\phantom{0}91.4\phantom{0}$ \\
 & \nopagebreak $\;J=200$  & ${-}0.9\phantom{0}$ & ${-}1.3\phantom{0}$ & ${-}1.3\phantom{0}$ & ${-}1.2\phantom{0}$ & ${-}4.2\phantom{0}$ & $\phantom{0}0.03\phantom{0}$ & $\phantom{0}0.03\phantom{0}$ & $\phantom{0}0.03\phantom{0}$ & $\phantom{0}0.03\phantom{0}$ & $\phantom{0}0.03\phantom{0}$ & $\phantom{0}92.6\phantom{0}$ & $\phantom{0}94.1\phantom{0}$ & $\phantom{0}94.4\phantom{0}$ & $\phantom{0}94.1\phantom{0}$ & $\phantom{0}93.9\phantom{0}$ \\
 & \nopagebreak $\;J=1000$  & ${-}0.1\phantom{0}$ & ${-}0.3\phantom{0}$ & ${-}0.2\phantom{0}$ & ${-}0.1\phantom{0}$ & ${-}0.7\phantom{0}$ & $\phantom{0}0.01\phantom{0}$ & $\phantom{0}0.01\phantom{0}$ & $\phantom{0}0.01\phantom{0}$ & $\phantom{0}0.01\phantom{0}$ & $\phantom{0}0.01\phantom{0}$ & $\phantom{0}94.4\phantom{0}$ & $\phantom{0}94.7\phantom{0}$ & $\phantom{0}94.2\phantom{0}$ & $\phantom{0}94.8\phantom{0}$ & $\phantom{0}95.0\phantom{0}$ \\
[0.5ex]\hline\\[-1.6ex] 
& & \multicolumn{15}{c}{Moderate intraclass correlation $(\rho_{Iy}=.30)$} \\[0.6ex]\hline\\[-1.8ex]
\multicolumn{4}{l}{$\bar{n}=5$} \\  & \nopagebreak $\;J=50$  & ${-}1.8\phantom{0}$ & ${-}1.4\phantom{0}$ & ${-}1.1\phantom{0}$ & ${-}0.5\phantom{0}$ & ${-}5.5\phantom{0}$ & $\phantom{0}0.10\phantom{0}$ & $\phantom{0}0.12\phantom{0}$ & $\phantom{0}0.12\phantom{0}$ & $\phantom{0}0.12\phantom{0}$ & $\phantom{0}0.11\phantom{0}$ & $\phantom{0}92.4\phantom{0}$ & $\phantom{0}94.0\phantom{0}$ & $\phantom{0}95.3\phantom{0}$ & $\phantom{0}94.5\phantom{0}$ & $\phantom{0}93.7\phantom{0}$ \\
 & \nopagebreak $\;J=200$  & ${-}1.1\phantom{0}$ & ${-}1.3\phantom{0}$ & ${-}1.1\phantom{0}$ & ${-}1.0\phantom{0}$ & ${-}2.4\phantom{0}$ & $\phantom{0}0.05\phantom{0}$ & $\phantom{0}0.06\phantom{0}$ & $\phantom{0}0.06\phantom{0}$ & $\phantom{0}0.06\phantom{0}$ & $\phantom{0}0.06\phantom{0}$ & $\phantom{0}93.6\phantom{0}$ & $\phantom{0}94.0\phantom{0}$ & $\phantom{0}93.8\phantom{0}$ & $\phantom{0}93.8\phantom{0}$ & $\phantom{0}94.2\phantom{0}$ \\
 & \nopagebreak $\;J=1000$  & $\phantom{-}0.1\phantom{0}$ & ${-}0.2\phantom{0}$ & ${-}0.1\phantom{0}$ & ${-}0.0\phantom{0}$ & ${-}0.2\phantom{0}$ & $\phantom{0}0.02\phantom{0}$ & $\phantom{0}0.03\phantom{0}$ & $\phantom{0}0.03\phantom{0}$ & $\phantom{0}0.03\phantom{0}$ & $\phantom{0}0.03\phantom{0}$ & $\phantom{0}95.2\phantom{0}$ & $\phantom{0}93.7\phantom{0}$ & $\phantom{0}95.0\phantom{0}$ & $\phantom{0}93.8\phantom{0}$ & $\phantom{0}94.5\phantom{0}$ \\
\multicolumn{4}{l}{$\bar{n}=20$} \\  & \nopagebreak $\;J=50$  & ${-}0.3\phantom{0}$ & ${-}0.4\phantom{0}$ & ${-}0.9\phantom{0}$ & ${-}0.5\phantom{0}$ & ${-}4.1\phantom{0}$ & $\phantom{0}0.09\phantom{0}$ & $\phantom{0}0.10\phantom{0}$ & $\phantom{0}0.11\phantom{0}$ & $\phantom{0}0.10\phantom{0}$ & $\phantom{0}0.10\phantom{0}$ & $\phantom{0}92.5\phantom{0}$ & $\phantom{0}93.0\phantom{0}$ & $\phantom{0}93.9\phantom{0}$ & $\phantom{0}93.2\phantom{0}$ & $\phantom{0}93.0\phantom{0}$ \\
 & \nopagebreak $\;J=200$  & $\phantom{-}0.4\phantom{0}$ & $\phantom{-}0.5\phantom{0}$ & $\phantom{-}0.5\phantom{0}$ & $\phantom{-}0.5\phantom{0}$ & ${-}0.3\phantom{0}$ & $\phantom{0}0.05\phantom{0}$ & $\phantom{0}0.05\phantom{0}$ & $\phantom{0}0.05\phantom{0}$ & $\phantom{0}0.05\phantom{0}$ & $\phantom{0}0.05\phantom{0}$ & $\phantom{0}94.8\phantom{0}$ & $\phantom{0}94.4\phantom{0}$ & $\phantom{0}94.3\phantom{0}$ & $\phantom{0}94.1\phantom{0}$ & $\phantom{0}93.4\phantom{0}$ \\
 & \nopagebreak $\;J=1000$  & ${-}0.2\phantom{0}$ & $\phantom{-}0.0\phantom{0}$ & ${-}0.0\phantom{0}$ & $\phantom{-}0.0\phantom{0}$ & ${-}0.2\phantom{0}$ & $\phantom{0}0.02\phantom{0}$ & $\phantom{0}0.02\phantom{0}$ & $\phantom{0}0.02\phantom{0}$ & $\phantom{0}0.02\phantom{0}$ & $\phantom{0}0.02\phantom{0}$ & $\phantom{0}94.6\phantom{0}$ & $\phantom{0}94.7\phantom{0}$ & $\phantom{0}94.9\phantom{0}$ & $\phantom{0}94.6\phantom{0}$ & $\phantom{0}94.9\phantom{0}$ \\
[0.5ex]\hline\\[-1.6ex] 
\end{tabular}
\begin{tablenotes}[para,flushleft]{\footnotesize \textit{Note.} $\bar{n}$ = average cluster size; $J$ = number of clusters; CD = complete data sets; LD = listwise deletion; FCS-SL = single-level FCS; FCS-MAN = two-level FCS with manifest cluster means; FCS-LAT = two-level FCS with latent cluster means; JM = joint modeling.}\end{tablenotes}
\end{threeparttable}
\end{sidewaystable}
\begin{sidewaystable}
\begin{threeparttable}
\setlength{\tabcolsep}{1.0pt}
\renewcommand{\arraystretch}{0.95}
\footnotesize
\caption{\small Study 2: Bias (in \%), Relative RMSE, and Coverage of the 95\% Confidence Interval for the Covariance of $y$ With $z$ ($\hat\sigma_{yz}$) With Strongly Unbalanced Data (Uniform, $\pm 80\%$) and 20\% Missing Data (MAR, $\lambda=0.5$)}
\begin{tabular}{llccccccccccccccc}
\hline\\[-1.8ex]
& & \multicolumn{5}{c}{Bias (\%)} & \multicolumn{5}{c}{Rel. RMSE} & \multicolumn{5}{c}{Coverage (\%)} \\ \cmidrule(r){3-7}\cmidrule(r){8-12}\cmidrule(r){13-17}
 &  & CD & \makecell{FCS-\\MAN} & \makecell{FCS-\\NJ} & \makecell{FCS-\\LAT} & JM & CD & \makecell{FCS-\\MAN} & \makecell{FCS-\\NJ} & \makecell{FCS-\\LAT} & JM & CD & \makecell{FCS-\\MAN} & \makecell{FCS-\\NJ} & \makecell{FCS-\\LAT} & \multicolumn{1}{c}{JM} \\ 
[0.4ex]\hline\\[-1.8ex]
& & \multicolumn{15}{c}{Small intraclass correlation $(\rho_{Iy}=.10)$} \\[0.6ex]\hline\\[-1.8ex]
\multicolumn{4}{l}{$\bar{n}=5$} \\  & \nopagebreak $\;J=50$  & $\phantom{0}{-}4.9\phantom{0}$ & $\phantom{0}{-}5.6\phantom{0}$ & $\phantom{0}{-}2.5\phantom{0}$ & $\phantom{0}{-}0.6\phantom{0}$ & ${-}12.0\phantom{0}$ & $\phantom{0}0.08\phantom{0}$ & $\phantom{0}0.09\phantom{0}$ & $\phantom{0}0.10\phantom{0}$ & $\phantom{0}0.09\phantom{0}$ & $\phantom{0}0.09\phantom{0}$ & $\phantom{0}93.4\phantom{0}$ & $\phantom{0}94.3\phantom{0}$ & $\phantom{0}96.1\phantom{0}$ & $\phantom{0}93.9\phantom{0}$ & $\phantom{0}94.2\phantom{0}$ \\
 & \nopagebreak $\;J=200$  & $\phantom{0}{-}1.9\phantom{0}$ & $\phantom{0}{-}4.2\phantom{0}$ & $\phantom{0}{-}1.5\phantom{0}$ & $\phantom{0}{-}0.2\phantom{0}$ & $\phantom{0}{-}5.5\phantom{0}$ & $\phantom{0}0.04\phantom{0}$ & $\phantom{0}0.05\phantom{0}$ & $\phantom{0}0.05\phantom{0}$ & $\phantom{0}0.05\phantom{0}$ & $\phantom{0}0.05\phantom{0}$ & $\phantom{0}94.6\phantom{0}$ & $\phantom{0}94.1\phantom{0}$ & $\phantom{0}94.1\phantom{0}$ & $\phantom{0}93.8\phantom{0}$ & $\phantom{0}93.6\phantom{0}$ \\
 & \nopagebreak $\;J=1000$  & $\phantom{0}\phantom{-}0.2\phantom{0}$ & $\phantom{0}{-}3.0\phantom{0}$ & $\phantom{0}{-}0.1\phantom{0}$ & $\phantom{0}\phantom{-}0.3\phantom{0}$ & $\phantom{0}{-}1.1\phantom{0}$ & $\phantom{0}0.02\phantom{0}$ & $\phantom{0}0.02\phantom{0}$ & $\phantom{0}0.02\phantom{0}$ & $\phantom{0}0.02\phantom{0}$ & $\phantom{0}0.02\phantom{0}$ & $\phantom{0}94.8\phantom{0}$ & $\phantom{0}94.2\phantom{0}$ & $\phantom{0}94.7\phantom{0}$ & $\phantom{0}94.1\phantom{0}$ & $\phantom{0}94.3\phantom{0}$ \\
\multicolumn{4}{l}{$\bar{n}=20$} \\  & \nopagebreak $\;J=50$  & $\phantom{0}{-}0.4\phantom{0}$ & $\phantom{0}\phantom{-}0.7\phantom{0}$ & $\phantom{0}\phantom{-}1.1\phantom{0}$ & $\phantom{0}\phantom{-}1.8\phantom{0}$ & $\phantom{0}{-}7.9\phantom{0}$ & $\phantom{0}0.06\phantom{0}$ & $\phantom{0}0.07\phantom{0}$ & $\phantom{0}0.07\phantom{0}$ & $\phantom{0}0.07\phantom{0}$ & $\phantom{0}0.06\phantom{0}$ & $\phantom{0}93.1\phantom{0}$ & $\phantom{0}94.1\phantom{0}$ & $\phantom{0}95.0\phantom{0}$ & $\phantom{0}93.7\phantom{0}$ & $\phantom{0}93.6\phantom{0}$ \\
 & \nopagebreak $\;J=200$  & $\phantom{0}{-}0.1\phantom{0}$ & $\phantom{0}{-}0.5\phantom{0}$ & $\phantom{0}\phantom{-}0.1\phantom{0}$ & $\phantom{0}\phantom{-}0.4\phantom{0}$ & $\phantom{0}{-}2.6\phantom{0}$ & $\phantom{0}0.03\phantom{0}$ & $\phantom{0}0.03\phantom{0}$ & $\phantom{0}0.03\phantom{0}$ & $\phantom{0}0.03\phantom{0}$ & $\phantom{0}0.03\phantom{0}$ & $\phantom{0}94.8\phantom{0}$ & $\phantom{0}94.3\phantom{0}$ & $\phantom{0}94.7\phantom{0}$ & $\phantom{0}94.7\phantom{0}$ & $\phantom{0}94.5\phantom{0}$ \\
 & \nopagebreak $\;J=1000$  & $\phantom{0}\phantom{-}0.1\phantom{0}$ & $\phantom{0}{-}0.8\phantom{0}$ & $\phantom{0}{-}0.2\phantom{0}$ & $\phantom{0}{-}0.0\phantom{0}$ & $\phantom{0}{-}0.7\phantom{0}$ & $\phantom{0}0.01\phantom{0}$ & $\phantom{0}0.02\phantom{0}$ & $\phantom{0}0.02\phantom{0}$ & $\phantom{0}0.02\phantom{0}$ & $\phantom{0}0.02\phantom{0}$ & $\phantom{0}94.4\phantom{0}$ & $\phantom{0}94.4\phantom{0}$ & $\phantom{0}94.2\phantom{0}$ & $\phantom{0}93.8\phantom{0}$ & $\phantom{0}94.1\phantom{0}$ \\
[0.5ex]\hline\\[-1.6ex] 
& & \multicolumn{15}{c}{Moderate intraclass correlation $(\rho_{Iy}=.30)$} \\[0.6ex]\hline\\[-1.8ex]
\multicolumn{4}{l}{$\bar{n}=5$} \\  & \nopagebreak $\;J=50$  & ${-}2.5\phantom{0}$ & ${-}3.4\phantom{0}$ & ${-}2.7\phantom{0}$ & ${-}2.5\phantom{0}$ & ${-}8.0\phantom{0}$ & $\phantom{0}0.10\phantom{0}$ & $\phantom{0}0.12\phantom{0}$ & $\phantom{0}0.12\phantom{0}$ & $\phantom{0}0.12\phantom{0}$ & $\phantom{0}0.12\phantom{0}$ & $\phantom{0}93.2\phantom{0}$ & $\phantom{0}93.7\phantom{0}$ & $\phantom{0}94.1\phantom{0}$ & $\phantom{0}93.3\phantom{0}$ & $\phantom{0}92.6\phantom{0}$ \\
 & \nopagebreak $\;J=200$  & $\phantom{-}0.1\phantom{0}$ & ${-}1.0\phantom{0}$ & ${-}0.2\phantom{0}$ & ${-}0.1\phantom{0}$ & ${-}1.2\phantom{0}$ & $\phantom{0}0.05\phantom{0}$ & $\phantom{0}0.06\phantom{0}$ & $\phantom{0}0.06\phantom{0}$ & $\phantom{0}0.06\phantom{0}$ & $\phantom{0}0.06\phantom{0}$ & $\phantom{0}93.4\phantom{0}$ & $\phantom{0}93.6\phantom{0}$ & $\phantom{0}93.5\phantom{0}$ & $\phantom{0}93.6\phantom{0}$ & $\phantom{0}93.9\phantom{0}$ \\
 & \nopagebreak $\;J=1000$  & $\phantom{-}0.1\phantom{0}$ & ${-}1.0\phantom{0}$ & ${-}0.1\phantom{0}$ & $\phantom{-}0.3\phantom{0}$ & $\phantom{-}0.1\phantom{0}$ & $\phantom{0}0.02\phantom{0}$ & $\phantom{0}0.03\phantom{0}$ & $\phantom{0}0.03\phantom{0}$ & $\phantom{0}0.03\phantom{0}$ & $\phantom{0}0.03\phantom{0}$ & $\phantom{0}95.2\phantom{0}$ & $\phantom{0}94.2\phantom{0}$ & $\phantom{0}94.6\phantom{0}$ & $\phantom{0}93.8\phantom{0}$ & $\phantom{0}93.9\phantom{0}$ \\
\multicolumn{4}{l}{$\bar{n}=20$} \\  & \nopagebreak $\;J=50$  & ${-}0.4\phantom{0}$ & $\phantom{-}0.3\phantom{0}$ & $\phantom{-}0.9\phantom{0}$ & $\phantom{-}0.9\phantom{0}$ & ${-}3.0\phantom{0}$ & $\phantom{0}0.09\phantom{0}$ & $\phantom{0}0.10\phantom{0}$ & $\phantom{0}0.10\phantom{0}$ & $\phantom{0}0.10\phantom{0}$ & $\phantom{0}0.10\phantom{0}$ & $\phantom{0}91.4\phantom{0}$ & $\phantom{0}93.5\phantom{0}$ & $\phantom{0}93.9\phantom{0}$ & $\phantom{0}93.5\phantom{0}$ & $\phantom{0}92.7\phantom{0}$ \\
 & \nopagebreak $\;J=200$  & ${-}0.9\phantom{0}$ & ${-}1.2\phantom{0}$ & ${-}0.7\phantom{0}$ & ${-}0.9\phantom{0}$ & ${-}1.6\phantom{0}$ & $\phantom{0}0.05\phantom{0}$ & $\phantom{0}0.05\phantom{0}$ & $\phantom{0}0.05\phantom{0}$ & $\phantom{0}0.05\phantom{0}$ & $\phantom{0}0.05\phantom{0}$ & $\phantom{0}94.8\phantom{0}$ & $\phantom{0}94.5\phantom{0}$ & $\phantom{0}94.7\phantom{0}$ & $\phantom{0}94.6\phantom{0}$ & $\phantom{0}94.0\phantom{0}$ \\
 & \nopagebreak $\;J=1000$  & ${-}0.2\phantom{0}$ & ${-}0.4\phantom{0}$ & ${-}0.4\phantom{0}$ & ${-}0.3\phantom{0}$ & ${-}0.5\phantom{0}$ & $\phantom{0}0.02\phantom{0}$ & $\phantom{0}0.02\phantom{0}$ & $\phantom{0}0.02\phantom{0}$ & $\phantom{0}0.02\phantom{0}$ & $\phantom{0}0.02\phantom{0}$ & $\phantom{0}93.7\phantom{0}$ & $\phantom{0}94.5\phantom{0}$ & $\phantom{0}94.3\phantom{0}$ & $\phantom{0}94.3\phantom{0}$ & $\phantom{0}94.1\phantom{0}$ \\
[0.5ex]\hline\\[-1.6ex] 
\end{tabular}
\begin{tablenotes}[para,flushleft]{\footnotesize \textit{Note.} $\bar{n}$ = average cluster size; $J$ = number of clusters; CD = complete data sets; LD = listwise deletion; FCS-SL = single-level FCS; FCS-MAN = two-level FCS with manifest cluster means; FCS-LAT = two-level FCS with latent cluster means; JM = joint modeling.}\end{tablenotes}
\end{threeparttable}
\end{sidewaystable}
\begin{sidewaystable}
\begin{threeparttable}
\setlength{\tabcolsep}{1.0pt}
\renewcommand{\arraystretch}{0.95}
\footnotesize
\caption{\small Study 2: Bias (in \%), Relative RMSE, and Coverage of the 95\% Confidence Interval for the Covariance of $y$ With $z$ ($\hat\sigma_{yz}$) With Moderately Unbalanced Data (Bimodal, $\pm 40\%$) and 20\% Missing Data (MAR, $\lambda=0.5$)}
\begin{tabular}{llccccccccccccccc}
\hline\\[-1.8ex]
& & \multicolumn{5}{c}{Bias (\%)} & \multicolumn{5}{c}{Rel. RMSE} & \multicolumn{5}{c}{Coverage (\%)} \\ \cmidrule(r){3-7}\cmidrule(r){8-12}\cmidrule(r){13-17}
 &  & CD & \makecell{FCS-\\MAN} & \makecell{FCS-\\NJ} & \makecell{FCS-\\LAT} & JM & CD & \makecell{FCS-\\MAN} & \makecell{FCS-\\NJ} & \makecell{FCS-\\LAT} & JM & CD & \makecell{FCS-\\MAN} & \makecell{FCS-\\NJ} & \makecell{FCS-\\LAT} & \multicolumn{1}{c}{JM} \\ 
[0.4ex]\hline\\[-1.8ex]
& & \multicolumn{15}{c}{Small intraclass correlation $(\rho_{Iy}=.10)$} \\[0.6ex]\hline\\[-1.8ex]
\multicolumn{4}{l}{$\bar{n}=5$} \\  & \nopagebreak $\;J=50$  & $\phantom{0}{-}3.8\phantom{0}$ & $\phantom{0}{-}4.2\phantom{0}$ & $\phantom{0}{-}2.4\phantom{0}$ & $\phantom{0}{-}0.2\phantom{0}$ & ${-}12.0\phantom{0}$ & $\phantom{0}0.08\phantom{0}$ & $\phantom{0}0.10\phantom{0}$ & $\phantom{0}0.10\phantom{0}$ & $\phantom{0}0.10\phantom{0}$ & $\phantom{0}0.09\phantom{0}$ & $\phantom{0}90.6\phantom{0}$ & $\phantom{0}92.9\phantom{0}$ & $\phantom{0}93.1\phantom{0}$ & $\phantom{0}92.4\phantom{0}$ & $\phantom{0}92.5\phantom{0}$ \\
 & \nopagebreak $\;J=200$  & $\phantom{0}{-}0.4\phantom{0}$ & $\phantom{0}{-}1.2\phantom{0}$ & $\phantom{0}{-}0.2\phantom{0}$ & $\phantom{0}\phantom{-}1.2\phantom{0}$ & $\phantom{0}{-}4.2\phantom{0}$ & $\phantom{0}0.04\phantom{0}$ & $\phantom{0}0.05\phantom{0}$ & $\phantom{0}0.05\phantom{0}$ & $\phantom{0}0.05\phantom{0}$ & $\phantom{0}0.04\phantom{0}$ & $\phantom{0}94.4\phantom{0}$ & $\phantom{0}93.8\phantom{0}$ & $\phantom{0}95.0\phantom{0}$ & $\phantom{0}94.0\phantom{0}$ & $\phantom{0}94.3\phantom{0}$ \\
 & \nopagebreak $\;J=1000$  & $\phantom{0}\phantom{-}0.4\phantom{0}$ & $\phantom{0}{-}0.7\phantom{0}$ & $\phantom{0}\phantom{-}0.5\phantom{0}$ & $\phantom{0}\phantom{-}0.7\phantom{0}$ & $\phantom{0}{-}0.7\phantom{0}$ & $\phantom{0}0.02\phantom{0}$ & $\phantom{0}0.02\phantom{0}$ & $\phantom{0}0.02\phantom{0}$ & $\phantom{0}0.02\phantom{0}$ & $\phantom{0}0.02\phantom{0}$ & $\phantom{0}94.6\phantom{0}$ & $\phantom{0}94.4\phantom{0}$ & $\phantom{0}94.2\phantom{0}$ & $\phantom{0}94.6\phantom{0}$ & $\phantom{0}94.3\phantom{0}$ \\
\multicolumn{4}{l}{$\bar{n}=20$} \\  & \nopagebreak $\;J=50$  & $\phantom{0}{-}1.1\phantom{0}$ & $\phantom{0}{-}0.6\phantom{0}$ & $\phantom{0}{-}0.6\phantom{0}$ & $\phantom{0}\phantom{-}0.2\phantom{0}$ & $\phantom{0}{-}9.5\phantom{0}$ & $\phantom{0}0.06\phantom{0}$ & $\phantom{0}0.07\phantom{0}$ & $\phantom{0}0.07\phantom{0}$ & $\phantom{0}0.07\phantom{0}$ & $\phantom{0}0.06\phantom{0}$ & $\phantom{0}93.5\phantom{0}$ & $\phantom{0}93.5\phantom{0}$ & $\phantom{0}94.0\phantom{0}$ & $\phantom{0}92.9\phantom{0}$ & $\phantom{0}92.4\phantom{0}$ \\
 & \nopagebreak $\;J=200$  & $\phantom{0}{-}0.5\phantom{0}$ & $\phantom{0}{-}1.3\phantom{0}$ & $\phantom{0}{-}1.2\phantom{0}$ & $\phantom{0}{-}1.1\phantom{0}$ & $\phantom{0}{-}4.0\phantom{0}$ & $\phantom{0}0.03\phantom{0}$ & $\phantom{0}0.03\phantom{0}$ & $\phantom{0}0.03\phantom{0}$ & $\phantom{0}0.03\phantom{0}$ & $\phantom{0}0.03\phantom{0}$ & $\phantom{0}94.8\phantom{0}$ & $\phantom{0}95.1\phantom{0}$ & $\phantom{0}95.5\phantom{0}$ & $\phantom{0}95.5\phantom{0}$ & $\phantom{0}95.2\phantom{0}$ \\
 & \nopagebreak $\;J=1000$  & $\phantom{0}\phantom{-}0.2\phantom{0}$ & $\phantom{0}\phantom{-}0.1\phantom{0}$ & $\phantom{0}\phantom{-}0.3\phantom{0}$ & $\phantom{0}\phantom{-}0.4\phantom{0}$ & $\phantom{0}{-}0.2\phantom{0}$ & $\phantom{0}0.01\phantom{0}$ & $\phantom{0}0.02\phantom{0}$ & $\phantom{0}0.02\phantom{0}$ & $\phantom{0}0.02\phantom{0}$ & $\phantom{0}0.02\phantom{0}$ & $\phantom{0}94.2\phantom{0}$ & $\phantom{0}94.3\phantom{0}$ & $\phantom{0}94.6\phantom{0}$ & $\phantom{0}94.5\phantom{0}$ & $\phantom{0}94.6\phantom{0}$ \\
[0.5ex]\hline\\[-1.6ex] 
& & \multicolumn{15}{c}{Moderate intraclass correlation $(\rho_{Iy}=.30)$} \\[0.6ex]\hline\\[-1.8ex]
\multicolumn{4}{l}{$\bar{n}=5$} \\  & \nopagebreak $\;J=50$  & ${-}2.5\phantom{0}$ & ${-}1.6\phantom{0}$ & ${-}1.5\phantom{0}$ & ${-}0.8\phantom{0}$ & ${-}5.7\phantom{0}$ & $\phantom{0}0.10\phantom{0}$ & $\phantom{0}0.12\phantom{0}$ & $\phantom{0}0.12\phantom{0}$ & $\phantom{0}0.12\phantom{0}$ & $\phantom{0}0.11\phantom{0}$ & $\phantom{0}91.8\phantom{0}$ & $\phantom{0}93.5\phantom{0}$ & $\phantom{0}94.3\phantom{0}$ & $\phantom{0}93.6\phantom{0}$ & $\phantom{0}93.2\phantom{0}$ \\
 & \nopagebreak $\;J=200$  & $\phantom{-}0.5\phantom{0}$ & $\phantom{-}0.2\phantom{0}$ & $\phantom{-}0.0\phantom{0}$ & $\phantom{-}0.6\phantom{0}$ & ${-}0.7\phantom{0}$ & $\phantom{0}0.05\phantom{0}$ & $\phantom{0}0.06\phantom{0}$ & $\phantom{0}0.06\phantom{0}$ & $\phantom{0}0.06\phantom{0}$ & $\phantom{0}0.06\phantom{0}$ & $\phantom{0}94.2\phantom{0}$ & $\phantom{0}95.2\phantom{0}$ & $\phantom{0}94.7\phantom{0}$ & $\phantom{0}94.6\phantom{0}$ & $\phantom{0}94.3\phantom{0}$ \\
 & \nopagebreak $\;J=1000$  & $\phantom{-}0.1\phantom{0}$ & ${-}0.3\phantom{0}$ & ${-}0.1\phantom{0}$ & $\phantom{-}0.1\phantom{0}$ & ${-}0.2\phantom{0}$ & $\phantom{0}0.02\phantom{0}$ & $\phantom{0}0.03\phantom{0}$ & $\phantom{0}0.03\phantom{0}$ & $\phantom{0}0.03\phantom{0}$ & $\phantom{0}0.03\phantom{0}$ & $\phantom{0}95.2\phantom{0}$ & $\phantom{0}95.0\phantom{0}$ & $\phantom{0}94.9\phantom{0}$ & $\phantom{0}94.2\phantom{0}$ & $\phantom{0}94.9\phantom{0}$ \\
\multicolumn{4}{l}{$\bar{n}=20$} \\  & \nopagebreak $\;J=50$  & ${-}1.9\phantom{0}$ & ${-}1.9\phantom{0}$ & ${-}1.6\phantom{0}$ & ${-}1.7\phantom{0}$ & ${-}5.0\phantom{0}$ & $\phantom{0}0.09\phantom{0}$ & $\phantom{0}0.11\phantom{0}$ & $\phantom{0}0.11\phantom{0}$ & $\phantom{0}0.11\phantom{0}$ & $\phantom{0}0.10\phantom{0}$ & $\phantom{0}91.0\phantom{0}$ & $\phantom{0}92.8\phantom{0}$ & $\phantom{0}93.7\phantom{0}$ & $\phantom{0}93.1\phantom{0}$ & $\phantom{0}92.2\phantom{0}$ \\
 & \nopagebreak $\;J=200$  & ${-}0.4\phantom{0}$ & $\phantom{-}0.0\phantom{0}$ & ${-}0.1\phantom{0}$ & ${-}0.0\phantom{0}$ & ${-}0.8\phantom{0}$ & $\phantom{0}0.05\phantom{0}$ & $\phantom{0}0.05\phantom{0}$ & $\phantom{0}0.05\phantom{0}$ & $\phantom{0}0.05\phantom{0}$ & $\phantom{0}0.05\phantom{0}$ & $\phantom{0}94.5\phantom{0}$ & $\phantom{0}94.5\phantom{0}$ & $\phantom{0}95.0\phantom{0}$ & $\phantom{0}94.9\phantom{0}$ & $\phantom{0}94.4\phantom{0}$ \\
 & \nopagebreak $\;J=1000$  & $\phantom{-}0.2\phantom{0}$ & $\phantom{-}0.4\phantom{0}$ & $\phantom{-}0.4\phantom{0}$ & $\phantom{-}0.4\phantom{0}$ & $\phantom{-}0.2\phantom{0}$ & $\phantom{0}0.02\phantom{0}$ & $\phantom{0}0.02\phantom{0}$ & $\phantom{0}0.02\phantom{0}$ & $\phantom{0}0.02\phantom{0}$ & $\phantom{0}0.02\phantom{0}$ & $\phantom{0}95.6\phantom{0}$ & $\phantom{0}94.8\phantom{0}$ & $\phantom{0}94.6\phantom{0}$ & $\phantom{0}94.7\phantom{0}$ & $\phantom{0}94.0\phantom{0}$ \\
[0.5ex]\hline\\[-1.6ex] 
\end{tabular}
\begin{tablenotes}[para,flushleft]{\footnotesize \textit{Note.} $\bar{n}$ = average cluster size; $J$ = number of clusters; CD = complete data sets; LD = listwise deletion; FCS-SL = single-level FCS; FCS-MAN = two-level FCS with manifest cluster means; FCS-LAT = two-level FCS with latent cluster means; JM = joint modeling.}\end{tablenotes}
\end{threeparttable}
\end{sidewaystable}
\begin{sidewaystable}
\begin{threeparttable}
\setlength{\tabcolsep}{1.0pt}
\renewcommand{\arraystretch}{0.95}
\footnotesize
\caption{\small Study 2: Bias (in \%), Relative RMSE, and Coverage of the 95\% Confidence Interval for the Covariance of $y$ With $z$ ($\hat\sigma_{yz}$) With Strongly Unbalanced Data (Bimodal, $\pm 80\%$) and 20\% Missing Data (MAR, $\lambda=0.5$)}
\begin{tabular}{llccccccccccccccc}
\hline\\[-1.8ex]
& & \multicolumn{5}{c}{Bias (\%)} & \multicolumn{5}{c}{Rel. RMSE} & \multicolumn{5}{c}{Coverage (\%)} \\ \cmidrule(r){3-7}\cmidrule(r){8-12}\cmidrule(r){13-17}
 &  & CD & \makecell{FCS-\\MAN} & \makecell{FCS-\\NJ} & \makecell{FCS-\\LAT} & JM & CD & \makecell{FCS-\\MAN} & \makecell{FCS-\\NJ} & \makecell{FCS-\\LAT} & JM & CD & \makecell{FCS-\\MAN} & \makecell{FCS-\\NJ} & \makecell{FCS-\\LAT} & \multicolumn{1}{c}{JM} \\ 
[0.4ex]\hline\\[-1.8ex]
& & \multicolumn{15}{c}{Small intraclass correlation $(\rho_{Iy}=.10)$} \\[0.6ex]\hline\\[-1.8ex]
\multicolumn{4}{l}{$\bar{n}=5$} \\  & \nopagebreak $\;J=50$  & ${-}2.6\phantom{0}$ & ${-}6.1\phantom{0}$ & ${-}0.8\phantom{0}$ & $\phantom{-}0.3\phantom{0}$ & ${-}9.6\phantom{0}$ & $\phantom{0}0.09\phantom{0}$ & $\phantom{0}0.10\phantom{0}$ & $\phantom{0}0.10\phantom{0}$ & $\phantom{0}0.10\phantom{0}$ & $\phantom{0}0.09\phantom{0}$ & $\phantom{0}92.3\phantom{0}$ & $\phantom{0}93.7\phantom{0}$ & $\phantom{0}94.2\phantom{0}$ & $\phantom{0}92.6\phantom{0}$ & $\phantom{0}92.7\phantom{0}$ \\
 & \nopagebreak $\;J=200$  & ${-}0.5\phantom{0}$ & ${-}6.5\phantom{0}$ & ${-}0.4\phantom{0}$ & $\phantom{-}1.0\phantom{0}$ & ${-}3.7\phantom{0}$ & $\phantom{0}0.04\phantom{0}$ & $\phantom{0}0.05\phantom{0}$ & $\phantom{0}0.05\phantom{0}$ & $\phantom{0}0.05\phantom{0}$ & $\phantom{0}0.05\phantom{0}$ & $\phantom{0}94.5\phantom{0}$ & $\phantom{0}93.8\phantom{0}$ & $\phantom{0}95.3\phantom{0}$ & $\phantom{0}95.3\phantom{0}$ & $\phantom{0}95.0\phantom{0}$ \\
 & \nopagebreak $\;J=1000$  & ${-}0.6\phantom{0}$ & ${-}7.6\phantom{0}$ & ${-}1.2\phantom{0}$ & ${-}0.5\phantom{0}$ & ${-}1.6\phantom{0}$ & $\phantom{0}0.02\phantom{0}$ & $\phantom{0}0.02\phantom{0}$ & $\phantom{0}0.02\phantom{0}$ & $\phantom{0}0.02\phantom{0}$ & $\phantom{0}0.02\phantom{0}$ & $\phantom{0}94.3\phantom{0}$ & $\phantom{0}91.0\phantom{0}$ & $\phantom{0}94.0\phantom{0}$ & $\phantom{0}94.0\phantom{0}$ & $\phantom{0}94.5\phantom{0}$ \\
\multicolumn{4}{l}{$\bar{n}=20$} \\  & \nopagebreak $\;J=50$  & ${-}2.3\phantom{0}$ & ${-}3.8\phantom{0}$ & ${-}1.6\phantom{0}$ & ${-}0.4\phantom{0}$ & ${-}9.8\phantom{0}$ & $\phantom{0}0.06\phantom{0}$ & $\phantom{0}0.07\phantom{0}$ & $\phantom{0}0.07\phantom{0}$ & $\phantom{0}0.07\phantom{0}$ & $\phantom{0}0.07\phantom{0}$ & $\phantom{0}92.0\phantom{0}$ & $\phantom{0}93.1\phantom{0}$ & $\phantom{0}94.2\phantom{0}$ & $\phantom{0}93.3\phantom{0}$ & $\phantom{0}92.6\phantom{0}$ \\
 & \nopagebreak $\;J=200$  & ${-}0.3\phantom{0}$ & ${-}2.9\phantom{0}$ & ${-}0.7\phantom{0}$ & ${-}0.2\phantom{0}$ & ${-}3.0\phantom{0}$ & $\phantom{0}0.03\phantom{0}$ & $\phantom{0}0.04\phantom{0}$ & $\phantom{0}0.04\phantom{0}$ & $\phantom{0}0.04\phantom{0}$ & $\phantom{0}0.04\phantom{0}$ & $\phantom{0}94.0\phantom{0}$ & $\phantom{0}94.4\phantom{0}$ & $\phantom{0}95.3\phantom{0}$ & $\phantom{0}94.3\phantom{0}$ & $\phantom{0}94.1\phantom{0}$ \\
 & \nopagebreak $\;J=1000$  & ${-}0.2\phantom{0}$ & ${-}3.0\phantom{0}$ & ${-}0.6\phantom{0}$ & $\phantom{-}0.1\phantom{0}$ & ${-}0.8\phantom{0}$ & $\phantom{0}0.01\phantom{0}$ & $\phantom{0}0.02\phantom{0}$ & $\phantom{0}0.02\phantom{0}$ & $\phantom{0}0.02\phantom{0}$ & $\phantom{0}0.02\phantom{0}$ & $\phantom{0}94.4\phantom{0}$ & $\phantom{0}92.9\phantom{0}$ & $\phantom{0}94.0\phantom{0}$ & $\phantom{0}93.6\phantom{0}$ & $\phantom{0}93.7\phantom{0}$ \\
[0.5ex]\hline\\[-1.6ex] 
& & \multicolumn{15}{c}{Moderate intraclass correlation $(\rho_{Iy}=.30)$} \\[0.6ex]\hline\\[-1.8ex]
\multicolumn{4}{l}{$\bar{n}=5$} \\  & \nopagebreak $\;J=50$  & ${-}3.9\phantom{0}$ & ${-}5.3\phantom{0}$ & ${-}3.6\phantom{0}$ & ${-}1.7\phantom{0}$ & ${-}6.8\phantom{0}$ & $\phantom{0}0.11\phantom{0}$ & $\phantom{0}0.12\phantom{0}$ & $\phantom{0}0.12\phantom{0}$ & $\phantom{0}0.12\phantom{0}$ & $\phantom{0}0.12\phantom{0}$ & $\phantom{0}94.3\phantom{0}$ & $\phantom{0}94.5\phantom{0}$ & $\phantom{0}95.3\phantom{0}$ & $\phantom{0}94.7\phantom{0}$ & $\phantom{0}94.2\phantom{0}$ \\
 & \nopagebreak $\;J=200$  & ${-}0.3\phantom{0}$ & ${-}3.2\phantom{0}$ & ${-}0.9\phantom{0}$ & ${-}0.1\phantom{0}$ & ${-}1.6\phantom{0}$ & $\phantom{0}0.05\phantom{0}$ & $\phantom{0}0.06\phantom{0}$ & $\phantom{0}0.06\phantom{0}$ & $\phantom{0}0.06\phantom{0}$ & $\phantom{0}0.06\phantom{0}$ & $\phantom{0}95.1\phantom{0}$ & $\phantom{0}94.7\phantom{0}$ & $\phantom{0}95.2\phantom{0}$ & $\phantom{0}95.0\phantom{0}$ & $\phantom{0}95.4\phantom{0}$ \\
 & \nopagebreak $\;J=1000$  & $\phantom{-}0.3\phantom{0}$ & ${-}2.7\phantom{0}$ & ${-}0.4\phantom{0}$ & $\phantom{-}0.2\phantom{0}$ & ${-}0.0\phantom{0}$ & $\phantom{0}0.02\phantom{0}$ & $\phantom{0}0.03\phantom{0}$ & $\phantom{0}0.03\phantom{0}$ & $\phantom{0}0.03\phantom{0}$ & $\phantom{0}0.03\phantom{0}$ & $\phantom{0}95.8\phantom{0}$ & $\phantom{0}95.1\phantom{0}$ & $\phantom{0}95.5\phantom{0}$ & $\phantom{0}95.0\phantom{0}$ & $\phantom{0}94.7\phantom{0}$ \\
\multicolumn{4}{l}{$\bar{n}=20$} \\  & \nopagebreak $\;J=50$  & ${-}2.6\phantom{0}$ & ${-}1.7\phantom{0}$ & ${-}1.9\phantom{0}$ & ${-}1.2\phantom{0}$ & ${-}5.1\phantom{0}$ & $\phantom{0}0.09\phantom{0}$ & $\phantom{0}0.11\phantom{0}$ & $\phantom{0}0.11\phantom{0}$ & $\phantom{0}0.11\phantom{0}$ & $\phantom{0}0.10\phantom{0}$ & $\phantom{0}93.4\phantom{0}$ & $\phantom{0}93.4\phantom{0}$ & $\phantom{0}94.2\phantom{0}$ & $\phantom{0}93.6\phantom{0}$ & $\phantom{0}92.8\phantom{0}$ \\
 & \nopagebreak $\;J=200$  & $\phantom{-}0.2\phantom{0}$ & ${-}0.5\phantom{0}$ & ${-}0.2\phantom{0}$ & $\phantom{-}0.0\phantom{0}$ & ${-}1.0\phantom{0}$ & $\phantom{0}0.05\phantom{0}$ & $\phantom{0}0.05\phantom{0}$ & $\phantom{0}0.05\phantom{0}$ & $\phantom{0}0.05\phantom{0}$ & $\phantom{0}0.05\phantom{0}$ & $\phantom{0}93.8\phantom{0}$ & $\phantom{0}94.5\phantom{0}$ & $\phantom{0}94.4\phantom{0}$ & $\phantom{0}94.4\phantom{0}$ & $\phantom{0}94.5\phantom{0}$ \\
 & \nopagebreak $\;J=1000$  & ${-}0.5\phantom{0}$ & ${-}0.9\phantom{0}$ & ${-}0.6\phantom{0}$ & ${-}0.4\phantom{0}$ & ${-}0.6\phantom{0}$ & $\phantom{0}0.02\phantom{0}$ & $\phantom{0}0.02\phantom{0}$ & $\phantom{0}0.02\phantom{0}$ & $\phantom{0}0.02\phantom{0}$ & $\phantom{0}0.02\phantom{0}$ & $\phantom{0}95.7\phantom{0}$ & $\phantom{0}95.8\phantom{0}$ & $\phantom{0}95.7\phantom{0}$ & $\phantom{0}95.4\phantom{0}$ & $\phantom{0}96.5\phantom{0}$ \\
[0.5ex]\hline\\[-1.6ex] 
\end{tabular}
\begin{tablenotes}[para,flushleft]{\footnotesize \textit{Note.} $\bar{n}$ = average cluster size; $J$ = number of clusters; CD = complete data sets; LD = listwise deletion; FCS-SL = single-level FCS; FCS-MAN = two-level FCS with manifest cluster means; FCS-LAT = two-level FCS with latent cluster means; JM = joint modeling.}\end{tablenotes}
\end{threeparttable}
\end{sidewaystable}
\begin{sidewaystable}
\begin{threeparttable}
\setlength{\tabcolsep}{1.0pt}
\renewcommand{\arraystretch}{0.95}
\footnotesize
\caption{\small Study 2: Bias (in \%), Relative RMSE, and Coverage of the 95\% Confidence Interval for the Covariance of $y$ With $z$ ($\hat\sigma_{yz}$) With Moderately Unbalanced Data (Uniform, $\pm 40\%$) and 40\% Missing Data (MAR, $\lambda=0.5$)}
\begin{tabular}{llccccccccccccccc}
\hline\\[-1.8ex]
& & \multicolumn{5}{c}{Bias (\%)} & \multicolumn{5}{c}{Rel. RMSE} & \multicolumn{5}{c}{Coverage (\%)} \\ \cmidrule(r){3-7}\cmidrule(r){8-12}\cmidrule(r){13-17}
 &  & CD & \makecell{FCS-\\MAN} & \makecell{FCS-\\NJ} & \makecell{FCS-\\LAT} & JM & CD & \makecell{FCS-\\MAN} & \makecell{FCS-\\NJ} & \makecell{FCS-\\LAT} & JM & CD & \makecell{FCS-\\MAN} & \makecell{FCS-\\NJ} & \makecell{FCS-\\LAT} & \multicolumn{1}{c}{JM} \\ 
[0.4ex]\hline\\[-1.8ex]
& & \multicolumn{15}{c}{Small intraclass correlation $(\rho_{Iy}=.10)$} \\[0.6ex]\hline\\[-1.8ex]
\multicolumn{4}{l}{$\bar{n}=5$} \\  & \nopagebreak $\;J=50$  & $\phantom{0}{-}0.5\phantom{0}$ & $\phantom{0}\phantom{-}0.2\phantom{0}$ & $\phantom{0}\phantom{-}1.4\phantom{0}$ & $\phantom{0}\phantom{-}7.7\phantom{0}$ & ${-}21.0\phantom{0}$ & $\phantom{0}0.08\phantom{0}$ & $\phantom{0}0.11\phantom{0}$ & $\phantom{0}0.12\phantom{0}$ & $\phantom{0}0.11\phantom{0}$ & $\phantom{0}0.09\phantom{0}$ & $\phantom{0}92.9\phantom{0}$ & $\phantom{0}96.2\phantom{0}$ & $\phantom{0}96.7\phantom{0}$ & $\phantom{0}95.1\phantom{0}$ & $\phantom{0}96.3\phantom{0}$ \\
 & \nopagebreak $\;J=200$  & $\phantom{0}\phantom{-}2.1\phantom{0}$ & $\phantom{0}{-}0.6\phantom{0}$ & $\phantom{0}\phantom{-}1.1\phantom{0}$ & $\phantom{0}\phantom{-}4.5\phantom{0}$ & $\phantom{0}{-}8.6\phantom{0}$ & $\phantom{0}0.04\phantom{0}$ & $\phantom{0}0.05\phantom{0}$ & $\phantom{0}0.05\phantom{0}$ & $\phantom{0}0.06\phantom{0}$ & $\phantom{0}0.05\phantom{0}$ & $\phantom{0}94.7\phantom{0}$ & $\phantom{0}93.7\phantom{0}$ & $\phantom{0}94.7\phantom{0}$ & $\phantom{0}93.2\phantom{0}$ & $\phantom{0}94.6\phantom{0}$ \\
 & \nopagebreak $\;J=1000$  & $\phantom{0}{-}0.3\phantom{0}$ & $\phantom{0}{-}2.2\phantom{0}$ & $\phantom{0}{-}0.8\phantom{0}$ & $\phantom{0}{-}0.2\phantom{0}$ & $\phantom{0}{-}3.3\phantom{0}$ & $\phantom{0}0.02\phantom{0}$ & $\phantom{0}0.02\phantom{0}$ & $\phantom{0}0.02\phantom{0}$ & $\phantom{0}0.02\phantom{0}$ & $\phantom{0}0.02\phantom{0}$ & $\phantom{0}95.6\phantom{0}$ & $\phantom{0}94.0\phantom{0}$ & $\phantom{0}94.4\phantom{0}$ & $\phantom{0}93.7\phantom{0}$ & $\phantom{0}94.1\phantom{0}$ \\
\multicolumn{4}{l}{$\bar{n}=20$} \\  & \nopagebreak $\;J=50$  & $\phantom{0}{-}4.1\phantom{0}$ & $\phantom{0}{-}2.9\phantom{0}$ & $\phantom{0}{-}2.2\phantom{0}$ & $\phantom{0}{-}0.1\phantom{0}$ & ${-}20.5\phantom{0}$ & $\phantom{0}0.06\phantom{0}$ & $\phantom{0}0.08\phantom{0}$ & $\phantom{0}0.08\phantom{0}$ & $\phantom{0}0.08\phantom{0}$ & $\phantom{0}0.07\phantom{0}$ & $\phantom{0}88.9\phantom{0}$ & $\phantom{0}93.2\phantom{0}$ & $\phantom{0}94.0\phantom{0}$ & $\phantom{0}91.7\phantom{0}$ & $\phantom{0}90.3\phantom{0}$ \\
 & \nopagebreak $\;J=200$  & $\phantom{0}\phantom{-}0.1\phantom{0}$ & $\phantom{0}\phantom{-}1.1\phantom{0}$ & $\phantom{0}\phantom{-}0.7\phantom{0}$ & $\phantom{0}\phantom{-}1.1\phantom{0}$ & $\phantom{0}{-}6.1\phantom{0}$ & $\phantom{0}0.03\phantom{0}$ & $\phantom{0}0.04\phantom{0}$ & $\phantom{0}0.04\phantom{0}$ & $\phantom{0}0.04\phantom{0}$ & $\phantom{0}0.04\phantom{0}$ & $\phantom{0}94.5\phantom{0}$ & $\phantom{0}96.2\phantom{0}$ & $\phantom{0}95.9\phantom{0}$ & $\phantom{0}95.5\phantom{0}$ & $\phantom{0}94.3\phantom{0}$ \\
 & \nopagebreak $\;J=1000$  & $\phantom{0}{-}0.3\phantom{0}$ & $\phantom{0}{-}0.6\phantom{0}$ & $\phantom{0}{-}0.5\phantom{0}$ & $\phantom{0}{-}0.3\phantom{0}$ & $\phantom{0}{-}1.8\phantom{0}$ & $\phantom{0}0.01\phantom{0}$ & $\phantom{0}0.02\phantom{0}$ & $\phantom{0}0.02\phantom{0}$ & $\phantom{0}0.02\phantom{0}$ & $\phantom{0}0.02\phantom{0}$ & $\phantom{0}94.6\phantom{0}$ & $\phantom{0}94.6\phantom{0}$ & $\phantom{0}93.9\phantom{0}$ & $\phantom{0}94.7\phantom{0}$ & $\phantom{0}94.7\phantom{0}$ \\
[0.5ex]\hline\\[-1.6ex] 
& & \multicolumn{15}{c}{Moderate intraclass correlation $(\rho_{Iy}=.30)$} \\[0.6ex]\hline\\[-1.8ex]
\multicolumn{4}{l}{$\bar{n}=5$} \\  & \nopagebreak $\;J=50$  & $\phantom{0}{-}1.0\phantom{0}$ & $\phantom{0}{-}1.1\phantom{0}$ & $\phantom{0}\phantom{-}0.2\phantom{0}$ & $\phantom{0}\phantom{-}2.3\phantom{0}$ & ${-}10.7\phantom{0}$ & $\phantom{0}0.10\phantom{0}$ & $\phantom{0}0.14\phantom{0}$ & $\phantom{0}0.15\phantom{0}$ & $\phantom{0}0.14\phantom{0}$ & $\phantom{0}0.13\phantom{0}$ & $\phantom{0}92.8\phantom{0}$ & $\phantom{0}94.7\phantom{0}$ & $\phantom{0}94.6\phantom{0}$ & $\phantom{0}94.5\phantom{0}$ & $\phantom{0}93.3\phantom{0}$ \\
 & \nopagebreak $\;J=200$  & $\phantom{0}{-}0.3\phantom{0}$ & $\phantom{0}{-}0.8\phantom{0}$ & $\phantom{0}{-}0.2\phantom{0}$ & $\phantom{0}{-}0.0\phantom{0}$ & $\phantom{0}{-}2.8\phantom{0}$ & $\phantom{0}0.05\phantom{0}$ & $\phantom{0}0.07\phantom{0}$ & $\phantom{0}0.07\phantom{0}$ & $\phantom{0}0.07\phantom{0}$ & $\phantom{0}0.07\phantom{0}$ & $\phantom{0}94.7\phantom{0}$ & $\phantom{0}94.0\phantom{0}$ & $\phantom{0}94.6\phantom{0}$ & $\phantom{0}93.6\phantom{0}$ & $\phantom{0}94.7\phantom{0}$ \\
 & \nopagebreak $\;J=1000$  & $\phantom{0}{-}0.3\phantom{0}$ & $\phantom{0}{-}0.6\phantom{0}$ & $\phantom{0}{-}0.3\phantom{0}$ & $\phantom{0}{-}0.2\phantom{0}$ & $\phantom{0}{-}0.7\phantom{0}$ & $\phantom{0}0.02\phantom{0}$ & $\phantom{0}0.03\phantom{0}$ & $\phantom{0}0.03\phantom{0}$ & $\phantom{0}0.03\phantom{0}$ & $\phantom{0}0.03\phantom{0}$ & $\phantom{0}95.0\phantom{0}$ & $\phantom{0}93.6\phantom{0}$ & $\phantom{0}93.5\phantom{0}$ & $\phantom{0}93.8\phantom{0}$ & $\phantom{0}94.7\phantom{0}$ \\
\multicolumn{4}{l}{$\bar{n}=20$} \\  & \nopagebreak $\;J=50$  & $\phantom{0}{-}3.1\phantom{0}$ & $\phantom{0}{-}4.8\phantom{0}$ & $\phantom{0}{-}4.3\phantom{0}$ & $\phantom{0}{-}4.0\phantom{0}$ & ${-}11.5\phantom{0}$ & $\phantom{0}0.09\phantom{0}$ & $\phantom{0}0.12\phantom{0}$ & $\phantom{0}0.13\phantom{0}$ & $\phantom{0}0.12\phantom{0}$ & $\phantom{0}0.12\phantom{0}$ & $\phantom{0}91.4\phantom{0}$ & $\phantom{0}92.3\phantom{0}$ & $\phantom{0}93.2\phantom{0}$ & $\phantom{0}93.1\phantom{0}$ & $\phantom{0}90.8\phantom{0}$ \\
 & \nopagebreak $\;J=200$  & $\phantom{0}{-}0.2\phantom{0}$ & $\phantom{0}{-}0.2\phantom{0}$ & $\phantom{0}{-}0.4\phantom{0}$ & $\phantom{0}{-}0.3\phantom{0}$ & $\phantom{0}{-}2.1\phantom{0}$ & $\phantom{0}0.05\phantom{0}$ & $\phantom{0}0.06\phantom{0}$ & $\phantom{0}0.06\phantom{0}$ & $\phantom{0}0.06\phantom{0}$ & $\phantom{0}0.06\phantom{0}$ & $\phantom{0}94.6\phantom{0}$ & $\phantom{0}93.2\phantom{0}$ & $\phantom{0}93.2\phantom{0}$ & $\phantom{0}92.6\phantom{0}$ & $\phantom{0}94.1\phantom{0}$ \\
 & \nopagebreak $\;J=1000$  & $\phantom{0}\phantom{-}0.0\phantom{0}$ & $\phantom{0}\phantom{-}0.0\phantom{0}$ & $\phantom{0}\phantom{-}0.0\phantom{0}$ & $\phantom{0}\phantom{-}0.1\phantom{0}$ & $\phantom{0}{-}0.3\phantom{0}$ & $\phantom{0}0.02\phantom{0}$ & $\phantom{0}0.03\phantom{0}$ & $\phantom{0}0.03\phantom{0}$ & $\phantom{0}0.03\phantom{0}$ & $\phantom{0}0.03\phantom{0}$ & $\phantom{0}94.3\phantom{0}$ & $\phantom{0}93.7\phantom{0}$ & $\phantom{0}94.0\phantom{0}$ & $\phantom{0}93.8\phantom{0}$ & $\phantom{0}93.5\phantom{0}$ \\
[0.5ex]\hline\\[-1.6ex] 
\end{tabular}
\begin{tablenotes}[para,flushleft]{\footnotesize \textit{Note.} $\bar{n}$ = average cluster size; $J$ = number of clusters; CD = complete data sets; LD = listwise deletion; FCS-SL = single-level FCS; FCS-MAN = two-level FCS with manifest cluster means; FCS-LAT = two-level FCS with latent cluster means; JM = joint modeling.}\end{tablenotes}
\end{threeparttable}
\end{sidewaystable}
\begin{sidewaystable}
\begin{threeparttable}
\setlength{\tabcolsep}{1.0pt}
\renewcommand{\arraystretch}{0.95}
\footnotesize
\caption{\small Study 2: Bias (in \%), Relative RMSE, and Coverage of the 95\% Confidence Interval for the Covariance of $y$ With $z$ ($\hat\sigma_{yz}$) With Strongly Unbalanced Data (Uniform, $\pm 80\%$) and 40\% Missing Data (MAR, $\lambda=0.5$)}
\begin{tabular}{llccccccccccccccc}
\hline\\[-1.8ex]
& & \multicolumn{5}{c}{Bias (\%)} & \multicolumn{5}{c}{Rel. RMSE} & \multicolumn{5}{c}{Coverage (\%)} \\ \cmidrule(r){3-7}\cmidrule(r){8-12}\cmidrule(r){13-17}
 &  & CD & \makecell{FCS-\\MAN} & \makecell{FCS-\\NJ} & \makecell{FCS-\\LAT} & JM & CD & \makecell{FCS-\\MAN} & \makecell{FCS-\\NJ} & \makecell{FCS-\\LAT} & JM & CD & \makecell{FCS-\\MAN} & \makecell{FCS-\\NJ} & \makecell{FCS-\\LAT} & \multicolumn{1}{c}{JM} \\ 
[0.4ex]\hline\\[-1.8ex]
& & \multicolumn{15}{c}{Small intraclass correlation $(\rho_{Iy}=.10)$} \\[0.6ex]\hline\\[-1.8ex]
\multicolumn{4}{l}{$\bar{n}=5$} \\  & \nopagebreak $\;J=50$  & $\phantom{0}{-}3.3\phantom{0}$ & ${-}11.8\phantom{0}$ & $\phantom{0}{-}3.6\phantom{0}$ & $\phantom{0}\phantom{-}1.8\phantom{0}$ & ${-}25.1\phantom{0}$ & $\phantom{0}0.08\phantom{0}$ & $\phantom{0}0.11\phantom{0}$ & $\phantom{0}0.12\phantom{0}$ & $\phantom{0}0.11\phantom{0}$ & $\phantom{0}0.10\phantom{0}$ & $\phantom{0}92.5\phantom{0}$ & $\phantom{0}94.4\phantom{0}$ & $\phantom{0}95.4\phantom{0}$ & $\phantom{0}92.4\phantom{0}$ & $\phantom{0}92.7\phantom{0}$ \\
 & \nopagebreak $\;J=200$  & $\phantom{0}{-}0.8\phantom{0}$ & $\phantom{0}{-}7.7\phantom{0}$ & $\phantom{0}{-}0.4\phantom{0}$ & $\phantom{0}\phantom{-}2.9\phantom{0}$ & $\phantom{0}{-}9.7\phantom{0}$ & $\phantom{0}0.04\phantom{0}$ & $\phantom{0}0.05\phantom{0}$ & $\phantom{0}0.05\phantom{0}$ & $\phantom{0}0.06\phantom{0}$ & $\phantom{0}0.05\phantom{0}$ & $\phantom{0}95.3\phantom{0}$ & $\phantom{0}94.1\phantom{0}$ & $\phantom{0}95.8\phantom{0}$ & $\phantom{0}94.9\phantom{0}$ & $\phantom{0}94.6\phantom{0}$ \\
 & \nopagebreak $\;J=1000$  & $\phantom{0}\phantom{-}0.3\phantom{0}$ & $\phantom{0}{-}7.5\phantom{0}$ & $\phantom{0}{-}0.4\phantom{0}$ & $\phantom{0}\phantom{-}0.6\phantom{0}$ & $\phantom{0}{-}2.7\phantom{0}$ & $\phantom{0}0.02\phantom{0}$ & $\phantom{0}0.03\phantom{0}$ & $\phantom{0}0.02\phantom{0}$ & $\phantom{0}0.02\phantom{0}$ & $\phantom{0}0.02\phantom{0}$ & $\phantom{0}95.0\phantom{0}$ & $\phantom{0}92.1\phantom{0}$ & $\phantom{0}95.5\phantom{0}$ & $\phantom{0}93.8\phantom{0}$ & $\phantom{0}94.6\phantom{0}$ \\
\multicolumn{4}{l}{$\bar{n}=20$} \\  & \nopagebreak $\;J=50$  & $\phantom{0}{-}1.3\phantom{0}$ & $\phantom{0}{-}1.9\phantom{0}$ & $\phantom{0}{-}0.1\phantom{0}$ & $\phantom{0}\phantom{-}1.5\phantom{0}$ & ${-}20.1\phantom{0}$ & $\phantom{0}0.06\phantom{0}$ & $\phantom{0}0.08\phantom{0}$ & $\phantom{0}0.08\phantom{0}$ & $\phantom{0}0.08\phantom{0}$ & $\phantom{0}0.07\phantom{0}$ & $\phantom{0}93.0\phantom{0}$ & $\phantom{0}93.7\phantom{0}$ & $\phantom{0}95.4\phantom{0}$ & $\phantom{0}94.5\phantom{0}$ & $\phantom{0}91.5\phantom{0}$ \\
 & \nopagebreak $\;J=200$  & $\phantom{0}{-}0.8\phantom{0}$ & $\phantom{0}{-}4.1\phantom{0}$ & $\phantom{0}{-}1.5\phantom{0}$ & $\phantom{0}{-}1.3\phantom{0}$ & $\phantom{0}{-}8.1\phantom{0}$ & $\phantom{0}0.03\phantom{0}$ & $\phantom{0}0.04\phantom{0}$ & $\phantom{0}0.04\phantom{0}$ & $\phantom{0}0.04\phantom{0}$ & $\phantom{0}0.04\phantom{0}$ & $\phantom{0}92.6\phantom{0}$ & $\phantom{0}94.1\phantom{0}$ & $\phantom{0}95.1\phantom{0}$ & $\phantom{0}93.2\phantom{0}$ & $\phantom{0}92.5\phantom{0}$ \\
 & \nopagebreak $\;J=1000$  & $\phantom{0}\phantom{-}0.2\phantom{0}$ & $\phantom{0}{-}1.6\phantom{0}$ & $\phantom{0}\phantom{-}0.1\phantom{0}$ & $\phantom{0}\phantom{-}0.5\phantom{0}$ & $\phantom{0}{-}1.1\phantom{0}$ & $\phantom{0}0.01\phantom{0}$ & $\phantom{0}0.02\phantom{0}$ & $\phantom{0}0.02\phantom{0}$ & $\phantom{0}0.02\phantom{0}$ & $\phantom{0}0.02\phantom{0}$ & $\phantom{0}95.4\phantom{0}$ & $\phantom{0}95.5\phantom{0}$ & $\phantom{0}96.0\phantom{0}$ & $\phantom{0}95.9\phantom{0}$ & $\phantom{0}96.2\phantom{0}$ \\
[0.5ex]\hline\\[-1.6ex] 
& & \multicolumn{15}{c}{Moderate intraclass correlation $(\rho_{Iy}=.30)$} \\[0.6ex]\hline\\[-1.8ex]
\multicolumn{4}{l}{$\bar{n}=5$} \\  & \nopagebreak $\;J=50$  & $\phantom{-}0.1\phantom{0}$ & ${-}0.9\phantom{0}$ & $\phantom{-}2.4\phantom{0}$ & $\phantom{-}3.6\phantom{0}$ & ${-}9.6\phantom{0}$ & $\phantom{0}0.11\phantom{0}$ & $\phantom{0}0.15\phantom{0}$ & $\phantom{0}0.15\phantom{0}$ & $\phantom{0}0.15\phantom{0}$ & $\phantom{0}0.13\phantom{0}$ & $\phantom{0}91.8\phantom{0}$ & $\phantom{0}94.0\phantom{0}$ & $\phantom{0}96.2\phantom{0}$ & $\phantom{0}93.5\phantom{0}$ & $\phantom{0}93.7\phantom{0}$ \\
 & \nopagebreak $\;J=200$  & ${-}1.5\phantom{0}$ & ${-}3.9\phantom{0}$ & ${-}1.5\phantom{0}$ & ${-}0.5\phantom{0}$ & ${-}3.6\phantom{0}$ & $\phantom{0}0.05\phantom{0}$ & $\phantom{0}0.07\phantom{0}$ & $\phantom{0}0.07\phantom{0}$ & $\phantom{0}0.07\phantom{0}$ & $\phantom{0}0.07\phantom{0}$ & $\phantom{0}93.8\phantom{0}$ & $\phantom{0}92.7\phantom{0}$ & $\phantom{0}93.6\phantom{0}$ & $\phantom{0}93.5\phantom{0}$ & $\phantom{0}94.1\phantom{0}$ \\
 & \nopagebreak $\;J=1000$  & ${-}0.4\phantom{0}$ & ${-}2.8\phantom{0}$ & ${-}0.8\phantom{0}$ & ${-}0.3\phantom{0}$ & ${-}0.7\phantom{0}$ & $\phantom{0}0.02\phantom{0}$ & $\phantom{0}0.03\phantom{0}$ & $\phantom{0}0.03\phantom{0}$ & $\phantom{0}0.03\phantom{0}$ & $\phantom{0}0.03\phantom{0}$ & $\phantom{0}95.2\phantom{0}$ & $\phantom{0}93.7\phantom{0}$ & $\phantom{0}94.5\phantom{0}$ & $\phantom{0}94.0\phantom{0}$ & $\phantom{0}95.3\phantom{0}$ \\
\multicolumn{4}{l}{$\bar{n}=20$} \\  & \nopagebreak $\;J=50$  & ${-}1.4\phantom{0}$ & ${-}1.0\phantom{0}$ & ${-}0.9\phantom{0}$ & ${-}0.6\phantom{0}$ & ${-}8.3\phantom{0}$ & $\phantom{0}0.09\phantom{0}$ & $\phantom{0}0.12\phantom{0}$ & $\phantom{0}0.13\phantom{0}$ & $\phantom{0}0.12\phantom{0}$ & $\phantom{0}0.12\phantom{0}$ & $\phantom{0}91.4\phantom{0}$ & $\phantom{0}93.1\phantom{0}$ & $\phantom{0}94.0\phantom{0}$ & $\phantom{0}92.5\phantom{0}$ & $\phantom{0}92.0\phantom{0}$ \\
 & \nopagebreak $\;J=200$  & $\phantom{-}0.2\phantom{0}$ & ${-}0.4\phantom{0}$ & $\phantom{-}0.1\phantom{0}$ & ${-}0.0\phantom{0}$ & ${-}2.1\phantom{0}$ & $\phantom{0}0.05\phantom{0}$ & $\phantom{0}0.06\phantom{0}$ & $\phantom{0}0.06\phantom{0}$ & $\phantom{0}0.06\phantom{0}$ & $\phantom{0}0.06\phantom{0}$ & $\phantom{0}93.8\phantom{0}$ & $\phantom{0}92.9\phantom{0}$ & $\phantom{0}94.1\phantom{0}$ & $\phantom{0}93.2\phantom{0}$ & $\phantom{0}93.1\phantom{0}$ \\
 & \nopagebreak $\;J=1000$  & ${-}0.2\phantom{0}$ & ${-}0.5\phantom{0}$ & ${-}0.4\phantom{0}$ & ${-}0.2\phantom{0}$ & ${-}0.6\phantom{0}$ & $\phantom{0}0.02\phantom{0}$ & $\phantom{0}0.03\phantom{0}$ & $\phantom{0}0.03\phantom{0}$ & $\phantom{0}0.03\phantom{0}$ & $\phantom{0}0.03\phantom{0}$ & $\phantom{0}94.7\phantom{0}$ & $\phantom{0}93.5\phantom{0}$ & $\phantom{0}95.1\phantom{0}$ & $\phantom{0}93.3\phantom{0}$ & $\phantom{0}94.7\phantom{0}$ \\
[0.5ex]\hline\\[-1.6ex] 
\end{tabular}
\begin{tablenotes}[para,flushleft]{\footnotesize \textit{Note.} $\bar{n}$ = average cluster size; $J$ = number of clusters; CD = complete data sets; LD = listwise deletion; FCS-SL = single-level FCS; FCS-MAN = two-level FCS with manifest cluster means; FCS-LAT = two-level FCS with latent cluster means; JM = joint modeling.}\end{tablenotes}
\end{threeparttable}
\end{sidewaystable}
\begin{sidewaystable}
\begin{threeparttable}
\setlength{\tabcolsep}{1.0pt}
\renewcommand{\arraystretch}{0.95}
\footnotesize
\caption{\small Study 2: Bias (in \%), Relative RMSE, and Coverage of the 95\% Confidence Interval for the Covariance of $y$ With $z$ ($\hat\sigma_{yz}$) With Moderately Unbalanced Data (Bimodal, $\pm 40\%$) and 40\% Missing Data (MAR, $\lambda=0.5$)}
\begin{tabular}{llccccccccccccccc}
\hline\\[-1.8ex]
& & \multicolumn{5}{c}{Bias (\%)} & \multicolumn{5}{c}{Rel. RMSE} & \multicolumn{5}{c}{Coverage (\%)} \\ \cmidrule(r){3-7}\cmidrule(r){8-12}\cmidrule(r){13-17}
 &  & CD & \makecell{FCS-\\MAN} & \makecell{FCS-\\NJ} & \makecell{FCS-\\LAT} & JM & CD & \makecell{FCS-\\MAN} & \makecell{FCS-\\NJ} & \makecell{FCS-\\LAT} & JM & CD & \makecell{FCS-\\MAN} & \makecell{FCS-\\NJ} & \makecell{FCS-\\LAT} & \multicolumn{1}{c}{JM} \\ 
[0.4ex]\hline\\[-1.8ex]
& & \multicolumn{15}{c}{Small intraclass correlation $(\rho_{Iy}=.10)$} \\[0.6ex]\hline\\[-1.8ex]
\multicolumn{4}{l}{$\bar{n}=5$} \\  & \nopagebreak $\;J=50$  & $\phantom{0}{-}2.9\phantom{0}$ & $\phantom{0}{-}3.4\phantom{0}$ & $\phantom{0}{-}2.2\phantom{0}$ & $\phantom{0}\phantom{-}5.6\phantom{0}$ & ${-}21.9\phantom{0}$ & $\phantom{0}0.08\phantom{0}$ & $\phantom{0}0.11\phantom{0}$ & $\phantom{0}0.12\phantom{0}$ & $\phantom{0}0.12\phantom{0}$ & $\phantom{0}0.10\phantom{0}$ & $\phantom{0}93.0\phantom{0}$ & $\phantom{0}94.6\phantom{0}$ & $\phantom{0}95.0\phantom{0}$ & $\phantom{0}93.6\phantom{0}$ & $\phantom{0}93.2\phantom{0}$ \\
 & \nopagebreak $\;J=200$  & $\phantom{0}\phantom{-}0.1\phantom{0}$ & $\phantom{0}{-}2.2\phantom{0}$ & $\phantom{0}\phantom{-}0.1\phantom{0}$ & $\phantom{0}\phantom{-}4.3\phantom{0}$ & $\phantom{0}{-}9.0\phantom{0}$ & $\phantom{0}0.04\phantom{0}$ & $\phantom{0}0.05\phantom{0}$ & $\phantom{0}0.06\phantom{0}$ & $\phantom{0}0.06\phantom{0}$ & $\phantom{0}0.05\phantom{0}$ & $\phantom{0}94.3\phantom{0}$ & $\phantom{0}94.2\phantom{0}$ & $\phantom{0}93.8\phantom{0}$ & $\phantom{0}93.2\phantom{0}$ & $\phantom{0}93.9\phantom{0}$ \\
 & \nopagebreak $\;J=1000$  & $\phantom{0}{-}0.7\phantom{0}$ & $\phantom{0}{-}3.2\phantom{0}$ & $\phantom{0}{-}0.6\phantom{0}$ & $\phantom{0}\phantom{-}0.4\phantom{0}$ & $\phantom{0}{-}3.0\phantom{0}$ & $\phantom{0}0.02\phantom{0}$ & $\phantom{0}0.02\phantom{0}$ & $\phantom{0}0.02\phantom{0}$ & $\phantom{0}0.02\phantom{0}$ & $\phantom{0}0.02\phantom{0}$ & $\phantom{0}94.3\phantom{0}$ & $\phantom{0}94.7\phantom{0}$ & $\phantom{0}93.3\phantom{0}$ & $\phantom{0}93.6\phantom{0}$ & $\phantom{0}93.6\phantom{0}$ \\
\multicolumn{4}{l}{$\bar{n}=20$} \\  & \nopagebreak $\;J=50$  & $\phantom{0}{-}3.0\phantom{0}$ & $\phantom{0}{-}1.8\phantom{0}$ & $\phantom{0}{-}2.2\phantom{0}$ & $\phantom{0}\phantom{-}0.4\phantom{0}$ & ${-}20.2\phantom{0}$ & $\phantom{0}0.06\phantom{0}$ & $\phantom{0}0.08\phantom{0}$ & $\phantom{0}0.08\phantom{0}$ & $\phantom{0}0.08\phantom{0}$ & $\phantom{0}0.07\phantom{0}$ & $\phantom{0}93.9\phantom{0}$ & $\phantom{0}94.8\phantom{0}$ & $\phantom{0}94.9\phantom{0}$ & $\phantom{0}93.6\phantom{0}$ & $\phantom{0}91.9\phantom{0}$ \\
 & \nopagebreak $\;J=200$  & $\phantom{0}{-}0.8\phantom{0}$ & $\phantom{0}{-}2.0\phantom{0}$ & $\phantom{0}{-}1.2\phantom{0}$ & $\phantom{0}{-}1.2\phantom{0}$ & $\phantom{0}{-}8.1\phantom{0}$ & $\phantom{0}0.03\phantom{0}$ & $\phantom{0}0.04\phantom{0}$ & $\phantom{0}0.04\phantom{0}$ & $\phantom{0}0.04\phantom{0}$ & $\phantom{0}0.04\phantom{0}$ & $\phantom{0}94.2\phantom{0}$ & $\phantom{0}94.1\phantom{0}$ & $\phantom{0}94.3\phantom{0}$ & $\phantom{0}93.9\phantom{0}$ & $\phantom{0}92.9\phantom{0}$ \\
 & \nopagebreak $\;J=1000$  & $\phantom{0}\phantom{-}0.0\phantom{0}$ & $\phantom{0}{-}1.0\phantom{0}$ & $\phantom{0}{-}0.4\phantom{0}$ & $\phantom{0}{-}0.2\phantom{0}$ & $\phantom{0}{-}2.0\phantom{0}$ & $\phantom{0}0.01\phantom{0}$ & $\phantom{0}0.02\phantom{0}$ & $\phantom{0}0.02\phantom{0}$ & $\phantom{0}0.02\phantom{0}$ & $\phantom{0}0.02\phantom{0}$ & $\phantom{0}94.8\phantom{0}$ & $\phantom{0}94.4\phantom{0}$ & $\phantom{0}94.1\phantom{0}$ & $\phantom{0}94.4\phantom{0}$ & $\phantom{0}94.8\phantom{0}$ \\
[0.5ex]\hline\\[-1.6ex] 
& & \multicolumn{15}{c}{Moderate intraclass correlation $(\rho_{Iy}=.30)$} \\[0.6ex]\hline\\[-1.8ex]
\multicolumn{4}{l}{$\bar{n}=5$} \\  & \nopagebreak $\;J=50$  & $\phantom{0}{-}1.4\phantom{0}$ & $\phantom{0}{-}2.2\phantom{0}$ & $\phantom{0}{-}2.1\phantom{0}$ & $\phantom{0}\phantom{-}1.2\phantom{0}$ & ${-}10.9\phantom{0}$ & $\phantom{0}0.10\phantom{0}$ & $\phantom{0}0.14\phantom{0}$ & $\phantom{0}0.15\phantom{0}$ & $\phantom{0}0.14\phantom{0}$ & $\phantom{0}0.13\phantom{0}$ & $\phantom{0}91.6\phantom{0}$ & $\phantom{0}92.6\phantom{0}$ & $\phantom{0}95.0\phantom{0}$ & $\phantom{0}92.8\phantom{0}$ & $\phantom{0}92.3\phantom{0}$ \\
 & \nopagebreak $\;J=200$  & $\phantom{0}{-}1.0\phantom{0}$ & $\phantom{0}{-}1.4\phantom{0}$ & $\phantom{0}{-}1.0\phantom{0}$ & $\phantom{0}{-}0.6\phantom{0}$ & $\phantom{0}{-}3.7\phantom{0}$ & $\phantom{0}0.05\phantom{0}$ & $\phantom{0}0.07\phantom{0}$ & $\phantom{0}0.07\phantom{0}$ & $\phantom{0}0.07\phantom{0}$ & $\phantom{0}0.07\phantom{0}$ & $\phantom{0}93.3\phantom{0}$ & $\phantom{0}94.4\phantom{0}$ & $\phantom{0}95.0\phantom{0}$ & $\phantom{0}94.4\phantom{0}$ & $\phantom{0}94.4\phantom{0}$ \\
 & \nopagebreak $\;J=1000$  & $\phantom{0}{-}0.1\phantom{0}$ & $\phantom{0}{-}0.7\phantom{0}$ & $\phantom{0}{-}0.0\phantom{0}$ & $\phantom{0}\phantom{-}0.3\phantom{0}$ & $\phantom{0}{-}0.2\phantom{0}$ & $\phantom{0}0.02\phantom{0}$ & $\phantom{0}0.03\phantom{0}$ & $\phantom{0}0.03\phantom{0}$ & $\phantom{0}0.03\phantom{0}$ & $\phantom{0}0.03\phantom{0}$ & $\phantom{0}94.3\phantom{0}$ & $\phantom{0}94.5\phantom{0}$ & $\phantom{0}94.7\phantom{0}$ & $\phantom{0}95.3\phantom{0}$ & $\phantom{0}94.7\phantom{0}$ \\
\multicolumn{4}{l}{$\bar{n}=20$} \\  & \nopagebreak $\;J=50$  & $\phantom{0}{-}1.5\phantom{0}$ & $\phantom{0}{-}2.8\phantom{0}$ & $\phantom{0}{-}2.7\phantom{0}$ & $\phantom{0}{-}2.1\phantom{0}$ & ${-}10.0\phantom{0}$ & $\phantom{0}0.09\phantom{0}$ & $\phantom{0}0.12\phantom{0}$ & $\phantom{0}0.13\phantom{0}$ & $\phantom{0}0.12\phantom{0}$ & $\phantom{0}0.12\phantom{0}$ & $\phantom{0}92.2\phantom{0}$ & $\phantom{0}92.9\phantom{0}$ & $\phantom{0}94.0\phantom{0}$ & $\phantom{0}93.9\phantom{0}$ & $\phantom{0}92.2\phantom{0}$ \\
 & \nopagebreak $\;J=200$  & $\phantom{0}{-}0.2\phantom{0}$ & $\phantom{0}\phantom{-}0.1\phantom{0}$ & $\phantom{0}\phantom{-}0.2\phantom{0}$ & $\phantom{0}\phantom{-}0.1\phantom{0}$ & $\phantom{0}{-}1.8\phantom{0}$ & $\phantom{0}0.05\phantom{0}$ & $\phantom{0}0.06\phantom{0}$ & $\phantom{0}0.06\phantom{0}$ & $\phantom{0}0.06\phantom{0}$ & $\phantom{0}0.06\phantom{0}$ & $\phantom{0}93.3\phantom{0}$ & $\phantom{0}94.5\phantom{0}$ & $\phantom{0}92.9\phantom{0}$ & $\phantom{0}94.5\phantom{0}$ & $\phantom{0}94.3\phantom{0}$ \\
 & \nopagebreak $\;J=1000$  & $\phantom{0}\phantom{-}0.1\phantom{0}$ & $\phantom{0}{-}0.0\phantom{0}$ & $\phantom{0}\phantom{-}0.1\phantom{0}$ & $\phantom{0}\phantom{-}0.2\phantom{0}$ & $\phantom{0}{-}0.2\phantom{0}$ & $\phantom{0}0.02\phantom{0}$ & $\phantom{0}0.03\phantom{0}$ & $\phantom{0}0.03\phantom{0}$ & $\phantom{0}0.03\phantom{0}$ & $\phantom{0}0.03\phantom{0}$ & $\phantom{0}94.5\phantom{0}$ & $\phantom{0}93.1\phantom{0}$ & $\phantom{0}94.6\phantom{0}$ & $\phantom{0}93.5\phantom{0}$ & $\phantom{0}94.5\phantom{0}$ \\
[0.5ex]\hline\\[-1.6ex] 
\end{tabular}
\begin{tablenotes}[para,flushleft]{\footnotesize \textit{Note.} $\bar{n}$ = average cluster size; $J$ = number of clusters; CD = complete data sets; LD = listwise deletion; FCS-SL = single-level FCS; FCS-MAN = two-level FCS with manifest cluster means; FCS-LAT = two-level FCS with latent cluster means; JM = joint modeling.}\end{tablenotes}
\end{threeparttable}
\end{sidewaystable}
\begin{sidewaystable}
\begin{threeparttable}
\setlength{\tabcolsep}{1.0pt}
\renewcommand{\arraystretch}{0.95}
\footnotesize
\caption{\small Study 2: Bias (in \%), Relative RMSE, and Coverage of the 95\% Confidence Interval for the Covariance of $y$ With $z$ ($\hat\sigma_{yz}$) With Strongly Unbalanced Data (Bimodal, $\pm 80\%$) and 40\% Missing Data (MAR, $\lambda=0.5$)}
\begin{tabular}{llccccccccccccccc}
\hline\\[-1.8ex]
& & \multicolumn{5}{c}{Bias (\%)} & \multicolumn{5}{c}{Rel. RMSE} & \multicolumn{5}{c}{Coverage (\%)} \\ \cmidrule(r){3-7}\cmidrule(r){8-12}\cmidrule(r){13-17}
 &  & CD & \makecell{FCS-\\MAN} & \makecell{FCS-\\NJ} & \makecell{FCS-\\LAT} & JM & CD & \makecell{FCS-\\MAN} & \makecell{FCS-\\NJ} & \makecell{FCS-\\LAT} & JM & CD & \makecell{FCS-\\MAN} & \makecell{FCS-\\NJ} & \makecell{FCS-\\LAT} & \multicolumn{1}{c}{JM} \\ 
[0.4ex]\hline\\[-1.8ex]
& & \multicolumn{15}{c}{Small intraclass correlation $(\rho_{Iy}=.10)$} \\[0.6ex]\hline\\[-1.8ex]
\multicolumn{4}{l}{$\bar{n}=5$} \\  & \nopagebreak $\;J=50$  & $\phantom{0}{-}2.5\phantom{0}$ & ${-}15.5\phantom{0}$ & $\phantom{0}{-}4.7\phantom{0}$ & $\phantom{0}{-}0.4\phantom{0}$ & ${-}24.5\phantom{0}$ & $\phantom{0}0.08\phantom{0}$ & $\phantom{0}0.11\phantom{0}$ & $\phantom{0}0.13\phantom{0}$ & $\phantom{0}0.12\phantom{0}$ & $\phantom{0}0.10\phantom{0}$ & $\phantom{0}93.1\phantom{0}$ & $\phantom{0}94.5\phantom{0}$ & $\phantom{0}95.5\phantom{0}$ & $\phantom{0}93.5\phantom{0}$ & $\phantom{0}92.8\phantom{0}$ \\
 & \nopagebreak $\;J=200$  & $\phantom{0}{-}0.7\phantom{0}$ & ${-}17.5\phantom{0}$ & $\phantom{0}{-}2.5\phantom{0}$ & $\phantom{0}\phantom{-}0.9\phantom{0}$ & $\phantom{0}{-}10.0\phantom{0}$ & $\phantom{0}0.04\phantom{0}$ & $\phantom{0}0.06\phantom{0}$ & $\phantom{0}0.05\phantom{0}$ & $\phantom{0}0.06\phantom{0}$ & $\phantom{0}0.05\phantom{0}$ & $\phantom{0}94.6\phantom{0}$ & $\phantom{0}92.8\phantom{0}$ & $\phantom{0}95.5\phantom{0}$ & $\phantom{0}94.2\phantom{0}$ & $\phantom{0}94.6\phantom{0}$ \\
 & \nopagebreak $\;J=1000$  & $\phantom{0}\phantom{-}0.2\phantom{0}$ & ${-}16.8\phantom{0}$ & $\phantom{0}{-}1.5\phantom{0}$ & $\phantom{0}\phantom{-}0.3\phantom{0}$ & $\phantom{0}{-}2.7\phantom{0}$ & $\phantom{0}0.02\phantom{0}$ & $\phantom{0}0.04\phantom{0}$ & $\phantom{0}0.02\phantom{0}$ & $\phantom{0}0.02\phantom{0}$ & $\phantom{0}0.02\phantom{0}$ & $\phantom{0}95.4\phantom{0}$ & $\phantom{0}81.4\phantom{0}$ & $\phantom{0}94.4\phantom{0}$ & $\phantom{0}93.3\phantom{0}$ & $\phantom{0}94.1\phantom{0}$ \\
\multicolumn{4}{l}{$\bar{n}=20$} \\  & \nopagebreak $\;J=50$  & $\phantom{0}{-}2.2\phantom{0}$ & $\phantom{0}{-}7.6\phantom{0}$ & $\phantom{0}{-}2.4\phantom{0}$ & $\phantom{0}{-}0.7\phantom{0}$ & ${-}20.2\phantom{0}$ & $\phantom{0}0.06\phantom{0}$ & $\phantom{0}0.09\phantom{0}$ & $\phantom{0}0.09\phantom{0}$ & $\phantom{0}0.08\phantom{0}$ & $\phantom{0}0.08\phantom{0}$ & $\phantom{0}91.4\phantom{0}$ & $\phantom{0}94.3\phantom{0}$ & $\phantom{0}95.9\phantom{0}$ & $\phantom{0}93.8\phantom{0}$ & $\phantom{0}92.3\phantom{0}$ \\
 & \nopagebreak $\;J=200$  & $\phantom{0}{-}0.1\phantom{0}$ & $\phantom{0}{-}6.8\phantom{0}$ & $\phantom{0}{-}1.1\phantom{0}$ & $\phantom{0}\phantom{-}0.2\phantom{0}$ & $\phantom{0}{-}7.2\phantom{0}$ & $\phantom{0}0.03\phantom{0}$ & $\phantom{0}0.04\phantom{0}$ & $\phantom{0}0.04\phantom{0}$ & $\phantom{0}0.04\phantom{0}$ & $\phantom{0}0.04\phantom{0}$ & $\phantom{0}95.6\phantom{0}$ & $\phantom{0}94.5\phantom{0}$ & $\phantom{0}95.3\phantom{0}$ & $\phantom{0}95.3\phantom{0}$ & $\phantom{0}94.9\phantom{0}$ \\
 & \nopagebreak $\;J=1000$  & $\phantom{0}{-}0.0\phantom{0}$ & $\phantom{0}{-}7.1\phantom{0}$ & $\phantom{0}{-}1.1\phantom{0}$ & $\phantom{0}{-}0.1\phantom{0}$ & $\phantom{0}{-}1.8\phantom{0}$ & $\phantom{0}0.01\phantom{0}$ & $\phantom{0}0.02\phantom{0}$ & $\phantom{0}0.02\phantom{0}$ & $\phantom{0}0.02\phantom{0}$ & $\phantom{0}0.02\phantom{0}$ & $\phantom{0}93.7\phantom{0}$ & $\phantom{0}90.0\phantom{0}$ & $\phantom{0}94.9\phantom{0}$ & $\phantom{0}94.4\phantom{0}$ & $\phantom{0}94.3\phantom{0}$ \\
[0.5ex]\hline\\[-1.6ex] 
& & \multicolumn{15}{c}{Moderate intraclass correlation $(\rho_{Iy}=.30)$} \\[0.6ex]\hline\\[-1.8ex]
\multicolumn{4}{l}{$\bar{n}=5$} \\  & \nopagebreak $\;J=50$  & $\phantom{0}{-}0.7\phantom{0}$ & $\phantom{0}{-}8.6\phantom{0}$ & $\phantom{0}{-}2.2\phantom{0}$ & $\phantom{0}{-}0.2\phantom{0}$ & ${-}13.0\phantom{0}$ & $\phantom{0}0.11\phantom{0}$ & $\phantom{0}0.16\phantom{0}$ & $\phantom{0}0.16\phantom{0}$ & $\phantom{0}0.16\phantom{0}$ & $\phantom{0}0.14\phantom{0}$ & $\phantom{0}92.9\phantom{0}$ & $\phantom{0}92.6\phantom{0}$ & $\phantom{0}94.0\phantom{0}$ & $\phantom{0}92.3\phantom{0}$ & $\phantom{0}92.4\phantom{0}$ \\
 & \nopagebreak $\;J=200$  & $\phantom{0}{-}0.9\phantom{0}$ & $\phantom{0}{-}8.9\phantom{0}$ & $\phantom{0}{-}2.6\phantom{0}$ & $\phantom{0}{-}1.6\phantom{0}$ & $\phantom{0}{-}4.9\phantom{0}$ & $\phantom{0}0.05\phantom{0}$ & $\phantom{0}0.08\phantom{0}$ & $\phantom{0}0.07\phantom{0}$ & $\phantom{0}0.07\phantom{0}$ & $\phantom{0}0.07\phantom{0}$ & $\phantom{0}94.8\phantom{0}$ & $\phantom{0}92.1\phantom{0}$ & $\phantom{0}94.4\phantom{0}$ & $\phantom{0}95.2\phantom{0}$ & $\phantom{0}94.3\phantom{0}$ \\
 & \nopagebreak $\;J=1000$  & $\phantom{0}{-}0.3\phantom{0}$ & $\phantom{0}{-}7.8\phantom{0}$ & $\phantom{0}{-}1.3\phantom{0}$ & $\phantom{0}{-}0.3\phantom{0}$ & $\phantom{0}{-}0.8\phantom{0}$ & $\phantom{0}0.02\phantom{0}$ & $\phantom{0}0.04\phantom{0}$ & $\phantom{0}0.03\phantom{0}$ & $\phantom{0}0.03\phantom{0}$ & $\phantom{0}0.03\phantom{0}$ & $\phantom{0}95.5\phantom{0}$ & $\phantom{0}89.7\phantom{0}$ & $\phantom{0}94.6\phantom{0}$ & $\phantom{0}94.5\phantom{0}$ & $\phantom{0}95.2\phantom{0}$ \\
\multicolumn{4}{l}{$\bar{n}=20$} \\  & \nopagebreak $\;J=50$  & $\phantom{0}{-}0.6\phantom{0}$ & $\phantom{0}{-}3.2\phantom{0}$ & $\phantom{0}{-}2.6\phantom{0}$ & $\phantom{0}{-}1.9\phantom{0}$ & ${-}10.7\phantom{0}$ & $\phantom{0}0.10\phantom{0}$ & $\phantom{0}0.12\phantom{0}$ & $\phantom{0}0.13\phantom{0}$ & $\phantom{0}0.13\phantom{0}$ & $\phantom{0}0.12\phantom{0}$ & $\phantom{0}90.6\phantom{0}$ & $\phantom{0}93.9\phantom{0}$ & $\phantom{0}95.0\phantom{0}$ & $\phantom{0}94.7\phantom{0}$ & $\phantom{0}93.7\phantom{0}$ \\
 & \nopagebreak $\;J=200$  & $\phantom{0}\phantom{-}0.1\phantom{0}$ & $\phantom{0}{-}2.1\phantom{0}$ & $\phantom{0}{-}0.7\phantom{0}$ & $\phantom{0}{-}0.2\phantom{0}$ & $\phantom{0}{-}2.5\phantom{0}$ & $\phantom{0}0.05\phantom{0}$ & $\phantom{0}0.06\phantom{0}$ & $\phantom{0}0.06\phantom{0}$ & $\phantom{0}0.06\phantom{0}$ & $\phantom{0}0.06\phantom{0}$ & $\phantom{0}94.6\phantom{0}$ & $\phantom{0}93.7\phantom{0}$ & $\phantom{0}93.8\phantom{0}$ & $\phantom{0}93.6\phantom{0}$ & $\phantom{0}93.9\phantom{0}$ \\
 & \nopagebreak $\;J=1000$  & $\phantom{0}{-}0.0\phantom{0}$ & $\phantom{0}{-}1.3\phantom{0}$ & $\phantom{0}{-}0.2\phantom{0}$ & $\phantom{0}\phantom{-}0.2\phantom{0}$ & $\phantom{0}{-}0.1\phantom{0}$ & $\phantom{0}0.02\phantom{0}$ & $\phantom{0}0.03\phantom{0}$ & $\phantom{0}0.03\phantom{0}$ & $\phantom{0}0.03\phantom{0}$ & $\phantom{0}0.03\phantom{0}$ & $\phantom{0}95.3\phantom{0}$ & $\phantom{0}94.3\phantom{0}$ & $\phantom{0}93.8\phantom{0}$ & $\phantom{0}94.6\phantom{0}$ & $\phantom{0}95.0\phantom{0}$ \\
[0.5ex]\hline\\[-1.6ex] 
\end{tabular}
\begin{tablenotes}[para,flushleft]{\footnotesize \textit{Note.} $\bar{n}$ = average cluster size; $J$ = number of clusters; CD = complete data sets; LD = listwise deletion; FCS-SL = single-level FCS; FCS-MAN = two-level FCS with manifest cluster means; FCS-LAT = two-level FCS with latent cluster means; JM = joint modeling.}\end{tablenotes}
\end{threeparttable}
\end{sidewaystable}
\begin{sidewaystable}
\begin{threeparttable}
\setlength{\tabcolsep}{1.0pt}
\renewcommand{\arraystretch}{0.95}
\footnotesize
\caption{\small Study 2: Bias (in \%), Relative RMSE, and Coverage of the 95\% Confidence Interval for the Regression Coefficient of $y$ on $z$ ($\hat\beta_{yz}$) With Moderately Unbalanced Data (Uniform, $\pm 40\%$) and 20\% Missing Data (MAR, $\lambda=0.5$)}
\begin{tabular}{llccccccccccccccc}
\hline\\[-1.8ex]
& & \multicolumn{5}{c}{Bias (\%)} & \multicolumn{5}{c}{Rel. RMSE} & \multicolumn{5}{c}{Coverage (\%)} \\ \cmidrule(r){3-7}\cmidrule(r){8-12}\cmidrule(r){13-17}
 &  & CD & \makecell{FCS-\\MAN} & \makecell{FCS-\\NJ} & \makecell{FCS-\\LAT} & JM & CD & \makecell{FCS-\\MAN} & \makecell{FCS-\\NJ} & \makecell{FCS-\\LAT} & JM & CD & \makecell{FCS-\\MAN} & \makecell{FCS-\\NJ} & \makecell{FCS-\\LAT} & \multicolumn{1}{c}{JM} \\ 
[0.4ex]\hline\\[-1.8ex]
& & \multicolumn{15}{c}{Small intraclass correlation $(\rho_{Iy}=.10)$} \\[0.6ex]\hline\\[-1.8ex]
\multicolumn{4}{l}{$\bar{n}=5$} \\  & \nopagebreak $\;J=50$  & $\phantom{-}0.6\phantom{0}$ & ${-}0.9\phantom{0}$ & ${-}3.4\phantom{0}$ & $\phantom{-}2.8\phantom{0}$ & ${-}7.3\phantom{0}$ & $\phantom{0}0.07\phantom{0}$ & $\phantom{0}0.09\phantom{0}$ & $\phantom{0}0.08\phantom{0}$ & $\phantom{0}0.09\phantom{0}$ & $\phantom{0}0.08\phantom{0}$ & $\phantom{0}91.9\phantom{0}$ & $\phantom{0}93.6\phantom{0}$ & $\phantom{0}94.3\phantom{0}$ & $\phantom{0}91.3\phantom{0}$ & $\phantom{0}95.7\phantom{0}$ \\
 & \nopagebreak $\;J=200$  & ${-}0.2\phantom{0}$ & ${-}1.0\phantom{0}$ & ${-}1.3\phantom{0}$ & $\phantom{-}0.9\phantom{0}$ & ${-}3.9\phantom{0}$ & $\phantom{0}0.04\phantom{0}$ & $\phantom{0}0.04\phantom{0}$ & $\phantom{0}0.04\phantom{0}$ & $\phantom{0}0.04\phantom{0}$ & $\phantom{0}0.04\phantom{0}$ & $\phantom{0}95.0\phantom{0}$ & $\phantom{0}94.8\phantom{0}$ & $\phantom{0}95.2\phantom{0}$ & $\phantom{0}93.8\phantom{0}$ & $\phantom{0}95.9\phantom{0}$ \\
 & \nopagebreak $\;J=1000$  & $\phantom{-}0.2\phantom{0}$ & ${-}0.8\phantom{0}$ & ${-}0.5\phantom{0}$ & $\phantom{-}0.0\phantom{0}$ & ${-}1.3\phantom{0}$ & $\phantom{0}0.02\phantom{0}$ & $\phantom{0}0.02\phantom{0}$ & $\phantom{0}0.02\phantom{0}$ & $\phantom{0}0.02\phantom{0}$ & $\phantom{0}0.02\phantom{0}$ & $\phantom{0}94.6\phantom{0}$ & $\phantom{0}94.3\phantom{0}$ & $\phantom{0}93.6\phantom{0}$ & $\phantom{0}94.2\phantom{0}$ & $\phantom{0}95.0\phantom{0}$ \\
\multicolumn{4}{l}{$\bar{n}=20$} \\  & \nopagebreak $\;J=50$  & ${-}0.1\phantom{0}$ & ${-}2.6\phantom{0}$ & ${-}5.0\phantom{0}$ & ${-}1.7\phantom{0}$ & ${-}7.8\phantom{0}$ & $\phantom{0}0.05\phantom{0}$ & $\phantom{0}0.06\phantom{0}$ & $\phantom{0}0.06\phantom{0}$ & $\phantom{0}0.06\phantom{0}$ & $\phantom{0}0.06\phantom{0}$ & $\phantom{0}91.6\phantom{0}$ & $\phantom{0}93.3\phantom{0}$ & $\phantom{0}92.8\phantom{0}$ & $\phantom{0}91.7\phantom{0}$ & $\phantom{0}94.0\phantom{0}$ \\
 & \nopagebreak $\;J=200$  & ${-}0.3\phantom{0}$ & ${-}1.4\phantom{0}$ & ${-}1.9\phantom{0}$ & ${-}1.2\phantom{0}$ & ${-}3.5\phantom{0}$ & $\phantom{0}0.03\phantom{0}$ & $\phantom{0}0.03\phantom{0}$ & $\phantom{0}0.03\phantom{0}$ & $\phantom{0}0.03\phantom{0}$ & $\phantom{0}0.03\phantom{0}$ & $\phantom{0}92.8\phantom{0}$ & $\phantom{0}94.6\phantom{0}$ & $\phantom{0}94.4\phantom{0}$ & $\phantom{0}93.9\phantom{0}$ & $\phantom{0}94.8\phantom{0}$ \\
 & \nopagebreak $\;J=1000$  & $\phantom{-}0.0\phantom{0}$ & ${-}0.3\phantom{0}$ & ${-}0.3\phantom{0}$ & ${-}0.1\phantom{0}$ & ${-}0.6\phantom{0}$ & $\phantom{0}0.01\phantom{0}$ & $\phantom{0}0.01\phantom{0}$ & $\phantom{0}0.01\phantom{0}$ & $\phantom{0}0.01\phantom{0}$ & $\phantom{0}0.01\phantom{0}$ & $\phantom{0}94.9\phantom{0}$ & $\phantom{0}94.9\phantom{0}$ & $\phantom{0}94.9\phantom{0}$ & $\phantom{0}94.4\phantom{0}$ & $\phantom{0}94.4\phantom{0}$ \\
[0.5ex]\hline\\[-1.6ex] 
& & \multicolumn{15}{c}{Moderate intraclass correlation $(\rho_{Iy}=.30)$} \\[0.6ex]\hline\\[-1.8ex]
\multicolumn{4}{l}{$\bar{n}=5$} \\  & \nopagebreak $\;J=50$  & $\phantom{-}1.9\phantom{0}$ & ${-}0.3\phantom{0}$ & ${-}2.7\phantom{0}$ & $\phantom{-}0.5\phantom{0}$ & ${-}2.6\phantom{0}$ & $\phantom{0}0.09\phantom{0}$ & $\phantom{0}0.10\phantom{0}$ & $\phantom{0}0.10\phantom{0}$ & $\phantom{0}0.10\phantom{0}$ & $\phantom{0}0.10\phantom{0}$ & $\phantom{0}92.3\phantom{0}$ & $\phantom{0}94.0\phantom{0}$ & $\phantom{0}94.2\phantom{0}$ & $\phantom{0}93.1\phantom{0}$ & $\phantom{0}94.5\phantom{0}$ \\
 & \nopagebreak $\;J=200$  & ${-}0.4\phantom{0}$ & ${-}1.2\phantom{0}$ & ${-}1.6\phantom{0}$ & ${-}1.0\phantom{0}$ & ${-}1.8\phantom{0}$ & $\phantom{0}0.04\phantom{0}$ & $\phantom{0}0.05\phantom{0}$ & $\phantom{0}0.05\phantom{0}$ & $\phantom{0}0.05\phantom{0}$ & $\phantom{0}0.05\phantom{0}$ & $\phantom{0}93.8\phantom{0}$ & $\phantom{0}94.4\phantom{0}$ & $\phantom{0}93.9\phantom{0}$ & $\phantom{0}94.4\phantom{0}$ & $\phantom{0}94.8\phantom{0}$ \\
 & \nopagebreak $\;J=1000$  & $\phantom{-}0.0\phantom{0}$ & ${-}0.4\phantom{0}$ & ${-}0.3\phantom{0}$ & ${-}0.2\phantom{0}$ & ${-}0.2\phantom{0}$ & $\phantom{0}0.02\phantom{0}$ & $\phantom{0}0.02\phantom{0}$ & $\phantom{0}0.02\phantom{0}$ & $\phantom{0}0.02\phantom{0}$ & $\phantom{0}0.02\phantom{0}$ & $\phantom{0}95.2\phantom{0}$ & $\phantom{0}94.3\phantom{0}$ & $\phantom{0}94.0\phantom{0}$ & $\phantom{0}94.3\phantom{0}$ & $\phantom{0}93.4\phantom{0}$ \\
\multicolumn{4}{l}{$\bar{n}=20$} \\  & \nopagebreak $\;J=50$  & $\phantom{-}1.5\phantom{0}$ & ${-}1.0\phantom{0}$ & ${-}3.9\phantom{0}$ & ${-}1.0\phantom{0}$ & ${-}3.2\phantom{0}$ & $\phantom{0}0.07\phantom{0}$ & $\phantom{0}0.08\phantom{0}$ & $\phantom{0}0.08\phantom{0}$ & $\phantom{0}0.08\phantom{0}$ & $\phantom{0}0.08\phantom{0}$ & $\phantom{0}92.7\phantom{0}$ & $\phantom{0}93.4\phantom{0}$ & $\phantom{0}93.6\phantom{0}$ & $\phantom{0}93.5\phantom{0}$ & $\phantom{0}93.5\phantom{0}$ \\
 & \nopagebreak $\;J=200$  & $\phantom{-}0.4\phantom{0}$ & ${-}0.2\phantom{0}$ & ${-}0.8\phantom{0}$ & ${-}0.2\phantom{0}$ & ${-}0.8\phantom{0}$ & $\phantom{0}0.04\phantom{0}$ & $\phantom{0}0.04\phantom{0}$ & $\phantom{0}0.04\phantom{0}$ & $\phantom{0}0.04\phantom{0}$ & $\phantom{0}0.04\phantom{0}$ & $\phantom{0}94.6\phantom{0}$ & $\phantom{0}95.2\phantom{0}$ & $\phantom{0}94.5\phantom{0}$ & $\phantom{0}95.2\phantom{0}$ & $\phantom{0}95.2\phantom{0}$ \\
 & \nopagebreak $\;J=1000$  & ${-}0.1\phantom{0}$ & ${-}0.0\phantom{0}$ & ${-}0.2\phantom{0}$ & ${-}0.1\phantom{0}$ & ${-}0.2\phantom{0}$ & $\phantom{0}0.02\phantom{0}$ & $\phantom{0}0.02\phantom{0}$ & $\phantom{0}0.02\phantom{0}$ & $\phantom{0}0.02\phantom{0}$ & $\phantom{0}0.02\phantom{0}$ & $\phantom{0}94.6\phantom{0}$ & $\phantom{0}95.1\phantom{0}$ & $\phantom{0}95.1\phantom{0}$ & $\phantom{0}94.9\phantom{0}$ & $\phantom{0}95.1\phantom{0}$ \\
[0.5ex]\hline\\[-1.6ex] 
\end{tabular}
\begin{tablenotes}[para,flushleft]{\footnotesize \textit{Note.} $\bar{n}$ = average cluster size; $J$ = number of clusters; CD = complete data sets; LD = listwise deletion; FCS-SL = single-level FCS; FCS-MAN = two-level FCS with manifest cluster means; FCS-LAT = two-level FCS with latent cluster means; JM = joint modeling.}\end{tablenotes}
\end{threeparttable}
\end{sidewaystable}
\begin{sidewaystable}
\begin{threeparttable}
\setlength{\tabcolsep}{1.0pt}
\renewcommand{\arraystretch}{0.95}
\footnotesize
\caption{\small Study 2: Bias (in \%), Relative RMSE, and Coverage of the 95\% Confidence Interval for the Regression Coefficient of $y$ on $z$ ($\hat\beta_{yz}$) With Strongly Unbalanced Data (Uniform, $\pm 80\%$) and 20\% Missing Data (MAR, $\lambda=0.5$)}
\begin{tabular}{llccccccccccccccc}
\hline\\[-1.8ex]
& & \multicolumn{5}{c}{Bias (\%)} & \multicolumn{5}{c}{Rel. RMSE} & \multicolumn{5}{c}{Coverage (\%)} \\ \cmidrule(r){3-7}\cmidrule(r){8-12}\cmidrule(r){13-17}
 &  & CD & \makecell{FCS-\\MAN} & \makecell{FCS-\\NJ} & \makecell{FCS-\\LAT} & JM & CD & \makecell{FCS-\\MAN} & \makecell{FCS-\\NJ} & \makecell{FCS-\\LAT} & JM & CD & \makecell{FCS-\\MAN} & \makecell{FCS-\\NJ} & \makecell{FCS-\\LAT} & \multicolumn{1}{c}{JM} \\ 
[0.4ex]\hline\\[-1.8ex]
& & \multicolumn{15}{c}{Small intraclass correlation $(\rho_{Iy}=.10)$} \\[0.6ex]\hline\\[-1.8ex]
\multicolumn{4}{l}{$\bar{n}=5$} \\  & \nopagebreak $\;J=50$  & ${-}1.7\phantom{0}$ & ${-}6.0\phantom{0}$ & ${-}5.7\phantom{0}$ & ${-}0.5\phantom{0}$ & ${-}9.6\phantom{0}$ & $\phantom{0}0.08\phantom{0}$ & $\phantom{0}0.09\phantom{0}$ & $\phantom{0}0.09\phantom{0}$ & $\phantom{0}0.09\phantom{0}$ & $\phantom{0}0.08\phantom{0}$ & $\phantom{0}92.1\phantom{0}$ & $\phantom{0}93.8\phantom{0}$ & $\phantom{0}93.8\phantom{0}$ & $\phantom{0}91.9\phantom{0}$ & $\phantom{0}94.5\phantom{0}$ \\
 & \nopagebreak $\;J=200$  & ${-}1.6\phantom{0}$ & ${-}4.7\phantom{0}$ & ${-}2.6\phantom{0}$ & ${-}0.6\phantom{0}$ & ${-}5.1\phantom{0}$ & $\phantom{0}0.04\phantom{0}$ & $\phantom{0}0.04\phantom{0}$ & $\phantom{0}0.04\phantom{0}$ & $\phantom{0}0.04\phantom{0}$ & $\phantom{0}0.04\phantom{0}$ & $\phantom{0}94.0\phantom{0}$ & $\phantom{0}95.1\phantom{0}$ & $\phantom{0}95.1\phantom{0}$ & $\phantom{0}92.6\phantom{0}$ & $\phantom{0}94.7\phantom{0}$ \\
 & \nopagebreak $\;J=1000$  & $\phantom{-}0.2\phantom{0}$ & ${-}3.3\phantom{0}$ & ${-}0.3\phantom{0}$ & $\phantom{-}0.2\phantom{0}$ & ${-}1.0\phantom{0}$ & $\phantom{0}0.02\phantom{0}$ & $\phantom{0}0.02\phantom{0}$ & $\phantom{0}0.02\phantom{0}$ & $\phantom{0}0.02\phantom{0}$ & $\phantom{0}0.02\phantom{0}$ & $\phantom{0}94.5\phantom{0}$ & $\phantom{0}94.3\phantom{0}$ & $\phantom{0}94.7\phantom{0}$ & $\phantom{0}94.1\phantom{0}$ & $\phantom{0}94.9\phantom{0}$ \\
\multicolumn{4}{l}{$\bar{n}=20$} \\  & \nopagebreak $\;J=50$  & $\phantom{-}1.1\phantom{0}$ & ${-}0.8\phantom{0}$ & ${-}3.2\phantom{0}$ & $\phantom{-}0.5\phantom{0}$ & ${-}5.9\phantom{0}$ & $\phantom{0}0.05\phantom{0}$ & $\phantom{0}0.06\phantom{0}$ & $\phantom{0}0.06\phantom{0}$ & $\phantom{0}0.06\phantom{0}$ & $\phantom{0}0.06\phantom{0}$ & $\phantom{0}92.5\phantom{0}$ & $\phantom{0}93.0\phantom{0}$ & $\phantom{0}93.5\phantom{0}$ & $\phantom{0}92.2\phantom{0}$ & $\phantom{0}94.7\phantom{0}$ \\
 & \nopagebreak $\;J=200$  & $\phantom{-}0.5\phantom{0}$ & ${-}1.0\phantom{0}$ & ${-}0.8\phantom{0}$ & $\phantom{-}0.0\phantom{0}$ & ${-}2.1\phantom{0}$ & $\phantom{0}0.03\phantom{0}$ & $\phantom{0}0.03\phantom{0}$ & $\phantom{0}0.03\phantom{0}$ & $\phantom{0}0.03\phantom{0}$ & $\phantom{0}0.03\phantom{0}$ & $\phantom{0}94.4\phantom{0}$ & $\phantom{0}94.9\phantom{0}$ & $\phantom{0}94.3\phantom{0}$ & $\phantom{0}93.8\phantom{0}$ & $\phantom{0}94.8\phantom{0}$ \\
 & \nopagebreak $\;J=1000$  & $\phantom{-}0.3\phantom{0}$ & ${-}0.9\phantom{0}$ & ${-}0.3\phantom{0}$ & ${-}0.0\phantom{0}$ & ${-}0.6\phantom{0}$ & $\phantom{0}0.01\phantom{0}$ & $\phantom{0}0.01\phantom{0}$ & $\phantom{0}0.01\phantom{0}$ & $\phantom{0}0.01\phantom{0}$ & $\phantom{0}0.01\phantom{0}$ & $\phantom{0}94.0\phantom{0}$ & $\phantom{0}94.6\phantom{0}$ & $\phantom{0}94.6\phantom{0}$ & $\phantom{0}93.8\phantom{0}$ & $\phantom{0}94.9\phantom{0}$ \\
[0.5ex]\hline\\[-1.6ex] 
& & \multicolumn{15}{c}{Moderate intraclass correlation $(\rho_{Iy}=.30)$} \\[0.6ex]\hline\\[-1.8ex]
\multicolumn{4}{l}{$\bar{n}=5$} \\  & \nopagebreak $\;J=50$  & $\phantom{-}0.2\phantom{0}$ & ${-}4.1\phantom{0}$ & ${-}5.8\phantom{0}$ & ${-}2.9\phantom{0}$ & ${-}6.3\phantom{0}$ & $\phantom{0}0.09\phantom{0}$ & $\phantom{0}0.11\phantom{0}$ & $\phantom{0}0.11\phantom{0}$ & $\phantom{0}0.11\phantom{0}$ & $\phantom{0}0.11\phantom{0}$ & $\phantom{0}92.5\phantom{0}$ & $\phantom{0}93.0\phantom{0}$ & $\phantom{0}93.0\phantom{0}$ & $\phantom{0}92.7\phantom{0}$ & $\phantom{0}93.0\phantom{0}$ \\
 & \nopagebreak $\;J=200$  & $\phantom{-}0.2\phantom{0}$ & ${-}1.7\phantom{0}$ & ${-}1.3\phantom{0}$ & ${-}0.5\phantom{0}$ & ${-}1.2\phantom{0}$ & $\phantom{0}0.05\phantom{0}$ & $\phantom{0}0.05\phantom{0}$ & $\phantom{0}0.05\phantom{0}$ & $\phantom{0}0.05\phantom{0}$ & $\phantom{0}0.05\phantom{0}$ & $\phantom{0}93.4\phantom{0}$ & $\phantom{0}93.8\phantom{0}$ & $\phantom{0}93.9\phantom{0}$ & $\phantom{0}92.9\phantom{0}$ & $\phantom{0}93.9\phantom{0}$ \\
 & \nopagebreak $\;J=1000$  & $\phantom{-}0.2\phantom{0}$ & ${-}1.0\phantom{0}$ & ${-}0.1\phantom{0}$ & $\phantom{-}0.3\phantom{0}$ & $\phantom{-}0.1\phantom{0}$ & $\phantom{0}0.02\phantom{0}$ & $\phantom{0}0.02\phantom{0}$ & $\phantom{0}0.02\phantom{0}$ & $\phantom{0}0.02\phantom{0}$ & $\phantom{0}0.02\phantom{0}$ & $\phantom{0}95.5\phantom{0}$ & $\phantom{0}95.1\phantom{0}$ & $\phantom{0}94.5\phantom{0}$ & $\phantom{0}94.6\phantom{0}$ & $\phantom{0}94.9\phantom{0}$ \\
\multicolumn{4}{l}{$\bar{n}=20$} \\  & \nopagebreak $\;J=50$  & $\phantom{-}0.9\phantom{0}$ & ${-}0.6\phantom{0}$ & ${-}2.8\phantom{0}$ & ${-}0.0\phantom{0}$ & ${-}2.2\phantom{0}$ & $\phantom{0}0.08\phantom{0}$ & $\phantom{0}0.08\phantom{0}$ & $\phantom{0}0.08\phantom{0}$ & $\phantom{0}0.08\phantom{0}$ & $\phantom{0}0.08\phantom{0}$ & $\phantom{0}92.6\phantom{0}$ & $\phantom{0}93.3\phantom{0}$ & $\phantom{0}93.8\phantom{0}$ & $\phantom{0}92.9\phantom{0}$ & $\phantom{0}93.9\phantom{0}$ \\
 & \nopagebreak $\;J=200$  & ${-}0.3\phantom{0}$ & ${-}1.2\phantom{0}$ & ${-}1.4\phantom{0}$ & ${-}0.9\phantom{0}$ & ${-}1.3\phantom{0}$ & $\phantom{0}0.04\phantom{0}$ & $\phantom{0}0.04\phantom{0}$ & $\phantom{0}0.04\phantom{0}$ & $\phantom{0}0.04\phantom{0}$ & $\phantom{0}0.04\phantom{0}$ & $\phantom{0}94.2\phantom{0}$ & $\phantom{0}93.9\phantom{0}$ & $\phantom{0}94.2\phantom{0}$ & $\phantom{0}94.4\phantom{0}$ & $\phantom{0}94.4\phantom{0}$ \\
 & \nopagebreak $\;J=1000$  & ${-}0.0\phantom{0}$ & ${-}0.3\phantom{0}$ & ${-}0.4\phantom{0}$ & ${-}0.2\phantom{0}$ & ${-}0.4\phantom{0}$ & $\phantom{0}0.02\phantom{0}$ & $\phantom{0}0.02\phantom{0}$ & $\phantom{0}0.02\phantom{0}$ & $\phantom{0}0.02\phantom{0}$ & $\phantom{0}0.02\phantom{0}$ & $\phantom{0}95.3\phantom{0}$ & $\phantom{0}95.3\phantom{0}$ & $\phantom{0}95.3\phantom{0}$ & $\phantom{0}95.3\phantom{0}$ & $\phantom{0}94.5\phantom{0}$ \\
[0.5ex]\hline\\[-1.6ex] 
\end{tabular}
\begin{tablenotes}[para,flushleft]{\footnotesize \textit{Note.} $\bar{n}$ = average cluster size; $J$ = number of clusters; CD = complete data sets; LD = listwise deletion; FCS-SL = single-level FCS; FCS-MAN = two-level FCS with manifest cluster means; FCS-LAT = two-level FCS with latent cluster means; JM = joint modeling.}\end{tablenotes}
\end{threeparttable}
\end{sidewaystable}
\begin{sidewaystable}
\begin{threeparttable}
\setlength{\tabcolsep}{1.0pt}
\renewcommand{\arraystretch}{0.95}
\footnotesize
\caption{\small Study 2: Bias (in \%), Relative RMSE, and Coverage of the 95\% Confidence Interval for the Regression Coefficient of $y$ on $z$ ($\hat\beta_{yz}$) With Moderately Unbalanced Data (Bimodal, $\pm 40\%$) and 20\% Missing Data (MAR, $\lambda=0.5$)}
\begin{tabular}{llccccccccccccccc}
\hline\\[-1.8ex]
& & \multicolumn{5}{c}{Bias (\%)} & \multicolumn{5}{c}{Rel. RMSE} & \multicolumn{5}{c}{Coverage (\%)} \\ \cmidrule(r){3-7}\cmidrule(r){8-12}\cmidrule(r){13-17}
 &  & CD & \makecell{FCS-\\MAN} & \makecell{FCS-\\NJ} & \makecell{FCS-\\LAT} & JM & CD & \makecell{FCS-\\MAN} & \makecell{FCS-\\NJ} & \makecell{FCS-\\LAT} & JM & CD & \makecell{FCS-\\MAN} & \makecell{FCS-\\NJ} & \makecell{FCS-\\LAT} & \multicolumn{1}{c}{JM} \\ 
[0.4ex]\hline\\[-1.8ex]
& & \multicolumn{15}{c}{Small intraclass correlation $(\rho_{Iy}=.10)$} \\[0.6ex]\hline\\[-1.8ex]
\multicolumn{4}{l}{$\bar{n}=5$} \\  & \nopagebreak $\;J=50$  & $\phantom{0}{-}2.2\phantom{0}$ & $\phantom{0}{-}5.1\phantom{0}$ & $\phantom{0}{-}6.0\phantom{0}$ & $\phantom{0}{-}1.3\phantom{0}$ & ${-}10.1\phantom{0}$ & $\phantom{0}0.08\phantom{0}$ & $\phantom{0}0.09\phantom{0}$ & $\phantom{0}0.09\phantom{0}$ & $\phantom{0}0.09\phantom{0}$ & $\phantom{0}0.08\phantom{0}$ & $\phantom{0}91.3\phantom{0}$ & $\phantom{0}91.8\phantom{0}$ & $\phantom{0}91.5\phantom{0}$ & $\phantom{0}90.4\phantom{0}$ & $\phantom{0}93.5\phantom{0}$ \\
 & \nopagebreak $\;J=200$  & $\phantom{0}\phantom{-}0.6\phantom{0}$ & $\phantom{0}{-}1.0\phantom{0}$ & $\phantom{0}{-}0.6\phantom{0}$ & $\phantom{0}\phantom{-}1.2\phantom{0}$ & $\phantom{0}{-}3.2\phantom{0}$ & $\phantom{0}0.04\phantom{0}$ & $\phantom{0}0.04\phantom{0}$ & $\phantom{0}0.04\phantom{0}$ & $\phantom{0}0.04\phantom{0}$ & $\phantom{0}0.04\phantom{0}$ & $\phantom{0}93.7\phantom{0}$ & $\phantom{0}94.0\phantom{0}$ & $\phantom{0}94.0\phantom{0}$ & $\phantom{0}93.7\phantom{0}$ & $\phantom{0}95.4\phantom{0}$ \\
 & \nopagebreak $\;J=1000$  & $\phantom{0}\phantom{-}0.4\phantom{0}$ & $\phantom{0}{-}1.0\phantom{0}$ & $\phantom{0}\phantom{-}0.1\phantom{0}$ & $\phantom{0}\phantom{-}0.4\phantom{0}$ & $\phantom{0}{-}0.8\phantom{0}$ & $\phantom{0}0.02\phantom{0}$ & $\phantom{0}0.02\phantom{0}$ & $\phantom{0}0.02\phantom{0}$ & $\phantom{0}0.02\phantom{0}$ & $\phantom{0}0.02\phantom{0}$ & $\phantom{0}95.8\phantom{0}$ & $\phantom{0}94.7\phantom{0}$ & $\phantom{0}94.5\phantom{0}$ & $\phantom{0}94.7\phantom{0}$ & $\phantom{0}94.9\phantom{0}$ \\
\multicolumn{4}{l}{$\bar{n}=20$} \\  & \nopagebreak $\;J=50$  & $\phantom{0}\phantom{-}1.6\phantom{0}$ & $\phantom{0}{-}1.3\phantom{0}$ & $\phantom{0}{-}3.8\phantom{0}$ & $\phantom{0}{-}0.6\phantom{0}$ & $\phantom{0}{-}7.0\phantom{0}$ & $\phantom{0}0.05\phantom{0}$ & $\phantom{0}0.06\phantom{0}$ & $\phantom{0}0.06\phantom{0}$ & $\phantom{0}0.06\phantom{0}$ & $\phantom{0}0.06\phantom{0}$ & $\phantom{0}93.7\phantom{0}$ & $\phantom{0}93.6\phantom{0}$ & $\phantom{0}93.8\phantom{0}$ & $\phantom{0}92.5\phantom{0}$ & $\phantom{0}95.2\phantom{0}$ \\
 & \nopagebreak $\;J=200$  & $\phantom{0}{-}0.1\phantom{0}$ & $\phantom{0}{-}1.2\phantom{0}$ & $\phantom{0}{-}1.5\phantom{0}$ & $\phantom{0}{-}0.9\phantom{0}$ & $\phantom{0}{-}2.9\phantom{0}$ & $\phantom{0}0.02\phantom{0}$ & $\phantom{0}0.03\phantom{0}$ & $\phantom{0}0.03\phantom{0}$ & $\phantom{0}0.03\phantom{0}$ & $\phantom{0}0.03\phantom{0}$ & $\phantom{0}95.9\phantom{0}$ & $\phantom{0}95.4\phantom{0}$ & $\phantom{0}96.3\phantom{0}$ & $\phantom{0}95.6\phantom{0}$ & $\phantom{0}95.6\phantom{0}$ \\
 & \nopagebreak $\;J=1000$  & $\phantom{0}{-}0.1\phantom{0}$ & $\phantom{0}{-}0.3\phantom{0}$ & $\phantom{0}{-}0.2\phantom{0}$ & $\phantom{0}{-}0.0\phantom{0}$ & $\phantom{0}{-}0.5\phantom{0}$ & $\phantom{0}0.01\phantom{0}$ & $\phantom{0}0.01\phantom{0}$ & $\phantom{0}0.01\phantom{0}$ & $\phantom{0}0.01\phantom{0}$ & $\phantom{0}0.01\phantom{0}$ & $\phantom{0}94.0\phantom{0}$ & $\phantom{0}94.6\phantom{0}$ & $\phantom{0}93.5\phantom{0}$ & $\phantom{0}94.1\phantom{0}$ & $\phantom{0}94.6\phantom{0}$ \\
[0.5ex]\hline\\[-1.6ex] 
& & \multicolumn{15}{c}{Moderate intraclass correlation $(\rho_{Iy}=.30)$} \\[0.6ex]\hline\\[-1.8ex]
\multicolumn{4}{l}{$\bar{n}=5$} \\  & \nopagebreak $\;J=50$  & $\phantom{-}1.0\phantom{0}$ & ${-}1.5\phantom{0}$ & ${-}3.8\phantom{0}$ & ${-}0.6\phantom{0}$ & ${-}3.7\phantom{0}$ & $\phantom{0}0.09\phantom{0}$ & $\phantom{0}0.10\phantom{0}$ & $\phantom{0}0.10\phantom{0}$ & $\phantom{0}0.10\phantom{0}$ & $\phantom{0}0.10\phantom{0}$ & $\phantom{0}92.7\phantom{0}$ & $\phantom{0}94.0\phantom{0}$ & $\phantom{0}94.3\phantom{0}$ & $\phantom{0}93.5\phantom{0}$ & $\phantom{0}94.1\phantom{0}$ \\
 & \nopagebreak $\;J=200$  & $\phantom{-}1.0\phantom{0}$ & ${-}0.1\phantom{0}$ & ${-}0.7\phantom{0}$ & $\phantom{-}0.3\phantom{0}$ & ${-}0.6\phantom{0}$ & $\phantom{0}0.04\phantom{0}$ & $\phantom{0}0.05\phantom{0}$ & $\phantom{0}0.05\phantom{0}$ & $\phantom{0}0.05\phantom{0}$ & $\phantom{0}0.05\phantom{0}$ & $\phantom{0}93.8\phantom{0}$ & $\phantom{0}94.6\phantom{0}$ & $\phantom{0}95.1\phantom{0}$ & $\phantom{0}94.5\phantom{0}$ & $\phantom{0}94.5\phantom{0}$ \\
 & \nopagebreak $\;J=1000$  & $\phantom{-}0.1\phantom{0}$ & ${-}0.5\phantom{0}$ & ${-}0.4\phantom{0}$ & ${-}0.1\phantom{0}$ & ${-}0.2\phantom{0}$ & $\phantom{0}0.02\phantom{0}$ & $\phantom{0}0.02\phantom{0}$ & $\phantom{0}0.02\phantom{0}$ & $\phantom{0}0.02\phantom{0}$ & $\phantom{0}0.02\phantom{0}$ & $\phantom{0}93.9\phantom{0}$ & $\phantom{0}94.6\phantom{0}$ & $\phantom{0}94.5\phantom{0}$ & $\phantom{0}94.3\phantom{0}$ & $\phantom{0}94.3\phantom{0}$ \\
\multicolumn{4}{l}{$\bar{n}=20$} \\  & \nopagebreak $\;J=50$  & $\phantom{-}0.1\phantom{0}$ & ${-}2.4\phantom{0}$ & ${-}5.0\phantom{0}$ & ${-}2.2\phantom{0}$ & ${-}4.1\phantom{0}$ & $\phantom{0}0.08\phantom{0}$ & $\phantom{0}0.08\phantom{0}$ & $\phantom{0}0.08\phantom{0}$ & $\phantom{0}0.08\phantom{0}$ & $\phantom{0}0.08\phantom{0}$ & $\phantom{0}92.8\phantom{0}$ & $\phantom{0}94.2\phantom{0}$ & $\phantom{0}94.5\phantom{0}$ & $\phantom{0}94.8\phantom{0}$ & $\phantom{0}95.2\phantom{0}$ \\
 & \nopagebreak $\;J=200$  & $\phantom{-}0.2\phantom{0}$ & ${-}0.1\phantom{0}$ & ${-}0.8\phantom{0}$ & ${-}0.2\phantom{0}$ & ${-}0.6\phantom{0}$ & $\phantom{0}0.04\phantom{0}$ & $\phantom{0}0.04\phantom{0}$ & $\phantom{0}0.04\phantom{0}$ & $\phantom{0}0.04\phantom{0}$ & $\phantom{0}0.04\phantom{0}$ & $\phantom{0}93.5\phantom{0}$ & $\phantom{0}95.2\phantom{0}$ & $\phantom{0}94.9\phantom{0}$ & $\phantom{0}94.7\phantom{0}$ & $\phantom{0}94.9\phantom{0}$ \\
 & \nopagebreak $\;J=1000$  & $\phantom{-}0.3\phantom{0}$ & $\phantom{-}0.1\phantom{0}$ & $\phantom{-}0.1\phantom{0}$ & $\phantom{-}0.2\phantom{0}$ & $\phantom{-}0.1\phantom{0}$ & $\phantom{0}0.02\phantom{0}$ & $\phantom{0}0.02\phantom{0}$ & $\phantom{0}0.02\phantom{0}$ & $\phantom{0}0.02\phantom{0}$ & $\phantom{0}0.02\phantom{0}$ & $\phantom{0}94.4\phantom{0}$ & $\phantom{0}94.5\phantom{0}$ & $\phantom{0}93.7\phantom{0}$ & $\phantom{0}94.3\phantom{0}$ & $\phantom{0}93.8\phantom{0}$ \\
[0.5ex]\hline\\[-1.6ex] 
\end{tabular}
\begin{tablenotes}[para,flushleft]{\footnotesize \textit{Note.} $\bar{n}$ = average cluster size; $J$ = number of clusters; CD = complete data sets; LD = listwise deletion; FCS-SL = single-level FCS; FCS-MAN = two-level FCS with manifest cluster means; FCS-LAT = two-level FCS with latent cluster means; JM = joint modeling.}\end{tablenotes}
\end{threeparttable}
\end{sidewaystable}
\begin{sidewaystable}
\begin{threeparttable}
\setlength{\tabcolsep}{1.0pt}
\renewcommand{\arraystretch}{0.95}
\footnotesize
\caption{\small Study 2: Bias (in \%), Relative RMSE, and Coverage of the 95\% Confidence Interval for the Regression Coefficient of $y$ on $z$ ($\hat\beta_{yz}$) With Strongly Unbalanced Data (Bimodal, $\pm 80\%$) and 20\% Missing Data (MAR, $\lambda=0.5$)}
\begin{tabular}{llccccccccccccccc}
\hline\\[-1.8ex]
& & \multicolumn{5}{c}{Bias (\%)} & \multicolumn{5}{c}{Rel. RMSE} & \multicolumn{5}{c}{Coverage (\%)} \\ \cmidrule(r){3-7}\cmidrule(r){8-12}\cmidrule(r){13-17}
 &  & CD & \makecell{FCS-\\MAN} & \makecell{FCS-\\NJ} & \makecell{FCS-\\LAT} & JM & CD & \makecell{FCS-\\MAN} & \makecell{FCS-\\NJ} & \makecell{FCS-\\LAT} & JM & CD & \makecell{FCS-\\MAN} & \makecell{FCS-\\NJ} & \makecell{FCS-\\LAT} & \multicolumn{1}{c}{JM} \\ 
[0.4ex]\hline\\[-1.8ex]
& & \multicolumn{15}{c}{Small intraclass correlation $(\rho_{Iy}=.10)$} \\[0.6ex]\hline\\[-1.8ex]
\multicolumn{4}{l}{$\bar{n}=5$} \\  & \nopagebreak $\;J=50$  & ${-}0.2\phantom{0}$ & ${-}7.7\phantom{0}$ & ${-}4.7\phantom{0}$ & ${-}0.7\phantom{0}$ & ${-}8.3\phantom{0}$ & $\phantom{0}0.08\phantom{0}$ & $\phantom{0}0.09\phantom{0}$ & $\phantom{0}0.09\phantom{0}$ & $\phantom{0}0.10\phantom{0}$ & $\phantom{0}0.09\phantom{0}$ & $\phantom{0}90.7\phantom{0}$ & $\phantom{0}93.0\phantom{0}$ & $\phantom{0}91.5\phantom{0}$ & $\phantom{0}89.3\phantom{0}$ & $\phantom{0}93.1\phantom{0}$ \\
 & \nopagebreak $\;J=200$  & $\phantom{-}0.4\phantom{0}$ & ${-}6.8\phantom{0}$ & ${-}0.9\phantom{0}$ & $\phantom{-}1.0\phantom{0}$ & ${-}3.0\phantom{0}$ & $\phantom{0}0.04\phantom{0}$ & $\phantom{0}0.04\phantom{0}$ & $\phantom{0}0.04\phantom{0}$ & $\phantom{0}0.04\phantom{0}$ & $\phantom{0}0.04\phantom{0}$ & $\phantom{0}93.4\phantom{0}$ & $\phantom{0}95.2\phantom{0}$ & $\phantom{0}95.2\phantom{0}$ & $\phantom{0}93.5\phantom{0}$ & $\phantom{0}95.4\phantom{0}$ \\
 & \nopagebreak $\;J=1000$  & ${-}0.5\phantom{0}$ & ${-}7.9\phantom{0}$ & ${-}1.4\phantom{0}$ & ${-}0.7\phantom{0}$ & ${-}1.5\phantom{0}$ & $\phantom{0}0.02\phantom{0}$ & $\phantom{0}0.02\phantom{0}$ & $\phantom{0}0.02\phantom{0}$ & $\phantom{0}0.02\phantom{0}$ & $\phantom{0}0.02\phantom{0}$ & $\phantom{0}94.5\phantom{0}$ & $\phantom{0}92.8\phantom{0}$ & $\phantom{0}95.2\phantom{0}$ & $\phantom{0}94.3\phantom{0}$ & $\phantom{0}95.4\phantom{0}$ \\
\multicolumn{4}{l}{$\bar{n}=20$} \\  & \nopagebreak $\;J=50$  & $\phantom{-}0.3\phantom{0}$ & ${-}4.8\phantom{0}$ & ${-}4.5\phantom{0}$ & ${-}1.2\phantom{0}$ & ${-}7.7\phantom{0}$ & $\phantom{0}0.06\phantom{0}$ & $\phantom{0}0.06\phantom{0}$ & $\phantom{0}0.06\phantom{0}$ & $\phantom{0}0.07\phantom{0}$ & $\phantom{0}0.06\phantom{0}$ & $\phantom{0}92.6\phantom{0}$ & $\phantom{0}94.1\phantom{0}$ & $\phantom{0}94.1\phantom{0}$ & $\phantom{0}92.9\phantom{0}$ & $\phantom{0}94.5\phantom{0}$ \\
 & \nopagebreak $\;J=200$  & $\phantom{-}0.5\phantom{0}$ & ${-}3.2\phantom{0}$ & ${-}1.1\phantom{0}$ & ${-}0.2\phantom{0}$ & ${-}2.3\phantom{0}$ & $\phantom{0}0.03\phantom{0}$ & $\phantom{0}0.03\phantom{0}$ & $\phantom{0}0.03\phantom{0}$ & $\phantom{0}0.03\phantom{0}$ & $\phantom{0}0.03\phantom{0}$ & $\phantom{0}94.0\phantom{0}$ & $\phantom{0}94.5\phantom{0}$ & $\phantom{0}93.8\phantom{0}$ & $\phantom{0}94.8\phantom{0}$ & $\phantom{0}95.7\phantom{0}$ \\
 & \nopagebreak $\;J=1000$  & ${-}0.0\phantom{0}$ & ${-}3.2\phantom{0}$ & ${-}0.6\phantom{0}$ & $\phantom{-}0.1\phantom{0}$ & ${-}0.6\phantom{0}$ & $\phantom{0}0.01\phantom{0}$ & $\phantom{0}0.02\phantom{0}$ & $\phantom{0}0.01\phantom{0}$ & $\phantom{0}0.01\phantom{0}$ & $\phantom{0}0.01\phantom{0}$ & $\phantom{0}92.9\phantom{0}$ & $\phantom{0}93.8\phantom{0}$ & $\phantom{0}94.9\phantom{0}$ & $\phantom{0}94.3\phantom{0}$ & $\phantom{0}94.8\phantom{0}$ \\
[0.5ex]\hline\\[-1.6ex] 
& & \multicolumn{15}{c}{Moderate intraclass correlation $(\rho_{Iy}=.30)$} \\[0.6ex]\hline\\[-1.8ex]
\multicolumn{4}{l}{$\bar{n}=5$} \\  & \nopagebreak $\;J=50$  & ${-}2.0\phantom{0}$ & ${-}6.8\phantom{0}$ & ${-}7.1\phantom{0}$ & ${-}2.6\phantom{0}$ & ${-}5.9\phantom{0}$ & $\phantom{0}0.10\phantom{0}$ & $\phantom{0}0.11\phantom{0}$ & $\phantom{0}0.11\phantom{0}$ & $\phantom{0}0.11\phantom{0}$ & $\phantom{0}0.10\phantom{0}$ & $\phantom{0}92.9\phantom{0}$ & $\phantom{0}94.1\phantom{0}$ & $\phantom{0}94.4\phantom{0}$ & $\phantom{0}93.4\phantom{0}$ & $\phantom{0}94.3\phantom{0}$ \\
 & \nopagebreak $\;J=200$  & $\phantom{-}0.3\phantom{0}$ & ${-}3.3\phantom{0}$ & ${-}1.3\phantom{0}$ & $\phantom{-}0.0\phantom{0}$ & ${-}1.0\phantom{0}$ & $\phantom{0}0.05\phantom{0}$ & $\phantom{0}0.05\phantom{0}$ & $\phantom{0}0.05\phantom{0}$ & $\phantom{0}0.05\phantom{0}$ & $\phantom{0}0.05\phantom{0}$ & $\phantom{0}95.5\phantom{0}$ & $\phantom{0}95.1\phantom{0}$ & $\phantom{0}95.3\phantom{0}$ & $\phantom{0}95.2\phantom{0}$ & $\phantom{0}95.6\phantom{0}$ \\
 & \nopagebreak $\;J=1000$  & $\phantom{-}0.5\phantom{0}$ & ${-}2.9\phantom{0}$ & ${-}0.3\phantom{0}$ & $\phantom{-}0.3\phantom{0}$ & $\phantom{-}0.1\phantom{0}$ & $\phantom{0}0.02\phantom{0}$ & $\phantom{0}0.02\phantom{0}$ & $\phantom{0}0.02\phantom{0}$ & $\phantom{0}0.02\phantom{0}$ & $\phantom{0}0.02\phantom{0}$ & $\phantom{0}96.1\phantom{0}$ & $\phantom{0}94.9\phantom{0}$ & $\phantom{0}96.0\phantom{0}$ & $\phantom{0}95.9\phantom{0}$ & $\phantom{0}95.9\phantom{0}$ \\
\multicolumn{4}{l}{$\bar{n}=20$} \\  & \nopagebreak $\;J=50$  & $\phantom{-}0.6\phantom{0}$ & ${-}2.0\phantom{0}$ & ${-}4.2\phantom{0}$ & ${-}1.2\phantom{0}$ & ${-}3.4\phantom{0}$ & $\phantom{0}0.08\phantom{0}$ & $\phantom{0}0.09\phantom{0}$ & $\phantom{0}0.09\phantom{0}$ & $\phantom{0}0.09\phantom{0}$ & $\phantom{0}0.09\phantom{0}$ & $\phantom{0}93.9\phantom{0}$ & $\phantom{0}93.6\phantom{0}$ & $\phantom{0}94.4\phantom{0}$ & $\phantom{0}93.9\phantom{0}$ & $\phantom{0}94.0\phantom{0}$ \\
 & \nopagebreak $\;J=200$  & $\phantom{-}0.8\phantom{0}$ & ${-}0.5\phantom{0}$ & ${-}0.6\phantom{0}$ & $\phantom{-}0.1\phantom{0}$ & ${-}0.5\phantom{0}$ & $\phantom{0}0.04\phantom{0}$ & $\phantom{0}0.04\phantom{0}$ & $\phantom{0}0.04\phantom{0}$ & $\phantom{0}0.04\phantom{0}$ & $\phantom{0}0.04\phantom{0}$ & $\phantom{0}93.9\phantom{0}$ & $\phantom{0}94.7\phantom{0}$ & $\phantom{0}95.0\phantom{0}$ & $\phantom{0}93.5\phantom{0}$ & $\phantom{0}94.5\phantom{0}$ \\
 & \nopagebreak $\;J=1000$  & ${-}0.3\phantom{0}$ & ${-}0.9\phantom{0}$ & ${-}0.7\phantom{0}$ & ${-}0.3\phantom{0}$ & ${-}0.5\phantom{0}$ & $\phantom{0}0.02\phantom{0}$ & $\phantom{0}0.02\phantom{0}$ & $\phantom{0}0.02\phantom{0}$ & $\phantom{0}0.02\phantom{0}$ & $\phantom{0}0.02\phantom{0}$ & $\phantom{0}95.2\phantom{0}$ & $\phantom{0}95.5\phantom{0}$ & $\phantom{0}95.3\phantom{0}$ & $\phantom{0}95.4\phantom{0}$ & $\phantom{0}95.6\phantom{0}$ \\
[0.5ex]\hline\\[-1.6ex] 
\end{tabular}
\begin{tablenotes}[para,flushleft]{\footnotesize \textit{Note.} $\bar{n}$ = average cluster size; $J$ = number of clusters; CD = complete data sets; LD = listwise deletion; FCS-SL = single-level FCS; FCS-MAN = two-level FCS with manifest cluster means; FCS-LAT = two-level FCS with latent cluster means; JM = joint modeling.}\end{tablenotes}
\end{threeparttable}
\end{sidewaystable}
\begin{sidewaystable}
\begin{threeparttable}
\setlength{\tabcolsep}{1.0pt}
\renewcommand{\arraystretch}{0.95}
\footnotesize
\caption{\small Study 2: Bias (in \%), Relative RMSE, and Coverage of the 95\% Confidence Interval for the Regression Coefficient of $y$ on $z$ ($\hat\beta_{yz}$) With Moderately Unbalanced Data (Uniform, $\pm 40\%$) and 40\% Missing Data (MAR, $\lambda=0.5$)}
\begin{tabular}{llccccccccccccccc}
\hline\\[-1.8ex]
& & \multicolumn{5}{c}{Bias (\%)} & \multicolumn{5}{c}{Rel. RMSE} & \multicolumn{5}{c}{Coverage (\%)} \\ \cmidrule(r){3-7}\cmidrule(r){8-12}\cmidrule(r){13-17}
 &  & CD & \makecell{FCS-\\MAN} & \makecell{FCS-\\NJ} & \makecell{FCS-\\LAT} & JM & CD & \makecell{FCS-\\MAN} & \makecell{FCS-\\NJ} & \makecell{FCS-\\LAT} & JM & CD & \makecell{FCS-\\MAN} & \makecell{FCS-\\NJ} & \makecell{FCS-\\LAT} & \multicolumn{1}{c}{JM} \\ 
[0.4ex]\hline\\[-1.8ex]
& & \multicolumn{15}{c}{Small intraclass correlation $(\rho_{Iy}=.10)$} \\[0.6ex]\hline\\[-1.8ex]
\multicolumn{4}{l}{$\bar{n}=5$} \\  & \nopagebreak $\;J=50$  & $\phantom{0}\phantom{-}1.9\phantom{0}$ & $\phantom{0}{-}3.6\phantom{0}$ & ${-}10.8\phantom{0}$ & $\phantom{0}\phantom{-}3.2\phantom{0}$ & ${-}18.4\phantom{0}$ & $\phantom{0}0.07\phantom{0}$ & $\phantom{0}0.10\phantom{0}$ & $\phantom{0}0.10\phantom{0}$ & $\phantom{0}0.10\phantom{0}$ & $\phantom{0}0.09\phantom{0}$ & $\phantom{0}92.3\phantom{0}$ & $\phantom{0}92.7\phantom{0}$ & $\phantom{0}93.4\phantom{0}$ & $\phantom{0}91.4\phantom{0}$ & $\phantom{0}96.3\phantom{0}$ \\
 & \nopagebreak $\;J=200$  & $\phantom{0}\phantom{-}2.4\phantom{0}$ & $\phantom{0}{-}1.7\phantom{0}$ & $\phantom{0}{-}1.8\phantom{0}$ & $\phantom{0}\phantom{-}3.1\phantom{0}$ & $\phantom{0}{-}8.0\phantom{0}$ & $\phantom{0}0.04\phantom{0}$ & $\phantom{0}0.05\phantom{0}$ & $\phantom{0}0.05\phantom{0}$ & $\phantom{0}0.05\phantom{0}$ & $\phantom{0}0.05\phantom{0}$ & $\phantom{0}94.0\phantom{0}$ & $\phantom{0}92.8\phantom{0}$ & $\phantom{0}93.6\phantom{0}$ & $\phantom{0}92.4\phantom{0}$ & $\phantom{0}94.1\phantom{0}$ \\
 & \nopagebreak $\;J=1000$  & $\phantom{0}\phantom{-}0.0\phantom{0}$ & $\phantom{0}{-}2.0\phantom{0}$ & $\phantom{0}{-}1.0\phantom{0}$ & $\phantom{0}{-}0.1\phantom{0}$ & $\phantom{0}{-}2.9\phantom{0}$ & $\phantom{0}0.02\phantom{0}$ & $\phantom{0}0.02\phantom{0}$ & $\phantom{0}0.02\phantom{0}$ & $\phantom{0}0.02\phantom{0}$ & $\phantom{0}0.02\phantom{0}$ & $\phantom{0}96.4\phantom{0}$ & $\phantom{0}94.7\phantom{0}$ & $\phantom{0}94.6\phantom{0}$ & $\phantom{0}94.2\phantom{0}$ & $\phantom{0}94.5\phantom{0}$ \\
\multicolumn{4}{l}{$\bar{n}=20$} \\  & \nopagebreak $\;J=50$  & $\phantom{0}{-}2.5\phantom{0}$ & $\phantom{0}{-}7.5\phantom{0}$ & ${-}13.7\phantom{0}$ & $\phantom{0}{-}5.5\phantom{0}$ & ${-}19.1\phantom{0}$ & $\phantom{0}0.05\phantom{0}$ & $\phantom{0}0.07\phantom{0}$ & $\phantom{0}0.07\phantom{0}$ & $\phantom{0}0.07\phantom{0}$ & $\phantom{0}0.07\phantom{0}$ & $\phantom{0}91.6\phantom{0}$ & $\phantom{0}93.3\phantom{0}$ & $\phantom{0}93.8\phantom{0}$ & $\phantom{0}91.8\phantom{0}$ & $\phantom{0}94.3\phantom{0}$ \\
 & \nopagebreak $\;J=200$  & $\phantom{0}\phantom{-}0.3\phantom{0}$ & $\phantom{0}{-}0.3\phantom{0}$ & $\phantom{0}{-}2.0\phantom{0}$ & $\phantom{0}{-}0.1\phantom{0}$ & $\phantom{0}{-}5.4\phantom{0}$ & $\phantom{0}0.02\phantom{0}$ & $\phantom{0}0.03\phantom{0}$ & $\phantom{0}0.03\phantom{0}$ & $\phantom{0}0.03\phantom{0}$ & $\phantom{0}0.03\phantom{0}$ & $\phantom{0}94.5\phantom{0}$ & $\phantom{0}95.1\phantom{0}$ & $\phantom{0}95.3\phantom{0}$ & $\phantom{0}94.7\phantom{0}$ & $\phantom{0}95.2\phantom{0}$ \\
 & \nopagebreak $\;J=1000$  & $\phantom{0}{-}0.2\phantom{0}$ & $\phantom{0}{-}0.9\phantom{0}$ & $\phantom{0}{-}1.1\phantom{0}$ & $\phantom{0}{-}0.6\phantom{0}$ & $\phantom{0}{-}1.8\phantom{0}$ & $\phantom{0}0.01\phantom{0}$ & $\phantom{0}0.01\phantom{0}$ & $\phantom{0}0.01\phantom{0}$ & $\phantom{0}0.01\phantom{0}$ & $\phantom{0}0.01\phantom{0}$ & $\phantom{0}94.9\phantom{0}$ & $\phantom{0}94.7\phantom{0}$ & $\phantom{0}94.6\phantom{0}$ & $\phantom{0}94.3\phantom{0}$ & $\phantom{0}94.1\phantom{0}$ \\
[0.5ex]\hline\\[-1.6ex] 
& & \multicolumn{15}{c}{Moderate intraclass correlation $(\rho_{Iy}=.30)$} \\[0.6ex]\hline\\[-1.8ex]
\multicolumn{4}{l}{$\bar{n}=5$} \\  & \nopagebreak $\;J=50$  & $\phantom{0}\phantom{-}1.0\phantom{0}$ & $\phantom{0}{-}5.4\phantom{0}$ & ${-}12.0\phantom{0}$ & $\phantom{0}{-}3.3\phantom{0}$ & ${-}10.7\phantom{0}$ & $\phantom{0}0.09\phantom{0}$ & $\phantom{0}0.12\phantom{0}$ & $\phantom{0}0.12\phantom{0}$ & $\phantom{0}0.12\phantom{0}$ & $\phantom{0}0.12\phantom{0}$ & $\phantom{0}91.6\phantom{0}$ & $\phantom{0}92.8\phantom{0}$ & $\phantom{0}92.9\phantom{0}$ & $\phantom{0}91.3\phantom{0}$ & $\phantom{0}95.0\phantom{0}$ \\
 & \nopagebreak $\;J=200$  & $\phantom{0}\phantom{-}0.1\phantom{0}$ & $\phantom{0}{-}2.1\phantom{0}$ & $\phantom{0}{-}3.2\phantom{0}$ & $\phantom{0}{-}1.4\phantom{0}$ & $\phantom{0}{-}3.3\phantom{0}$ & $\phantom{0}0.04\phantom{0}$ & $\phantom{0}0.06\phantom{0}$ & $\phantom{0}0.06\phantom{0}$ & $\phantom{0}0.06\phantom{0}$ & $\phantom{0}0.06\phantom{0}$ & $\phantom{0}94.0\phantom{0}$ & $\phantom{0}95.0\phantom{0}$ & $\phantom{0}94.3\phantom{0}$ & $\phantom{0}93.6\phantom{0}$ & $\phantom{0}94.7\phantom{0}$ \\
 & \nopagebreak $\;J=1000$  & $\phantom{0}{-}0.3\phantom{0}$ & $\phantom{0}{-}1.0\phantom{0}$ & $\phantom{0}{-}1.0\phantom{0}$ & $\phantom{0}{-}0.5\phantom{0}$ & $\phantom{0}{-}0.9\phantom{0}$ & $\phantom{0}0.02\phantom{0}$ & $\phantom{0}0.03\phantom{0}$ & $\phantom{0}0.03\phantom{0}$ & $\phantom{0}0.02\phantom{0}$ & $\phantom{0}0.03\phantom{0}$ & $\phantom{0}95.3\phantom{0}$ & $\phantom{0}95.4\phantom{0}$ & $\phantom{0}94.3\phantom{0}$ & $\phantom{0}95.0\phantom{0}$ & $\phantom{0}95.1\phantom{0}$ \\
\multicolumn{4}{l}{$\bar{n}=20$} \\  & \nopagebreak $\;J=50$  & $\phantom{0}{-}1.2\phantom{0}$ & $\phantom{0}{-}7.6\phantom{0}$ & ${-}13.5\phantom{0}$ & $\phantom{0}{-}7.3\phantom{0}$ & ${-}11.2\phantom{0}$ & $\phantom{0}0.08\phantom{0}$ & $\phantom{0}0.10\phantom{0}$ & $\phantom{0}0.10\phantom{0}$ & $\phantom{0}0.10\phantom{0}$ & $\phantom{0}0.10\phantom{0}$ & $\phantom{0}90.9\phantom{0}$ & $\phantom{0}93.8\phantom{0}$ & $\phantom{0}93.2\phantom{0}$ & $\phantom{0}93.8\phantom{0}$ & $\phantom{0}94.4\phantom{0}$ \\
 & \nopagebreak $\;J=200$  & $\phantom{0}\phantom{-}0.4\phantom{0}$ & $\phantom{0}{-}1.0\phantom{0}$ & $\phantom{0}{-}2.6\phantom{0}$ & $\phantom{0}{-}1.1\phantom{0}$ & $\phantom{0}{-}2.2\phantom{0}$ & $\phantom{0}0.04\phantom{0}$ & $\phantom{0}0.05\phantom{0}$ & $\phantom{0}0.05\phantom{0}$ & $\phantom{0}0.05\phantom{0}$ & $\phantom{0}0.05\phantom{0}$ & $\phantom{0}94.2\phantom{0}$ & $\phantom{0}94.1\phantom{0}$ & $\phantom{0}95.2\phantom{0}$ & $\phantom{0}94.7\phantom{0}$ & $\phantom{0}95.0\phantom{0}$ \\
 & \nopagebreak $\;J=1000$  & $\phantom{0}\phantom{-}0.1\phantom{0}$ & $\phantom{0}{-}0.1\phantom{0}$ & $\phantom{0}{-}0.4\phantom{0}$ & $\phantom{0}{-}0.0\phantom{0}$ & $\phantom{0}{-}0.3\phantom{0}$ & $\phantom{0}0.02\phantom{0}$ & $\phantom{0}0.02\phantom{0}$ & $\phantom{0}0.02\phantom{0}$ & $\phantom{0}0.02\phantom{0}$ & $\phantom{0}0.02\phantom{0}$ & $\phantom{0}94.7\phantom{0}$ & $\phantom{0}94.7\phantom{0}$ & $\phantom{0}95.0\phantom{0}$ & $\phantom{0}95.0\phantom{0}$ & $\phantom{0}95.3\phantom{0}$ \\
[0.5ex]\hline\\[-1.6ex] 
\end{tabular}
\begin{tablenotes}[para,flushleft]{\footnotesize \textit{Note.} $\bar{n}$ = average cluster size; $J$ = number of clusters; CD = complete data sets; LD = listwise deletion; FCS-SL = single-level FCS; FCS-MAN = two-level FCS with manifest cluster means; FCS-LAT = two-level FCS with latent cluster means; JM = joint modeling.}\end{tablenotes}
\end{threeparttable}
\end{sidewaystable}
\begin{sidewaystable}
\begin{threeparttable}
\setlength{\tabcolsep}{1.0pt}
\renewcommand{\arraystretch}{0.95}
\footnotesize
\caption{\small Study 2: Bias (in \%), Relative RMSE, and Coverage of the 95\% Confidence Interval for the Regression Coefficient of $y$ on $z$ ($\hat\beta_{yz}$) With Strongly Unbalanced Data (Uniform, $\pm 80\%$) and 40\% Missing Data (MAR, $\lambda=0.5$)}
\begin{tabular}{llccccccccccccccc}
\hline\\[-1.8ex]
& & \multicolumn{5}{c}{Bias (\%)} & \multicolumn{5}{c}{Rel. RMSE} & \multicolumn{5}{c}{Coverage (\%)} \\ \cmidrule(r){3-7}\cmidrule(r){8-12}\cmidrule(r){13-17}
 &  & CD & \makecell{FCS-\\MAN} & \makecell{FCS-\\NJ} & \makecell{FCS-\\LAT} & JM & CD & \makecell{FCS-\\MAN} & \makecell{FCS-\\NJ} & \makecell{FCS-\\LAT} & JM & CD & \makecell{FCS-\\MAN} & \makecell{FCS-\\NJ} & \makecell{FCS-\\LAT} & \multicolumn{1}{c}{JM} \\ 
[0.4ex]\hline\\[-1.8ex]
& & \multicolumn{15}{c}{Small intraclass correlation $(\rho_{Iy}=.10)$} \\[0.6ex]\hline\\[-1.8ex]
\multicolumn{4}{l}{$\bar{n}=5$} \\  & \nopagebreak $\;J=50$  & $\phantom{0}{-}1.2\phantom{0}$ & ${-}16.6\phantom{0}$ & ${-}16.8\phantom{0}$ & $\phantom{0}{-}3.5\phantom{0}$ & ${-}24.1\phantom{0}$ & $\phantom{0}0.08\phantom{0}$ & $\phantom{0}0.10\phantom{0}$ & $\phantom{0}0.10\phantom{0}$ & $\phantom{0}0.10\phantom{0}$ & $\phantom{0}0.10\phantom{0}$ & $\phantom{0}91.3\phantom{0}$ & $\phantom{0}93.3\phantom{0}$ & $\phantom{0}94.0\phantom{0}$ & $\phantom{0}91.4\phantom{0}$ & $\phantom{0}95.6\phantom{0}$ \\
 & \nopagebreak $\;J=200$  & $\phantom{0}{-}0.0\phantom{0}$ & $\phantom{0}{-}8.9\phantom{0}$ & $\phantom{0}{-}3.2\phantom{0}$ & $\phantom{0}\phantom{-}1.6\phantom{0}$ & $\phantom{0}{-}8.9\phantom{0}$ & $\phantom{0}0.04\phantom{0}$ & $\phantom{0}0.05\phantom{0}$ & $\phantom{0}0.05\phantom{0}$ & $\phantom{0}0.05\phantom{0}$ & $\phantom{0}0.05\phantom{0}$ & $\phantom{0}94.9\phantom{0}$ & $\phantom{0}94.9\phantom{0}$ & $\phantom{0}94.0\phantom{0}$ & $\phantom{0}92.8\phantom{0}$ & $\phantom{0}95.6\phantom{0}$ \\
 & \nopagebreak $\;J=1000$  & $\phantom{0}\phantom{-}0.6\phantom{0}$ & $\phantom{0}{-}7.7\phantom{0}$ & $\phantom{0}{-}0.8\phantom{0}$ & $\phantom{0}\phantom{-}0.5\phantom{0}$ & $\phantom{0}{-}2.4\phantom{0}$ & $\phantom{0}0.02\phantom{0}$ & $\phantom{0}0.02\phantom{0}$ & $\phantom{0}0.02\phantom{0}$ & $\phantom{0}0.02\phantom{0}$ & $\phantom{0}0.02\phantom{0}$ & $\phantom{0}94.5\phantom{0}$ & $\phantom{0}92.6\phantom{0}$ & $\phantom{0}95.2\phantom{0}$ & $\phantom{0}93.1\phantom{0}$ & $\phantom{0}95.0\phantom{0}$ \\
\multicolumn{4}{l}{$\bar{n}=20$} \\  & \nopagebreak $\;J=50$  & $\phantom{0}\phantom{-}0.8\phantom{0}$ & $\phantom{0}{-}6.3\phantom{0}$ & ${-}11.3\phantom{0}$ & $\phantom{0}{-}2.6\phantom{0}$ & ${-}17.6\phantom{0}$ & $\phantom{0}0.05\phantom{0}$ & $\phantom{0}0.07\phantom{0}$ & $\phantom{0}0.07\phantom{0}$ & $\phantom{0}0.07\phantom{0}$ & $\phantom{0}0.07\phantom{0}$ & $\phantom{0}93.0\phantom{0}$ & $\phantom{0}93.4\phantom{0}$ & $\phantom{0}94.6\phantom{0}$ & $\phantom{0}93.2\phantom{0}$ & $\phantom{0}95.7\phantom{0}$ \\
 & \nopagebreak $\;J=200$  & $\phantom{0}{-}0.0\phantom{0}$ & $\phantom{0}{-}4.2\phantom{0}$ & $\phantom{0}{-}3.3\phantom{0}$ & $\phantom{0}{-}1.4\phantom{0}$ & $\phantom{0}{-}6.5\phantom{0}$ & $\phantom{0}0.03\phantom{0}$ & $\phantom{0}0.03\phantom{0}$ & $\phantom{0}0.03\phantom{0}$ & $\phantom{0}0.03\phantom{0}$ & $\phantom{0}0.03\phantom{0}$ & $\phantom{0}93.8\phantom{0}$ & $\phantom{0}94.7\phantom{0}$ & $\phantom{0}94.6\phantom{0}$ & $\phantom{0}94.1\phantom{0}$ & $\phantom{0}94.3\phantom{0}$ \\
 & \nopagebreak $\;J=1000$  & $\phantom{0}\phantom{-}0.3\phantom{0}$ & $\phantom{0}{-}1.9\phantom{0}$ & $\phantom{0}{-}0.6\phantom{0}$ & $\phantom{0}\phantom{-}0.1\phantom{0}$ & $\phantom{0}{-}1.1\phantom{0}$ & $\phantom{0}0.01\phantom{0}$ & $\phantom{0}0.01\phantom{0}$ & $\phantom{0}0.01\phantom{0}$ & $\phantom{0}0.01\phantom{0}$ & $\phantom{0}0.01\phantom{0}$ & $\phantom{0}95.3\phantom{0}$ & $\phantom{0}95.6\phantom{0}$ & $\phantom{0}95.0\phantom{0}$ & $\phantom{0}94.6\phantom{0}$ & $\phantom{0}94.9\phantom{0}$ \\
[0.5ex]\hline\\[-1.6ex] 
& & \multicolumn{15}{c}{Moderate intraclass correlation $(\rho_{Iy}=.30)$} \\[0.6ex]\hline\\[-1.8ex]
\multicolumn{4}{l}{$\bar{n}=5$} \\  & \nopagebreak $\;J=50$  & $\phantom{0}\phantom{-}1.3\phantom{0}$ & $\phantom{0}{-}7.6\phantom{0}$ & ${-}11.4\phantom{0}$ & $\phantom{0}{-}4.0\phantom{0}$ & ${-}11.2\phantom{0}$ & $\phantom{0}0.09\phantom{0}$ & $\phantom{0}0.12\phantom{0}$ & $\phantom{0}0.12\phantom{0}$ & $\phantom{0}0.12\phantom{0}$ & $\phantom{0}0.12\phantom{0}$ & $\phantom{0}91.8\phantom{0}$ & $\phantom{0}93.5\phantom{0}$ & $\phantom{0}93.6\phantom{0}$ & $\phantom{0}91.3\phantom{0}$ & $\phantom{0}94.2\phantom{0}$ \\
 & \nopagebreak $\;J=200$  & $\phantom{0}{-}0.5\phantom{0}$ & $\phantom{0}{-}4.7\phantom{0}$ & $\phantom{0}{-}3.7\phantom{0}$ & $\phantom{0}{-}1.3\phantom{0}$ & $\phantom{0}{-}3.3\phantom{0}$ & $\phantom{0}0.05\phantom{0}$ & $\phantom{0}0.06\phantom{0}$ & $\phantom{0}0.06\phantom{0}$ & $\phantom{0}0.06\phantom{0}$ & $\phantom{0}0.06\phantom{0}$ & $\phantom{0}93.9\phantom{0}$ & $\phantom{0}95.1\phantom{0}$ & $\phantom{0}95.0\phantom{0}$ & $\phantom{0}94.5\phantom{0}$ & $\phantom{0}95.3\phantom{0}$ \\
 & \nopagebreak $\;J=1000$  & $\phantom{0}{-}0.1\phantom{0}$ & $\phantom{0}{-}2.9\phantom{0}$ & $\phantom{0}{-}1.0\phantom{0}$ & $\phantom{0}{-}0.2\phantom{0}$ & $\phantom{0}{-}0.4\phantom{0}$ & $\phantom{0}0.02\phantom{0}$ & $\phantom{0}0.03\phantom{0}$ & $\phantom{0}0.02\phantom{0}$ & $\phantom{0}0.02\phantom{0}$ & $\phantom{0}0.02\phantom{0}$ & $\phantom{0}94.8\phantom{0}$ & $\phantom{0}95.2\phantom{0}$ & $\phantom{0}95.9\phantom{0}$ & $\phantom{0}96.0\phantom{0}$ & $\phantom{0}95.4\phantom{0}$ \\
\multicolumn{4}{l}{$\bar{n}=20$} \\  & \nopagebreak $\;J=50$  & $\phantom{0}{-}0.4\phantom{0}$ & $\phantom{0}{-}6.2\phantom{0}$ & ${-}12.6\phantom{0}$ & $\phantom{0}{-}5.4\phantom{0}$ & $\phantom{0}{-}9.9\phantom{0}$ & $\phantom{0}0.08\phantom{0}$ & $\phantom{0}0.10\phantom{0}$ & $\phantom{0}0.10\phantom{0}$ & $\phantom{0}0.10\phantom{0}$ & $\phantom{0}0.10\phantom{0}$ & $\phantom{0}91.2\phantom{0}$ & $\phantom{0}93.2\phantom{0}$ & $\phantom{0}92.5\phantom{0}$ & $\phantom{0}93.0\phantom{0}$ & $\phantom{0}93.9\phantom{0}$ \\
 & \nopagebreak $\;J=200$  & $\phantom{0}\phantom{-}0.7\phantom{0}$ & $\phantom{0}{-}0.8\phantom{0}$ & $\phantom{0}{-}1.9\phantom{0}$ & $\phantom{0}{-}0.5\phantom{0}$ & $\phantom{0}{-}1.9\phantom{0}$ & $\phantom{0}0.04\phantom{0}$ & $\phantom{0}0.05\phantom{0}$ & $\phantom{0}0.05\phantom{0}$ & $\phantom{0}0.05\phantom{0}$ & $\phantom{0}0.05\phantom{0}$ & $\phantom{0}92.7\phantom{0}$ & $\phantom{0}94.1\phantom{0}$ & $\phantom{0}93.8\phantom{0}$ & $\phantom{0}93.6\phantom{0}$ & $\phantom{0}94.5\phantom{0}$ \\
 & \nopagebreak $\;J=1000$  & $\phantom{0}{-}0.1\phantom{0}$ & $\phantom{0}{-}0.6\phantom{0}$ & $\phantom{0}{-}0.8\phantom{0}$ & $\phantom{0}{-}0.3\phantom{0}$ & $\phantom{0}{-}0.6\phantom{0}$ & $\phantom{0}0.02\phantom{0}$ & $\phantom{0}0.02\phantom{0}$ & $\phantom{0}0.02\phantom{0}$ & $\phantom{0}0.02\phantom{0}$ & $\phantom{0}0.02\phantom{0}$ & $\phantom{0}95.2\phantom{0}$ & $\phantom{0}94.9\phantom{0}$ & $\phantom{0}94.8\phantom{0}$ & $\phantom{0}94.3\phantom{0}$ & $\phantom{0}95.2\phantom{0}$ \\
[0.5ex]\hline\\[-1.6ex] 
\end{tabular}
\begin{tablenotes}[para,flushleft]{\footnotesize \textit{Note.} $\bar{n}$ = average cluster size; $J$ = number of clusters; CD = complete data sets; LD = listwise deletion; FCS-SL = single-level FCS; FCS-MAN = two-level FCS with manifest cluster means; FCS-LAT = two-level FCS with latent cluster means; JM = joint modeling.}\end{tablenotes}
\end{threeparttable}
\end{sidewaystable}
\begin{sidewaystable}
\begin{threeparttable}
\setlength{\tabcolsep}{1.0pt}
\renewcommand{\arraystretch}{0.95}
\footnotesize
\caption{\small Study 2: Bias (in \%), Relative RMSE, and Coverage of the 95\% Confidence Interval for the Regression Coefficient of $y$ on $z$ ($\hat\beta_{yz}$) With Moderately Unbalanced Data (Bimodal, $\pm 40\%$) and 40\% Missing Data (MAR, $\lambda=0.5$)}
\begin{tabular}{llccccccccccccccc}
\hline\\[-1.8ex]
& & \multicolumn{5}{c}{Bias (\%)} & \multicolumn{5}{c}{Rel. RMSE} & \multicolumn{5}{c}{Coverage (\%)} \\ \cmidrule(r){3-7}\cmidrule(r){8-12}\cmidrule(r){13-17}
 &  & CD & \makecell{FCS-\\MAN} & \makecell{FCS-\\NJ} & \makecell{FCS-\\LAT} & JM & CD & \makecell{FCS-\\MAN} & \makecell{FCS-\\NJ} & \makecell{FCS-\\LAT} & JM & CD & \makecell{FCS-\\MAN} & \makecell{FCS-\\NJ} & \makecell{FCS-\\LAT} & \multicolumn{1}{c}{JM} \\ 
[0.4ex]\hline\\[-1.8ex]
& & \multicolumn{15}{c}{Small intraclass correlation $(\rho_{Iy}=.10)$} \\[0.6ex]\hline\\[-1.8ex]
\multicolumn{4}{l}{$\bar{n}=5$} \\  & \nopagebreak $\;J=50$  & $\phantom{0}{-}0.9\phantom{0}$ & $\phantom{0}{-}9.2\phantom{0}$ & ${-}14.9\phantom{0}$ & $\phantom{0}{-}1.1\phantom{0}$ & ${-}21.2\phantom{0}$ & $\phantom{0}0.07\phantom{0}$ & $\phantom{0}0.10\phantom{0}$ & $\phantom{0}0.10\phantom{0}$ & $\phantom{0}0.10\phantom{0}$ & $\phantom{0}0.09\phantom{0}$ & $\phantom{0}92.5\phantom{0}$ & $\phantom{0}92.6\phantom{0}$ & $\phantom{0}93.0\phantom{0}$ & $\phantom{0}90.8\phantom{0}$ & $\phantom{0}95.0\phantom{0}$ \\
 & \nopagebreak $\;J=200$  & $\phantom{0}\phantom{-}0.4\phantom{0}$ & $\phantom{0}{-}3.5\phantom{0}$ & $\phantom{0}{-}2.7\phantom{0}$ & $\phantom{0}\phantom{-}2.6\phantom{0}$ & $\phantom{0}{-}8.5\phantom{0}$ & $\phantom{0}0.04\phantom{0}$ & $\phantom{0}0.05\phantom{0}$ & $\phantom{0}0.05\phantom{0}$ & $\phantom{0}0.05\phantom{0}$ & $\phantom{0}0.05\phantom{0}$ & $\phantom{0}94.2\phantom{0}$ & $\phantom{0}94.8\phantom{0}$ & $\phantom{0}94.7\phantom{0}$ & $\phantom{0}92.1\phantom{0}$ & $\phantom{0}95.2\phantom{0}$ \\
 & \nopagebreak $\;J=1000$  & $\phantom{0}{-}0.4\phantom{0}$ & $\phantom{0}{-}3.2\phantom{0}$ & $\phantom{0}{-}0.9\phantom{0}$ & $\phantom{0}\phantom{-}0.3\phantom{0}$ & $\phantom{0}{-}2.6\phantom{0}$ & $\phantom{0}0.02\phantom{0}$ & $\phantom{0}0.02\phantom{0}$ & $\phantom{0}0.02\phantom{0}$ & $\phantom{0}0.02\phantom{0}$ & $\phantom{0}0.02\phantom{0}$ & $\phantom{0}94.6\phantom{0}$ & $\phantom{0}94.5\phantom{0}$ & $\phantom{0}94.2\phantom{0}$ & $\phantom{0}93.7\phantom{0}$ & $\phantom{0}95.0\phantom{0}$ \\
\multicolumn{4}{l}{$\bar{n}=20$} \\  & \nopagebreak $\;J=50$  & $\phantom{0}{-}0.4\phantom{0}$ & $\phantom{0}{-}6.2\phantom{0}$ & ${-}12.6\phantom{0}$ & $\phantom{0}{-}4.7\phantom{0}$ & ${-}18.4\phantom{0}$ & $\phantom{0}0.05\phantom{0}$ & $\phantom{0}0.07\phantom{0}$ & $\phantom{0}0.07\phantom{0}$ & $\phantom{0}0.07\phantom{0}$ & $\phantom{0}0.07\phantom{0}$ & $\phantom{0}92.3\phantom{0}$ & $\phantom{0}92.6\phantom{0}$ & $\phantom{0}92.5\phantom{0}$ & $\phantom{0}90.8\phantom{0}$ & $\phantom{0}93.7\phantom{0}$ \\
 & \nopagebreak $\;J=200$  & $\phantom{0}{-}0.4\phantom{0}$ & $\phantom{0}{-}2.7\phantom{0}$ & $\phantom{0}{-}3.6\phantom{0}$ & $\phantom{0}{-}1.9\phantom{0}$ & $\phantom{0}{-}6.9\phantom{0}$ & $\phantom{0}0.03\phantom{0}$ & $\phantom{0}0.03\phantom{0}$ & $\phantom{0}0.03\phantom{0}$ & $\phantom{0}0.03\phantom{0}$ & $\phantom{0}0.03\phantom{0}$ & $\phantom{0}94.7\phantom{0}$ & $\phantom{0}94.7\phantom{0}$ & $\phantom{0}94.4\phantom{0}$ & $\phantom{0}94.4\phantom{0}$ & $\phantom{0}95.4\phantom{0}$ \\
 & \nopagebreak $\;J=1000$  & $\phantom{0}\phantom{-}0.6\phantom{0}$ & $\phantom{0}{-}0.8\phantom{0}$ & $\phantom{0}{-}0.5\phantom{0}$ & $\phantom{0}\phantom{-}0.1\phantom{0}$ & $\phantom{0}{-}1.3\phantom{0}$ & $\phantom{0}0.01\phantom{0}$ & $\phantom{0}0.01\phantom{0}$ & $\phantom{0}0.01\phantom{0}$ & $\phantom{0}0.01\phantom{0}$ & $\phantom{0}0.01\phantom{0}$ & $\phantom{0}94.2\phantom{0}$ & $\phantom{0}94.0\phantom{0}$ & $\phantom{0}95.0\phantom{0}$ & $\phantom{0}93.7\phantom{0}$ & $\phantom{0}94.5\phantom{0}$ \\
[0.5ex]\hline\\[-1.6ex] 
& & \multicolumn{15}{c}{Moderate intraclass correlation $(\rho_{Iy}=.30)$} \\[0.6ex]\hline\\[-1.8ex]
\multicolumn{4}{l}{$\bar{n}=5$} \\  & \nopagebreak $\;J=50$  & $\phantom{0}{-}0.5\phantom{0}$ & $\phantom{0}{-}6.2\phantom{0}$ & ${-}12.9\phantom{0}$ & $\phantom{0}{-}3.9\phantom{0}$ & ${-}11.2\phantom{0}$ & $\phantom{0}0.09\phantom{0}$ & $\phantom{0}0.12\phantom{0}$ & $\phantom{0}0.12\phantom{0}$ & $\phantom{0}0.12\phantom{0}$ & $\phantom{0}0.12\phantom{0}$ & $\phantom{0}91.6\phantom{0}$ & $\phantom{0}94.3\phantom{0}$ & $\phantom{0}93.8\phantom{0}$ & $\phantom{0}92.0\phantom{0}$ & $\phantom{0}95.3\phantom{0}$ \\
 & \nopagebreak $\;J=200$  & $\phantom{0}{-}0.5\phantom{0}$ & $\phantom{0}{-}2.5\phantom{0}$ & $\phantom{0}{-}3.7\phantom{0}$ & $\phantom{0}{-}1.6\phantom{0}$ & $\phantom{0}{-}3.6\phantom{0}$ & $\phantom{0}0.05\phantom{0}$ & $\phantom{0}0.06\phantom{0}$ & $\phantom{0}0.06\phantom{0}$ & $\phantom{0}0.06\phantom{0}$ & $\phantom{0}0.06\phantom{0}$ & $\phantom{0}93.5\phantom{0}$ & $\phantom{0}94.1\phantom{0}$ & $\phantom{0}93.8\phantom{0}$ & $\phantom{0}94.0\phantom{0}$ & $\phantom{0}94.6\phantom{0}$ \\
 & \nopagebreak $\;J=1000$  & $\phantom{0}\phantom{-}0.1\phantom{0}$ & $\phantom{0}{-}1.1\phantom{0}$ & $\phantom{0}{-}0.6\phantom{0}$ & $\phantom{0}\phantom{-}0.0\phantom{0}$ & $\phantom{0}{-}0.4\phantom{0}$ & $\phantom{0}0.02\phantom{0}$ & $\phantom{0}0.02\phantom{0}$ & $\phantom{0}0.02\phantom{0}$ & $\phantom{0}0.02\phantom{0}$ & $\phantom{0}0.02\phantom{0}$ & $\phantom{0}94.1\phantom{0}$ & $\phantom{0}95.9\phantom{0}$ & $\phantom{0}94.6\phantom{0}$ & $\phantom{0}95.2\phantom{0}$ & $\phantom{0}95.9\phantom{0}$ \\
\multicolumn{4}{l}{$\bar{n}=20$} \\  & \nopagebreak $\;J=50$  & $\phantom{0}{-}0.1\phantom{0}$ & $\phantom{0}{-}5.8\phantom{0}$ & ${-}12.4\phantom{0}$ & $\phantom{0}{-}5.7\phantom{0}$ & ${-}10.3\phantom{0}$ & $\phantom{0}0.08\phantom{0}$ & $\phantom{0}0.10\phantom{0}$ & $\phantom{0}0.11\phantom{0}$ & $\phantom{0}0.10\phantom{0}$ & $\phantom{0}0.10\phantom{0}$ & $\phantom{0}91.7\phantom{0}$ & $\phantom{0}91.6\phantom{0}$ & $\phantom{0}92.5\phantom{0}$ & $\phantom{0}93.1\phantom{0}$ & $\phantom{0}94.8\phantom{0}$ \\
 & \nopagebreak $\;J=200$  & $\phantom{0}\phantom{-}0.4\phantom{0}$ & $\phantom{0}{-}0.8\phantom{0}$ & $\phantom{0}{-}2.4\phantom{0}$ & $\phantom{0}{-}0.8\phantom{0}$ & $\phantom{0}{-}2.0\phantom{0}$ & $\phantom{0}0.04\phantom{0}$ & $\phantom{0}0.05\phantom{0}$ & $\phantom{0}0.05\phantom{0}$ & $\phantom{0}0.05\phantom{0}$ & $\phantom{0}0.05\phantom{0}$ & $\phantom{0}93.6\phantom{0}$ & $\phantom{0}93.7\phantom{0}$ & $\phantom{0}93.5\phantom{0}$ & $\phantom{0}93.5\phantom{0}$ & $\phantom{0}94.1\phantom{0}$ \\
 & \nopagebreak $\;J=1000$  & $\phantom{0}\phantom{-}0.1\phantom{0}$ & $\phantom{0}{-}0.2\phantom{0}$ & $\phantom{0}{-}0.4\phantom{0}$ & $\phantom{0}{-}0.1\phantom{0}$ & $\phantom{0}{-}0.3\phantom{0}$ & $\phantom{0}0.02\phantom{0}$ & $\phantom{0}0.02\phantom{0}$ & $\phantom{0}0.02\phantom{0}$ & $\phantom{0}0.02\phantom{0}$ & $\phantom{0}0.02\phantom{0}$ & $\phantom{0}95.8\phantom{0}$ & $\phantom{0}95.0\phantom{0}$ & $\phantom{0}95.2\phantom{0}$ & $\phantom{0}94.2\phantom{0}$ & $\phantom{0}95.3\phantom{0}$ \\
[0.5ex]\hline\\[-1.6ex] 
\end{tabular}
\begin{tablenotes}[para,flushleft]{\footnotesize \textit{Note.} $\bar{n}$ = average cluster size; $J$ = number of clusters; CD = complete data sets; LD = listwise deletion; FCS-SL = single-level FCS; FCS-MAN = two-level FCS with manifest cluster means; FCS-LAT = two-level FCS with latent cluster means; JM = joint modeling.}\end{tablenotes}
\end{threeparttable}
\end{sidewaystable}
\begin{sidewaystable}
\begin{threeparttable}
\setlength{\tabcolsep}{1.0pt}
\renewcommand{\arraystretch}{0.95}
\footnotesize
\caption{\small Study 2: Bias (in \%), Relative RMSE, and Coverage of the 95\% Confidence Interval for the Regression Coefficient of $y$ on $z$ ($\hat\beta_{yz}$) With Strongly Unbalanced Data (Bimodal, $\pm 80\%$) and 40\% Missing Data (MAR, $\lambda=0.5$)}
\begin{tabular}{llccccccccccccccc}
\hline\\[-1.8ex]
& & \multicolumn{5}{c}{Bias (\%)} & \multicolumn{5}{c}{Rel. RMSE} & \multicolumn{5}{c}{Coverage (\%)} \\ \cmidrule(r){3-7}\cmidrule(r){8-12}\cmidrule(r){13-17}
 &  & CD & \makecell{FCS-\\MAN} & \makecell{FCS-\\NJ} & \makecell{FCS-\\LAT} & JM & CD & \makecell{FCS-\\MAN} & \makecell{FCS-\\NJ} & \makecell{FCS-\\LAT} & JM & CD & \makecell{FCS-\\MAN} & \makecell{FCS-\\NJ} & \makecell{FCS-\\LAT} & \multicolumn{1}{c}{JM} \\ 
[0.4ex]\hline\\[-1.8ex]
& & \multicolumn{15}{c}{Small intraclass correlation $(\rho_{Iy}=.10)$} \\[0.6ex]\hline\\[-1.8ex]
\multicolumn{4}{l}{$\bar{n}=5$} \\  & \nopagebreak $\;J=50$  & $\phantom{0}{-}0.1\phantom{0}$ & ${-}19.5\phantom{0}$ & ${-}14.1\phantom{0}$ & $\phantom{0}{-}2.7\phantom{0}$ & ${-}21.9\phantom{0}$ & $\phantom{0}0.08\phantom{0}$ & $\phantom{0}0.11\phantom{0}$ & $\phantom{0}0.11\phantom{0}$ & $\phantom{0}0.12\phantom{0}$ & $\phantom{0}0.10\phantom{0}$ & $\phantom{0}90.4\phantom{0}$ & $\phantom{0}93.2\phantom{0}$ & $\phantom{0}91.8\phantom{0}$ & $\phantom{0}89.0\phantom{0}$ & $\phantom{0}94.5\phantom{0}$ \\
 & \nopagebreak $\;J=200$  & $\phantom{0}\phantom{-}0.2\phantom{0}$ & ${-}18.1\phantom{0}$ & $\phantom{0}{-}4.9\phantom{0}$ & $\phantom{0}{-}0.0\phantom{0}$ & $\phantom{0}{-}9.4\phantom{0}$ & $\phantom{0}0.04\phantom{0}$ & $\phantom{0}0.05\phantom{0}$ & $\phantom{0}0.05\phantom{0}$ & $\phantom{0}0.05\phantom{0}$ & $\phantom{0}0.05\phantom{0}$ & $\phantom{0}94.1\phantom{0}$ & $\phantom{0}94.3\phantom{0}$ & $\phantom{0}95.1\phantom{0}$ & $\phantom{0}93.0\phantom{0}$ & $\phantom{0}94.8\phantom{0}$ \\
 & \nopagebreak $\;J=1000$  & $\phantom{0}\phantom{-}0.3\phantom{0}$ & ${-}17.0\phantom{0}$ & $\phantom{0}{-}1.8\phantom{0}$ & $\phantom{0}\phantom{-}0.2\phantom{0}$ & $\phantom{0}{-}2.5\phantom{0}$ & $\phantom{0}0.02\phantom{0}$ & $\phantom{0}0.03\phantom{0}$ & $\phantom{0}0.02\phantom{0}$ & $\phantom{0}0.02\phantom{0}$ & $\phantom{0}0.02\phantom{0}$ & $\phantom{0}95.8\phantom{0}$ & $\phantom{0}81.2\phantom{0}$ & $\phantom{0}94.1\phantom{0}$ & $\phantom{0}93.3\phantom{0}$ & $\phantom{0}94.5\phantom{0}$ \\
\multicolumn{4}{l}{$\bar{n}=20$} \\  & \nopagebreak $\;J=50$  & $\phantom{0}\phantom{-}0.9\phantom{0}$ & $\phantom{0}{-}9.3\phantom{0}$ & ${-}10.5\phantom{0}$ & $\phantom{0}{-}2.1\phantom{0}$ & ${-}16.3\phantom{0}$ & $\phantom{0}0.06\phantom{0}$ & $\phantom{0}0.08\phantom{0}$ & $\phantom{0}0.08\phantom{0}$ & $\phantom{0}0.08\phantom{0}$ & $\phantom{0}0.07\phantom{0}$ & $\phantom{0}92.3\phantom{0}$ & $\phantom{0}94.8\phantom{0}$ & $\phantom{0}94.3\phantom{0}$ & $\phantom{0}90.6\phantom{0}$ & $\phantom{0}96.0\phantom{0}$ \\
 & \nopagebreak $\;J=200$  & $\phantom{0}\phantom{-}0.2\phantom{0}$ & $\phantom{0}{-}8.7\phantom{0}$ & $\phantom{0}{-}4.4\phantom{0}$ & $\phantom{0}{-}1.7\phantom{0}$ & $\phantom{0}{-}7.1\phantom{0}$ & $\phantom{0}0.03\phantom{0}$ & $\phantom{0}0.04\phantom{0}$ & $\phantom{0}0.03\phantom{0}$ & $\phantom{0}0.04\phantom{0}$ & $\phantom{0}0.04\phantom{0}$ & $\phantom{0}95.5\phantom{0}$ & $\phantom{0}94.8\phantom{0}$ & $\phantom{0}94.2\phantom{0}$ & $\phantom{0}94.0\phantom{0}$ & $\phantom{0}95.6\phantom{0}$ \\
 & \nopagebreak $\;J=1000$  & $\phantom{0}{-}0.0\phantom{0}$ & $\phantom{0}{-}7.5\phantom{0}$ & $\phantom{0}{-}1.7\phantom{0}$ & $\phantom{0}{-}0.3\phantom{0}$ & $\phantom{0}{-}1.7\phantom{0}$ & $\phantom{0}0.01\phantom{0}$ & $\phantom{0}0.02\phantom{0}$ & $\phantom{0}0.02\phantom{0}$ & $\phantom{0}0.02\phantom{0}$ & $\phantom{0}0.02\phantom{0}$ & $\phantom{0}93.0\phantom{0}$ & $\phantom{0}89.6\phantom{0}$ & $\phantom{0}95.0\phantom{0}$ & $\phantom{0}94.3\phantom{0}$ & $\phantom{0}94.6\phantom{0}$ \\
[0.5ex]\hline\\[-1.6ex] 
& & \multicolumn{15}{c}{Moderate intraclass correlation $(\rho_{Iy}=.30)$} \\[0.6ex]\hline\\[-1.8ex]
\multicolumn{4}{l}{$\bar{n}=5$} \\  & \nopagebreak $\;J=50$  & $\phantom{0}{-}0.2\phantom{0}$ & ${-}14.3\phantom{0}$ & ${-}14.6\phantom{0}$ & $\phantom{0}{-}6.4\phantom{0}$ & ${-}13.5\phantom{0}$ & $\phantom{0}0.10\phantom{0}$ & $\phantom{0}0.14\phantom{0}$ & $\phantom{0}0.13\phantom{0}$ & $\phantom{0}0.14\phantom{0}$ & $\phantom{0}0.13\phantom{0}$ & $\phantom{0}93.2\phantom{0}$ & $\phantom{0}93.8\phantom{0}$ & $\phantom{0}93.6\phantom{0}$ & $\phantom{0}90.4\phantom{0}$ & $\phantom{0}93.9\phantom{0}$ \\
 & \nopagebreak $\;J=200$  & $\phantom{0}{-}0.3\phantom{0}$ & $\phantom{0}{-}9.3\phantom{0}$ & $\phantom{0}{-}4.5\phantom{0}$ & $\phantom{0}{-}1.8\phantom{0}$ & $\phantom{0}{-}4.1\phantom{0}$ & $\phantom{0}0.05\phantom{0}$ & $\phantom{0}0.07\phantom{0}$ & $\phantom{0}0.06\phantom{0}$ & $\phantom{0}0.06\phantom{0}$ & $\phantom{0}0.06\phantom{0}$ & $\phantom{0}95.3\phantom{0}$ & $\phantom{0}94.5\phantom{0}$ & $\phantom{0}94.8\phantom{0}$ & $\phantom{0}94.9\phantom{0}$ & $\phantom{0}95.6\phantom{0}$ \\
 & \nopagebreak $\;J=1000$  & $\phantom{0}{-}0.3\phantom{0}$ & $\phantom{0}{-}7.9\phantom{0}$ & $\phantom{0}{-}1.8\phantom{0}$ & $\phantom{0}{-}0.4\phantom{0}$ & $\phantom{0}{-}0.7\phantom{0}$ & $\phantom{0}0.02\phantom{0}$ & $\phantom{0}0.03\phantom{0}$ & $\phantom{0}0.03\phantom{0}$ & $\phantom{0}0.03\phantom{0}$ & $\phantom{0}0.03\phantom{0}$ & $\phantom{0}95.2\phantom{0}$ & $\phantom{0}89.5\phantom{0}$ & $\phantom{0}95.1\phantom{0}$ & $\phantom{0}94.0\phantom{0}$ & $\phantom{0}95.8\phantom{0}$ \\
\multicolumn{4}{l}{$\bar{n}=20$} \\  & \nopagebreak $\;J=50$  & $\phantom{0}\phantom{-}1.7\phantom{0}$ & $\phantom{0}{-}5.7\phantom{0}$ & ${-}11.7\phantom{0}$ & $\phantom{0}{-}4.4\phantom{0}$ & $\phantom{0}{-}9.4\phantom{0}$ & $\phantom{0}0.08\phantom{0}$ & $\phantom{0}0.10\phantom{0}$ & $\phantom{0}0.11\phantom{0}$ & $\phantom{0}0.11\phantom{0}$ & $\phantom{0}0.10\phantom{0}$ & $\phantom{0}91.0\phantom{0}$ & $\phantom{0}95.0\phantom{0}$ & $\phantom{0}94.6\phantom{0}$ & $\phantom{0}93.1\phantom{0}$ & $\phantom{0}95.4\phantom{0}$ \\
 & \nopagebreak $\;J=200$  & $\phantom{0}\phantom{-}0.7\phantom{0}$ & $\phantom{0}{-}3.1\phantom{0}$ & $\phantom{0}{-}3.3\phantom{0}$ & $\phantom{0}{-}1.4\phantom{0}$ & $\phantom{0}{-}2.8\phantom{0}$ & $\phantom{0}0.04\phantom{0}$ & $\phantom{0}0.05\phantom{0}$ & $\phantom{0}0.05\phantom{0}$ & $\phantom{0}0.05\phantom{0}$ & $\phantom{0}0.05\phantom{0}$ & $\phantom{0}94.7\phantom{0}$ & $\phantom{0}94.5\phantom{0}$ & $\phantom{0}94.1\phantom{0}$ & $\phantom{0}94.7\phantom{0}$ & $\phantom{0}94.6\phantom{0}$ \\
 & \nopagebreak $\;J=1000$  & $\phantom{0}{-}0.1\phantom{0}$ & $\phantom{0}{-}1.6\phantom{0}$ & $\phantom{0}{-}0.8\phantom{0}$ & $\phantom{0}{-}0.2\phantom{0}$ & $\phantom{0}{-}0.4\phantom{0}$ & $\phantom{0}0.02\phantom{0}$ & $\phantom{0}0.02\phantom{0}$ & $\phantom{0}0.02\phantom{0}$ & $\phantom{0}0.02\phantom{0}$ & $\phantom{0}0.02\phantom{0}$ & $\phantom{0}95.1\phantom{0}$ & $\phantom{0}95.4\phantom{0}$ & $\phantom{0}95.7\phantom{0}$ & $\phantom{0}95.1\phantom{0}$ & $\phantom{0}95.5\phantom{0}$ \\
[0.5ex]\hline\\[-1.6ex] 
\end{tabular}
\begin{tablenotes}[para,flushleft]{\footnotesize \textit{Note.} $\bar{n}$ = average cluster size; $J$ = number of clusters; CD = complete data sets; LD = listwise deletion; FCS-SL = single-level FCS; FCS-MAN = two-level FCS with manifest cluster means; FCS-LAT = two-level FCS with latent cluster means; JM = joint modeling.}\end{tablenotes}
\end{threeparttable}
\end{sidewaystable}
\begin{sidewaystable}
\begin{threeparttable}
\setlength{\tabcolsep}{1.0pt}
\renewcommand{\arraystretch}{0.95}
\footnotesize
\caption{\small Study 2: Bias (in \%), Relative RMSE, and Coverage of the 95\% Confidence Interval for the Regression Coefficient of $z$ on $y$ ($\hat\beta_{zy}$) With Moderately Unbalanced Data (Uniform, $\pm 40\%$) and 20\% Missing Data (MAR, $\lambda=0.5$)}
\begin{tabular}{llccccccccccccccc}
\hline\\[-1.8ex]
& & \multicolumn{5}{c}{Bias (\%)} & \multicolumn{5}{c}{Rel. RMSE} & \multicolumn{5}{c}{Coverage (\%)} \\ \cmidrule(r){3-7}\cmidrule(r){8-12}\cmidrule(r){13-17}
 &  & CD & \makecell{FCS-\\MAN} & \makecell{FCS-\\NJ} & \makecell{FCS-\\LAT} & JM & CD & \makecell{FCS-\\MAN} & \makecell{FCS-\\NJ} & \makecell{FCS-\\LAT} & JM & CD & \makecell{FCS-\\MAN} & \makecell{FCS-\\NJ} & \makecell{FCS-\\LAT} & \multicolumn{1}{c}{JM} \\ 
[0.4ex]\hline\\[-1.8ex]
& & \multicolumn{15}{c}{Small intraclass correlation $(\rho_{Iy}=.10)$} \\[0.6ex]\hline\\[-1.8ex]
\multicolumn{4}{l}{$\bar{n}=5$} \\  & \nopagebreak $\;J=50$  & $\phantom{-}32.5\phantom{0}$ & $\phantom{-}27.3\phantom{0}$ & $\phantom{-}28.3\phantom{0}$ & $\phantom{-}33.3\phantom{0}$ & $\phantom{-}21.3\phantom{0}$ & $\phantom{0}1.80\phantom{0}$ & $\phantom{0}1.67\phantom{0}$ & $\phantom{0}1.72\phantom{0}$ & $\phantom{0}1.77\phantom{0}$ & $\phantom{0}1.71\phantom{0}$ & $\phantom{0}91.2\phantom{0}$ & $\phantom{0}92.2\phantom{0}$ & $\phantom{0}93.3\phantom{0}$ & $\phantom{0}92.4\phantom{0}$ & $\phantom{0}92.6\phantom{0}$ \\
 & \nopagebreak $\;J=200$  & $\phantom{0}\phantom{-}9.6\phantom{0}$ & $\phantom{0}\phantom{-}8.7\phantom{0}$ & $\phantom{0}\phantom{-}10.0\phantom{0}$ & $\phantom{-}12.1\phantom{0}$ & $\phantom{0}\phantom{-}5.2\phantom{0}$ & $\phantom{0}0.62\phantom{0}$ & $\phantom{0}0.67\phantom{0}$ & $\phantom{0}0.69\phantom{0}$ & $\phantom{0}0.71\phantom{0}$ & $\phantom{0}0.63\phantom{0}$ & $\phantom{0}95.9\phantom{0}$ & $\phantom{0}96.6\phantom{0}$ & $\phantom{0}96.5\phantom{0}$ & $\phantom{0}96.3\phantom{0}$ & $\phantom{0}95.9\phantom{0}$ \\
 & \nopagebreak $\;J=1000$  & $\phantom{0}\phantom{-}1.4\phantom{0}$ & $\phantom{0}{-}0.0\phantom{0}$ & $\phantom{0}\phantom{-}1.0\phantom{0}$ & $\phantom{0}\phantom{-}1.4\phantom{0}$ & $\phantom{0}{-}0.1\phantom{0}$ & $\phantom{0}0.22\phantom{0}$ & $\phantom{0}0.25\phantom{0}$ & $\phantom{0}0.25\phantom{0}$ & $\phantom{0}0.25\phantom{0}$ & $\phantom{0}0.25\phantom{0}$ & $\phantom{0}95.5\phantom{0}$ & $\phantom{0}95.5\phantom{0}$ & $\phantom{0}95.4\phantom{0}$ & $\phantom{0}95.2\phantom{0}$ & $\phantom{0}95.3\phantom{0}$ \\
\multicolumn{4}{l}{$\bar{n}=20$} \\  & \nopagebreak $\;J=50$  & $\phantom{0}\phantom{-}4.5\phantom{0}$ & $\phantom{0}\phantom{-}4.2\phantom{0}$ & $\phantom{0}\phantom{-}5.3\phantom{0}$ & $\phantom{0}\phantom{-}5.6\phantom{0}$ & $\phantom{0}{-}4.2\phantom{0}$ & $\phantom{0}0.61\phantom{0}$ & $\phantom{0}0.72\phantom{0}$ & $\phantom{0}0.74\phantom{0}$ & $\phantom{0}0.74\phantom{0}$ & $\phantom{0}0.65\phantom{0}$ & $\phantom{0}93.9\phantom{0}$ & $\phantom{0}94.4\phantom{0}$ & $\phantom{0}94.5\phantom{0}$ & $\phantom{0}93.0\phantom{0}$ & $\phantom{0}94.5\phantom{0}$ \\
 & \nopagebreak $\;J=200$  & $\phantom{0}\phantom{-}1.1\phantom{0}$ & $\phantom{0}\phantom{-}0.4\phantom{0}$ & $\phantom{0}\phantom{-}0.6\phantom{0}$ & $\phantom{0}\phantom{-}0.7\phantom{0}$ & $\phantom{0}{-}2.5\phantom{0}$ & $\phantom{0}0.27\phantom{0}$ & $\phantom{0}0.31\phantom{0}$ & $\phantom{0}0.31\phantom{0}$ & $\phantom{0}0.31\phantom{0}$ & $\phantom{0}0.30\phantom{0}$ & $\phantom{0}94.3\phantom{0}$ & $\phantom{0}94.6\phantom{0}$ & $\phantom{0}94.0\phantom{0}$ & $\phantom{0}94.0\phantom{0}$ & $\phantom{0}94.6\phantom{0}$ \\
 & \nopagebreak $\;J=1000$  & $\phantom{0}{-}0.0\phantom{0}$ & $\phantom{0}{-}0.3\phantom{0}$ & $\phantom{0}{-}0.1\phantom{0}$ & $\phantom{0}\phantom{-}0.0\phantom{0}$ & $\phantom{0}{-}0.6\phantom{0}$ & $\phantom{0}0.12\phantom{0}$ & $\phantom{0}0.13\phantom{0}$ & $\phantom{0}0.14\phantom{0}$ & $\phantom{0}0.13\phantom{0}$ & $\phantom{0}0.13\phantom{0}$ & $\phantom{0}95.1\phantom{0}$ & $\phantom{0}95.1\phantom{0}$ & $\phantom{0}94.4\phantom{0}$ & $\phantom{0}94.6\phantom{0}$ & $\phantom{0}95.6\phantom{0}$ \\
[0.5ex]\hline\\[-1.6ex] 
& & \multicolumn{15}{c}{Moderate intraclass correlation $(\rho_{Iy}=.30)$} \\[0.6ex]\hline\\[-1.8ex]
\multicolumn{4}{l}{$\bar{n}=5$} \\  & \nopagebreak $\;J=50$  & $\phantom{-}3.4\phantom{0}$ & $\phantom{-}3.6\phantom{0}$ & $\phantom{-}4.3\phantom{0}$ & $\phantom{-}5.0\phantom{0}$ & ${-}0.7\phantom{0}$ & $\phantom{0}0.35\phantom{0}$ & $\phantom{0}0.40\phantom{0}$ & $\phantom{0}0.41\phantom{0}$ & $\phantom{0}0.41\phantom{0}$ & $\phantom{0}0.37\phantom{0}$ & $\phantom{0}93.4\phantom{0}$ & $\phantom{0}94.2\phantom{0}$ & $\phantom{0}95.1\phantom{0}$ & $\phantom{0}93.8\phantom{0}$ & $\phantom{0}94.4\phantom{0}$ \\
 & \nopagebreak $\;J=200$  & $\phantom{-}0.4\phantom{0}$ & ${-}0.1\phantom{0}$ & $\phantom{-}0.3\phantom{0}$ & $\phantom{-}0.4\phantom{0}$ & ${-}1.0\phantom{0}$ & $\phantom{0}0.16\phantom{0}$ & $\phantom{0}0.18\phantom{0}$ & $\phantom{0}0.18\phantom{0}$ & $\phantom{0}0.18\phantom{0}$ & $\phantom{0}0.18\phantom{0}$ & $\phantom{0}94.2\phantom{0}$ & $\phantom{0}94.4\phantom{0}$ & $\phantom{0}94.9\phantom{0}$ & $\phantom{0}94.2\phantom{0}$ & $\phantom{0}95.3\phantom{0}$ \\
 & \nopagebreak $\;J=1000$  & $\phantom{-}0.3\phantom{0}$ & ${-}0.2\phantom{0}$ & $\phantom{-}0.1\phantom{0}$ & $\phantom{-}0.2\phantom{0}$ & ${-}0.0\phantom{0}$ & $\phantom{0}0.07\phantom{0}$ & $\phantom{0}0.08\phantom{0}$ & $\phantom{0}0.08\phantom{0}$ & $\phantom{0}0.08\phantom{0}$ & $\phantom{0}0.08\phantom{0}$ & $\phantom{0}94.7\phantom{0}$ & $\phantom{0}95.4\phantom{0}$ & $\phantom{0}96.2\phantom{0}$ & $\phantom{0}95.0\phantom{0}$ & $\phantom{0}95.2\phantom{0}$ \\
\multicolumn{4}{l}{$\bar{n}=20$} \\  & \nopagebreak $\;J=50$  & $\phantom{-}2.0\phantom{0}$ & $\phantom{-}1.6\phantom{0}$ & $\phantom{-}1.2\phantom{0}$ & $\phantom{-}1.5\phantom{0}$ & ${-}2.2\phantom{0}$ & $\phantom{0}0.25\phantom{0}$ & $\phantom{0}0.30\phantom{0}$ & $\phantom{0}0.31\phantom{0}$ & $\phantom{0}0.30\phantom{0}$ & $\phantom{0}0.29\phantom{0}$ & $\phantom{0}90.8\phantom{0}$ & $\phantom{0}92.7\phantom{0}$ & $\phantom{0}93.1\phantom{0}$ & $\phantom{0}92.7\phantom{0}$ & $\phantom{0}93.8\phantom{0}$ \\
 & \nopagebreak $\;J=200$  & $\phantom{-}0.6\phantom{0}$ & $\phantom{-}0.7\phantom{0}$ & $\phantom{-}0.8\phantom{0}$ & $\phantom{-}0.7\phantom{0}$ & ${-}0.1\phantom{0}$ & $\phantom{0}0.12\phantom{0}$ & $\phantom{0}0.14\phantom{0}$ & $\phantom{0}0.14\phantom{0}$ & $\phantom{0}0.15\phantom{0}$ & $\phantom{0}0.14\phantom{0}$ & $\phantom{0}93.7\phantom{0}$ & $\phantom{0}94.2\phantom{0}$ & $\phantom{0}94.4\phantom{0}$ & $\phantom{0}94.4\phantom{0}$ & $\phantom{0}94.7\phantom{0}$ \\
 & \nopagebreak $\;J=1000$  & ${-}0.2\phantom{0}$ & $\phantom{-}0.1\phantom{0}$ & $\phantom{-}0.0\phantom{0}$ & $\phantom{-}0.1\phantom{0}$ & ${-}0.1\phantom{0}$ & $\phantom{0}0.05\phantom{0}$ & $\phantom{0}0.06\phantom{0}$ & $\phantom{0}0.06\phantom{0}$ & $\phantom{0}0.06\phantom{0}$ & $\phantom{0}0.06\phantom{0}$ & $\phantom{0}95.1\phantom{0}$ & $\phantom{0}95.1\phantom{0}$ & $\phantom{0}95.5\phantom{0}$ & $\phantom{0}95.2\phantom{0}$ & $\phantom{0}95.1\phantom{0}$ \\
[0.5ex]\hline\\[-1.6ex] 
\end{tabular}
\begin{tablenotes}[para,flushleft]{\footnotesize \textit{Note.} $\bar{n}$ = average cluster size; $J$ = number of clusters; CD = complete data sets; LD = listwise deletion; FCS-SL = single-level FCS; FCS-MAN = two-level FCS with manifest cluster means; FCS-LAT = two-level FCS with latent cluster means; JM = joint modeling.}\end{tablenotes}
\end{threeparttable}
\end{sidewaystable}
\begin{sidewaystable}
\begin{threeparttable}
\setlength{\tabcolsep}{1.0pt}
\renewcommand{\arraystretch}{0.95}
\footnotesize
\caption{\small Study 2: Bias (in \%), Relative RMSE, and Coverage of the 95\% Confidence Interval for the Regression Coefficient of $z$ on $y$ ($\hat\beta_{zy}$) With Strongly Unbalanced Data (Uniform, $\pm 80\%$) and 20\% Missing Data (MAR, $\lambda=0.5$)}
\begin{tabular}{llccccccccccccccc}
\hline\\[-1.8ex]
& & \multicolumn{5}{c}{Bias (\%)} & \multicolumn{5}{c}{Rel. RMSE} & \multicolumn{5}{c}{Coverage (\%)} \\ \cmidrule(r){3-7}\cmidrule(r){8-12}\cmidrule(r){13-17}
 &  & CD & \makecell{FCS-\\MAN} & \makecell{FCS-\\NJ} & \makecell{FCS-\\LAT} & JM & CD & \makecell{FCS-\\MAN} & \makecell{FCS-\\NJ} & \makecell{FCS-\\LAT} & JM & CD & \makecell{FCS-\\MAN} & \makecell{FCS-\\NJ} & \makecell{FCS-\\LAT} & \multicolumn{1}{c}{JM} \\ 
[0.4ex]\hline\\[-1.8ex]
& & \multicolumn{15}{c}{Small intraclass correlation $(\rho_{Iy}=.10)$} \\[0.6ex]\hline\\[-1.8ex]
\multicolumn{4}{l}{$\bar{n}=5$} \\  & \nopagebreak $\;J=50$  & $\phantom{-}35.2\phantom{0}$ & $\phantom{-}25.5\phantom{0}$ & $\phantom{-}32.9\phantom{0}$ & $\phantom{-}35.0\phantom{0}$ & $\phantom{-}24.5\phantom{0}$ & $\phantom{0}2.29\phantom{0}$ & $\phantom{0}1.77\phantom{0}$ & $\phantom{0}1.89\phantom{0}$ & $\phantom{0}1.89\phantom{0}$ & $\phantom{0}1.82\phantom{0}$ & $\phantom{0}92.2\phantom{0}$ & $\phantom{0}93.8\phantom{0}$ & $\phantom{0}94.8\phantom{0}$ & $\phantom{0}93.3\phantom{0}$ & $\phantom{0}93.5\phantom{0}$ \\
 & \nopagebreak $\;J=200$  & $\phantom{0}\phantom{-}9.2\phantom{0}$ & $\phantom{0}\phantom{-}3.8\phantom{0}$ & $\phantom{0}\phantom{-}9.6\phantom{0}$ & $\phantom{-}11.3\phantom{0}$ & $\phantom{0}\phantom{-}5.0\phantom{0}$ & $\phantom{0}0.64\phantom{0}$ & $\phantom{0}0.63\phantom{0}$ & $\phantom{0}0.70\phantom{0}$ & $\phantom{0}0.72\phantom{0}$ & $\phantom{0}0.65\phantom{0}$ & $\phantom{0}94.7\phantom{0}$ & $\phantom{0}94.9\phantom{0}$ & $\phantom{0}95.1\phantom{0}$ & $\phantom{0}95.4\phantom{0}$ & $\phantom{0}94.7\phantom{0}$ \\
 & \nopagebreak $\;J=1000$  & $\phantom{0}\phantom{-}1.4\phantom{0}$ & $\phantom{0}{-}3.7\phantom{0}$ & $\phantom{0}\phantom{-}1.1\phantom{0}$ & $\phantom{0}\phantom{-}1.5\phantom{0}$ & $\phantom{0}\phantom{-}0.1\phantom{0}$ & $\phantom{0}0.22\phantom{0}$ & $\phantom{0}0.23\phantom{0}$ & $\phantom{0}0.24\phantom{0}$ & $\phantom{0}0.25\phantom{0}$ & $\phantom{0}0.24\phantom{0}$ & $\phantom{0}95.3\phantom{0}$ & $\phantom{0}93.0\phantom{0}$ & $\phantom{0}94.7\phantom{0}$ & $\phantom{0}94.8\phantom{0}$ & $\phantom{0}95.0\phantom{0}$ \\
\multicolumn{4}{l}{$\bar{n}=20$} \\  & \nopagebreak $\;J=50$  & $\phantom{0}\phantom{-}8.2\phantom{0}$ & $\phantom{0}\phantom{-}7.6\phantom{0}$ & $\phantom{0}\phantom{-}9.0\phantom{0}$ & $\phantom{-}10.3\phantom{0}$ & $\phantom{0}{-}0.6\phantom{0}$ & $\phantom{0}0.63\phantom{0}$ & $\phantom{0}0.70\phantom{0}$ & $\phantom{0}0.74\phantom{0}$ & $\phantom{0}0.74\phantom{0}$ & $\phantom{0}0.64\phantom{0}$ & $\phantom{0}93.2\phantom{0}$ & $\phantom{0}93.9\phantom{0}$ & $\phantom{0}94.6\phantom{0}$ & $\phantom{0}93.0\phantom{0}$ & $\phantom{0}94.9\phantom{0}$ \\
 & \nopagebreak $\;J=200$  & $\phantom{0}\phantom{-}1.5\phantom{0}$ & $\phantom{0}\phantom{-}0.4\phantom{0}$ & $\phantom{0}\phantom{-}1.7\phantom{0}$ & $\phantom{0}\phantom{-}2.0\phantom{0}$ & $\phantom{0}{-}1.1\phantom{0}$ & $\phantom{0}0.27\phantom{0}$ & $\phantom{0}0.32\phantom{0}$ & $\phantom{0}0.32\phantom{0}$ & $\phantom{0}0.32\phantom{0}$ & $\phantom{0}0.31\phantom{0}$ & $\phantom{0}94.8\phantom{0}$ & $\phantom{0}94.4\phantom{0}$ & $\phantom{0}95.5\phantom{0}$ & $\phantom{0}94.3\phantom{0}$ & $\phantom{0}95.0\phantom{0}$ \\
 & \nopagebreak $\;J=1000$  & $\phantom{0}\phantom{-}0.2\phantom{0}$ & $\phantom{0}{-}1.3\phantom{0}$ & $\phantom{0}{-}0.1\phantom{0}$ & $\phantom{0}\phantom{-}0.1\phantom{0}$ & $\phantom{0}{-}0.6\phantom{0}$ & $\phantom{0}0.12\phantom{0}$ & $\phantom{0}0.14\phantom{0}$ & $\phantom{0}0.14\phantom{0}$ & $\phantom{0}0.14\phantom{0}$ & $\phantom{0}0.14\phantom{0}$ & $\phantom{0}94.8\phantom{0}$ & $\phantom{0}94.1\phantom{0}$ & $\phantom{0}94.3\phantom{0}$ & $\phantom{0}93.9\phantom{0}$ & $\phantom{0}95.1\phantom{0}$ \\
[0.5ex]\hline\\[-1.6ex] 
& & \multicolumn{15}{c}{Moderate intraclass correlation $(\rho_{Iy}=.30)$} \\[0.6ex]\hline\\[-1.8ex]
\multicolumn{4}{l}{$\bar{n}=5$} \\  & \nopagebreak $\;J=50$  & $\phantom{-}5.5\phantom{0}$ & $\phantom{-}2.9\phantom{0}$ & $\phantom{-}5.4\phantom{0}$ & $\phantom{-}6.2\phantom{0}$ & ${-}0.7\phantom{0}$ & $\phantom{0}0.39\phantom{0}$ & $\phantom{0}0.44\phantom{0}$ & $\phantom{0}0.48\phantom{0}$ & $\phantom{0}0.47\phantom{0}$ & $\phantom{0}0.42\phantom{0}$ & $\phantom{0}93.2\phantom{0}$ & $\phantom{0}93.4\phantom{0}$ & $\phantom{0}93.7\phantom{0}$ & $\phantom{0}92.5\phantom{0}$ & $\phantom{0}94.3\phantom{0}$ \\
 & \nopagebreak $\;J=200$  & $\phantom{-}1.7\phantom{0}$ & ${-}0.6\phantom{0}$ & $\phantom{-}1.3\phantom{0}$ & $\phantom{-}1.4\phantom{0}$ & $\phantom{-}0.3\phantom{0}$ & $\phantom{0}0.17\phantom{0}$ & $\phantom{0}0.19\phantom{0}$ & $\phantom{0}0.19\phantom{0}$ & $\phantom{0}0.19\phantom{0}$ & $\phantom{0}0.19\phantom{0}$ & $\phantom{0}93.7\phantom{0}$ & $\phantom{0}94.5\phantom{0}$ & $\phantom{0}93.8\phantom{0}$ & $\phantom{0}93.5\phantom{0}$ & $\phantom{0}95.1\phantom{0}$ \\
 & \nopagebreak $\;J=1000$  & $\phantom{-}0.4\phantom{0}$ & ${-}1.6\phantom{0}$ & $\phantom{-}0.2\phantom{0}$ & $\phantom{-}0.5\phantom{0}$ & $\phantom{-}0.3\phantom{0}$ & $\phantom{0}0.07\phantom{0}$ & $\phantom{0}0.08\phantom{0}$ & $\phantom{0}0.08\phantom{0}$ & $\phantom{0}0.08\phantom{0}$ & $\phantom{0}0.08\phantom{0}$ & $\phantom{0}95.5\phantom{0}$ & $\phantom{0}95.1\phantom{0}$ & $\phantom{0}95.3\phantom{0}$ & $\phantom{0}95.0\phantom{0}$ & $\phantom{0}94.9\phantom{0}$ \\
\multicolumn{4}{l}{$\bar{n}=20$} \\  & \nopagebreak $\;J=50$  & $\phantom{-}2.9\phantom{0}$ & $\phantom{-}3.5\phantom{0}$ & $\phantom{-}4.3\phantom{0}$ & $\phantom{-}4.4\phantom{0}$ & $\phantom{-}0.3\phantom{0}$ & $\phantom{0}0.26\phantom{0}$ & $\phantom{0}0.31\phantom{0}$ & $\phantom{0}0.33\phantom{0}$ & $\phantom{0}0.32\phantom{0}$ & $\phantom{0}0.30\phantom{0}$ & $\phantom{0}91.8\phantom{0}$ & $\phantom{0}92.7\phantom{0}$ & $\phantom{0}93.0\phantom{0}$ & $\phantom{0}92.7\phantom{0}$ & $\phantom{0}93.4\phantom{0}$ \\
 & \nopagebreak $\;J=200$  & ${-}0.3\phantom{0}$ & ${-}0.7\phantom{0}$ & ${-}0.2\phantom{0}$ & ${-}0.3\phantom{0}$ & ${-}1.0\phantom{0}$ & $\phantom{0}0.12\phantom{0}$ & $\phantom{0}0.15\phantom{0}$ & $\phantom{0}0.15\phantom{0}$ & $\phantom{0}0.15\phantom{0}$ & $\phantom{0}0.15\phantom{0}$ & $\phantom{0}94.3\phantom{0}$ & $\phantom{0}94.0\phantom{0}$ & $\phantom{0}94.2\phantom{0}$ & $\phantom{0}94.1\phantom{0}$ & $\phantom{0}94.6\phantom{0}$ \\
 & \nopagebreak $\;J=1000$  & $\phantom{-}0.0\phantom{0}$ & ${-}0.3\phantom{0}$ & ${-}0.2\phantom{0}$ & ${-}0.0\phantom{0}$ & ${-}0.3\phantom{0}$ & $\phantom{0}0.06\phantom{0}$ & $\phantom{0}0.07\phantom{0}$ & $\phantom{0}0.07\phantom{0}$ & $\phantom{0}0.07\phantom{0}$ & $\phantom{0}0.07\phantom{0}$ & $\phantom{0}94.5\phantom{0}$ & $\phantom{0}94.5\phantom{0}$ & $\phantom{0}94.7\phantom{0}$ & $\phantom{0}94.9\phantom{0}$ & $\phantom{0}94.2\phantom{0}$ \\
[0.5ex]\hline\\[-1.6ex] 
\end{tabular}
\begin{tablenotes}[para,flushleft]{\footnotesize \textit{Note.} $\bar{n}$ = average cluster size; $J$ = number of clusters; CD = complete data sets; LD = listwise deletion; FCS-SL = single-level FCS; FCS-MAN = two-level FCS with manifest cluster means; FCS-LAT = two-level FCS with latent cluster means; JM = joint modeling.}\end{tablenotes}
\end{threeparttable}
\end{sidewaystable}
\begin{sidewaystable}
\begin{threeparttable}
\setlength{\tabcolsep}{1.0pt}
\renewcommand{\arraystretch}{0.95}
\footnotesize
\caption{\small Study 2: Bias (in \%), Relative RMSE, and Coverage of the 95\% Confidence Interval for the Regression Coefficient of $z$ on $y$ ($\hat\beta_{zy}$) With Moderately Unbalanced Data (Bimodal, $\pm 40\%$) and 20\% Missing Data (MAR, $\lambda=0.5$)}
\begin{tabular}{llccccccccccccccc}
\hline\\[-1.8ex]
& & \multicolumn{5}{c}{Bias (\%)} & \multicolumn{5}{c}{Rel. RMSE} & \multicolumn{5}{c}{Coverage (\%)} \\ \cmidrule(r){3-7}\cmidrule(r){8-12}\cmidrule(r){13-17}
 &  & CD & \makecell{FCS-\\MAN} & \makecell{FCS-\\NJ} & \makecell{FCS-\\LAT} & JM & CD & \makecell{FCS-\\MAN} & \makecell{FCS-\\NJ} & \makecell{FCS-\\LAT} & JM & CD & \makecell{FCS-\\MAN} & \makecell{FCS-\\NJ} & \makecell{FCS-\\LAT} & \multicolumn{1}{c}{JM} \\ 
[0.4ex]\hline\\[-1.8ex]
& & \multicolumn{15}{c}{Small intraclass correlation $(\rho_{Iy}=.10)$} \\[0.6ex]\hline\\[-1.8ex]
\multicolumn{4}{l}{$\bar{n}=5$} \\  & \nopagebreak $\;J=50$  & $\phantom{-}26.7\phantom{0}$ & $\phantom{-}24.8\phantom{0}$ & $\phantom{-}26.3\phantom{0}$ & $\phantom{-}30.1\phantom{0}$ & $\phantom{-}20.0\phantom{0}$ & $\phantom{0}2.56\phantom{0}$ & $\phantom{0}1.70\phantom{0}$ & $\phantom{0}1.80\phantom{0}$ & $\phantom{0}1.80\phantom{0}$ & $\phantom{0}1.77\phantom{0}$ & $\phantom{0}92.1\phantom{0}$ & $\phantom{0}94.5\phantom{0}$ & $\phantom{0}94.7\phantom{0}$ & $\phantom{0}94.1\phantom{0}$ & $\phantom{0}94.7\phantom{0}$ \\
 & \nopagebreak $\;J=200$  & $\phantom{0}\phantom{-}9.0\phantom{0}$ & $\phantom{0}\phantom{-}6.3\phantom{0}$ & $\phantom{0}\phantom{-}9.2\phantom{0}$ & $\phantom{-}10.9\phantom{0}$ & $\phantom{0}\phantom{-}4.4\phantom{0}$ & $\phantom{0}0.64\phantom{0}$ & $\phantom{0}0.65\phantom{0}$ & $\phantom{0}0.69\phantom{0}$ & $\phantom{0}0.72\phantom{0}$ & $\phantom{0}0.63\phantom{0}$ & $\phantom{0}94.8\phantom{0}$ & $\phantom{0}94.9\phantom{0}$ & $\phantom{0}95.2\phantom{0}$ & $\phantom{0}95.8\phantom{0}$ & $\phantom{0}94.7\phantom{0}$ \\
 & \nopagebreak $\;J=1000$  & $\phantom{0}\phantom{-}1.2\phantom{0}$ & $\phantom{0}{-}1.1\phantom{0}$ & $\phantom{0}\phantom{-}1.2\phantom{0}$ & $\phantom{0}\phantom{-}1.5\phantom{0}$ & $\phantom{0}\phantom{-}0.0\phantom{0}$ & $\phantom{0}0.22\phantom{0}$ & $\phantom{0}0.24\phantom{0}$ & $\phantom{0}0.25\phantom{0}$ & $\phantom{0}0.25\phantom{0}$ & $\phantom{0}0.24\phantom{0}$ & $\phantom{0}95.1\phantom{0}$ & $\phantom{0}94.5\phantom{0}$ & $\phantom{0}95.1\phantom{0}$ & $\phantom{0}95.1\phantom{0}$ & $\phantom{0}94.3\phantom{0}$ \\
\multicolumn{4}{l}{$\bar{n}=20$} \\  & \nopagebreak $\;J=50$  & $\phantom{0}\phantom{-}4.7\phantom{0}$ & $\phantom{0}\phantom{-}4.7\phantom{0}$ & $\phantom{0}\phantom{-}5.5\phantom{0}$ & $\phantom{0}\phantom{-}6.3\phantom{0}$ & $\phantom{0}{-}4.6\phantom{0}$ & $\phantom{0}0.60\phantom{0}$ & $\phantom{0}0.71\phantom{0}$ & $\phantom{0}0.75\phantom{0}$ & $\phantom{0}0.75\phantom{0}$ & $\phantom{0}0.65\phantom{0}$ & $\phantom{0}93.5\phantom{0}$ & $\phantom{0}93.9\phantom{0}$ & $\phantom{0}94.8\phantom{0}$ & $\phantom{0}92.7\phantom{0}$ & $\phantom{0}94.6\phantom{0}$ \\
 & \nopagebreak $\;J=200$  & $\phantom{0}\phantom{-}1.1\phantom{0}$ & $\phantom{0}{-}0.1\phantom{0}$ & $\phantom{0}\phantom{-}0.3\phantom{0}$ & $\phantom{0}\phantom{-}0.5\phantom{0}$ & $\phantom{0}{-}2.5\phantom{0}$ & $\phantom{0}0.26\phantom{0}$ & $\phantom{0}0.30\phantom{0}$ & $\phantom{0}0.30\phantom{0}$ & $\phantom{0}0.31\phantom{0}$ & $\phantom{0}0.29\phantom{0}$ & $\phantom{0}94.9\phantom{0}$ & $\phantom{0}96.0\phantom{0}$ & $\phantom{0}96.0\phantom{0}$ & $\phantom{0}95.2\phantom{0}$ & $\phantom{0}95.9\phantom{0}$ \\
 & \nopagebreak $\;J=1000$  & $\phantom{0}\phantom{-}0.2\phantom{0}$ & $\phantom{0}{-}0.2\phantom{0}$ & $\phantom{0}\phantom{-}0.3\phantom{0}$ & $\phantom{0}\phantom{-}0.4\phantom{0}$ & $\phantom{0}{-}0.2\phantom{0}$ & $\phantom{0}0.12\phantom{0}$ & $\phantom{0}0.14\phantom{0}$ & $\phantom{0}0.14\phantom{0}$ & $\phantom{0}0.14\phantom{0}$ & $\phantom{0}0.14\phantom{0}$ & $\phantom{0}94.8\phantom{0}$ & $\phantom{0}94.4\phantom{0}$ & $\phantom{0}94.3\phantom{0}$ & $\phantom{0}94.9\phantom{0}$ & $\phantom{0}94.5\phantom{0}$ \\
[0.5ex]\hline\\[-1.6ex] 
& & \multicolumn{15}{c}{Moderate intraclass correlation $(\rho_{Iy}=.30)$} \\[0.6ex]\hline\\[-1.8ex]
\multicolumn{4}{l}{$\bar{n}=5$} \\  & \nopagebreak $\;J=50$  & $\phantom{-}4.7\phantom{0}$ & $\phantom{-}4.5\phantom{0}$ & $\phantom{-}5.4\phantom{0}$ & $\phantom{-}6.4\phantom{0}$ & $\phantom{-}0.6\phantom{0}$ & $\phantom{0}0.37\phantom{0}$ & $\phantom{0}0.40\phantom{0}$ & $\phantom{0}0.42\phantom{0}$ & $\phantom{0}0.42\phantom{0}$ & $\phantom{0}0.38\phantom{0}$ & $\phantom{0}93.5\phantom{0}$ & $\phantom{0}95.1\phantom{0}$ & $\phantom{0}95.2\phantom{0}$ & $\phantom{0}94.6\phantom{0}$ & $\phantom{0}95.7\phantom{0}$ \\
 & \nopagebreak $\;J=200$  & $\phantom{-}1.4\phantom{0}$ & $\phantom{-}0.7\phantom{0}$ & $\phantom{-}1.0\phantom{0}$ & $\phantom{-}1.5\phantom{0}$ & $\phantom{-}0.2\phantom{0}$ & $\phantom{0}0.16\phantom{0}$ & $\phantom{0}0.18\phantom{0}$ & $\phantom{0}0.18\phantom{0}$ & $\phantom{0}0.19\phantom{0}$ & $\phantom{0}0.18\phantom{0}$ & $\phantom{0}95.4\phantom{0}$ & $\phantom{0}93.9\phantom{0}$ & $\phantom{0}95.0\phantom{0}$ & $\phantom{0}93.6\phantom{0}$ & $\phantom{0}94.8\phantom{0}$ \\
 & \nopagebreak $\;J=1000$  & $\phantom{-}0.3\phantom{0}$ & ${-}0.4\phantom{0}$ & $\phantom{-}0.2\phantom{0}$ & $\phantom{-}0.4\phantom{0}$ & $\phantom{-}0.1\phantom{0}$ & $\phantom{0}0.07\phantom{0}$ & $\phantom{0}0.08\phantom{0}$ & $\phantom{0}0.08\phantom{0}$ & $\phantom{0}0.08\phantom{0}$ & $\phantom{0}0.08\phantom{0}$ & $\phantom{0}96.5\phantom{0}$ & $\phantom{0}95.0\phantom{0}$ & $\phantom{0}96.2\phantom{0}$ & $\phantom{0}95.3\phantom{0}$ & $\phantom{0}95.4\phantom{0}$ \\
\multicolumn{4}{l}{$\bar{n}=20$} \\  & \nopagebreak $\;J=50$  & $\phantom{-}0.2\phantom{0}$ & $\phantom{-}0.1\phantom{0}$ & $\phantom{-}0.4\phantom{0}$ & $\phantom{-}0.4\phantom{0}$ & ${-}3.2\phantom{0}$ & $\phantom{0}0.26\phantom{0}$ & $\phantom{0}0.31\phantom{0}$ & $\phantom{0}0.32\phantom{0}$ & $\phantom{0}0.31\phantom{0}$ & $\phantom{0}0.30\phantom{0}$ & $\phantom{0}90.9\phantom{0}$ & $\phantom{0}92.2\phantom{0}$ & $\phantom{0}92.3\phantom{0}$ & $\phantom{0}92.4\phantom{0}$ & $\phantom{0}92.9\phantom{0}$ \\
 & \nopagebreak $\;J=200$  & $\phantom{-}0.6\phantom{0}$ & $\phantom{-}1.0\phantom{0}$ & $\phantom{-}0.9\phantom{0}$ & $\phantom{-}1.0\phantom{0}$ & $\phantom{-}0.2\phantom{0}$ & $\phantom{0}0.12\phantom{0}$ & $\phantom{0}0.15\phantom{0}$ & $\phantom{0}0.15\phantom{0}$ & $\phantom{0}0.15\phantom{0}$ & $\phantom{0}0.14\phantom{0}$ & $\phantom{0}94.5\phantom{0}$ & $\phantom{0}94.6\phantom{0}$ & $\phantom{0}93.9\phantom{0}$ & $\phantom{0}93.0\phantom{0}$ & $\phantom{0}94.8\phantom{0}$ \\
 & \nopagebreak $\;J=1000$  & $\phantom{-}0.3\phantom{0}$ & $\phantom{-}0.3\phantom{0}$ & $\phantom{-}0.4\phantom{0}$ & $\phantom{-}0.4\phantom{0}$ & $\phantom{-}0.2\phantom{0}$ & $\phantom{0}0.05\phantom{0}$ & $\phantom{0}0.07\phantom{0}$ & $\phantom{0}0.07\phantom{0}$ & $\phantom{0}0.07\phantom{0}$ & $\phantom{0}0.07\phantom{0}$ & $\phantom{0}95.5\phantom{0}$ & $\phantom{0}94.2\phantom{0}$ & $\phantom{0}94.6\phantom{0}$ & $\phantom{0}94.8\phantom{0}$ & $\phantom{0}94.0\phantom{0}$ \\
[0.5ex]\hline\\[-1.6ex] 
\end{tabular}
\begin{tablenotes}[para,flushleft]{\footnotesize \textit{Note.} $\bar{n}$ = average cluster size; $J$ = number of clusters; CD = complete data sets; LD = listwise deletion; FCS-SL = single-level FCS; FCS-MAN = two-level FCS with manifest cluster means; FCS-LAT = two-level FCS with latent cluster means; JM = joint modeling.}\end{tablenotes}
\end{threeparttable}
\end{sidewaystable}
\begin{sidewaystable}
\begin{threeparttable}
\setlength{\tabcolsep}{1.0pt}
\renewcommand{\arraystretch}{0.95}
\footnotesize
\caption{\small Study 2: Bias (in \%), Relative RMSE, and Coverage of the 95\% Confidence Interval for the Regression Coefficient of $z$ on $y$ ($\hat\beta_{zy}$) With Strongly Unbalanced Data (Bimodal, $\pm 80\%$) and 20\% Missing Data (MAR, $\lambda=0.5$)}
\begin{tabular}{llccccccccccccccc}
\hline\\[-1.8ex]
& & \multicolumn{5}{c}{Bias (\%)} & \multicolumn{5}{c}{Rel. RMSE} & \multicolumn{5}{c}{Coverage (\%)} \\ \cmidrule(r){3-7}\cmidrule(r){8-12}\cmidrule(r){13-17}
 &  & CD & \makecell{FCS-\\MAN} & \makecell{FCS-\\NJ} & \makecell{FCS-\\LAT} & JM & CD & \makecell{FCS-\\MAN} & \makecell{FCS-\\NJ} & \makecell{FCS-\\LAT} & JM & CD & \makecell{FCS-\\MAN} & \makecell{FCS-\\NJ} & \makecell{FCS-\\LAT} & \multicolumn{1}{c}{JM} \\ 
[0.4ex]\hline\\[-1.8ex]
& & \multicolumn{15}{c}{Small intraclass correlation $(\rho_{Iy}=.10)$} \\[0.6ex]\hline\\[-1.8ex]
\multicolumn{4}{l}{$\bar{n}=5$} \\  & \nopagebreak $\;J=50$  & $\phantom{-}31.4\phantom{0}$ & $\phantom{-}16.0\phantom{0}$ & $\phantom{-}25.3\phantom{0}$ & $\phantom{-}29.7\phantom{0}$ & $\phantom{-}17.3\phantom{0}$ & $\phantom{0}2.14\phantom{0}$ & $\phantom{0}1.72\phantom{0}$ & $\phantom{0}1.69\phantom{0}$ & $\phantom{0}1.77\phantom{0}$ & $\phantom{0}1.67\phantom{0}$ & $\phantom{0}92.0\phantom{0}$ & $\phantom{0}94.1\phantom{0}$ & $\phantom{0}94.5\phantom{0}$ & $\phantom{0}93.6\phantom{0}$ & $\phantom{0}93.8\phantom{0}$ \\
 & \nopagebreak $\;J=200$  & $\phantom{0}\phantom{-}8.0\phantom{0}$ & $\phantom{0}{-}2.2\phantom{0}$ & $\phantom{0}\phantom{-}7.7\phantom{0}$ & $\phantom{0}\phantom{-}9.5\phantom{0}$ & $\phantom{0}\phantom{-}3.9\phantom{0}$ & $\phantom{0}0.59\phantom{0}$ & $\phantom{0}0.56\phantom{0}$ & $\phantom{0}0.63\phantom{0}$ & $\phantom{0}0.65\phantom{0}$ & $\phantom{0}0.59\phantom{0}$ & $\phantom{0}95.3\phantom{0}$ & $\phantom{0}93.8\phantom{0}$ & $\phantom{0}95.4\phantom{0}$ & $\phantom{0}95.3\phantom{0}$ & $\phantom{0}95.2\phantom{0}$ \\
 & \nopagebreak $\;J=1000$  & $\phantom{0}\phantom{-}0.8\phantom{0}$ & $\phantom{0}{-}8.6\phantom{0}$ & $\phantom{0}\phantom{-}0.2\phantom{0}$ & $\phantom{0}\phantom{-}0.8\phantom{0}$ & $\phantom{0}{-}0.3\phantom{0}$ & $\phantom{0}0.21\phantom{0}$ & $\phantom{0}0.25\phantom{0}$ & $\phantom{0}0.23\phantom{0}$ & $\phantom{0}0.23\phantom{0}$ & $\phantom{0}0.23\phantom{0}$ & $\phantom{0}95.7\phantom{0}$ & $\phantom{0}91.2\phantom{0}$ & $\phantom{0}95.7\phantom{0}$ & $\phantom{0}94.8\phantom{0}$ & $\phantom{0}95.0\phantom{0}$ \\
\multicolumn{4}{l}{$\bar{n}=20$} \\  & \nopagebreak $\;J=50$  & $\phantom{0}\phantom{-}5.7\phantom{0}$ & $\phantom{0}{-}0.8\phantom{0}$ & $\phantom{0}\phantom{-}5.4\phantom{0}$ & $\phantom{0}\phantom{-}6.7\phantom{0}$ & $\phantom{0}{-}4.1\phantom{0}$ & $\phantom{0}0.69\phantom{0}$ & $\phantom{0}0.71\phantom{0}$ & $\phantom{0}0.81\phantom{0}$ & $\phantom{0}0.81\phantom{0}$ & $\phantom{0}0.69\phantom{0}$ & $\phantom{0}92.4\phantom{0}$ & $\phantom{0}95.1\phantom{0}$ & $\phantom{0}94.1\phantom{0}$ & $\phantom{0}93.5\phantom{0}$ & $\phantom{0}94.5\phantom{0}$ \\
 & \nopagebreak $\;J=200$  & $\phantom{0}\phantom{-}1.9\phantom{0}$ & $\phantom{0}{-}3.0\phantom{0}$ & $\phantom{0}\phantom{-}1.6\phantom{0}$ & $\phantom{0}\phantom{-}1.9\phantom{0}$ & $\phantom{0}{-}1.0\phantom{0}$ & $\phantom{0}0.29\phantom{0}$ & $\phantom{0}0.32\phantom{0}$ & $\phantom{0}0.34\phantom{0}$ & $\phantom{0}0.34\phantom{0}$ & $\phantom{0}0.33\phantom{0}$ & $\phantom{0}94.9\phantom{0}$ & $\phantom{0}96.4\phantom{0}$ & $\phantom{0}95.0\phantom{0}$ & $\phantom{0}94.1\phantom{0}$ & $\phantom{0}95.3\phantom{0}$ \\
 & \nopagebreak $\;J=1000$  & $\phantom{0}\phantom{-}0.3\phantom{0}$ & $\phantom{0}{-}4.3\phantom{0}$ & $\phantom{0}{-}0.1\phantom{0}$ & $\phantom{0}\phantom{-}0.5\phantom{0}$ & $\phantom{0}{-}0.3\phantom{0}$ & $\phantom{0}0.13\phantom{0}$ & $\phantom{0}0.16\phantom{0}$ & $\phantom{0}0.15\phantom{0}$ & $\phantom{0}0.15\phantom{0}$ & $\phantom{0}0.15\phantom{0}$ & $\phantom{0}94.3\phantom{0}$ & $\phantom{0}94.1\phantom{0}$ & $\phantom{0}94.5\phantom{0}$ & $\phantom{0}93.8\phantom{0}$ & $\phantom{0}94.5\phantom{0}$ \\
[0.5ex]\hline\\[-1.6ex] 
& & \multicolumn{15}{c}{Moderate intraclass correlation $(\rho_{Iy}=.30)$} \\[0.6ex]\hline\\[-1.8ex]
\multicolumn{4}{l}{$\bar{n}=5$} \\  & \nopagebreak $\;J=50$  & $\phantom{-}7.4\phantom{0}$ & $\phantom{-}1.9\phantom{0}$ & $\phantom{-}8.0\phantom{0}$ & $\phantom{-}9.9\phantom{0}$ & $\phantom{-}4.0\phantom{0}$ & $\phantom{0}0.45\phantom{0}$ & $\phantom{0}0.46\phantom{0}$ & $\phantom{0}0.51\phantom{0}$ & $\phantom{0}0.51\phantom{0}$ & $\phantom{0}0.47\phantom{0}$ & $\phantom{0}91.8\phantom{0}$ & $\phantom{0}94.1\phantom{0}$ & $\phantom{0}94.0\phantom{0}$ & $\phantom{0}92.7\phantom{0}$ & $\phantom{0}94.3\phantom{0}$ \\
 & \nopagebreak $\;J=200$  & $\phantom{-}1.1\phantom{0}$ & ${-}4.0\phantom{0}$ & $\phantom{-}0.6\phantom{0}$ & $\phantom{-}1.3\phantom{0}$ & ${-}0.2\phantom{0}$ & $\phantom{0}0.17\phantom{0}$ & $\phantom{0}0.19\phantom{0}$ & $\phantom{0}0.20\phantom{0}$ & $\phantom{0}0.20\phantom{0}$ & $\phantom{0}0.19\phantom{0}$ & $\phantom{0}93.4\phantom{0}$ & $\phantom{0}94.1\phantom{0}$ & $\phantom{0}94.1\phantom{0}$ & $\phantom{0}93.2\phantom{0}$ & $\phantom{0}94.6\phantom{0}$ \\
 & \nopagebreak $\;J=1000$  & $\phantom{-}0.2\phantom{0}$ & ${-}4.7\phantom{0}$ & ${-}0.4\phantom{0}$ & $\phantom{-}0.1\phantom{0}$ & ${-}0.0\phantom{0}$ & $\phantom{0}0.07\phantom{0}$ & $\phantom{0}0.09\phantom{0}$ & $\phantom{0}0.08\phantom{0}$ & $\phantom{0}0.08\phantom{0}$ & $\phantom{0}0.09\phantom{0}$ & $\phantom{0}96.1\phantom{0}$ & $\phantom{0}92.4\phantom{0}$ & $\phantom{0}94.0\phantom{0}$ & $\phantom{0}94.8\phantom{0}$ & $\phantom{0}93.7\phantom{0}$ \\
\multicolumn{4}{l}{$\bar{n}=20$} \\  & \nopagebreak $\;J=50$  & $\phantom{-}1.4\phantom{0}$ & $\phantom{-}1.2\phantom{0}$ & $\phantom{-}2.1\phantom{0}$ & $\phantom{-}2.7\phantom{0}$ & ${-}1.5\phantom{0}$ & $\phantom{0}0.27\phantom{0}$ & $\phantom{0}0.32\phantom{0}$ & $\phantom{0}0.34\phantom{0}$ & $\phantom{0}0.33\phantom{0}$ & $\phantom{0}0.32\phantom{0}$ & $\phantom{0}93.2\phantom{0}$ & $\phantom{0}93.0\phantom{0}$ & $\phantom{0}93.2\phantom{0}$ & $\phantom{0}92.3\phantom{0}$ & $\phantom{0}94.1\phantom{0}$ \\
 & \nopagebreak $\;J=200$  & $\phantom{-}1.2\phantom{0}$ & ${-}0.1\phantom{0}$ & $\phantom{-}0.8\phantom{0}$ & $\phantom{-}1.0\phantom{0}$ & ${-}0.0\phantom{0}$ & $\phantom{0}0.13\phantom{0}$ & $\phantom{0}0.15\phantom{0}$ & $\phantom{0}0.16\phantom{0}$ & $\phantom{0}0.16\phantom{0}$ & $\phantom{0}0.15\phantom{0}$ & $\phantom{0}95.1\phantom{0}$ & $\phantom{0}94.3\phantom{0}$ & $\phantom{0}93.8\phantom{0}$ & $\phantom{0}93.9\phantom{0}$ & $\phantom{0}94.6\phantom{0}$ \\
 & \nopagebreak $\;J=1000$  & ${-}0.2\phantom{0}$ & ${-}1.2\phantom{0}$ & ${-}0.3\phantom{0}$ & ${-}0.1\phantom{0}$ & ${-}0.3\phantom{0}$ & $\phantom{0}0.06\phantom{0}$ & $\phantom{0}0.07\phantom{0}$ & $\phantom{0}0.07\phantom{0}$ & $\phantom{0}0.07\phantom{0}$ & $\phantom{0}0.07\phantom{0}$ & $\phantom{0}95.9\phantom{0}$ & $\phantom{0}95.1\phantom{0}$ & $\phantom{0}94.8\phantom{0}$ & $\phantom{0}95.6\phantom{0}$ & $\phantom{0}95.6\phantom{0}$ \\
[0.5ex]\hline\\[-1.6ex] 
\end{tabular}
\begin{tablenotes}[para,flushleft]{\footnotesize \textit{Note.} $\bar{n}$ = average cluster size; $J$ = number of clusters; CD = complete data sets; LD = listwise deletion; FCS-SL = single-level FCS; FCS-MAN = two-level FCS with manifest cluster means; FCS-LAT = two-level FCS with latent cluster means; JM = joint modeling.}\end{tablenotes}
\end{threeparttable}
\end{sidewaystable}
\begin{sidewaystable}
\begin{threeparttable}
\setlength{\tabcolsep}{1.0pt}
\renewcommand{\arraystretch}{0.95}
\footnotesize
\caption{\small Study 2: Bias (in \%), Relative RMSE, and Coverage of the 95\% Confidence Interval for the Regression Coefficient of $z$ on $y$ ($\hat\beta_{zy}$) With Moderately Unbalanced Data (Uniform, $\pm 40\%$) and 40\% Missing Data (MAR, $\lambda=0.5$)}
\begin{tabular}{llccccccccccccccc}
\hline\\[-1.8ex]
& & \multicolumn{5}{c}{Bias (\%)} & \multicolumn{5}{c}{Rel. RMSE} & \multicolumn{5}{c}{Coverage (\%)} \\ \cmidrule(r){3-7}\cmidrule(r){8-12}\cmidrule(r){13-17}
 &  & CD & \makecell{FCS-\\MAN} & \makecell{FCS-\\NJ} & \makecell{FCS-\\LAT} & JM & CD & \makecell{FCS-\\MAN} & \makecell{FCS-\\NJ} & \makecell{FCS-\\LAT} & JM & CD & \makecell{FCS-\\MAN} & \makecell{FCS-\\NJ} & \makecell{FCS-\\LAT} & \multicolumn{1}{c}{JM} \\ 
[0.4ex]\hline\\[-1.8ex]
& & \multicolumn{15}{c}{Small intraclass correlation $(\rho_{Iy}=.10)$} \\[0.6ex]\hline\\[-1.8ex]
\multicolumn{4}{l}{$\bar{n}=5$} \\  & \nopagebreak $\;J=50$  & $\phantom{-}40.0\phantom{0}$ & $\phantom{-}23.0\phantom{0}$ & $\phantom{-}25.6\phantom{0}$ & $\phantom{-}35.2\phantom{0}$ & $\phantom{0}\phantom{-}9.0\phantom{0}$ & $\phantom{0}2.52\phantom{0}$ & $\phantom{0}1.60\phantom{0}$ & $\phantom{0}1.72\phantom{0}$ & $\phantom{0}1.76\phantom{0}$ & $\phantom{0}1.70\phantom{0}$ & $\phantom{0}92.2\phantom{0}$ & $\phantom{0}94.2\phantom{0}$ & $\phantom{0}94.1\phantom{0}$ & $\phantom{0}95.0\phantom{0}$ & $\phantom{0}94.9\phantom{0}$ \\
 & \nopagebreak $\;J=200$  & $\phantom{-}11.3\phantom{0}$ & $\phantom{0}\phantom{-}6.0\phantom{0}$ & $\phantom{0}\phantom{-}10.0\phantom{0}$ & $\phantom{-}14.0\phantom{0}$ & $\phantom{0}{-}1.2\phantom{0}$ & $\phantom{0}0.70\phantom{0}$ & $\phantom{0}0.74\phantom{0}$ & $\phantom{0}0.80\phantom{0}$ & $\phantom{0}0.84\phantom{0}$ & $\phantom{0}0.69\phantom{0}$ & $\phantom{0}95.4\phantom{0}$ & $\phantom{0}95.0\phantom{0}$ & $\phantom{0}95.7\phantom{0}$ & $\phantom{0}95.0\phantom{0}$ & $\phantom{0}94.3\phantom{0}$ \\
 & \nopagebreak $\;J=1000$  & $\phantom{0}\phantom{-}1.9\phantom{0}$ & $\phantom{0}{-}1.0\phantom{0}$ & $\phantom{0}\phantom{-}1.4\phantom{0}$ & $\phantom{0}\phantom{-}2.1\phantom{0}$ & $\phantom{0}{-}1.3\phantom{0}$ & $\phantom{0}0.22\phantom{0}$ & $\phantom{0}0.26\phantom{0}$ & $\phantom{0}0.27\phantom{0}$ & $\phantom{0}0.28\phantom{0}$ & $\phantom{0}0.26\phantom{0}$ & $\phantom{0}95.7\phantom{0}$ & $\phantom{0}95.2\phantom{0}$ & $\phantom{0}95.3\phantom{0}$ & $\phantom{0}95.3\phantom{0}$ & $\phantom{0}95.0\phantom{0}$ \\
\multicolumn{4}{l}{$\bar{n}=20$} \\  & \nopagebreak $\;J=50$  & $\phantom{0}\phantom{-}2.4\phantom{0}$ & $\phantom{0}\phantom{-}3.0\phantom{0}$ & $\phantom{0}\phantom{-}4.2\phantom{0}$ & $\phantom{0}\phantom{-}6.9\phantom{0}$ & ${-}16.1\phantom{0}$ & $\phantom{0}0.58\phantom{0}$ & $\phantom{0}0.82\phantom{0}$ & $\phantom{0}0.88\phantom{0}$ & $\phantom{0}0.87\phantom{0}$ & $\phantom{0}0.70\phantom{0}$ & $\phantom{0}93.0\phantom{0}$ & $\phantom{0}93.2\phantom{0}$ & $\phantom{0}95.0\phantom{0}$ & $\phantom{0}91.5\phantom{0}$ & $\phantom{0}94.2\phantom{0}$ \\
 & \nopagebreak $\;J=200$  & $\phantom{0}\phantom{-}1.1\phantom{0}$ & $\phantom{0}\phantom{-}1.9\phantom{0}$ & $\phantom{0}\phantom{-}1.7\phantom{0}$ & $\phantom{0}\phantom{-}2.1\phantom{0}$ & $\phantom{0}{-}5.2\phantom{0}$ & $\phantom{0}0.26\phantom{0}$ & $\phantom{0}0.36\phantom{0}$ & $\phantom{0}0.36\phantom{0}$ & $\phantom{0}0.36\phantom{0}$ & $\phantom{0}0.34\phantom{0}$ & $\phantom{0}94.7\phantom{0}$ & $\phantom{0}94.5\phantom{0}$ & $\phantom{0}95.5\phantom{0}$ & $\phantom{0}95.4\phantom{0}$ & $\phantom{0}95.0\phantom{0}$ \\
 & \nopagebreak $\;J=1000$  & $\phantom{0}{-}0.3\phantom{0}$ & $\phantom{0}{-}0.7\phantom{0}$ & $\phantom{0}{-}0.4\phantom{0}$ & $\phantom{0}{-}0.2\phantom{0}$ & $\phantom{0}{-}1.8\phantom{0}$ & $\phantom{0}0.12\phantom{0}$ & $\phantom{0}0.16\phantom{0}$ & $\phantom{0}0.17\phantom{0}$ & $\phantom{0}0.17\phantom{0}$ & $\phantom{0}0.16\phantom{0}$ & $\phantom{0}94.8\phantom{0}$ & $\phantom{0}94.1\phantom{0}$ & $\phantom{0}93.5\phantom{0}$ & $\phantom{0}93.9\phantom{0}$ & $\phantom{0}93.7\phantom{0}$ \\
[0.5ex]\hline\\[-1.6ex] 
& & \multicolumn{15}{c}{Moderate intraclass correlation $(\rho_{Iy}=.30)$} \\[0.6ex]\hline\\[-1.8ex]
\multicolumn{4}{l}{$\bar{n}=5$} \\  & \nopagebreak $\;J=50$  & $\phantom{-}5.3\phantom{0}$ & $\phantom{-}4.0\phantom{0}$ & $\phantom{-}6.0\phantom{0}$ & $\phantom{-}8.9\phantom{0}$ & ${-}5.8\phantom{0}$ & $\phantom{0}0.35\phantom{0}$ & $\phantom{0}0.47\phantom{0}$ & $\phantom{0}0.53\phantom{0}$ & $\phantom{0}0.51\phantom{0}$ & $\phantom{0}0.43\phantom{0}$ & $\phantom{0}94.4\phantom{0}$ & $\phantom{0}95.0\phantom{0}$ & $\phantom{0}95.5\phantom{0}$ & $\phantom{0}94.0\phantom{0}$ & $\phantom{0}96.6\phantom{0}$ \\
 & \nopagebreak $\;J=200$  & $\phantom{-}1.1\phantom{0}$ & $\phantom{-}0.1\phantom{0}$ & $\phantom{-}1.0\phantom{0}$ & $\phantom{-}1.3\phantom{0}$ & ${-}1.6\phantom{0}$ & $\phantom{0}0.16\phantom{0}$ & $\phantom{0}0.21\phantom{0}$ & $\phantom{0}0.22\phantom{0}$ & $\phantom{0}0.22\phantom{0}$ & $\phantom{0}0.21\phantom{0}$ & $\phantom{0}94.3\phantom{0}$ & $\phantom{0}94.4\phantom{0}$ & $\phantom{0}95.1\phantom{0}$ & $\phantom{0}94.0\phantom{0}$ & $\phantom{0}95.2\phantom{0}$ \\
 & \nopagebreak $\;J=1000$  & $\phantom{-}0.1\phantom{0}$ & ${-}0.4\phantom{0}$ & $\phantom{-}0.1\phantom{0}$ & $\phantom{-}0.2\phantom{0}$ & ${-}0.2\phantom{0}$ & $\phantom{0}0.07\phantom{0}$ & $\phantom{0}0.09\phantom{0}$ & $\phantom{0}0.10\phantom{0}$ & $\phantom{0}0.10\phantom{0}$ & $\phantom{0}0.10\phantom{0}$ & $\phantom{0}95.6\phantom{0}$ & $\phantom{0}93.4\phantom{0}$ & $\phantom{0}94.1\phantom{0}$ & $\phantom{0}94.9\phantom{0}$ & $\phantom{0}94.7\phantom{0}$ \\
\multicolumn{4}{l}{$\bar{n}=20$} \\  & \nopagebreak $\;J=50$  & $\phantom{-}0.0\phantom{0}$ & ${-}1.7\phantom{0}$ & ${-}1.4\phantom{0}$ & ${-}0.9\phantom{0}$ & ${-}8.9\phantom{0}$ & $\phantom{0}0.26\phantom{0}$ & $\phantom{0}0.38\phantom{0}$ & $\phantom{0}0.41\phantom{0}$ & $\phantom{0}0.39\phantom{0}$ & $\phantom{0}0.36\phantom{0}$ & $\phantom{0}92.1\phantom{0}$ & $\phantom{0}92.4\phantom{0}$ & $\phantom{0}93.6\phantom{0}$ & $\phantom{0}92.8\phantom{0}$ & $\phantom{0}93.9\phantom{0}$ \\
 & \nopagebreak $\;J=200$  & $\phantom{-}0.3\phantom{0}$ & $\phantom{-}0.1\phantom{0}$ & ${-}0.1\phantom{0}$ & $\phantom{-}0.0\phantom{0}$ & ${-}1.8\phantom{0}$ & $\phantom{0}0.12\phantom{0}$ & $\phantom{0}0.18\phantom{0}$ & $\phantom{0}0.18\phantom{0}$ & $\phantom{0}0.17\phantom{0}$ & $\phantom{0}0.17\phantom{0}$ & $\phantom{0}94.7\phantom{0}$ & $\phantom{0}94.2\phantom{0}$ & $\phantom{0}93.4\phantom{0}$ & $\phantom{0}94.3\phantom{0}$ & $\phantom{0}95.7\phantom{0}$ \\
 & \nopagebreak $\;J=1000$  & ${-}0.0\phantom{0}$ & ${-}0.1\phantom{0}$ & ${-}0.0\phantom{0}$ & $\phantom{-}0.1\phantom{0}$ & ${-}0.4\phantom{0}$ & $\phantom{0}0.05\phantom{0}$ & $\phantom{0}0.08\phantom{0}$ & $\phantom{0}0.08\phantom{0}$ & $\phantom{0}0.08\phantom{0}$ & $\phantom{0}0.08\phantom{0}$ & $\phantom{0}95.7\phantom{0}$ & $\phantom{0}94.9\phantom{0}$ & $\phantom{0}94.4\phantom{0}$ & $\phantom{0}94.6\phantom{0}$ & $\phantom{0}93.9\phantom{0}$ \\
[0.5ex]\hline\\[-1.6ex] 
\end{tabular}
\begin{tablenotes}[para,flushleft]{\footnotesize \textit{Note.} $\bar{n}$ = average cluster size; $J$ = number of clusters; CD = complete data sets; LD = listwise deletion; FCS-SL = single-level FCS; FCS-MAN = two-level FCS with manifest cluster means; FCS-LAT = two-level FCS with latent cluster means; JM = joint modeling.}\end{tablenotes}
\end{threeparttable}
\end{sidewaystable}
\begin{sidewaystable}
\begin{threeparttable}
\setlength{\tabcolsep}{1.0pt}
\renewcommand{\arraystretch}{0.95}
\footnotesize
\caption{\small Study 2: Bias (in \%), Relative RMSE, and Coverage of the 95\% Confidence Interval for the Regression Coefficient of $z$ on $y$ ($\hat\beta_{zy}$) With Strongly Unbalanced Data (Uniform, $\pm 80\%$) and 40\% Missing Data (MAR, $\lambda=0.5$)}
\begin{tabular}{llccccccccccccccc}
\hline\\[-1.8ex]
& & \multicolumn{5}{c}{Bias (\%)} & \multicolumn{5}{c}{Rel. RMSE} & \multicolumn{5}{c}{Coverage (\%)} \\ \cmidrule(r){3-7}\cmidrule(r){8-12}\cmidrule(r){13-17}
 &  & CD & \makecell{FCS-\\MAN} & \makecell{FCS-\\NJ} & \makecell{FCS-\\LAT} & JM & CD & \makecell{FCS-\\MAN} & \makecell{FCS-\\NJ} & \makecell{FCS-\\LAT} & JM & CD & \makecell{FCS-\\MAN} & \makecell{FCS-\\NJ} & \makecell{FCS-\\LAT} & \multicolumn{1}{c}{JM} \\ 
[0.4ex]\hline\\[-1.8ex]
& & \multicolumn{15}{c}{Small intraclass correlation $(\rho_{Iy}=.10)$} \\[0.6ex]\hline\\[-1.8ex]
\multicolumn{4}{l}{$\bar{n}=5$} \\  & \nopagebreak $\;J=50$  & $\phantom{-}38.2\phantom{0}$ & $\phantom{-}11.5\phantom{0}$ & $\phantom{-}24.0\phantom{0}$ & $\phantom{-}32.3\phantom{0}$ & $\phantom{0}\phantom{-}5.6\phantom{0}$ & $\phantom{0}1.90\phantom{0}$ & $\phantom{0}1.86\phantom{0}$ & $\phantom{0}1.84\phantom{0}$ & $\phantom{0}1.90\phantom{0}$ & $\phantom{0}1.70\phantom{0}$ & $\phantom{0}92.5\phantom{0}$ & $\phantom{0}93.1\phantom{0}$ & $\phantom{0}94.1\phantom{0}$ & $\phantom{0}94.3\phantom{0}$ & $\phantom{0}94.0\phantom{0}$ \\
 & \nopagebreak $\;J=200$  & $\phantom{0}\phantom{-}6.6\phantom{0}$ & $\phantom{0}{-}5.8\phantom{0}$ & $\phantom{0}\phantom{-}6.6\phantom{0}$ & $\phantom{-}10.1\phantom{0}$ & $\phantom{0}{-}4.6\phantom{0}$ & $\phantom{0}0.60\phantom{0}$ & $\phantom{0}0.59\phantom{0}$ & $\phantom{0}0.69\phantom{0}$ & $\phantom{0}0.73\phantom{0}$ & $\phantom{0}0.58\phantom{0}$ & $\phantom{0}96.4\phantom{0}$ & $\phantom{0}95.3\phantom{0}$ & $\phantom{0}97.2\phantom{0}$ & $\phantom{0}96.8\phantom{0}$ & $\phantom{0}95.2\phantom{0}$ \\
 & \nopagebreak $\;J=1000$  & $\phantom{0}\phantom{-}2.4\phantom{0}$ & $\phantom{0}{-}8.1\phantom{0}$ & $\phantom{0}\phantom{-}1.9\phantom{0}$ & $\phantom{0}\phantom{-}2.7\phantom{0}$ & $\phantom{0}{-}0.8\phantom{0}$ & $\phantom{0}0.22\phantom{0}$ & $\phantom{0}0.27\phantom{0}$ & $\phantom{0}0.27\phantom{0}$ & $\phantom{0}0.28\phantom{0}$ & $\phantom{0}0.25\phantom{0}$ & $\phantom{0}96.1\phantom{0}$ & $\phantom{0}91.7\phantom{0}$ & $\phantom{0}95.3\phantom{0}$ & $\phantom{0}95.1\phantom{0}$ & $\phantom{0}95.2\phantom{0}$ \\
\multicolumn{4}{l}{$\bar{n}=20$} \\  & \nopagebreak $\;J=50$  & $\phantom{0}\phantom{-}4.8\phantom{0}$ & $\phantom{0}\phantom{-}1.5\phantom{0}$ & $\phantom{0}\phantom{-}5.8\phantom{0}$ & $\phantom{0}\phantom{-}8.1\phantom{0}$ & ${-}15.9\phantom{0}$ & $\phantom{0}0.60\phantom{0}$ & $\phantom{0}0.78\phantom{0}$ & $\phantom{0}0.87\phantom{0}$ & $\phantom{0}0.88\phantom{0}$ & $\phantom{0}0.70\phantom{0}$ & $\phantom{0}93.8\phantom{0}$ & $\phantom{0}94.2\phantom{0}$ & $\phantom{0}94.9\phantom{0}$ & $\phantom{0}92.8\phantom{0}$ & $\phantom{0}95.0\phantom{0}$ \\
 & \nopagebreak $\;J=200$  & $\phantom{0}\phantom{-}0.7\phantom{0}$ & $\phantom{0}{-}3.8\phantom{0}$ & $\phantom{0}\phantom{-}0.1\phantom{0}$ & $\phantom{0}\phantom{-}0.3\phantom{0}$ & $\phantom{0}{-}6.8\phantom{0}$ & $\phantom{0}0.27\phantom{0}$ & $\phantom{0}0.37\phantom{0}$ & $\phantom{0}0.38\phantom{0}$ & $\phantom{0}0.38\phantom{0}$ & $\phantom{0}0.36\phantom{0}$ & $\phantom{0}94.9\phantom{0}$ & $\phantom{0}94.2\phantom{0}$ & $\phantom{0}94.6\phantom{0}$ & $\phantom{0}93.5\phantom{0}$ & $\phantom{0}93.7\phantom{0}$ \\
 & \nopagebreak $\;J=1000$  & $\phantom{0}\phantom{-}0.1\phantom{0}$ & $\phantom{0}{-}2.6\phantom{0}$ & $\phantom{0}{-}0.0\phantom{0}$ & $\phantom{0}\phantom{-}0.5\phantom{0}$ & $\phantom{0}{-}1.2\phantom{0}$ & $\phantom{0}0.12\phantom{0}$ & $\phantom{0}0.16\phantom{0}$ & $\phantom{0}0.16\phantom{0}$ & $\phantom{0}0.16\phantom{0}$ & $\phantom{0}0.15\phantom{0}$ & $\phantom{0}95.8\phantom{0}$ & $\phantom{0}95.6\phantom{0}$ & $\phantom{0}96.0\phantom{0}$ & $\phantom{0}94.8\phantom{0}$ & $\phantom{0}95.1\phantom{0}$ \\
[0.5ex]\hline\\[-1.6ex] 
& & \multicolumn{15}{c}{Moderate intraclass correlation $(\rho_{Iy}=.30)$} \\[0.6ex]\hline\\[-1.8ex]
\multicolumn{4}{l}{$\bar{n}=5$} \\  & \nopagebreak $\;J=50$  & $\phantom{0}\phantom{-}6.9\phantom{0}$ & $\phantom{0}\phantom{-}2.0\phantom{0}$ & $\phantom{0}\phantom{-}8.8\phantom{0}$ & $\phantom{-}10.7\phantom{0}$ & $\phantom{0}{-}4.7\phantom{0}$ & $\phantom{0}0.38\phantom{0}$ & $\phantom{0}0.47\phantom{0}$ & $\phantom{0}0.54\phantom{0}$ & $\phantom{0}0.55\phantom{0}$ & $\phantom{0}0.44\phantom{0}$ & $\phantom{0}92.6\phantom{0}$ & $\phantom{0}94.7\phantom{0}$ & $\phantom{0}95.1\phantom{0}$ & $\phantom{0}91.8\phantom{0}$ & $\phantom{0}95.6\phantom{0}$ \\
 & \nopagebreak $\;J=200$  & $\phantom{0}\phantom{-}0.6\phantom{0}$ & $\phantom{0}{-}3.5\phantom{0}$ & $\phantom{0}\phantom{-}0.7\phantom{0}$ & $\phantom{0}\phantom{-}1.6\phantom{0}$ & $\phantom{0}{-}1.6\phantom{0}$ & $\phantom{0}0.16\phantom{0}$ & $\phantom{0}0.22\phantom{0}$ & $\phantom{0}0.23\phantom{0}$ & $\phantom{0}0.23\phantom{0}$ & $\phantom{0}0.22\phantom{0}$ & $\phantom{0}94.0\phantom{0}$ & $\phantom{0}93.2\phantom{0}$ & $\phantom{0}94.1\phantom{0}$ & $\phantom{0}93.5\phantom{0}$ & $\phantom{0}93.8\phantom{0}$ \\
 & \nopagebreak $\;J=1000$  & $\phantom{0}\phantom{-}0.0\phantom{0}$ & $\phantom{0}{-}3.7\phantom{0}$ & $\phantom{0}{-}0.4\phantom{0}$ & $\phantom{0}\phantom{-}0.2\phantom{0}$ & $\phantom{0}{-}0.2\phantom{0}$ & $\phantom{0}0.07\phantom{0}$ & $\phantom{0}0.10\phantom{0}$ & $\phantom{0}0.09\phantom{0}$ & $\phantom{0}0.09\phantom{0}$ & $\phantom{0}0.09\phantom{0}$ & $\phantom{0}95.4\phantom{0}$ & $\phantom{0}94.0\phantom{0}$ & $\phantom{0}95.2\phantom{0}$ & $\phantom{0}94.4\phantom{0}$ & $\phantom{0}96.0\phantom{0}$ \\
\multicolumn{4}{l}{$\bar{n}=20$} \\  & \nopagebreak $\;J=50$  & $\phantom{0}\phantom{-}1.4\phantom{0}$ & $\phantom{0}\phantom{-}1.4\phantom{0}$ & $\phantom{0}\phantom{-}1.7\phantom{0}$ & $\phantom{0}\phantom{-}2.3\phantom{0}$ & $\phantom{0}{-}6.0\phantom{0}$ & $\phantom{0}0.27\phantom{0}$ & $\phantom{0}0.39\phantom{0}$ & $\phantom{0}0.41\phantom{0}$ & $\phantom{0}0.39\phantom{0}$ & $\phantom{0}0.36\phantom{0}$ & $\phantom{0}91.0\phantom{0}$ & $\phantom{0}92.3\phantom{0}$ & $\phantom{0}93.7\phantom{0}$ & $\phantom{0}92.4\phantom{0}$ & $\phantom{0}94.3\phantom{0}$ \\
 & \nopagebreak $\;J=200$  & $\phantom{0}\phantom{-}0.5\phantom{0}$ & $\phantom{0}{-}0.4\phantom{0}$ & $\phantom{0}\phantom{-}0.3\phantom{0}$ & $\phantom{0}\phantom{-}0.2\phantom{0}$ & $\phantom{0}{-}1.9\phantom{0}$ & $\phantom{0}0.13\phantom{0}$ & $\phantom{0}0.18\phantom{0}$ & $\phantom{0}0.18\phantom{0}$ & $\phantom{0}0.18\phantom{0}$ & $\phantom{0}0.18\phantom{0}$ & $\phantom{0}93.2\phantom{0}$ & $\phantom{0}93.2\phantom{0}$ & $\phantom{0}93.8\phantom{0}$ & $\phantom{0}93.1\phantom{0}$ & $\phantom{0}94.6\phantom{0}$ \\
 & \nopagebreak $\;J=1000$  & $\phantom{0}\phantom{-}0.1\phantom{0}$ & $\phantom{0}{-}0.4\phantom{0}$ & $\phantom{0}{-}0.1\phantom{0}$ & $\phantom{0}\phantom{-}0.1\phantom{0}$ & $\phantom{0}{-}0.3\phantom{0}$ & $\phantom{0}0.06\phantom{0}$ & $\phantom{0}0.08\phantom{0}$ & $\phantom{0}0.08\phantom{0}$ & $\phantom{0}0.08\phantom{0}$ & $\phantom{0}0.08\phantom{0}$ & $\phantom{0}93.5\phantom{0}$ & $\phantom{0}92.9\phantom{0}$ & $\phantom{0}94.6\phantom{0}$ & $\phantom{0}92.6\phantom{0}$ & $\phantom{0}94.2\phantom{0}$ \\
[0.5ex]\hline\\[-1.6ex] 
\end{tabular}
\begin{tablenotes}[para,flushleft]{\footnotesize \textit{Note.} $\bar{n}$ = average cluster size; $J$ = number of clusters; CD = complete data sets; LD = listwise deletion; FCS-SL = single-level FCS; FCS-MAN = two-level FCS with manifest cluster means; FCS-LAT = two-level FCS with latent cluster means; JM = joint modeling.}\end{tablenotes}
\end{threeparttable}
\end{sidewaystable}
\begin{sidewaystable}
\begin{threeparttable}
\setlength{\tabcolsep}{1.0pt}
\renewcommand{\arraystretch}{0.95}
\footnotesize
\caption{\small Study 2: Bias (in \%), Relative RMSE, and Coverage of the 95\% Confidence Interval for the Regression Coefficient of $z$ on $y$ ($\hat\beta_{zy}$) With Moderately Unbalanced Data (Bimodal, $\pm 40\%$) and 40\% Missing Data (MAR, $\lambda=0.5$)}
\begin{tabular}{llccccccccccccccc}
\hline\\[-1.8ex]
& & \multicolumn{5}{c}{Bias (\%)} & \multicolumn{5}{c}{Rel. RMSE} & \multicolumn{5}{c}{Coverage (\%)} \\ \cmidrule(r){3-7}\cmidrule(r){8-12}\cmidrule(r){13-17}
 &  & CD & \makecell{FCS-\\MAN} & \makecell{FCS-\\NJ} & \makecell{FCS-\\LAT} & JM & CD & \makecell{FCS-\\MAN} & \makecell{FCS-\\NJ} & \makecell{FCS-\\LAT} & JM & CD & \makecell{FCS-\\MAN} & \makecell{FCS-\\NJ} & \makecell{FCS-\\LAT} & \multicolumn{1}{c}{JM} \\ 
[0.4ex]\hline\\[-1.8ex]
& & \multicolumn{15}{c}{Small intraclass correlation $(\rho_{Iy}=.10)$} \\[0.6ex]\hline\\[-1.8ex]
\multicolumn{4}{l}{$\bar{n}=5$} \\  & \nopagebreak $\;J=50$  & $\phantom{-}38.0\phantom{0}$ & $\phantom{-}18.3\phantom{0}$ & $\phantom{-}21.2\phantom{0}$ & $\phantom{-}29.0\phantom{0}$ & $\phantom{0}\phantom{-}6.1\phantom{0}$ & $\phantom{0}2.19\phantom{0}$ & $\phantom{0}1.75\phantom{0}$ & $\phantom{0}1.80\phantom{0}$ & $\phantom{0}1.82\phantom{0}$ & $\phantom{0}1.71\phantom{0}$ & $\phantom{0}92.1\phantom{0}$ & $\phantom{0}94.4\phantom{0}$ & $\phantom{0}94.5\phantom{0}$ & $\phantom{0}94.6\phantom{0}$ & $\phantom{0}93.8\phantom{0}$ \\
 & \nopagebreak $\;J=200$  & $\phantom{0}\phantom{-}6.8\phantom{0}$ & $\phantom{0}\phantom{-}1.1\phantom{0}$ & $\phantom{0}\phantom{-}6.9\phantom{0}$ & $\phantom{-}11.5\phantom{0}$ & $\phantom{0}{-}3.6\phantom{0}$ & $\phantom{0}0.59\phantom{0}$ & $\phantom{0}0.66\phantom{0}$ & $\phantom{0}0.73\phantom{0}$ & $\phantom{0}0.76\phantom{0}$ & $\phantom{0}0.62\phantom{0}$ & $\phantom{0}95.8\phantom{0}$ & $\phantom{0}95.2\phantom{0}$ & $\phantom{0}96.6\phantom{0}$ & $\phantom{0}96.1\phantom{0}$ & $\phantom{0}95.1\phantom{0}$ \\
 & \nopagebreak $\;J=1000$  & $\phantom{0}\phantom{-}0.9\phantom{0}$ & $\phantom{0}{-}3.4\phantom{0}$ & $\phantom{0}\phantom{-}1.0\phantom{0}$ & $\phantom{0}\phantom{-}2.1\phantom{0}$ & $\phantom{0}{-}1.5\phantom{0}$ & $\phantom{0}0.21\phantom{0}$ & $\phantom{0}0.25\phantom{0}$ & $\phantom{0}0.27\phantom{0}$ & $\phantom{0}0.27\phantom{0}$ & $\phantom{0}0.25\phantom{0}$ & $\phantom{0}96.4\phantom{0}$ & $\phantom{0}94.8\phantom{0}$ & $\phantom{0}95.4\phantom{0}$ & $\phantom{0}95.8\phantom{0}$ & $\phantom{0}95.4\phantom{0}$ \\
\multicolumn{4}{l}{$\bar{n}=20$} \\  & \nopagebreak $\;J=50$  & $\phantom{0}\phantom{-}4.4\phantom{0}$ & $\phantom{0}\phantom{-}3.8\phantom{0}$ & $\phantom{0}\phantom{-}4.5\phantom{0}$ & $\phantom{0}\phantom{-}7.6\phantom{0}$ & ${-}15.0\phantom{0}$ & $\phantom{0}0.57\phantom{0}$ & $\phantom{0}0.81\phantom{0}$ & $\phantom{0}0.89\phantom{0}$ & $\phantom{0}0.88\phantom{0}$ & $\phantom{0}0.70\phantom{0}$ & $\phantom{0}95.7\phantom{0}$ & $\phantom{0}94.3\phantom{0}$ & $\phantom{0}95.0\phantom{0}$ & $\phantom{0}93.5\phantom{0}$ & $\phantom{0}95.7\phantom{0}$ \\
 & \nopagebreak $\;J=200$  & $\phantom{0}\phantom{-}0.8\phantom{0}$ & $\phantom{0}{-}1.0\phantom{0}$ & $\phantom{0}\phantom{-}0.5\phantom{0}$ & $\phantom{0}\phantom{-}0.5\phantom{0}$ & $\phantom{0}{-}6.7\phantom{0}$ & $\phantom{0}0.26\phantom{0}$ & $\phantom{0}0.36\phantom{0}$ & $\phantom{0}0.37\phantom{0}$ & $\phantom{0}0.37\phantom{0}$ & $\phantom{0}0.35\phantom{0}$ & $\phantom{0}95.0\phantom{0}$ & $\phantom{0}94.6\phantom{0}$ & $\phantom{0}94.3\phantom{0}$ & $\phantom{0}93.5\phantom{0}$ & $\phantom{0}95.4\phantom{0}$ \\
 & \nopagebreak $\;J=1000$  & $\phantom{0}\phantom{-}0.0\phantom{0}$ & $\phantom{0}{-}1.5\phantom{0}$ & $\phantom{0}{-}0.4\phantom{0}$ & $\phantom{0}{-}0.2\phantom{0}$ & $\phantom{0}{-}2.0\phantom{0}$ & $\phantom{0}0.12\phantom{0}$ & $\phantom{0}0.16\phantom{0}$ & $\phantom{0}0.17\phantom{0}$ & $\phantom{0}0.17\phantom{0}$ & $\phantom{0}0.17\phantom{0}$ & $\phantom{0}94.3\phantom{0}$ & $\phantom{0}94.6\phantom{0}$ & $\phantom{0}94.6\phantom{0}$ & $\phantom{0}94.3\phantom{0}$ & $\phantom{0}93.8\phantom{0}$ \\
[0.5ex]\hline\\[-1.6ex] 
& & \multicolumn{15}{c}{Moderate intraclass correlation $(\rho_{Iy}=.30)$} \\[0.6ex]\hline\\[-1.8ex]
\multicolumn{4}{l}{$\bar{n}=5$} \\  & \nopagebreak $\;J=50$  & $\phantom{-}5.7\phantom{0}$ & $\phantom{-}3.2\phantom{0}$ & $\phantom{-}5.4\phantom{0}$ & $\phantom{-}8.6\phantom{0}$ & ${-}5.3\phantom{0}$ & $\phantom{0}0.37\phantom{0}$ & $\phantom{0}0.47\phantom{0}$ & $\phantom{0}0.55\phantom{0}$ & $\phantom{0}0.53\phantom{0}$ & $\phantom{0}0.43\phantom{0}$ & $\phantom{0}94.1\phantom{0}$ & $\phantom{0}94.0\phantom{0}$ & $\phantom{0}95.7\phantom{0}$ & $\phantom{0}92.6\phantom{0}$ & $\phantom{0}95.1\phantom{0}$ \\
 & \nopagebreak $\;J=200$  & $\phantom{-}1.0\phantom{0}$ & ${-}0.0\phantom{0}$ & $\phantom{-}1.0\phantom{0}$ & $\phantom{-}1.6\phantom{0}$ & ${-}1.7\phantom{0}$ & $\phantom{0}0.17\phantom{0}$ & $\phantom{0}0.22\phantom{0}$ & $\phantom{0}0.23\phantom{0}$ & $\phantom{0}0.23\phantom{0}$ & $\phantom{0}0.22\phantom{0}$ & $\phantom{0}94.5\phantom{0}$ & $\phantom{0}94.2\phantom{0}$ & $\phantom{0}94.7\phantom{0}$ & $\phantom{0}94.0\phantom{0}$ & $\phantom{0}94.7\phantom{0}$ \\
 & \nopagebreak $\;J=1000$  & $\phantom{-}0.1\phantom{0}$ & ${-}1.0\phantom{0}$ & $\phantom{-}0.2\phantom{0}$ & $\phantom{-}0.6\phantom{0}$ & $\phantom{-}0.1\phantom{0}$ & $\phantom{0}0.07\phantom{0}$ & $\phantom{0}0.10\phantom{0}$ & $\phantom{0}0.10\phantom{0}$ & $\phantom{0}0.10\phantom{0}$ & $\phantom{0}0.10\phantom{0}$ & $\phantom{0}94.7\phantom{0}$ & $\phantom{0}93.3\phantom{0}$ & $\phantom{0}94.2\phantom{0}$ & $\phantom{0}94.4\phantom{0}$ & $\phantom{0}94.0\phantom{0}$ \\
\multicolumn{4}{l}{$\bar{n}=20$} \\  & \nopagebreak $\;J=50$  & ${-}0.2\phantom{0}$ & ${-}1.9\phantom{0}$ & ${-}1.6\phantom{0}$ & ${-}1.0\phantom{0}$ & ${-}9.2\phantom{0}$ & $\phantom{0}0.26\phantom{0}$ & $\phantom{0}0.37\phantom{0}$ & $\phantom{0}0.40\phantom{0}$ & $\phantom{0}0.37\phantom{0}$ & $\phantom{0}0.35\phantom{0}$ & $\phantom{0}92.7\phantom{0}$ & $\phantom{0}91.9\phantom{0}$ & $\phantom{0}95.1\phantom{0}$ & $\phantom{0}93.3\phantom{0}$ & $\phantom{0}94.1\phantom{0}$ \\
 & \nopagebreak $\;J=200$  & $\phantom{-}0.2\phantom{0}$ & $\phantom{-}0.4\phantom{0}$ & $\phantom{-}0.5\phantom{0}$ & $\phantom{-}0.5\phantom{0}$ & ${-}1.5\phantom{0}$ & $\phantom{0}0.12\phantom{0}$ & $\phantom{0}0.18\phantom{0}$ & $\phantom{0}0.19\phantom{0}$ & $\phantom{0}0.18\phantom{0}$ & $\phantom{0}0.18\phantom{0}$ & $\phantom{0}93.6\phantom{0}$ & $\phantom{0}93.6\phantom{0}$ & $\phantom{0}92.6\phantom{0}$ & $\phantom{0}92.6\phantom{0}$ & $\phantom{0}94.4\phantom{0}$ \\
 & \nopagebreak $\;J=1000$  & $\phantom{-}0.2\phantom{0}$ & $\phantom{-}0.0\phantom{0}$ & $\phantom{-}0.2\phantom{0}$ & $\phantom{-}0.3\phantom{0}$ & ${-}0.1\phantom{0}$ & $\phantom{0}0.06\phantom{0}$ & $\phantom{0}0.08\phantom{0}$ & $\phantom{0}0.08\phantom{0}$ & $\phantom{0}0.08\phantom{0}$ & $\phantom{0}0.08\phantom{0}$ & $\phantom{0}94.8\phantom{0}$ & $\phantom{0}93.5\phantom{0}$ & $\phantom{0}93.7\phantom{0}$ & $\phantom{0}93.5\phantom{0}$ & $\phantom{0}95.5\phantom{0}$ \\
[0.5ex]\hline\\[-1.6ex] 
\end{tabular}
\begin{tablenotes}[para,flushleft]{\footnotesize \textit{Note.} $\bar{n}$ = average cluster size; $J$ = number of clusters; CD = complete data sets; LD = listwise deletion; FCS-SL = single-level FCS; FCS-MAN = two-level FCS with manifest cluster means; FCS-LAT = two-level FCS with latent cluster means; JM = joint modeling.}\end{tablenotes}
\end{threeparttable}
\end{sidewaystable}
\begin{sidewaystable}
\begin{threeparttable}
\setlength{\tabcolsep}{1.0pt}
\renewcommand{\arraystretch}{0.95}
\footnotesize
\caption{\small Study 2: Bias (in \%), Relative RMSE, and Coverage of the 95\% Confidence Interval for the Regression Coefficient of $z$ on $y$ ($\hat\beta_{zy}$) With Strongly Unbalanced Data (Bimodal, $\pm 80\%$) and 40\% Missing Data (MAR, $\lambda=0.5$)}
\begin{tabular}{llccccccccccccccc}
\hline\\[-1.8ex]
& & \multicolumn{5}{c}{Bias (\%)} & \multicolumn{5}{c}{Rel. RMSE} & \multicolumn{5}{c}{Coverage (\%)} \\ \cmidrule(r){3-7}\cmidrule(r){8-12}\cmidrule(r){13-17}
 &  & CD & \makecell{FCS-\\MAN} & \makecell{FCS-\\NJ} & \makecell{FCS-\\LAT} & JM & CD & \makecell{FCS-\\MAN} & \makecell{FCS-\\NJ} & \makecell{FCS-\\LAT} & JM & CD & \makecell{FCS-\\MAN} & \makecell{FCS-\\NJ} & \makecell{FCS-\\LAT} & \multicolumn{1}{c}{JM} \\ 
[0.4ex]\hline\\[-1.8ex]
& & \multicolumn{15}{c}{Small intraclass correlation $(\rho_{Iy}=.10)$} \\[0.6ex]\hline\\[-1.8ex]
\multicolumn{4}{l}{$\bar{n}=5$} \\  & \nopagebreak $\;J=50$  & $\phantom{-}21.1\phantom{0}$ & $\phantom{0}{-}9.2\phantom{0}$ & $\phantom{0}\phantom{-}6.5\phantom{0}$ & $\phantom{-}14.6\phantom{0}$ & $\phantom{0}{-}8.8\phantom{0}$ & $\phantom{0}1.76\phantom{0}$ & $\phantom{0}1.42\phantom{0}$ & $\phantom{0}1.69\phantom{0}$ & $\phantom{0}1.74\phantom{0}$ & $\phantom{0}1.55\phantom{0}$ & $\phantom{0}92.1\phantom{0}$ & $\phantom{0}94.5\phantom{0}$ & $\phantom{0}94.7\phantom{0}$ & $\phantom{0}94.0\phantom{0}$ & $\phantom{0}94.7\phantom{0}$ \\
 & \nopagebreak $\;J=200$  & $\phantom{0}\phantom{-}6.1\phantom{0}$ & ${-}16.8\phantom{0}$ & $\phantom{0}\phantom{-}4.6\phantom{0}$ & $\phantom{0}\phantom{-}7.8\phantom{0}$ & $\phantom{0}{-}4.9\phantom{0}$ & $\phantom{0}0.62\phantom{0}$ & $\phantom{0}0.63\phantom{0}$ & $\phantom{0}0.71\phantom{0}$ & $\phantom{0}0.73\phantom{0}$ & $\phantom{0}0.62\phantom{0}$ & $\phantom{0}94.9\phantom{0}$ & $\phantom{0}90.6\phantom{0}$ & $\phantom{0}95.6\phantom{0}$ & $\phantom{0}94.3\phantom{0}$ & $\phantom{0}94.7\phantom{0}$ \\
 & \nopagebreak $\;J=1000$  & $\phantom{0}\phantom{-}1.4\phantom{0}$ & ${-}19.1\phantom{0}$ & $\phantom{0}{-}0.1\phantom{0}$ & $\phantom{0}\phantom{-}1.5\phantom{0}$ & $\phantom{0}{-}1.6\phantom{0}$ & $\phantom{0}0.21\phantom{0}$ & $\phantom{0}0.37\phantom{0}$ & $\phantom{0}0.26\phantom{0}$ & $\phantom{0}0.27\phantom{0}$ & $\phantom{0}0.25\phantom{0}$ & $\phantom{0}95.5\phantom{0}$ & $\phantom{0}76.4\phantom{0}$ & $\phantom{0}94.2\phantom{0}$ & $\phantom{0}94.2\phantom{0}$ & $\phantom{0}94.8\phantom{0}$ \\
\multicolumn{4}{l}{$\bar{n}=20$} \\  & \nopagebreak $\;J=50$  & $\phantom{0}\phantom{-}3.0\phantom{0}$ & $\phantom{0}{-}9.5\phantom{0}$ & $\phantom{0}\phantom{-}2.6\phantom{0}$ & $\phantom{0}\phantom{-}4.4\phantom{0}$ & ${-}17.4\phantom{0}$ & $\phantom{0}0.64\phantom{0}$ & $\phantom{0}0.79\phantom{0}$ & $\phantom{0}0.94\phantom{0}$ & $\phantom{0}0.91\phantom{0}$ & $\phantom{0}0.74\phantom{0}$ & $\phantom{0}92.7\phantom{0}$ & $\phantom{0}94.8\phantom{0}$ & $\phantom{0}95.2\phantom{0}$ & $\phantom{0}92.4\phantom{0}$ & $\phantom{0}94.8\phantom{0}$ \\
 & \nopagebreak $\;J=200$  & $\phantom{0}\phantom{-}1.7\phantom{0}$ & $\phantom{0}{-}8.8\phantom{0}$ & $\phantom{0}\phantom{-}0.7\phantom{0}$ & $\phantom{0}\phantom{-}1.7\phantom{0}$ & $\phantom{0}{-}5.9\phantom{0}$ & $\phantom{0}0.29\phantom{0}$ & $\phantom{0}0.38\phantom{0}$ & $\phantom{0}0.40\phantom{0}$ & $\phantom{0}0.40\phantom{0}$ & $\phantom{0}0.37\phantom{0}$ & $\phantom{0}94.0\phantom{0}$ & $\phantom{0}93.9\phantom{0}$ & $\phantom{0}94.3\phantom{0}$ & $\phantom{0}93.3\phantom{0}$ & $\phantom{0}95.1\phantom{0}$ \\
 & \nopagebreak $\;J=1000$  & $\phantom{0}\phantom{-}0.4\phantom{0}$ & $\phantom{0}{-}9.4\phantom{0}$ & $\phantom{0}{-}0.6\phantom{0}$ & $\phantom{0}\phantom{-}0.3\phantom{0}$ & $\phantom{0}{-}1.4\phantom{0}$ & $\phantom{0}0.13\phantom{0}$ & $\phantom{0}0.22\phantom{0}$ & $\phantom{0}0.17\phantom{0}$ & $\phantom{0}0.17\phantom{0}$ & $\phantom{0}0.17\phantom{0}$ & $\phantom{0}95.2\phantom{0}$ & $\phantom{0}86.1\phantom{0}$ & $\phantom{0}95.1\phantom{0}$ & $\phantom{0}94.3\phantom{0}$ & $\phantom{0}94.3\phantom{0}$ \\
[0.5ex]\hline\\[-1.6ex] 
& & \multicolumn{15}{c}{Moderate intraclass correlation $(\rho_{Iy}=.30)$} \\[0.6ex]\hline\\[-1.8ex]
\multicolumn{4}{l}{$\bar{n}=5$} \\  & \nopagebreak $\;J=50$  & $\phantom{0}\phantom{-}9.2\phantom{0}$ & $\phantom{0}{-}8.1\phantom{0}$ & $\phantom{0}\phantom{-}6.3\phantom{0}$ & $\phantom{0}\phantom{-}7.8\phantom{0}$ & $\phantom{0}{-}6.9\phantom{0}$ & $\phantom{0}0.42\phantom{0}$ & $\phantom{0}0.50\phantom{0}$ & $\phantom{0}0.60\phantom{0}$ & $\phantom{0}0.59\phantom{0}$ & $\phantom{0}0.48\phantom{0}$ & $\phantom{0}93.0\phantom{0}$ & $\phantom{0}93.3\phantom{0}$ & $\phantom{0}94.1\phantom{0}$ & $\phantom{0}91.3\phantom{0}$ & $\phantom{0}94.7\phantom{0}$ \\
 & \nopagebreak $\;J=200$  & $\phantom{0}\phantom{-}1.4\phantom{0}$ & ${-}10.6\phantom{0}$ & $\phantom{0}{-}0.5\phantom{0}$ & $\phantom{0}\phantom{-}0.4\phantom{0}$ & $\phantom{0}{-}3.1\phantom{0}$ & $\phantom{0}0.16\phantom{0}$ & $\phantom{0}0.22\phantom{0}$ & $\phantom{0}0.22\phantom{0}$ & $\phantom{0}0.22\phantom{0}$ & $\phantom{0}0.22\phantom{0}$ & $\phantom{0}95.9\phantom{0}$ & $\phantom{0}94.1\phantom{0}$ & $\phantom{0}95.5\phantom{0}$ & $\phantom{0}95.7\phantom{0}$ & $\phantom{0}95.7\phantom{0}$ \\
 & \nopagebreak $\;J=1000$  & $\phantom{0}\phantom{-}0.2\phantom{0}$ & ${-}10.2\phantom{0}$ & $\phantom{0}{-}0.8\phantom{0}$ & $\phantom{0}\phantom{-}0.1\phantom{0}$ & $\phantom{0}{-}0.4\phantom{0}$ & $\phantom{0}0.07\phantom{0}$ & $\phantom{0}0.13\phantom{0}$ & $\phantom{0}0.10\phantom{0}$ & $\phantom{0}0.10\phantom{0}$ & $\phantom{0}0.10\phantom{0}$ & $\phantom{0}94.2\phantom{0}$ & $\phantom{0}84.5\phantom{0}$ & $\phantom{0}94.8\phantom{0}$ & $\phantom{0}94.2\phantom{0}$ & $\phantom{0}94.4\phantom{0}$ \\
\multicolumn{4}{l}{$\bar{n}=20$} \\  & \nopagebreak $\;J=50$  & $\phantom{0}\phantom{-}2.2\phantom{0}$ & $\phantom{0}{-}1.9\phantom{0}$ & $\phantom{0}\phantom{-}0.9\phantom{0}$ & $\phantom{0}\phantom{-}1.2\phantom{0}$ & $\phantom{0}{-}8.3\phantom{0}$ & $\phantom{0}0.28\phantom{0}$ & $\phantom{0}0.38\phantom{0}$ & $\phantom{0}0.43\phantom{0}$ & $\phantom{0}0.40\phantom{0}$ & $\phantom{0}0.36\phantom{0}$ & $\phantom{0}91.0\phantom{0}$ & $\phantom{0}93.4\phantom{0}$ & $\phantom{0}93.8\phantom{0}$ & $\phantom{0}93.4\phantom{0}$ & $\phantom{0}94.8\phantom{0}$ \\
 & \nopagebreak $\;J=200$  & $\phantom{0}\phantom{-}0.6\phantom{0}$ & $\phantom{0}{-}2.7\phantom{0}$ & $\phantom{0}{-}0.2\phantom{0}$ & $\phantom{0}\phantom{-}0.3\phantom{0}$ & $\phantom{0}{-}2.0\phantom{0}$ & $\phantom{0}0.13\phantom{0}$ & $\phantom{0}0.19\phantom{0}$ & $\phantom{0}0.19\phantom{0}$ & $\phantom{0}0.20\phantom{0}$ & $\phantom{0}0.19\phantom{0}$ & $\phantom{0}95.1\phantom{0}$ & $\phantom{0}94.4\phantom{0}$ & $\phantom{0}94.2\phantom{0}$ & $\phantom{0}93.2\phantom{0}$ & $\phantom{0}94.5\phantom{0}$ \\
 & \nopagebreak $\;J=1000$  & $\phantom{0}\phantom{-}0.2\phantom{0}$ & $\phantom{0}{-}1.9\phantom{0}$ & $\phantom{0}\phantom{-}0.0\phantom{0}$ & $\phantom{0}\phantom{-}0.4\phantom{0}$ & $\phantom{0}\phantom{-}0.1\phantom{0}$ & $\phantom{0}0.06\phantom{0}$ & $\phantom{0}0.08\phantom{0}$ & $\phantom{0}0.08\phantom{0}$ & $\phantom{0}0.08\phantom{0}$ & $\phantom{0}0.08\phantom{0}$ & $\phantom{0}95.9\phantom{0}$ & $\phantom{0}95.0\phantom{0}$ & $\phantom{0}95.1\phantom{0}$ & $\phantom{0}94.3\phantom{0}$ & $\phantom{0}95.1\phantom{0}$ \\
[0.5ex]\hline\\[-1.6ex] 
\end{tabular}
\begin{tablenotes}[para,flushleft]{\footnotesize \textit{Note.} $\bar{n}$ = average cluster size; $J$ = number of clusters; CD = complete data sets; LD = listwise deletion; FCS-SL = single-level FCS; FCS-MAN = two-level FCS with manifest cluster means; FCS-LAT = two-level FCS with latent cluster means; JM = joint modeling.}\end{tablenotes}
\end{threeparttable}
\end{sidewaystable}


\clearpage
\restoregeometry

\bibliographystyle{apacite}
\bibliography{../zotero_fullbib}

\end{document}
